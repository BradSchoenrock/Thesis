\chapter{Object \& Event Reconstruction}
\label{SECTION-OBJ}

High energy physics is messy. Having particles that interact with several different layers of the detector makes our particle identification and reconstruction complex but possible. By using tracking from the inner detector, energy measurements from the calorimeters, more tracking from the muon spectrometers, and by looking at global variables that have to do with event kinematics we can ensure quality physics objects for analysis. This analysis in particular uses a wide variety of objects including electrons, muons, light jets, heavy jets, \met, reconstructed $W$-bosons and $Z$-bosons, and the heaviest of all fundamental particles with one of the most unique signatures, the top quark. 


\section{Electron Reconstruction}
\label{SECTION-OBJ-EL}

Electrons use information from the inner detector and the electromagnetic calorimeter. Electrons are  among the more scrutinized reconstructed objects because hadronic jets, photons, and taus can fake electrons heightening the importance of quality control. To achieve this electron candidates are required to be within $|(\eta)| < 2.47$ as measured by the tracks in the ID with \PT~$>$ 7 GeV as measured by the calorimeter. Requirements on the transverse impact parameter (d0) and the longitudinal impact parameter (z0) constrain the degree to which tracks in the ID are allowed to vary from the interaction point while still being counted as part of the electron object. Ratios of energy measured in the electromagnetic calorimeter and the hadronic calorimeter are also used to reject jets that interact with the electromagnetic calorimeter and could fake electrons. Ratios of energy in varying window sizes in the electromagnetic calorimeter are also used to help distinguish other activity such as pions from electrons. These reconstructed electrons are then trigger matched with L1 EM objects and HLT electrons~\cite{Electrons}~\cite{ElEffiRecomm}~\cite{ELECTRON-RECO}.


\section{Muon Reconstruction}
\label{SECTION-OBJ-MU}

Muons can use information from the inner detector, the muon systems, and to a lesser extent the calorimeters. Muons are not stopped by the detector making full calorimetry impossible so their energy must be determined by their curvature in the magnetic field set up by ATLAS's namesake toroidal magnet system. 

There are four algorithms that are used in muon reconstruction and define the tightness of the muons provided to analyzers. The Moore algorithm/Muid standalone method starts from hits in the muon spectrometer and traces them back to a primary vertex to create a standalone muon. The Muid Combined method combines an inner detector track with a muon spectrometer track to produce a combined muon. The MuGirl method performs a search for segments and tracks in the muon spectrometer using an inner detector track as a seed. If the refit performed by the MuGirl method is successful a Combined muon is made, if not then a tagged muon is made. The fourth is the MuTagIMO method which identifies muons by associating an inner detector track with a standalone muon in a similar way as the Moore/Muid standalone method to produce a tagged muon. Muons from all of these algorithms are added after overlapping definitions are accounted for. 

The tightness of the muon is defined based on which of the four algorithms have been successful in constructing the muon. A tight muon is one where the Muid combined and MuGirl muons have been successfully combined. A medium muon includes all standalone muons. Loose muons are all muons found by tagging algorithms and have an inner detector track with silicon hits associated with it. Lastly a very loose muon includes MuTagIMO muons which are allowed to not have any silicon hits, but have TRT only tracks in the inner detector~\cite{Muons}~\cite{Aad:2016jkr}.


\section{Jet Reconstruction}
\label{SECTION-OBJ-JET}

Hadronizing quarks and gluons interact with the inner tracker, the electromagnetic calorimeter, and the hadronic calorimeter. We use this information to reconstruct the location and energy of the jets. 

There are several jet reconstruction algorithms but the most common are described by equations \ref{EQ-OBJ-ANTIKT} and \ref{EQ-OBJ-ANTIKTBEAM}

\begin{equation}
\label{EQ-OBJ-ANTIKT}
d_{ij} = min(p^{2p}_{T,i},p^{2p}_{T,j})\frac{\Delta\eta^{2}_{ij}+\Delta\phi^{2}_{ij}}{R^{2}}
\end{equation}

\begin{equation}
\label{EQ-OBJ-ANTIKTBEAM}
d_{i} = p^{2p}_{T}
\end{equation}

where the exponent p is either 1, 0, or -1 which correspond to the kt, Cambridge Achen, or anti-kt algorithm~\cite{AntiKt}. These algorithms work by considering every cell of the detector as an object in a list enumerated by $i$ and $j$ and considering pairs of these cells for combination by this equation. If the minimum $d_{ij}$ is smaller than $d_i$ then the two objects are combined into one object. If a $d_i$ is smaller then that object is removed and considered a jet. This continues until all objects are removed from the list. The parameter R sets the separation distance between two jets.  

The first of these three algorithms is the kt algorithm which gives irregular jets, but is more theoretically sound than simple cone drawing. The Cambridge Aachen algorithm gives jets that are slightly more regular but are larger than their kt counterparts, while the anti-kt algorithm gives jets that are more regular than the kt. This is the algorithm of choice for ATLAS and this analysis considers anti-kt jets with an R parameter of 0.4. 

After the selection of jets with the anti-kt algorithm corrections are applied to each jet based on the jet's position and $p_t$ to correct for specific detector effects. These corrected jets have a series of quality cuts applied including a check for unphysical negative energy jets and electron isolation requirements to ensure that an electron is not being double counted as a jet. 


\subsection{Jet \btag ging}
\label{SECTION-OBJ-JET-BTAG}

b-jets are unique because the b quark decays after the hadronization process begins but before it interacts with the detector. This creates a secondary vertex (displaced by a few millimeters) which can be found by looking at the tracking information from the inner detector. This is the heart of \btag ging algorithms used in \atlas. The \btag ging algorithm used in \atlas~ for run 2 is the MV2c20 algorithm. When this algorithm is applied to anti-kt jets with an R parameter of 0.4 the $p_t$ of the jet is required to be greater than 20GeV, the $\eta$ is required to be less than 2.5, and for high \PT~jets (greater than 50GeV) the Jet Vertex Tagger (JVT) output variable is required to be greater than 0.64. The MV2c20 algorithm is a neural network analysis of \btag ging algorithms used in \atlas~ to create a single discriminating variable that can distinguish between jets originating from a b-quark and all other jets.~\cite{ATL-PHYS-PUB-2015-022}. 



\section{$Z$ boson}
\label{SECTION-OBJ-Z}

Now that we have the leptons defined we can reconstruct the intermediate $Z$ boson. To do this we require that the $Z$-boson be constructed from an Opposite-Sign Same-Flavor (OSSF) lepton pair. With three leptons (which can be electrons or muons) we can have 1, 2, or 0 OSSF pairs. These are two fundametal cuts we make in Section~\ref{SECTION-PRESELECTION}. When there is one OSSF pair the $Z$-boson is reconstructed with those, and the remaining lepton is used to reconstruct the $W$-boson. When there are two OSSF pairs then the OSSF pair which reconstructs the $Z$-boson mass more closely is considered, and the remaining lepton reconstructs the $W$-boson. The last case with no OSSF pair represents either a charge misidentification of a lepton or a fake lepton whose charge couldn't be properly reconstructed. In this case the event is rejected. 



\section{Missing Transverse Energy and the $W$-boson}
\label{SECTION-OBJ-MET}

For each event we apply conservation of momentum in the transverse plane of the detector to obtain what is known as missing transverse energy (\met). \met is commonly used as a stand in for some information on neutrinos. While we can not apply the same method to find the neutrino $p_z$, because the colliding partons do not necessarily have balanced z momenta, we can make the assumption that it came from a $W$-boson and begin with conservation of momentum of the $W$-boson decay vertex. 

\begin{equation}
\label{EQ-MET-one}
p_{w}^{\mu}=p_{\nu}^{\mu}+p_{l}^{\mu}
\end{equation}

The lepton up for consideration is the one that did not come from the $Z$-boson. Solving for the $p_z$ of the neutrino we come to the following quadratic.

\begin{equation}
\label{EQ-MET-two}
p_{z\nu}=\frac{\alpha  p_{zl}}{p_{tl}^{2}} \pm \sqrt{\frac{\alpha^2 p_{zl}^2}{p_{tl}^4} - \frac{E_l^2 p_{t\nu}^2-\alpha^2}{p_{tl}^2}}
\end{equation}

\begin{equation}
\label{EQ-MET-three}
\alpha = \frac{m_w^2}{2}+cos(\Delta \phi) p_{t\nu} p_{tl}
\end{equation}

Equation~\ref{EQ-MET-two} can have two, one, or no real solutions. If it has two real solutions the one with lower $p_z$ is chosen. In the case where there are no real solutions the measured \met is scaled to the point where one real solution is found. The measured \met and azimuthal angle ($\phi$) of the \met and the reconstructed $p_z$, along with the fact that neutrinos are functionally massless, define the neutrino four vector. 

Now that the neutrino is defined we have reconstructed the four vectors for all of our final state particles. We can reconstruct the $W$-boson, but the mass will already be defined because we assumed the $W$-boson mass in reconstructing the neutrino. For this reason the experimental variable $m_{T}^W$, which is the transverse mass of the $W$-boson, is defined in equation~\ref{EQ-MET-four}. It is used to help distinguish events that have a real $W$-boson from events that don't~\cite{MET}. 

\begin{equation}
\label{EQ-MET-four}
m_{tw}^2 = 2 E_{Tl} E_{Tv} (1-cos(\Delta \phi))
\end{equation}


\section{Reconstructing the top-quark}
\label{SECTION-OBJ-TOP}

Once the $W$-boson is reconstructed, and the b-jet selected the reconstruction of the top-quark is simply the result of the addition of the four vectors of the two. The top-quark decay has properties that are used to help distinguish signal and backgrounds even though this analysis has significant top-quark backgrounds as seen in Chapter~\ref{SECTION-ANALYSIS}. 

