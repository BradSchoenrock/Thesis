\chapter{Object \& Event Reconstruction}
\label{SECTION-OBJ}

I was always interested in figuring things out. I'd do experiments, like combining things I found around the house to see what would happen if I put them together. - Alan Alda

\vspace{5mm} %5mm vertical space

In practice, high energy physics is messy. Having particles that interact with several different layers of the detector makes our particle identification and reconstruction complex but possible. By using tracking from the inner detector, energy measurements from the calorimeters, more tracking from the muon spectrometers, and by looking at global variables that have to do with event kinematic properties we can ensure quality physics objects for analysis. This analysis in particular uses a wide variety of objects including electrons, muons, jets, b-jets, \met, reconstructed \aw s and \az s, and the heaviest of all fundamental particles with one of the most unique signatures, the \at. 


\section{Electron Reconstruction}
\label{SECTION-OBJ-EL}

Electrons use information from the inner detector and the electromagnetic calorimeter. Electrons are among the more scrutinized reconstructed objects because hadronic jets, photons, and taus can fake electrons heightening the importance of quality control. Electron candidates are required to be within $|\eta | < 2.47$ as measured by the tracks in the inner detector with \PT~$>$ 7~GeV as measured by an energy cluster (energy deposits within dR of 0.4) in the calorimeter. Requirements on the transverse impact parameter (d0) and the longitudinal impact parameter (z0) constrain the degree to which tracks in the ID are allowed to vary from the interaction point while still being counted as part of the electron object. Ratios of energy measured in the electromagnetic calorimeter and the hadronic calorimeter are also used to reject jets that interact within the electromagnetic calorimeter and could fake electrons. Ratios of energy in varying window sizes in the electromagnetic calorimeter are also used to help distinguish other activity such as pions from electrons. These reconstructed electrons are required to have \PT~$>$ 25~GeV, then trigger matched with L1 EM objects and HLT electrons and required to be isolated from hadronic activity~\cite{Electrons,ElEffiRecomm,ELECTRON-RECO}.


\section{Muon Reconstruction}
\label{SECTION-OBJ-MU}

Muons use information from the inner detector, the muon systems, and to a lesser extent the calorimeters. Muons are not stopped by the detector, making full calorimetry impossible, so their energy must be determined by their curvature in the magnetic field set up by \atlas 's namesake toroidal magnet system. 

There are four algorithms that are used in muon reconstruction and define the requirements for various levels of the muon reconstruction provided to analyzers. One method used starts from hits in the muon spectrometer and traces them back to a primary vertex to create a standalone muon. Another method combines an inner detector track with a muon spectrometer track to produce a combined muon. Yet another method performs a search for segments and tracks in the muon spectrometer using an inner detector track as a seed, and if the refit performed is successful, then a Combined muon is made, if not then a tagged muon is made. The fourth method identifies muons by associating an inner detector track with a standalone muon in a similar way to the first  method to produce a tagged muon. Muons from all of these algorithms are added after overlapping definitions are accounted for~\cite{Muons,Aad:2016jkr}. 

%The tightness of the muon is defined based on which of the four algorithms have been successful in constructing the muon. A tight muon is one where the Muid combined and MuGirl muons have been successfully combined. A medium muon includes all standalone muons. Loose muons are all muons found by tagging algorithms and have an inner detector track with silicon hits associated with it. Lastly a very loose muon includes MuTagIMO muons which are allowed to not have any silicon hits, but have TRT only tracks in the inner detector. Reconstructed muons are required to have \PT~$>$ 25~GeV and be isolated from hadronic activity to avoid counting muons from $b$-meson decays~\cite{Muons,Aad:2016jkr}.


\section{Jet Reconstruction}
\label{SECTION-OBJ-JET}

Hadronizing quarks and gluons interact with the inner detector, the electromagnetic calorimeter, and the hadronic calorimeter. We use this information to reconstruct the location and energy of the jets. 

There are several jet reconstruction algorithms but the most common are described by equations \ref{EQ-OBJ-ANTIKT} and \ref{EQ-OBJ-ANTIKTBEAM}, 

\begin{equation}
\label{EQ-OBJ-ANTIKT}
d_{ij} = min(p^{2p}_{T,i},p^{2p}_{T,j})\frac{\Delta\eta^{2}_{ij}+\Delta\phi^{2}_{ij}}{R^{2}},
\end{equation}

\begin{equation}
\label{EQ-OBJ-ANTIKTBEAM}
d_{i} = p^{2p}_{T},
\end{equation}

\noindent where \PT is the transverse momentum to the power of $2p$ which is either 1, 0, or -1. These three values correspond to the kt, Cambridge Achen, or anti-kt algorithm~\cite{AntiKt}. These algorithms work by considering every cell of the detector as an object in a list enumerated by $i$ and $j$ and considering pairs of these cells for combination by this equation. If the minimum $d_{ij}$ is smaller than $d_i$ then the two objects are combined into one object. If a $d_i$ is smaller then that object is removed and considered a jet. This continues until all objects are removed from the list. The parameter $R$ sets the separation distance between two jets.  

The first of these three algorithms is the kt algorithm which gives irregular jets, but is more theoretically sound than simple cone drawing. The Cambridge Aachen algorithm gives jets that are slightly more regular but are larger than their kt counterparts, while the anti-kt algorithm gives jets that are more regular than the kt. The anti-kt algorithm is the one chosen for \atlas~and this analysis considers anti-kt jets with an $R$ parameter of 0.4, and for high \PT~jets (greater than 50~GeV) the Jet Vertex Tagger (JVT) output variable is required to be greater than 0.64~\cite{ATLAS-CONF-2014-018}. 

After the selection of jets with the anti-kt algorithm, corrections are applied to each jet based on the jet's position and \PT~to correct for specific detector effects. These corrected jets have a series of quality cuts applied including a minimum \PT~threshold of 30~GeV, a check for unphysical negative energy jets, an eta range $|\eta| < 4.5$ and electron isolation requirements to ensure that an electron is not being double counted as a jet. 


\subsection{Jet \btag ging}
\label{SECTION-OBJ-JET-BTAG}

Jets that are \btag ed are unique because the \ab~decays after the hadronization process begins but before it interacts with the detector. This creates a secondary vertex (displaced by a few millimeters) which can be found by looking at the tracking information from the inner detector~\cite{Coccaro:2011np}. This can be seen diagrammatically in Figure~\ref{FIGURE-bjet}. The \btag ging algorithm used in \atlas~for Run 2 is the MV2c20 algorithm~\cite{Hetherly:2032280}. When this algorithm is applied to anti-kt jets with an $R$ parameter of 0.4 the \PT~of the jet is required to be greater than 20~GeV, the $\eta$ is required to be less than 2.5. The MV2c20 algorithm is a neural network analysis of \btag ging algorithms used in \atlas~to create a single discriminating variable that can distinguish between jets originating from a \ab~and all other jets.~\cite{ATL-PHYS-PUB-2015-022}. 

\LARGEFIG{Picture-b-tagging-2}{Diagram illustrating a displaced vertex.~\cite{Coccaro:2011np}.}{FIGURE-bjet}


\section{\az}
\label{SECTION-OBJ-Z}

Now that we have the leptons defined we can reconstruct the intermediate \az. We require that the \az~be constructed from an Opposite-Sign Same-Flavor (OSSF) lepton pair. With three leptons (which can be electrons or muons) we can have 0, 1, or 2 OSSF pairs. There are two fundamental cuts we make in Section~\ref{SECTION-PRESELECTION}. When there is one OSSF pair, the \az~is reconstructed with those, and the remaining lepton is used to reconstruct the \aw. When there are two OSSF pairs, then the OSSF pair which reconstructs the \azhyph~mass more closely is considered, and the remaining lepton reconstructs the \aw. The last case with no OSSF pair represents either a charge misidentification of a lepton or a jet mis-reconstructed as a lepton whose charge couldn't be properly reconstructed. In this case the event is rejected. 



\section{Missing Transverse Energy (\met ) and the \aw}
\label{SECTION-OBJ-MET}

For each event we apply conservation of momentum in the transverse plane of the detector to obtain what is known as missing transverse energy (\met). We often use \met~as as a stand in for some information on neutrinos. While we can not apply the same method to find the neutrino $p_z$, because the colliding partons do not necessarily have balanced $z$ momenta, we can make the assumption that it came from a \aw~and begin with conservation of four-momentum of the \aw~decay vertex with the momentum of the \aw~$p_{w}^{\mu}$, the momentum of the neutrino $p_{\nu}^{\mu}$, and the momentum of the lepton $p_{l}^{\mu}$. 


\begin{equation}
\label{EQ-MET-one}
p_{w}^{\mu}=p_{\nu}^{\mu}+p_{l}^{\mu}
\end{equation}

The lepton up for consideration is the one that did not come from the \az. Solving for the $p_z$ of the neutrino we obtain the following quadratic:

\begin{equation}
\label{EQ-MET-two}
p_{z\nu}=\frac{\alpha  p_{zl}}{p_{tl}^{2}} \pm \sqrt{\frac{\alpha^2 p_{zl}^2}{p_{tl}^4} - \frac{E_l^2 p_{t\nu}^2-\alpha^2}{p_{tl}^2}}
\end{equation}

\begin{equation}
\label{EQ-MET-three}
\alpha = \frac{m_w^2}{2}+cos(\Delta \phi) p_{t\nu} p_{tl}.
\end{equation}

Equation~\ref{EQ-MET-two} can have two, one, or no real solutions. If it has two real solutions, the one with lower $p_z$ is chosen. In the case where there are no real solutions the measured \met~is scaled to the point where one real solution is found. The measured \met~and azimuthal angle ($\phi$) of the \met~and the reconstructed $p_z$, and because neutrinos are functionally massless, define the neutrino four vector. 

Now that the neutrino is defined, we have reconstructed the four vectors for all of our final state particles. We can reconstruct the \aw, but the mass will already be defined because we assumed the \awhyph~mass in reconstructing the neutrino. For this reason, and to remain independent of any assumptions regarding $z$ momenta, the experimental variable \wtm is used. This is the transverse mass of the \aw, which as defined by the energies of the lepton and neutrino ($E_{Tl}, E_{Tv}$) and the angle between them($\Delta \phi$) in equation~\ref{EQ-wtm}. It is used to help distinguish events that have a real \aw~from events that don't~\cite{MET,TOPMET}. 

\begin{equation}
\label{EQ-wtm}
m_{\textrm{T}}^{W^{2}} = 2 E_{Tl} E_{Tv} (1-cos(\Delta \phi))
\end{equation}

The helicity of the \aw~is another variable which can be used to distinguish events with a real \aw~from ones that do not. Furthermore, \aw s from top decays have correlations with the \bjet~because they carry forward information about the spin of the \at. Helicity is defined as the projection of the spin vector($s$) onto the momentum vector($p$) as defined in equation~\ref{EQ-hel}.


\begin{equation}
\label{EQ-hel}
H = \frac{s \cdot p }{|s \cdot p|}
\end{equation}



\section{Reconstructing the \at}
\label{SECTION-OBJ-TOP}

Once the \aw~is reconstructed and the \bjet~selected, the reconstruction of the \at~is simply the result of the addition of the four vectors of the two objects. The \at~decay has properties that are used to help distinguish signal and backgrounds, even though this analysis has a large background contribution from \athyph~pair production as seen in Chapter~\ref{SECTION-ANALYSIS}. One of those properties is the \athyph~polarization, which is unique because the \at~decays before its spin can be flipped by the strong interaction allowing this to be measured in single top production. The \athyph~polarization is evaluated as the angle between the lepton from the \at~decay and the polarization axis, in the \at~rest frame~\cite{Schwienhorst:2010je}. The polarization of the \at~is dependent on what reference frame you are measuring from. Two common reference frames for this are the Optimal Basis and the Helicity Basis. The Helicity Basis, which is the most common basis of consideration, takes the \at~rest frame. The Optimal basis is the helicity basis measurement boosted into the reference frame of the jet that does not come from the top decay in the case of single top \tchan. 
