\chapter{Introduction}
\label{SECTION-INTRO}

If you have knowledge, let others light their candles in it. -Margaret Fuller

\vspace{5mm} %5mm vertical space

High energy physics is concerned with obtaining the most fundamental understanding of the universe. In practice this means categorizing all fundamental particles and their interactions in order to understand what the world is made of. Assorted scientific fields question what the world is made of in various detail. Chemistry asks which atoms and molecules comprise the things around us, nuclear physics investigates what makes up the nuclei of atoms and how nuclei are formed, and high energy physics studies what we currently think are the most fundamental particles in existence. In order to understand high energy physics we need a framework to describe the elementary particles and their interactions. This framework is referred to as the Standard Model (SM). 


\section{The Standard Model}
\label{SECTION-THEORY-SM}

The SM of high energy physics has been among the most successful theories of the past century. It has been tested again and again and has encountered few unexplained anomalies. It started as an effort to combine the fundamental forces we know into one overarching theory. Electricity and magnetism had been combined into electromagnetism long ago and in the last century the SM was developed. Electromagnetism was combined with weak interactions, followed by the inclusion of the Higgs mechanism and strong interactions to form the SM we know today~\cite{Griffiths,QFT-PS,QFT-IZ}. 

\LARGEFIG{StdMdl-wheel}{The SM of high energy physics~\cite{SMPic}.}{FIGURE-STDMDL}
%\LARGEFIG{StdMdl}{The SM of high energy physics~\cite{Figure-StdMdl}.}{FIGURE-STDMDL}

The SM particles are classified based on their properties and interactions and are shown in Figure~\ref{FIGURE-STDMDL}. One way we can classify particles is by their spin. A particle with half integer spin is called a fermion (colored red or green in Figure~\ref{FIGURE-STDMDL}) while a particle with integer spin is a boson (colored blue or black in Figure~\ref{FIGURE-STDMDL}). All discovered fundamental particles are either spin 0, spin $\frac{1}{2}$ or spin 1. We further break down the fermions into two categories, the first set are the leptons which have an electric charge $\pm 1$ (electron, muon, and tau) to interact with the electroweak force, and three neutral neutrinos which only interact via the weak force. The other type of fermion is the quark. Quarks interact via the weak, electromagnetic, and strong forces carrying half integer spins, $\pm \frac{1}{3}$ or $\pm \frac{2}{3}$ electrical charges, and color charges. The strong force, at low energies, imparts color confinement onto individual quarks which binds them together in mesons (quark antiquark pairs) or baryons (three quark systems such as the proton or neutron). If quarks are high enough energy they undergo a process known as hadronization where new quark-antiquark pairs are created from that energy until all that remain are many mesons and baryons. Quarks also interact electromagnetically and weakly like their charged leptonic counterparts. The vector bosons (spin 1) moderate the forces involved in the Standard Model. The gluon interacts via the strong force, the photon and the $W^\pm$ interact electromagnetically, and the $W^\pm$ and $Z$~bosons interact weakly. The final particle we have is the recently discovered Higgs boson~\cite{Aad:2012tfa} which took nearly 50~years to discover. A history of particle discovery can be seen in Figure~\ref{FIGURE-TIMELINE}. 

\VLARGEFIG{Timeline}{History of high energy physics illustrating the time it took from theorizing the existence of the particles until discovery~\cite{Timeline}.}{FIGURE-TIMELINE}

A deeper understanding of the SM can be obtained through the Lagrange density~\cite{QFT-PS}, 
 
%\mathscr{L}
\begin{equation}
\begin{split}%\begin{multline}
\mathscr{L} = & -\frac{1}{2}tr[G_{\mu \nu}G^{\mu \nu}] -\frac{1}{2}tr[W_{\mu \nu}W^{\mu \nu}] -\frac{1}{4} B_{\mu \nu} B^{\mu \nu} \\ 
& + i \bar{\psi} [ \cancel{D} -m] \psi + \bar{\psi}_{iL}y_{ij}\psi_{jR}\phi  +  h.c. +|D_\mu \phi|^2 - V(\phi)
\end{split}%\end{multline}
\label{EQUATION-STDLAG}
\end{equation}

\noindent where $\psi$ is the Dirac field with a sum over the matter particles with $L$ denoting left-handed particles and $R$ denoting right-handed particles, $\phi$ is the Higgs field, $y_{ij}$ are the Yukawa couplings, $\cancel{D}$ is the covariant derivative defined through Dirac slash notation as 

\begin{equation}
\cancel{D}=\gamma_\mu D^\mu
\end{equation}

\begin{equation}
D^\mu=\partial^\mu - i g_S T^a G^{a\mu} - i Y g_Y B^\mu - \frac{i g_L}{2}\sigma ^a W^{a \mu} 
\end{equation}


\noindent where Y is the hypercharge of a particle. Hypercharge for left-handed particles are $-\frac{1}{2}$ for leptons and $\frac{1}{6}$ for quarks while for right-handed particles hypercharge is the electric charge of the particle. The tensor $T^a$ is defined as half of $\lambda_a$ (which are the 8 Gell Mann matrices) for quarks, and is zero for leptons. The $\sigma_a$ matrices are the 3 Pauli matrices. The covariant derivative also applies to the Higgs boson, with $T^a=0$ (no coupling to strong force) and $Y=-\frac{1}{2}$. The gauge field strength tensor is denoted by $B_{\mu \nu}$ and is defined by 
 
\begin{equation}
B_{\mu \nu} = \frac{\partial B_\nu}{\partial \mu} - \frac{\partial B_\mu}{\partial \nu}
\end{equation}


\noindent where $B_\nu$ is the hypercharge gauge potential. The QCD field tensor $G_{\mu\nu}$ defines the gluon fields and are defined as 


\begin{equation}
G_{\mu\nu}=\frac{\lambda_a}{2} G_{\mu\nu}^a=\frac{i}{g_s}[D_\mu ,D_\nu]
\end{equation}
\begin{equation}
G_{\mu\nu}^a=\partial_\mu G_\nu^ a - \partial_\nu G_\mu^a + g_s f^{abc} G_{\mu b} G_{\nu c}
\end{equation}

\noindent where $G_\nu$ is the strong gauge potential. The weak tensor $W_{\mu\nu}$ is defined as 

\begin{equation}
W_{\mu\nu}=\frac{\sigma_a}{2} W_{\mu\nu}^a=\frac{i}{g}[D_\mu ,D_\nu]
\end{equation}
\begin{equation}
W_{\mu\nu}^a=\partial_\mu W_\nu^ a - \partial_\nu W_\mu^a + g f^{abc} W_{\mu b} W_{\nu c}
\end{equation}

\noindent where $W_\nu$ is the weak gauge potential. Note that the $W^{a \mu}$ term in equation 1.3 only couples to left-handed particles. 

The SM Lagrangian in equation \ref{EQUATION-STDLAG} contains a lot of information on the SM in one concise equation. The first three terms in the Lagrange density formula contains the strong and electroweak forces, the fourth term describes how the particles interact with these fields, the fifth term and its Hermitian conjugate ($h.c.$) describes how the fermions get their masses (note the $\phi$ dependence means that the Higgs contributes but does not determine the value of their masses), the next to last term describes how the Higgs gives mass to the bosons, and the final term is the Higgs potential~\cite{Griffiths,QFT-PS,QFT-IZ}. 

This formulation represents a group with a $SU(3) \times SU(2) \times U(1)$ symmetry. The $SU(3)$ represents the strong force, with the threefold symmetry in color charge. The eight generators of this $SU(3)$ symmetry correspond to the various color combinations of the gluon which can be mathematically represented by the Gell-Mann matrices. The $SU(2) \times U(1)$ represents the electroweak force which unified electricity, magnetism, and the weak forces whose generators can be represented by the Pauli matrices. The Higgs mechanism breaks this symmetry and this phenomenon is known as electroweak symmetry breaking. By breaking this symmetry the massless electroweak bosons ($W_1$, $W_2$, $W_3$), and the hypercharge boson ($B$) are recombined as the massive $W^+$, $W^-$ (which are linear combinations of $W_1$ and $W_2$), the massive $Z^0$ (which is a linear combination of $W_3$ and $B$) and the massless photon (which is a combination of $W_3$ and $B$ as well). The Higgs doublet has 4 degrees of freedom, three of which are consumed by the longitudinal components of the massive $W^+$, $W^-$, and $Z^0$. The remaining degree of freedom is a neutral scalar particle, the Higgs boson~\cite{GROUPTHEORY,GROUPTHEORY2}. 

\section{Feynman Diagrams}
\label{SECTION-FEYNMANN-DIAGRAMS}

Thanks to Richard Feynman we can obtain an intuitive understanding of particles and their interactions through Feynman diagrams~\cite{QFT-PS}. We can view these pictures as having direct correlation with the processes involved and even set up the relevant equations to compute the scattering amplitude of a particular process. In these diagrams we compact the spacial dimensions into one vertical axis while time is represented on the horizontal axis. %A classic example is Bhabha scattering in Figure~\ref{FIGURE-baba}.

%\MEDIUMFIG{BaBa}{Bhabha scattering~\cite{JAXO}.}{FIGURE-baba}

%In Figure~\ref{FIGURE-baba} the electron and positron meet, annihilate into a photon or $Z$~boson, and then decay into an electron positron pair. Notice that the electron lines have a directionality to them. The arrow pointing forward in time is the particle (electron) while the arrow pointing backward in time is interpreted as the antiparticle (the positron). Diagrams with loops and/or diagrams which have gluons radiating from the incoming quarks (known as initial-state radiation or ISR) or have gluons radiating from the outgoing particles (known as final-state radiation FSR) that still have the same underlying event structure are known as higher order diagrams, which can be seen in Figure~\ref{FIGURE-babaloop} or~\ref{FIGURE-OneLoopbaba}. 

%This Feynman diagram represents the annihilation term in the Bhabha scattering process but, when doing calculations, there is another term that corresponds to one electron emitting a photon and the other electron absorbing it (Moller scattering). Together these diagrams will give a LO calculation of the cross-section for Bhabha scattering. There are also diagrams with loops and/or diagrams which have initial-state radiation(ISR) or final-state radiation(FSR) that still have the same underlying event structure as seen in Figure~\ref{FIGURE-babaloop} or~\ref{FIGURE-OneLoopbaba}. 

%\SMALLFIG{babaloop}{An example of an NLO diagram for Bhabha scattering~\cite{Actis:2009uq}.}{FIGURE-babaloop}

%\FIG{OneLoopbaba}{Examples of NNLO diagrams for Bhabha scattering with loops and real emissions~\cite{Actis:2009uq}.}{FIGURE-OneLoopbaba}

The \xs~for a particular scattering process is defined as the ratio of number of particles scattered per unit time ($dN(t)$) to number of particles passing through a defined area per unit time ($n$), see equation~\ref{EQUATION-xsecs}. 

\begin{equation}
d\sigma =dN(t)/n 
\label{EQUATION-xsecs}
\end{equation}

\begin{equation}
N_{events}=\sigma \int L(t) dt
\label{EQUATION-NEV}
\end{equation}

\begin{equation}
 L(t)= \frac{n_1 n_2}{4\pi \sigma_x \sigma_y} 
\label{EQUATION-LUMIN}
\end{equation}

Informally this is how probable that process is to occur in each interaction. The standard unit for \xs~($\sigma$) is the Barn ($10^{-24}\ $cm$^{2}$) but we commonly use picobarn or femtobarn to describe \xs s. Consequently we define the beam intensity, or luminosity ($L$), in inverse picobarns or inverse femtobarns. That way we can easily calculate the expected number of events from equation \ref{EQUATION-NEV}. Luminosity can be calculated from the beam parameters of the accelerator by equation~\ref{EQUATION-LUMIN} where $n_1$ and $n_2$ are the number of particles in each beam, and $\sigma_x$ and $\sigma_y$ are the Gaussian RMS beam sizes in their respective directions~\cite{QFT-PS}.


%\subsection{The Higgs Boson}
%\label{SECTION-HIGGS}

%The most recently discovered particle was the Higgs Boson. on July 2, 2012 D0 and CDF at the Tevatron announced they had evidence of a particle resembling the SM Higgs boson with a mass of approximately 125 GeV~\cite{d0higgs}~\cite{tevatron-bbbar}. Two days later the \atlas and CMS experiments jointly announced the discovery of a particle resembling the SM Higgs boson with a mass of approximately 125 GeV that has since been verified to be the SM Higgs boson. The excess in \atlas data (solid line) at 125 GeV can be seen as distinctly above the one standard deviation band (green) and the two standard deviation band (yellow) away from the expected background only hypothesis (dashed line inside the bands) in Figure~\ref{FIGURE-HIGGS-DISC}~\cite{Higgs\atlas}~\cite{HiggsCMS}. 

%\VLARGEFIG{HiggsDisc}{A graph showing an excess in \atlas data above the expected limit without the Higgs boson at 125 GeV~\cite{HiggsATLAS}.}{FIGURE-HIGGS-DISC}

%\VLARGEFIG{Timeline}{History of high energy physics illustrating the time it took from theorizing the existence of the particles until discovery~\cite{Timeline}.}{FIGURE-TIMELINE}

%This concluded a nearly five decade search that stands as the longest search for a fundamental particle that has been discovered. The history of fundamental particle discovery can be seen in Figure~\ref{FIGURE-TIMELINE}. The Theory of electroweak symmetry breaking and the subsequent search for the Higgs started with the work by Nambu and Goldstone in 1960 which predicted the massless Nambu-Goldstone boson as a consequence of electroweak symmetry breaking~\cite{PhysRev.117.648}. Anderson pointed out in 1963 that in non-relativistic theories these massless Nambu-Goldstone bosons could be reparametrize to give rise to massive bosons~\cite{PhysRev.130.439}. All that was left was to show that this could be done in relativistic theories as well, and that was accomplished independently by Higgs~\cite{Higgs:1964ia}~\cite{Higgs:1964pj}~\cite{Higgs:1966ev}, Englert and Brout~\cite{Englert:1964et}, and Guralnik Hagen and Kibble~\cite{Guralnik:1964eu} in 1964. In 2013 Peter Higgs and Francois Englert shared the Nobel prize ``for the theoretical discovery of a mechanism that contributes to our understanding of the origin of mass of subatomic particles, and which recently was confirmed through the discovery of the predicted fundamental particle, by the \atlas and CMS experiments at CERN's Large Hadron Collider''. 


\section{Top Quark Physics}
\label{SECTION-TOP-PHYSICS}

The \at~is of specific interest to high energy physics and in particular this thesis. It has a mass that makes it the heaviest fundamental particle that we know today, 173.2~GeV, which is about the mass of a gold atom~\cite{topmass}.

 Due to the \at's large natural width, which is defined as the probability per unit time that a particle decays, it is the only quark with an observed decay lifetime ($10^{-25}$~s) shorter than the timescale for strong interactions ($10^{-24}$~s)~\cite{TOPWidth:1993,CDF-topwidth,D0-topwidth,D0TopWidth:2010}. 

\noindent Because of this, and that the \ckm~matrix element \vtb~(\vtb~corresponds to the strength of the \at~flavor changing to bottom quark through a weak decay) is approximately equal to 1, the \at~almost always decays into a \aw~and a \ab~before it hadronizes into a jet~\cite{sgtopvtb,QFT-PS}. 

The \at~was originally discovered through pair production at the Tevatron in 1995~\cite{Top-CDF,Top-D0}. Later the production of a single \at~was discovered at the Tevatron~\cite{SGTOP-D0,SGTOP-CDF} and its width measured~\cite{CDF-topwidth,D0-topwidth,D0TopWidth:2010}. These production channels have also been investigated at the LHC~\cite{Aad:2015yem,Aad:2014fwa,Aad:2012ux,Chatrchyan:2011vp,Schilling:2012dx}.

There are three channels of single \at~physics that have been studied at the LHC. They are \tchan, \schan, and associated production (also referred to as \Wt). The largest contribution to single top at the LHC is \tchan, followed by \Wt, with \schan~being the smallest of the three. Being the largest, \tchan~was observed first and has been observed independent of the other single \athyph~production modes~\cite{TCHAN-ATLAS}. \Wt~has also been observed in \atlas~\cite{Aad:2015eto}and CMS~\cite{Chatrchyan:2014tua}. Cross-sections for the different single \athyph~processes at a proton-proton collider with $\sqrt{s} =$ 8~TeV are given in Table~\ref{TABLE-THEORY-SGTOP-XS}. The center-of-mass energy is denoted as $\sqrt{s}$ for the proton-proton collision. The LHC's high beam energies make gluons in the proton more prevalent then when compared to energetic quarks so a look into the initial states of these processes shown in Figures~\ref{FIGURE-tchan},~\ref{FIGURE-Wtchan}, and~\ref{FIGURE-schan} reveal the hierarchical nature of their \xs s. 

\SMALLFIG{tchan}{Representative Feynman diagram for the \tchan~single \at~process~\cite{SGTOP-DIAG}.}{FIGURE-tchan}
\MEDIUMFIG{Wtchan}{Representative Feynman diagrams for the \Wt~single \at~process~\cite{SGTOP-DIAG}.}{FIGURE-Wtchan}
\SMALLFIG{schan}{Representative Feynman diagram for the \schan~single \at~process~\cite{SGTOP-DIAG}.}{FIGURE-schan}

The \tchan~process has an initial state of an energetic gluon as well as a light quark, \Wt~has an initial state of an energetic gluon as well as an energetic \ab~(which will be harder to get from a proton when compared to a light quark which is naturally in a proton), and \schan~has an energetic antiquark in its initial state making it difficult to produce at the LHC. While \schan~has a comparatively small \xs~at the LHC it was not so disfavored at the Tevatron because the Tevatron was a proton anti-proton collider, making  energetic anti-quarks more prevalent. 

\begin{table}[!h!tbp] 
\begin{center}
\begin{tabular}{|l|r|}
\hline
\tchan &   216.99 +9.04 -7.71 pb\\
\hline
\Wt  &  84.4 +5.00 -6.80 pb \\
\hline
\schan &  10.32  +0.40 -0.36  pb \\
\hline
\end{tabular}
\label{TABLE-THEORY-SGTOP-XS}
\caption{The \xs~for different modes of single \athyph~production at the LHC at $\sqrt{s} = 8$ TeV~\cite{SGTOP-XS}~\cite{Kant:2014oha}.}
\end{center}
\end{table}

\section{tZ Associated Production}
\label{SECTION-TZ}

The production of a \at~in association with a \az~has not been considered at the LHC until now. The Feynman diagram for \tz~can be seen in Figure~\ref{FIGURE-tZ}. The related \ttz~\xs~has been measured, and although the uncertainty is quite high, the \at~+ \az~processes are a potentially fruitful one to investigate~\cite{ATLAS-CONF-2016-003}. The rate of the \tz~process suggests that it should be visible in the 8 TeV data set as seen in Figure~\ref{FIGURE-TZRATES} which shows NLO \xs s for the processes shown at various energies. The \tz~signature investigated includes three charged leptons, missing transverse energy, and two jets, one of which may be identified as a \ab~\cite{Campbell:2013yla}. 

Several histograms can be seen in Figures~\ref{TruthJetFig} and~\ref{TruthObjFig} which show simulations of particles before any detector interaction or decays. Some notable features of \tz~are the disparity between the $\eta$ ($\eta$ is defined in Section~\ref{SECTION-ATLAS-DET}) of the light jet vs. \ab, the higher transverse momentum (\PT) of the light jet compared to the \ab, the similarity in \PT~of leptons from the \az~and \aw~, and the \PT~of the neutrino which will manifest as \met. The variable \met~is discussed in more detail in Section~\ref{SECTION-OBJ-MET}.

\LARGEFIG{tZ}{Representative Feynman diagram for the \tz~associated production decaying to three leptons via a \az~and a \aw~\cite{JAXO}.}{FIGURE-tZ}

\VLARGEFIG{TZRATES}{Top-Quark pair and single \at~\xs s with and without accompanying \az~\cite{Campbell:2013yla}.}{FIGURE-TZRATES}

\QUAFIG{LightJetPt}{LightJetEta}{bQuarkPt}{bQuarkEta}{Information drawn from simulation of $tZ$. Light quark \PT~and $\eta$ as well as \abhyph~\PT~and $\eta$. This simulation is described in detail in Section~\ref{SECTION-MC-SIG} with added simulation steps taken for a more complete analysis. }{TruthJetFig}

\SIXFIG{ZLepPt}{WLepPt}{ZbosonPt}{WbosonPt}{TopQuarkPt}{NuPt}{Information drawn from simulation of $tZ$. The \PT~of other objects in \tz~including the lepton from the decay of the \az~and the lepton from the decay of the \aw. This simulation is described in detail in Section~\ref{SECTION-MC-SIG} with added simulation steps taken for a more complete analysis.}{TruthObjFig}


Standard model \tz~is important to measure because it is able to probe the coupling of the \at~with a \az~\cite{Campbell:2013yla}. Standard model \tz~is also a background to several SM processes and Beyond the SM processes. Anomalous \tz~couplings are one model that are of interest~\cite{Dror:2015nkp}. Monotop-quark production is one of these involving a \at~and large missing transverse energy coming from theorized dark mater candidates. Single \at~production in association with a Higgs boson is important to look for to probe the coupling of a Higgs boson to the \at. One can also consider \tz~as a background to Flavor Changing Neutral Current (FCNC) decays from \TTB~where one of the \at s decays to a \az~and a light quark which would enhance the cross section for this analysis. 

For this analysis a cut and count method is used. By examining the kinematic properties of the particles, as we have begun to do in Figures~\ref{TruthJetFig} and~\ref{TruthObjFig}, regions of phase space can be created to isolate backgrounds to ensure proper data modeling through simulation as well as isolating the \tz~signal to improve sensitivity for a statistical analysis. 


%\section{\tz~Beyond The Standard Model}
%\label{SECTION-THEORY-BSM}



%\section{Beyond The Standard Model}
%\label{SECTION-THEORY-BSM}

%The Standard Model, through all its successes,does not explain all phenomena observed. Perhaps the oldest phenomena not explained by the SM is gravity. The best theory of gravity to date is general relativity, which is incompatible with the Standard Model. More recently dark matter and dark energy have been found to comprise approximately 95\% of all matter in the universe. We currently do not know what dark matter is, but there is a popular opinion that it takes the form of a gravity only interacting particle. This is the Weakly Interacting Massive Particle (WIMP) theory. The neutrino sector also holds mysteries. According to the SM neutrinos are massless, but the observation of neutrino oscillation confirms that neutrinos do have mass. In the SM neutrino masses can be added, but they must be extremely small and its not clear if their masses come from the Higgs mechanism as the other particles do.~\cite{2009APS..HAW.KD010C}

%Another confusing observation is that our universe has a lot of matter, but not so much antimatter. The SM predicts equal parts matter and antimatter upon creation, and has no sufficient explanation concerning the lack of antimatter observed. The final phenomena i will discuss here is the top-quark anti-top-quark forward backward asymmetry (denoted $A_{FB}$). D0 and CDF at the Tevatron noticed that $A_{FB}$ was dramatically higher than the SM predicted value.~\cite{Aaltonen:2008hc}~\cite{Berger:2011ua}~\cite{Baumgart:2013yra}. The LHC is a proton-proton collider as opposed to the Tevatron's proton-antiproton collider, meaning this could not be studied further at the LHC. 

%There are several theories to account for the shortcomings of the SM as well as to extend it to cover other hypothetical particles. Some of these include SUSY, technicolor, Kaluza-Klien theory, string theory, M-theory, extra dimensions, and modified gravity. Each of these attempt to move one step closer to a Theory Of Everything(TOE) that will describe every aspect of the composition of the universe.  
%(more here)
