\chapter{CERN, the LHC, and \atlas~}
\label{SECTION-EXPERIMENT}

In 1954 the Conseil Europ�en pour la Recherche Nucl�aire(CERN) formed a nuclear physics laboratory just outside of Geneva Switzerland to provide a laboratory needed for physics research. CERN has delivered on that promise to give us dozens of experiments that study everything from meteorology to biology. Some of the labs accomplishments include: the discovery of the $W$-boson and $Z$-boson; the determination of the number of light neutrino families; the creation of the world wide web; the creation, isolation, and stabilization of anti-hydrogen for up to 15 minutes; and the discovery of the Higgs boson. 

Over the past few decades CERN has focused on accelerator physics, housing the Large Electron-Positron Collider (LEP) which ran from 1989 until 2000. LEP was then replaced with the Large Hadron Collider(LHC) starting operations in 2009 after a faulty start in 2008 due to a failure in an electrical connection leading to a rupture of the liquid helium enclosure of one of the superconducting magnets. The LHC and LEP are often thought of hand in hand because they both used the same 27 km tunnel.


\section{The Accelerator Chain}
\label{SECTION-ACCELERATORCHAIN}

The LHC is capable of colliding protons as well as heavy ions, although we focus primarily on the proton accelerator chain shown in Figure~\ref{FIGURE-LHCchain}. The protons used in the LHC start from a hydrogen bottle where a magnetic field strips the electrons from $H_2$ and the resulting protons are sent through linear accelerator 3(Linac3). Linac3 uses radio-frequency cavities which charge cylindrical conductors which are alternately positively or negatively charged. The conductors directly behind the protons are positively charged while the conductors in front of the protons are negatively charged, with both working to accelerate the protons. Once the protons are through Linac3 they will be bunched with 100ms bunch spacing and will be up to 50MeV in energy~\cite{Linac2}. From here it is sent through the 157m circumference Proton Synchrotron Booster(PSB) which accelerate the protons to an energy of 1.4GeV in only 530 ms~\cite{PSB}. From there the protons go to the 628m circumference Proton Synchrotron(PS) for tighter bunching of 25ns, and are accelerated to 25GeV~\cite{PS}. The final step before the LHC is the Super Proton Synchrotron(SPS) which is 7km in circumference. The SPS can accelerate protons to 450GeV in 4.3 seconds~\cite{SPS}. The SPS is notable for the 1983 Nobel prize winning discovery of the $W$-boson and $Z$-boson.

\VLARGEFIG{LHCchain}{Diagram of the accelerator complex for protons to get to the LHC~\cite{LHCchain}.}{FIGURE-LHCchain}

Finally the protons make it to the LHC to be ramped up to the desired energy for collision. A segment of the LHC beampipe can be seen in Figure~\ref{FIGURE-LHC-Tunnel}.

\VLARGEFIG{LHC-Tunnel}{A Segment of the LHC beampipe~\cite{LHC-Tunnel}.}{FIGURE-LHC-Tunnel}


\section{The Large Hadron Collider}
\label{SECTION-LHC}

It takes approximately 4min 20sec to fill each LHC ring (one in each direction) forming the beams of 2,808 bunches each. After a 20 minute wait time after injection to stabilize and tighten the beams they are accelerated over 25 minutes to an energy of 4TeV per beam. All in all it takes between 5 and 20 seconds to get the protons from Linac2 to the LHC, then 45 minutes to get them up to energy. Once at energy they can be stored for collisions for around 10 hours limited mostly by protons in the beam exchanging momentum between the transverse and longitudinal directions which is enhanced relativistically by a factor of $\gamma$. This is known as the Touschek effect. Particles are lost from the beam if their longitudinal momentum deviation is greater that the RF bucket or the momentum aperture of the lattice. After approximately 10 hours of beam collisions the beam is exhausted and is dumped and the injection process is repeated~\cite{LHC-TDR}~\cite{LHC}. 

Given the environment necessary for the discovery of new physics the LHC was designed with unprecedented capabilities. While most people think of the LHC as the highest energy collider in the world (which it is) there are more considerations when building an accelerator. In order to discover rare processes we consider instantaneous luminosity in order to collect as many interesting events as can be produced as quickly as possible. Peak \atlas~ online luminosity is around $8*10^{33}cm^{-2}s^{-1}$ (as seen in Figure~\ref{FIGURE-LUMI1}) which is around 20 times the peak Tevatron luminosity~\cite{TevatronLumi}~\cite{LUMIPLOTS}. A greater instantaneous luminosity leads to a greater integrated luminosity, which is a measure of how much data has been collected over time, as seen in Figure~\ref{FIGURE-LUMI2} for previous runs and Figure~\ref{FIGURE-LUMI3} for run 2. The generic term luminosity will usually refer to integrated luminosity in this thesis unless otherwise stated. 

\VLARGEFIG{peakLumiByFill}{Peak instantaneous luminosity over time. As run 2 went on instantaneous luminosity was increased to maximize data collection showing the dramatic increase in data collection in August and September~\cite{ATLAS-LUMI}.}{FIGURE-LUMI1}

~\DBLFIG{LUMI-INTEG}{Total LHC delivered integrated luminosity over time for run 1~\cite{ATLAS-LUMI}}{FIGURE-LUMI2}{intlumivstime2015DQ}{Total LHC delivered integrated luminosity over time for run 2~\cite{ATLAS-LUMI}}{FIGURE-LUMI3}

%\VLARGEFIG{intlumivstime2015DQ}{Total LHC delivered integrated luminosity over time for run 2~\cite{ATLAS-LUMI}}{FIGURE-LUMI3}

In order to keep the beams on track and together the LHC has 1232 dipole magnets to steer the beam and 392 quadrupole magnets for focusing and a total of around 9600 superconducting magnets. The beams are segmented into 2808 buckets which can be filled with bunches of protons or not. The LHC was designed to deliver bunches that are spaced so that the resulting collisions are 25 ns apart (corresponding to approximately 10 meters between bunches). In practice the LHC operates at 50 ns bunch spacing (leaving every other bucket empty) to help with pile up. Pile up is when two separate proton proton collisions interfere with each other and comes in two forms. Out of time pile up refers to two different bunch crossings interact with the detector more quickly than the detectors response time. Running with 50ns bunch spacing helps with out of time pile up while running with 25ns bunch spacing helps with the other type of pile up, in time pile up. In time pile up is when two parton collisions happen within the same bunch crossing and both interact with the detector at the same time. This is designed for and has been increased from the 7~TeV run with an average of 9.1 interactions per bunch crossing to the 8~TeV run with 20.7 interactions per bunch crossing as seen in Figure~\ref{FIGURE-NUM-INTERACTIONS} with a more conservative 13.7 interactions per bunch crossing for run 2 as seen in Figure~\ref{FIGURE-NUM-INTERACTIONS13}. Pile up is an important consideration in triggering and is discussed in this capacity in chapter~\ref{SECTION-TRIGGERS}~\cite{LHC-TDR}~\cite{LHC}. 

~\DBLFIG{NUM-INTERACTIONS}{Number of interactions per crossing is a measure of in time pile up. This is comparing the number of interactions per crossing in the LHC 7~TeV run and the 8~TeV run~\cite{ATLAS-LUMI}.}{FIGURE-NUM-INTERACTIONS}{mu_2015}{Number of interactions per crossing is a measure of in time pile up. This figure is for run 2 which is in comparison to Figure~\ref{FIGURE-NUM-INTERACTIONS} which is for run 1~\cite{ATLAS-LUMI}.}{FIGURE-NUM-INTERACTIONS13}

%~\VLARGEFIG{mu_2015}{Number of interactions per crossing is a measure of in time pile up. This figure is for run 2 which is in comparison to Figure~\ref{FIGURE-NUM-INTERACTIONS} which is for run 1~\cite{ATLAS-LUMI}.}{FIGURE-NUM-INTERACTIONS13}

 With an accelerator of this magnitude and a diversity of possible research topics, investigation through multiple experiments is merited. CMS and \atlas~ are the largest general purpose detectors designed to search for the broadest range of possible new physics models and precision measurements. MoEDAL searches for magnetic monopoles. TOTEM and LHCf are looking for forward particles and are positioned near CMS and \atlas~ respectively. ALICE was specially designed to study heavy ion collisions at the LHC to search for a state of matter known as quark gluon plasma. LHCb is an asymmetric detector studying the effects of matter antimatter asymmetry in proton proton collisions. 

With the LHC at such high energies and luminosities detectors had to be designed like never before to be fast, radiation hard, and finely segmented all while maintaining a sensible budget. 


\section{\atlas~}
\label{SECTION-ATLAS-DET}

\VLARGEFIG{ATLAS-OPEN}{\atlas~with its namesake toroidal magnets prominently visible~\cite{ATLAS-OPEN}.}{FIGURE-ATLAS-OPEN}

A Large Toroidal LHC AparatuS (also known as \atlas~ or the \atlas~ detector) is among the largest and most complex particle detectors in the world and can be seen schematically in Figure~\ref{FIGURE-ATLAS} but a more impressive view in Figure~\ref{FIGURE-ATLAS-OPEN} shows its namesake toroidal magnets in full view before much of the detector was added. It utilizes a multilayer design which has become ubiquitous in high energy physics. With this multilayer design comes a coordinate system that is vital to the design and use of the detector. There is a Cartesian coordinate system superimposed in \atlas~ with the $\hat{y}$ coordinate running vertically to the surface, the $\hat{x}$ coordinate running toward the center of the LHC ring, and the $\hat{z}$ coordinate running the length of \atlas~ pointing in the counter clockwise direction around the LHC ring when viewed from above. With this there is also a spherical coordinate system defined with $\phi$ running around the detector sweeping from the $\hat{x}$ axis toward the $\hat{y}$ axis while $\theta$ runs away from the $\hat{z}$ axis.While useful for construction and planning purposes, these variables are not as useful for analysis. For analysis a Lorentz invariant variable is desirable so a particles properties in the detector can be measured in any reference frame. One is rapidity, defined as

\begin{equation}
y = \frac{1}{2}ln\left(\frac{E+p_z}{E-p_z}\right)
\end{equation}

\noindent which has the unfortunate property of being dependent on the particle mass. Another is the widely used pseudorapidity, defined as 

\begin{equation}
 \eta = \frac{1}{2}ln\left(\frac{\left|\vec{p}\right|+p_z}{\left|\vec{p}\right|-p_z}\right)
\end{equation}

\noindent which can be rewritten in terms of detector geometry variables as 

\begin{equation}
 \eta = -ln\left(tan\left(\frac{\theta}{2}\right)\right)
\end{equation}

\noindent where $\eta=\infty$ corresponds to the beamline. $\eta$ is Lorentz invariant as long as $m << E$ which is in the low mass regime. In this regime pseudorapidity approximates rapidity. Pseudorapidity, therefore, has both the properties of describing detector geometry and describing particles in the detector that have boosts along the $\hat{z}$ axis~\cite{ATLAS-EXP}. 

\VLARGEFIG{IDWEDGE}{A figure diagramming how particle identification can be achieved using multiple layers of the detector~\cite{Pequenao:1505342}.}{FIGURE-IDWEDGE}

\atlas~ can be segmented into several parts. The inner detector, the calorimeters, the muon spectrometer, and the magnets. Overall these systems are designed to work together to give measurements of particle energies as well as particle identification as diagrammed in Figure~\ref{FIGURE-IDWEDGE}. An electron can be identified by tracks in the inner detector and a shower in the electromagnetic calorimeter, distinguished from the photon by the tracks in the inner detector. Meanwhile jets get stopped in the hadronic instead of the electromagnetic calorimeter and muons will go all the way through the detector leaving tracks all the way. Many particles like the $Z$-boson and top-quark decay before reaching the detector. These objects must be reconstructed from their decay products. Neutrinos can be difficult to reconstruct because they go through the entire detector without interacting at all. More on object reconstruction is described in chapter~\ref{SECTION-OBJ} but for now we can take a deeper look into the subsystems of \atlas~.

\VLARGEFIG{Atlas}{Cutaway diagram of \atlas~\cite{Figure-Atlas}.}{FIGURE-ATLAS}


\subsection{Magnet System}
\label{SECTION-ATLAS-MAGNETS}

\atlas~ has a magnet system designed to assist in particle identification by curving the path of charged particles through the detector systems. There are three parts to the magnet systems; the solenoidal magnet around the inner detector, the barrel toroids, and the endcap toroids. A schematic diagram of the layouts of these magnets can be seen in Figure~\ref{FIGURE-MAGNETS}. The magnetic field can be seen in Figure~\ref{FIGURE-BFIELD}.

\MEDIUMFIG{Magnets}{Illustration of the \atlas~ magnet system, showing the barrel solenoid, barrel toroid, and endcap toroid coils~\cite{ATLAS-EXP}.}{FIGURE-MAGNETS}

\VLARGEFIG{ATLAS-BField}{A mapping of the magnetic fields in atlas. Note the inhomogeneous nature of the toroidal field and the comparative constant nature of the solenoidal field~\cite{ASalzburger}.}{FIGURE-BFIELD}

The solenoidal magnet provides a nearly uniform 2T magnetic field for the inner detector. The Solenoid is designed to be as thin as possible to minimize the interaction of the particles from physics events to aid in calorimetry. Any interaction in the solenoid will begin the showering process which means energy from the interacting particle will be lost and will have to be accounted for in calorimetry. 

The eight barrel toroids are visible in Figure~\ref{FIGURE-ATLAS-OPEN} and run $\eta < 1.6$ providing a peak magnetic field of 3.9T around the muon spectrometers and are highly irregular as seen in Figure~\ref{FIGURE-BFIELD}. Because of this irregularity the magnetic field must be mapped carefully for accurate muon tracking. 

The endcap toroids complete the \atlas~ magnet systems providing a peak magnetic field of 4.1T for the forward detectors at $1.4 < \eta < 2.7$. The endcap magnets are offset from the barrel toroids by $\frac{1}{16}$ of a turn so that they bisect the angle (in $\phi$) between the barrel toroids seen in Figure~\ref{FIGURE-BFIELD}.

\VLARGEFIG{ATLAS-Endcap-magnet}{Preparation for lowering an endcap magnet into the \atlas~ cavern~\cite{ATLAS-Endcap-magnet}.}{FIGURE-MAGNETS-ENDCAP}

These magnets are crucial for particle identification and momentum measurements of charged particles and are strategically placed around the detector subsystems described hereafter.~\cite{MAGNET}


\subsection{Inner Detector}
\label{SECTION-ATLAS-ID}

The inner detector provides tracking information for tracked particles close to the beamline. Track reconstruction is finding sets of measurements coming from one charged particle and building the associated trajectory through the detector.
 In order to achieve this the inner detector was designed to be as hermetic as possible with high granularity as close to the beamline as possible.

The inner detector has three parts and can be seen in its entirety in Figure~\ref{FIGURE-ATLAS-ID}. Those parts are the pixel detector, the SemiConducting Tracker (SCT), and the Transition Radiation Tracker (TRT)~\cite{INNERDET}.

\VLARGEFIG{InnerDetector}{Cutaway diagram of the \atlas~ inner detector~\cite{Figure-InnerDetector}.}{FIGURE-ATLAS-ID}

\VLARGEFIG{ATLAS-ID}{A diagram of the sub detectors of the inner detector with relative radial positions~\cite{ATLAS-EXP}}{FIGURE-ATLAS-ID2}

The pixel detector and the SCT work on ionization of silicon which is separated into positive and negative charges which can be separated by an electric field into read out electronics. The read out can be either binary or non-binary. A binary read out registers a hit over some threshold or doesn't. The non-binary readout registers the charge collected over time over some threshold and reports the amount of charge collected to assist in track reconstruction. Non-binary readouts give better tracks, but are more expensive and the read out electronics take up more space in the valuable real estate near the beamline. The pixel detector and SCT cover $\eta < 2.5$.~\cite{PIXEL}~\cite{PIXEL-DET}

The SCT strip detectors use a stereo-angle technique to get position measurements where concentric layers are constructed so a small angle ($40$ mRad) gives position measurement. Without an angle between layers of the strip detector only a $\phi$ could be read out, but with the angle a $\eta$ can be read out as well. By using strips of silicon a lot of money and space can be saved in comparison to pixel detectors mostly in read out electronics and there wont be as much supporting material in the way for the calorimeters.~\cite{SCT}~\cite{SCT_Barrel}~\cite{SCT_Endcap}

The TRT works on the principle of transition radiation. When a high energy particle goes between media with differing dielectric constants, the result will be the emission of radiation or as Jackson puts it ``the fields must reorganize themselves as the particle approaches and passes through the interface. In this process, some pieces of the fields are shaken off as transition radiation''.~\cite{jackson91} The TRT uses this by filling tubes with a gas of $Xe$, $CO_2$, and $O_2$ which is ionized by charged particles passing through. All charged particles will interact with the TRT giving tracking information, but the TRT has two separate thresholds for readout. The first threshold tracks charged particles while the second higher threshold determines if transition radiation is being detected. Because the electron participates in transition radiation more strongly than the pion we can obtain good pion rejection while maintaining electron reconstruction efficiency.~\cite{TRT}~\cite{TRT_Barrel}~\cite{TRT_Endcap}

The largest source of track reconstruction inefficiency is hadronic interaction. When a hadron interacts with the nucleus of the detector material it is usually destroyed, and creates a hadronic and electromagnetic shower. When this happens the primary track stops and a series of other tracks begin, but unfortunately the track can not be reconstructed. Another problem is electron bremsstrahlung. When a charged particle passes near the nucleus of the detector material it will radiate, loosing energy. This effects electrons more than other particles because the energy loss as it traverses the detector is proportional to energy over mass squared ($\frac{dE}{dx} \alpha \frac{E}{m^2}$) so the light electron will brem more strongly than its heavier counterparts. Another consideration in tracking is multiple scattering which can cause a random change in direction not caused by curvature in the magnetic field. The method of detection can actually be a problem as well because when the particle ionizes the atoms of the detector it looses energy~\cite{ASalzburger}. 


\subsection{Calorimeters}
\label{SECTION-ATLAS-CALO}

There are two calorimeter systems for detecting electromagnetically interacting particles and strongly interacting particles referred to as the electromagnetic calorimeter and hadronic calorimeter respectively as seen in Figure~\ref{FIGURE-ATLAS-CALO}. One high energy particle from the hard interaction of an event will shower into many particles creating a wave of energy deposition in the calorimeters. This process is particularly useful for neutral particles that can not be tracked in the inner detector, but is also useful for a more complete picture of a particular object. 

\VLARGEFIG{Calorimeters}{Cutaway diagram of the \atlas~ calorimeter systems~\cite{Figure-Calo}.}{FIGURE-ATLAS-CALO}

Both the electromagnetic and hadronic calorimeters work on several different types of interactions. The first is radiative interactions where the incoming particle will scatter off a constituent atom creating a Rutherford scattering type interaction. When they aren't hard scattering off a nucleus they can also Compton scatter off atomic electrons, ionize the atoms of the detector, or have other similar low momentum transfer interactions. After this particle energies fall to when they are absorbed by atomic interactions and the number of particles in the shower begins to fall. It is often noted that muons and protons are minimally ionizing in the electromagnetic calorimeters. This is because the radiative interactions which begin the particle cascade fall by $m^2$, so their large mass gets them through the electromagnetic calorimeter~\cite{Wigmans}. 

In \atlas~ our Liquid Argon (LAr) calorimeters use sampling calorimetry. Because the primary interactions are radiative in nature it is desirable to have high atomic number in the calorimeter material. The principal of sampling calorimetry is to have two materials in the calorimeter; one to facilitate the radiative interaction and begin the showering process, and another to accommodate the lower energy interactions that provide the signal that is read out. In \atlas~ the LAr calorimeter is accordion shaped lead coated with stainless steel for radiative interactions with liquid argon to collect the resulting shower. The accordion shape is useful to increase the path length of particles in the material, thereby increasing the probability of an interaction and lowering the total amount of material needed. In the liquid argon are wires held at high voltage to attract the ionized particles and read out the charge current generated~\cite{Wigmans}. 

In \atlas~ we also have tile calorimeters(TileCal). They work on the same principles as the LAr calorimeters. The sampler for the TileCal is sheet steel without the accordion shape, and the readout material is a collection of plastic scintillators that emit light when hit with the resulting shower. The plastic scintillators are coupled to optical wavelength shifting fibers to redirect light to photomultiplier tubes. The light emitted from the scintillating plastic is typically in the UV range, and is shifted into the blue or green visible wavelengths to help limit attenuation while propagating~\cite{TILE}~\cite{Proudfoot:2006tr}. 

The geometry of the calorimeter systems are complex, but a simplified version is given here and can be seen in Figure~\ref{FIGURE-ATLAS-CALO}. The barrel region of the detector (with $\eta < 1.475$) has both LAr calorimetry and TileCal~\cite{EMCAL_Barrel}. The endcap calorimeters cover $1.375 < \eta < 3.2$ ~\cite{EMCAL_Endcap}. There is also a LAr forward calorimeter(FCal) that covers the extremely forward region $3.1 < \eta < 4.9$ with a copper absorber for the electromagnetic portion and a tungsten absorber for the hadronic part. There is also an inner presampler to catch how intensively radiative interactions from interactions with the inner detector took place~\cite{EMCAL_Presampler}.


\subsection{Muon Systems}
\label{SECTION-ATLAS-MUON}

Muons provide an interesting challenge because they do not interact strongly with the electromagnetic or hadronic calorimeters and pass through the detector. The muon systems have four layers shown in Figure~\ref{FIGURE-ATLAS-MUON}; they are the Monitored Drift Tubes (MDT), the Cathode Strip Chambers (CSC), the Resistive Plate Chambers (RPC) and the Thin Gap Chambers (TGC). 

\VLARGEFIG{Muon}{Cutaway diagram of the \atlas~ muon spectrometer and toroid magnet systems~\cite{Figure-Muon}.}{FIGURE-ATLAS-MUON}

The MDT provides most of the precision muon tracking in \atlas. The MDT has a barrel and endcap section. The barrel section covers $\eta < 1.0$ while the endcaps cover $1.0 < \eta < 2.7$. The MDT works on ionizing gases and is composed of straw tubes filled with gas which is composed of argon, nitrogen, and methane. The muon traverses the straw tube and ionizes the gas and the resulting charged particles are collected on a wire in the center of the tube which is held at high voltage. The amount of time it takes from the first current to reach the center wire until the last current reaches the center wire tells us how far away the muon came to the center of the tube. With this information we can make measurements on the path of the muon. 

The CSC are multiwire proportional chambers used in high radiation zones around \atlas~at$2.0 < eta < 2.7$. They work on the same principles as the MDT, but with a different gas mixture. 

The RPC is designed to supplement the MDT in the barrel region $\eta < 1.05$ and has a fairly simple design. Each RPC chamber is two resistive plates held at 8900V across a 2mm gap. An incoming charged particle will ionize the gas and cause a localized discharge of the capacitor. The location of this discharge can then be read out. This method does not give very good spacial resolution (approximately 1 cm) but is quite fast with a timing uncertainty of 1.5 ns. Because of this it is utilized primarily by the Level 1 trigger. 

The TGC (as seen in Figure~\ref{FIGURE-ATLAS-TGC-wheel}) is also designed to supplement the MDT but in the forward region of the detector $1.05 < \eta < 2.4$. It uses the same technology as the RPC and consequently is used for triggering as well. The TGC has a different gas mixture in order to decrease spacial resolution for bunch identification (down to 9 mm) but suffers in timing response (7ns) which is still fast enough to be used by the level 1 trigger system.~\cite{Green:1221848}~\cite{TDR1}~\cite{ATLAS-TDR}~\cite{DETECTORS}
 
\VLARGEFIG{ATLAS-TGC-wheel}{The TGC wheel~\cite{ATLAS-TGC-wheel}.}{FIGURE-ATLAS-TGC-wheel}

