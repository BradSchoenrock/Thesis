\chapter{Multivariate Analysis}\label{app:MultivariateApproach}
Although this document has focused on a simple, cut-based analysis approach, there are more complicated analysis options.  These include multivariate techniques such as boosted decision trees.  Multivariate techniques use computer algorithms to determine several dimensions of selection sequences, making use of events which both pass and fail individual selections.  The result is in an output related to the probability of an event being signal or background.  In this chapter we review the boosted decision tree technique and then suggest options for future analyses using the variables from the main document and an additional set of variables.  As this section is intended as suggestions for future work and another view point of the t-channel single-top analysis, only the large uncertainties from the cut-based analysis are considered.

%citation for SPR and the different classifier algorithms

\section{Boosted Decision Tree Overview}\label{app:bdtoverview}
The boosted decision tree (BDT)~\cite{BDT2} method has traditionally been used in single-top analyses~\cite{singletopdiscovery:D0, singletopdiscovery:CDF, CMStchannel} and this method is used here, as provided in a statistical package called StatPatternRecognition~\cite{SPR}.  A boosted decision tree is based on a collection of decision trees, and an example of one is pictured in Figure~\ref{fig:bdtex}.  A decision tree is a cut-flow diagram, where selections are applied that eventually result in sets of mostly signal or mostly background events.  These final sets of events are the leaves or terminal nodes of the tree (each selection has an associated node that is not necessarily terminal).  When a decision tree is applied to a new event, if this event passes background-like selections it is probably background, whereas if it passes signal-like selections it is probably signal.  The ultimate output of the multivariate classifier indicates how likely it is that an event is background or signal.  One can then take a simple cut on the classifier output, to select for high signal probability, or do a fitting technique to determine how much signal is present in the data.

 \begin{figure}[!h!tpb]
 \centering
 \includegraphics[width=0.60\textwidth]{figures/appendix/BDTXY.eps}%bdt2 has green background.  image looks ok in acrobat reader
 \caption{A pictoral representation of a single decision tree, where A and C are variables values, X and Y are selection thresholds.  The node is designated as S for signal and B for background.  The S and B circles are the final nodes, or leaves, in the tree.}
 \label{fig:bdtex}
 \end{figure}

The tree itself is formed using an optimization criterion to determine which variable to use for each selection (or node split) and what threshold to take.  The goal of such a criterion is to optimize the signal and background separation.  For this study we use the default optimization criterion, which is the Gini index.  This is the purity times 1 minus the purity, p(1-p).  The purity is the signal events divided by all of the events considered in that node, so if a node would have only signal events, the purity would be 1, giving a Gini index of 0.  If there are only background events, the purity is 0, and the Gini index is again 0.  The goal of the splitting is to obtain nodes that are background or signal dominated, so nodes are optimized to obtain a Gini index of 0 (or close to 0).  Unlike the cut-based analysis in Section~\ref{sec:optimization}, systematic uncertainties cannot be considered when determining each individual selection of a tree.

The ``boosted'' portion of the name boosted decision tree refers to an algorithm which reweights events based on whether or not they were mis-classified as signal and background in the previous tree.  These weights then affect how the performance is evaluated in the training of the next tree.  The boosting algorithm used in this study is called $\epsilon$-boost~\cite{SPR, BDT2}.  This particular algorithm increases the weights of incorrectly classified events by a factor of $e^{2\epsilon}$.  The $\epsilon$ value may be set to different values, but here we use the default value of 0.01.  In the end, the various trees are all averaged to give a final boosted tree.  The classifiers for each tree, the functions which are averaged, are a formed by minimizing what is called a quadratic loss criterion.  This is the average of the square of the difference between the true and predicted classifications for all events.  

\subsection{Classifier Formation and Parameter Optimization}
When forming a classifier, there are three different MC samples considered.  These are formed by taking the modulus of the event number and are called training, validation, and yield.  The yield sample is only used for the final analysis evaluation to ensure an unbiased result.  The training sample is used to form various classifiers and the validation sample is used to evaluate if some particular classifier is the one we desire.  The one exception to this sample division in this study is the multijets process. The statistics are quite low for this sample, so it is divided in half to form a training and yield sample, again using the modulus of the event number.  Additionally, because of limitations in the statistical package, negatively weighted events cannot be used during the training phase when the classifier is generated.  There are such events in most of the MC samples for the various top-quark processes.  However, the proportion of negatively weighted events is low (about 7\% in the training sample for the signal).  Even if this did have some effect, it would simply result in a less than optimally trained classifier, not a biased result, because the sample used for the result includes both negative and positively weighted events.

It is possible to train different combinations of channels: each channel separately, the number of jet channels separately (but with both lepton charges allowed), and all four channels combined.  For this study we use two BDT's for the final result, each with a different number of jets (2 or 3).  When samples are split in this way, the classifiers can take advantage of the different kinematics in each number of jets channel.  However, the MC statistics will be lower after splitting the sample, potentially causing the kinematics and events to be unevenly distributed.  A multivariate classifier can be particularly sensitive to this, especially if a tight cut on the classifier is taken, as we do here.  This is one reason why we do not further divide this sample in additional kinematic regions and combine several classifiers for the final result.

There are several different classifier settings to choose from when forming a BDT.  In this study, we vary the number of decision trees the BDT uses and the minimum number of events in the leaves for each tree.  We use the default settings for other parameters, including the type of boost ($\epsilon$-boost with $\epsilon$ of 0.01), the per event loss (quadratic), and the optimization criterion (Gini index), discussed at the beginning of Section~\ref{app:bdtoverview}.  The number of variables considered is another parameter of the BDT and we consider several different combinations of variables.

Many different trees are generated using the training sample with a variety of classifier settings.  We choose the trained BDT classifier and cut threshold for that classifier by using a criteria that includes systematic uncertainties, as we do in the main text in Section~\ref{sec:optimization}.  In this case, we found a few classifiers that had consistent distributions in the training and validation samples and were continuous. We then determine which of these would give the best expected result, based on the significance calculated using validation sample information.  Multijets are not considered during the significance calculation.

It is possible to overtrain a BDT during the generation of the classifier, which means it is too tuned to the particular MC sample's kinematics subtleties, like being trained on noise.  Overtraining results in a BDT that is sub-optimal, which we would like to avoid, but doesn't invalidate the analysis.  The $\chi^2$~\cite{Lyons} and Kolmogorov-Smirnov (KS)~\cite{KS} tests are used to check the training and validation sample agreements.  Classifiers are chosen which have good agreement ($>5\%$).  Additionally, because the validation sample is used to determine the BDT settings and threshold, it is also possible to be sensitive to this sample's particular distributions.  We save a yield sample for the cross-section calculation to ensure that such a sensitivity won't impact the final result.

\subsection{Cut-based Analysis Variables}\label{app:sbdt}
Because this dissertation focuses on a cut-based analysis, it is interesting to consider what would happen if we train BDTs using only the variables from the cut based analysis.  For this, we use all four variables considered in Section~\ref{sec:selectionchoices} for each number of jet channel, as well as lepton charge for a total of five variables in each channel.  The variables used for both channels are: sum of the transverse momenta of all jets, lepton, and \met; leading untagged jet $\eta$; top quark mass reconstructed using the b-tagged jet, lepton, and reconstructed neutrino; and lepton charge.  Additionally, $\Delta\eta$ between the b-tagged jet and leading untagged jet is used for the 2 jet selection and the invariant mass of all jets is used for the 3 jet selection.  

For the 2 jet selection, the classifier parameters and cut threshold are: 250 trees, 1500 events minimum per leaf, and 0.74 cut threshold.  For the 3 jet selection, these are: 150 trees, 1250 events minimum per leaf, and 0.41 cut threshold.  The BDT classifier distribution before and after the selection for each channel is shown in Figure~\ref{fig:Plot_SBDT}, normalized to the observed t-channel cross-section.  The variable distributions after this cut threshold for each channel are given in Figures~\ref{fig:Plot_2SBDT} and~\ref{fig:Plot_3SBDT} for the 2 and 3 jet selections respectively, also normalized to the observed t-channel cross-section.  Notice that after the selections, the kinematic regions chosen for the distributions look similar to those in Figures~\ref{fig:Plot_2TagCuts} and~\ref{fig:Plot_3TagCuts}, particularly the reconstructed top mass, leading untagged jet $\eta$ and the invariant mass of all of the jets.  Overall the agreement is fairly good between data and MC in these plots, keeping in mind the lower MC statistics from splitting the MC into thirds and also the somewhat large systematic uncertainties.

 \begin{figure}[!h!tpb]
 \centering
% \includegraphics[width=0.46\textwidth]{figures/appendix/bdtvar/BDTqcd_test_nocut_2jshort_NoSigScale_classiAdaBoost.log.data.eps}
%\includegraphics[width=0.46\textwidth]{figures/appendix/bdtvar/BDTqcd_test_nocut_3jshort_NoSigScale_classiAdaBoost.log.data.eps}
 \includegraphics[width=0.46\textwidth]{figures/appendix/bdtvar/BDTqcd_test_nocut_2jshort_classiAdaBoost.log.data.eps}
 \includegraphics[width=0.46\textwidth]{figures/appendix/bdtvar/BDTqcd_test_cut_2jshort_classiAdaBoost.data.eps}
\includegraphics[width=0.46\textwidth]{figures/appendix/bdtvar/BDTqcd_test_nocut_3jshort_classiAdaBoost.log.data.eps}
  \includegraphics[width=0.46\textwidth]{figures/appendix/bdtvar/BDTqcd_test_cut_3jshort_classiAdaBoost.data.eps}
 \includegraphics[width=0.46\textwidth]{figures/variables/Plot_Legend.eps}
\vspace{-0.5cm}
 \caption{BDT classifier distributions for the 2 jet (top) and 3 jet (bottom) selections, formed using cut-based analysis variables.  The left column is before the selection on the BDT classifier in a log scale, and the right column is after.  The $t$-channel single-top contribution is normalized to the observed cross-section determined using all four channels. Other top refers to the $s$-channel and $Wt$ single-top contributions.}
 \label{fig:Plot_SBDT}
 \end{figure}

The yields after the selections on the BDT thresholds are given in Table~\ref{tab:tch_eventyields_sbdt}.  Overall the yields a little lower than the cut-based analysis (see Section~\ref{sec:selectionchoices}).  The signal to background ratios are much higher in the 2 jet channel but a little lower or the same in the 3 jet channel.  This indicates that the BDT for the 2 jet selection in particular has better separating power between signal and background than the cut-based analysis cuts.
\begin{table}[!h!tpb]
  \begin{center}
     \begin{tabular}{lrr|rr}
    \hline \hline
        &\multicolumn{2}{c|}{BDT 5 Variables 2 Jets} &\multicolumn{2}{c} { BDT 5 Variables 3 Jets}  \\
        & Lepton + & Lepton -  & Lepton + & Lepton -  \\

    \hline \hline
    $t$-channel            & $ 27.7 $ & $ 7.6 $ & $ 56.9$ & $ 12.8 $ \\
    \hline                                                                       
    $t\bar t$, Other top   & $ 2.2 $ & $ 1.0 $ & $ 25.9$ & $ 6.9$ \\
    $W$+light jets         & $ 0.8 $   & $ < 0.1$ & $ 4.4 $ & $ < 0.1 $ \\
    $W$+heavy flavour jets & $ 7.5$  & $ 2.0$ & $ 23.6 $ & $ 4.5 $ \\
    $Z$+jets, Diboson      & $ < 0.1$  & $ < 0.1 $ & $ 1.1 $ & $ < 0.1$ \\
    Multijets              & $ 0.3 $ & $ < 0.1 $ & $ 7.8  $ & $ < 0.1 $ \\
    \hline    
    TOTAL Exp              & $ 38.5 $ & $ 10.7$ & $ 119.7$ & $ 24.2 $ \\
    S/B                    &  2.6  & 2.4 &  0.9 &   1.1  \\
    \hline \hline
    DATA                   &  60  &   16   &   115  &   24   \\
     \hline \hline
    \end{tabular}
 \caption{Event yield for the two-jets and three-jets tag positive and negative lepton-charge channels after the selection on the BDT formed using the cut-based analysis variables. The multijets and $W$+jets backgrounds are normalized to the data; all other samples are normalized to theory cross-sections.   The $t$-channel single-top contribution is normalized to the observed cross-section determined using all four channels. Other top refers to the $s$-channel and $Wt$ single-top contributions.
\label{tab:tch_eventyields_sbdt}}
  \end{center}
\end{table}

The expected cross-section was also calculated for the BDT distributions.  The 2 and 3 jet channels were split into negative and positive lepton charge channels, after selecting the desired region of the BDT classifier, and the combination was calculated using all four channels.  The expected cross-section calculation used the systematic shifts and statistical methods from the cut-based analysis (Section~\ref{Sys}) except in the cases of the largest systematic uncertainties, which were estimated using the shifts in the yields of the BDT classifier distributions after the selection on it.  These systematic uncertainties that were re-estimated are the statistical, b-tagging scale factor, mis-tagging scale factor, jet energy scale, generator, parton shower, and ISR/FSR uncertainties.  The MC statistical uncertainty is not changed here, but we might expect it to be about 1.7 times as large as in the cut-based analysis, if the proportion of events from the different processes is relatively unchanged.  This is because the MC event weights increase by a factor of 3 and there are 1/3 as many events, giving a factor of $\sqrt{3}$ multiplied with the square root of the sum of the squares of the weights of the events (the MC statistical uncertainty).  The uncertainties which are re-estimated and the total uncertainties using both re-estimated and cut-based analysis values are given in Table~\ref{tab:xs-uncertainty-exp-sbdt}.  

The uncertainties are generally comparable with the cut-based analysis except for the ISR/FSR uncertainty, which is much larger.  This may be due to the ISR/FSR uncertainty not being considered during the optimization of the classifier with the validation sample, leading to the selection of events that happen to have a larger uncertainty.  This is something that could be added to the classifier optimization in a future study.  Additionally, the jet energy scale uncertainty is higher and the b-tagging scale factor uncertainty is lower versus the cut-based analysis, reflecting some differences in the selected events in the BDT versus the cut-based analysis.  If we use the cut-based ISR/FSR uncertainty value, assuming the BDT analysis could be improved to reduce this uncertainty, the expected cross-section uncertainty would be $\sigma_{t}= 65^{+21}_{-19}$~pb, compared to $\sigma_{t}= 65^{+22}_{-20}$~pb from the cut-based analysis in Section~\ref{sec:fourchanresult}.  The expected cross-section uncertainty with the re-estimated ISR/FSR is $\sigma_{t}= 65^{+34}_{-26}$~pb.  The observed cross-section value is $\sigma_{t}= 82.9^{+36}_{-28}$~pb, which is consistent with the cut-based analysis result within uncertainties.

\begin{table}[htdp]
\begin{center}
   \begin{tabular}{l|c}
    \hline
    Source & $\Delta\sigma/\sigma$ (\%) \\
    \hline \hline
    Expected statistics              & +13/-13  \\
     \hline  
    $b$ tagging scale factor     & +6/-6  \\
    Mistag scale factor          & +1/-1  \\
    Jet energy scale             & +7/-8  \\
    Generator                    & +7/-7  \\
    Shower                       & +14/-14  \\ 
    ISR/FSR                      & +40/-29  \\
    \hline  
    Total Systematics            & +50/-38  \\
    Total                        & +52/-40  \\
    \hline
   \end{tabular}
\caption{Systematic uncertainties for the expected $t$-channel cross-section measurement for the BDT formed using cut-based analysis variables, where the final line includes all systematic uncertainties and the statistical uncertainty of the data.  Uncertainties that were re-estimated versus the cut-based analysis (Section~\ref{sec:xs}) are listed individually.  Others are not listed but are included in the totals.}
\label{tab:xs-uncertainty-exp-sbdt}
\end{center}
\end{table}

%     SysName     up[%]  down[%]  bias[%]
%  Data stat.    13.044  -13.044    0.001
%----------------------------------------      
%      btagSF     5.926   -5.926   -0.181
%         JES     6.775   -7.666   -3.587
%    partshow    13.635  -13.635   -0.127
%      isrfsr    40.222  -29.300   27.555
%       Total sys   50.337  -38.143   32.847
% 
%     ttgen     4.618   -4.618   -0.019
%      stgen     6.207   -6.207   -0.071
%
%       Total sys   31.129  -27.431   14.717
%       Total    33  30

 \begin{figure}[!h!tpb]
 \centering
 \includegraphics[width=0.46\textwidth]{figures/appendix/bdtvar/BDTqcd_test_cut_2jshort_classiTopBJet1_mass.data.eps}
 \includegraphics[width=0.46\textwidth]{figures/appendix/bdtvar/BDTqcd_test_cut_2jshort_classiUJet1_Jet_eta.data.eps}
 \includegraphics[width=0.46\textwidth]{figures/appendix/bdtvar/BDTqcd_test_cut_2jshort_classiLepton_charge.data.eps}
 \includegraphics[width=0.46\textwidth]{figures/appendix/bdtvar/BDTqcd_test_cut_2jshort_classiHt_AllJetsLeptonMissingEt.data.eps}
 \includegraphics[width=0.46\textwidth]{figures/appendix/bdtvar/BDTqcd_test_cut_2jshort_classiDeltaEta_BJet1UJet1.data.eps}
 \includegraphics[width=0.46\textwidth]{figures/variables/Plot_Legend.eps}
\vspace{-0.5cm}
 \caption{Discriminating variables for the 2 jets selection after a selection on the BDT classifier formed using cut-based analysis variables.  The last bin contains the sum of the events in that bin or higher.  The $t$-channel single-top contribution is normalized to the observed cross-section determined using all four channels. Other top refers to the $s$-channel and $Wt$ single-top contributions.}
 \label{fig:Plot_2SBDT}
 \end{figure}

\begin{figure}[!h!tpb]
 \centering
 \includegraphics[width=0.46\textwidth]{figures/appendix/bdtvar/BDTqcd_test_cut_3jshort_classiTopBJet1_mass.data.eps}
 \includegraphics[width=0.46\textwidth]{figures/appendix/bdtvar/BDTqcd_test_cut_3jshort_classiUJet1_Jet_eta.data.eps}
 \includegraphics[width=0.46\textwidth]{figures/appendix/bdtvar/BDTqcd_test_cut_3jshort_classiLepton_charge.data.eps}
 \includegraphics[width=0.46\textwidth]{figures/appendix/bdtvar/BDTqcd_test_cut_3jshort_classiHt_AllJetsLeptonMissingEt.data.eps}
 \includegraphics[width=0.46\textwidth]{figures/appendix/bdtvar/BDTqcd_test_cut_3jshort_classiInvariantMass_AllJets.data.eps}
 \includegraphics[width=0.46\textwidth]{figures/variables/Plot_Legend.eps}
\vspace{-0.5cm}
 \caption{Discriminating variables for the 3 jets selection after a selection on the BDT classifier formed using cut-based analysis variables.  The last bin contains the sum of the events in that bin or higher.  The $t$-channel single-top contribution is normalized to the observed cross-section determined using all four channels. Other top refers to the $s$-channel and $Wt$ single-top contributions.}
 \label{fig:Plot_3SBDT}
 \end{figure}

%There were some studies done before the detector running that looked at what performance one might expect with a BDT.  In these studies a selection of variables was proposed, and we again consider this list and.  Ideally, you would form many different BDT's using various combinations of these variables and find which BDT gives the best separation.   Adding more variables to the truly optimized BDT would give a similar performance as the optimal one.  Extra variables beyond this point won't hurt performance, but they are not necessary.  This can be seen in a distribution of the number of variables versus significance using the validation sample, for various classifiers that were trained.  

\subsection{Additional Variables}
Of course, there is no reason to choose only the variables used for the cut based analysis when generating the BDTs.  Starting from these variables, we considered many additional variable combinations, using variables that were considered for the cut-based analysis but not used (see Section~\ref{chap:variables}).  After many options were considered, BDT classifiers were chosen for the 2 jet and 3 jet selections which happen to use the same variables.  These BDT classifiers were chosen to have a large significance in the validation sample, a relatively low number of variables, and good agreement between the training and validation BDT distributions.

The best classifiers had 10 variables, including the lepton charge variable.  In addition to the 6 variables considered in this analysis (listed in Section~\ref{app:sbdt}), the following were used for both jet number selection channels: $\eta$ of the lepton; cosine of the angle between the lepton and the untagged jet, both in the rest frame of the top quark reconstructed using the leading b-tagged jet; W transverse mass; and $\Delta\eta$ between the b-tagged jet and the lepton.  Note that the $\Delta\eta$ between the b-tagged jet and leading untagged jet and the invariant mass of all jets are now used for both the 2 and 3 jet selections, unlike in Section~\ref{app:sbdt}.  The distributions of the additional variables with the preselection applied for the full MC set are shown in Figure~\ref{fig:preselbdt2} for the 2 jet selection and Figure~\ref{fig:preselbdt3} for the 3 jet selection, showing good agreement between the data and the MC.

%The cosine variable is also known as the spin correlation variable for single-top.  The spin for the top is almost 100% polarized with the forward light jet in the standard model and this cosine variable is directly related to the spin~\ref{spin}.  We look a the angle between the decay products and this forward jet with this variable, and single-top tends to prefer a region where the cosine 

\begin{figure}[!h!tpb]
 \centering
 \includegraphics[width=0.46\textwidth]{figures/appendix/bdtvar/PaperFinal_MCtchannorm_2jet1tag__wflavorDeltaEta_BJet1Lepton.eps}
 \includegraphics[width=0.46\textwidth]{figures/appendix/bdtvar/PaperFinal_MCtchannorm_2jet1tag__wflavorCos_Lepton_UJet1_RBJet1Top.eps}
\includegraphics[width=0.46\textwidth]{figures/appendix/bdtvar/PaperFinal_MCtchannorm_2jet1tag__wflavorLepton_eta.eps}
\includegraphics[width=0.46\textwidth]{figures/appendix/bdtvar/PaperFinal_MCtchannorm_2jet1tag__wflavorW_TransverseMass.eps}
 \includegraphics[width=0.46\textwidth]{figures/appendix/bdtvar/PaperFinal_MCtchannorm_2jet1tag__wflavorInvariantMass_AllJets.eps}
 \includegraphics[width=0.46\textwidth]{figures/variables/Plot_Legend.eps}
\vspace{-0.5cm}
 \caption{Discriminating variables for the 2 jets selection before any BDT classifier selection.  The last bin contains the sum of the events in that bin or higher. Other top refers to the $s$-channel and $Wt$ single-top contributions.}
 \label{fig:preselbdt2}
 \end{figure}

\begin{figure}[!h!tpb]
 \centering
 \includegraphics[width=0.46\textwidth]{figures/appendix/bdtvar/PaperFinal_MCtchannorm_3jet1tag__wflavorDeltaEta_BJet1Lepton.eps}
 \includegraphics[width=0.46\textwidth]{figures/appendix/bdtvar/PaperFinal_MCtchannorm_3jet1tag__wflavorCos_Lepton_UJet1_RBJet1Top.eps}
\includegraphics[width=0.46\textwidth]{figures/appendix/bdtvar/PaperFinal_MCtchannorm_3jet1tag__wflavorLepton_eta.eps}
\includegraphics[width=0.46\textwidth]{figures/appendix/bdtvar/PaperFinal_MCtchannorm_3jet1tag__wflavorW_TransverseMass.eps}
 \includegraphics[width=0.46\textwidth]{figures/appendix/bdtvar/PaperFinal_MCtchannorm_3jet1tag__wflavorDeltaEta_BJet1UJet1.eps}
 \includegraphics[width=0.46\textwidth]{figures/variables/Plot_Legend.eps}
\vspace{-0.5cm}
 \caption{Discriminating variables for the 3 jets selection before any BDT classifier selection.  The last bin contains the sum of the events in that bin or higher. Other top refers to the $s$-channel and $Wt$ single-top contributions.}
 \label{fig:preselbdt3}
 \end{figure}

For the 2 jet selection, the classifier parameters and cut threshold are: 150 trees, 2500 events minimum per leaf, and 0.64 cut threshold.  For the 3 jet selection, these are: 150 trees, 1500 events minimum per leaf, and 0.42 cut threshold.  The BDT classifier distributions before and after the selection for each channel are shown in Figure~\ref{fig:Plot_ABDT}, normalized to the observed t-channel cross-section.  The variable distributions after this cut threshold for each channel are given in Figures~\ref{fig:Plot_2ABDT} and~\ref{fig:Plot_2ABDT2} for the 2 jet selection, and Figures~\ref{fig:Plot_3ABDT} and~\ref{fig:Plot_3ABDT2} for the 3 jet selection, with all normalized to the observed t-channel cross-section.  Again, notice that after the selections, the kinematic regions selected in the distributions look similar to those in Figures~\ref{fig:Plot_2TagCuts} and Figures~\ref{fig:Plot_3TagCuts}, particularly the reconstructed top mass, invariant mass and leading untagged jet $\eta$ distributions.  

 \begin{figure}[!h!tpb]
 \centering
 \includegraphics[width=0.46\textwidth]{figures/appendix/bdtvar/BDTqcd_test_nocut_2jall_classiAdaBoost.log.data.eps}
 \includegraphics[width=0.46\textwidth]{figures/appendix/bdtvar/BDTqcd_test_cut_2jall_classiAdaBoost.data.eps}
 \includegraphics[width=0.46\textwidth]{figures/appendix/bdtvar/BDTqcd_test_nocut_3jall_classiAdaBoost.log.data.eps}
 \includegraphics[width=0.46\textwidth]{figures/appendix/bdtvar/BDTqcd_test_cut_3jall_classiAdaBoost.data.eps}
 \includegraphics[width=0.46\textwidth]{figures/variables/Plot_Legend.eps}
\vspace{-0.5cm}
 \caption{BDT classifier distributions for the 2 jet selection on the top line and the 3 jet selection on the next line, for the BDT formed using 10 analysis variables.  The left figures are before the selection on the BDT classifier, the right figures are after.  Note that the BDT distributions before selections are in a log scale.   The $t$-channel single-top contribution is normalized to the observed cross-section determined using all four channels. Other top refers to the $s$-channel and $Wt$ single-top contributions.}
 \label{fig:Plot_ABDT}
 \end{figure}

The yields after the selections on the BDT thresholds are given in Table~\ref{tab:tch_eventyields_abdt}.  The signal to background ratios here are about the same as those from Section~\ref{app:sbdt} for the 3 jet bin but are much improved for the 2 jet selection.  This indicates that the extra variables have particularly helped the signal versus background discrimination in the 2 jet bin.
\begin{table}[!h!tpb]
  \begin{center}
     \begin{tabular}{lrr|rr}
    \hline \hline
        &\multicolumn{2}{c|}{BDT 10 Variables 2 Jets} &\multicolumn{2}{c} {BDT 10 Variables 3 Jets}  \\
        & Lepton + & Lepton -  & Lepton + & Lepton -  \\

    \hline \hline
    $t$-channel            & $ 45.9 $ & $ 19.7 $ & $ 46.0$ & $ 16.7 $ \\
    \hline                                                                       
    $t\bar t$, Other top   & $ 2.2 $ & $ 2.1 $ & $ 16.6$ & $ 7.5 $ \\
    $W$+light jets         & $ 0.8 $ & $ < 0.1$ & $ 2.4 $ & $ < 0.1 $ \\
    $W$+heavy flavour jets & $ 9.7$  & $ 6.5 $ & $ 25.4 $ & $ 8.3 $ \\
    $Z$+jets, Diboson      & $ < 0.1$ & $ < 0.1 $ & $ 0.3 $ & $ 0.6$ \\
    Multijets              & $ 1.9 $ & $ 1.6 $ & $ 3.6  $ & $ < 0.1 $ \\
    \hline    
    TOTAL Exp              & $ 60.5 $ & $ 29.9$ & $ 94.3$ & $ 33.1 $ \\
    S/B                    &  3.1  & 1.9 &  0.9 &   1.0  \\
    \hline \hline
    DATA                   &  94  &   33   &   82  &   38   \\
     \hline \hline
    \end{tabular}
 \caption{Event yield for the two-jets and three-jets tag positive and negative lepton-charge channels after the selection on the BDT formed using 10 analysis variables. The multijets and $W$+jets backgrounds are normalized to the data; all other samples are normalized to theory cross-sections.   The $t$-channel single-top contribution is normalized to the observed cross-section determined using all four channels. Other top refers to the $s$-channel and $Wt$ single-top contributions.
\label{tab:tch_eventyields_abdt}}
  \end{center}
\end{table}

The additional variables are used to try to improve the uncertainty on the cross-section measurement versus the BDT formed with cut-based variables only.  The extra information should improve the signal and background separation, and may also help to improve the selection of low uncertainty kinematic regions.  As in Section~\ref{app:sbdt}, the 2 and 3 jet channels were split into negative and positive lepton charge channels, after selecting the desired region of the BDT classifier, and the combination was calculated using all four channels.  The expected cross-section uncertainties for the combination is given in Table~\ref{tab:xs-uncertainty-exp-abdt}.  Also, as in Section~\ref{app:sbdt}, only the statistical, b-tagging scale factor, mis-tagging scale factor, jet energy scale, generator, parton shower, and ISR/FSR uncertainties are re-estimated.  Again, the ISR/FSR uncertainty is very high here, and including this uncertainty in the BDT optimization might improve the result in future studies.  The other uncertainties are generally similar to the cut-based result, although we again see that the b-tagging scale factor uncertainty is lower than it was in the cut-based analysis from Section~\ref{sec:xs}.  The overall uncertainty is lower than the BDT cross-section expectation using only cut-based analysis variables, indicating the usefulness of the additional variables.  The expected cross-section is $\sigma_{t}= 65^{+30}_{-21}$~pb, while combined result from the BDTs using only cut-based variables had an expected cross-section of $\sigma_{t}= 65^{+34}_{-26}$~pb.  The observed cross-section value is $\sigma_{t}= 83.8^{+34}_{-25}$~pb, which has a lower uncertainty than, and is consistent with, the value found from the BDTs using only cut-based analysis variables.  In particular the central value is very similar.  This cross-section is also consistent with the cut-based analyis value.

\begin{table}[htdp]
\begin{center}
   \begin{tabular}{l|c}
    \hline
    Source & $\Delta\sigma/\sigma$ (\%) \\
    \hline \hline
    Expected statistics              & +11/-11  \\
     \hline  
    $b$ tagging scale factor     & +5/-5  \\
    Mistag scale factor          & +2/-2  \\
    Jet energy scale             & +4/-4  \\
    Generator                    & +8/-8  \\
    Shower                       & +11/-10  \\ 
    ISR/FSR                      & +34/-24  \\
    \hline  
    Total Systematics            & +45/-30  \\
    Total                        & +46/-32  \\
    \hline
   \end{tabular}
\caption{Systematic uncertainties for the expected $t$-channel cross-section measurement determined using the BDT created with 10 analysis variables, where the final line includes all systematic uncertainties and the statistical uncertainty of the data.  Uncertainties that were re-estimated versus the cut-based analysis (Section~\ref{sec:xs}) are listed individually.  Others are not listed but are included in the totals.}
\label{tab:xs-uncertainty-exp-abdt}
\end{center}
\end{table}
%Process: tchan
%     SysName     up[%]  down[%]  bias[%]
%  Data stat.    11.264  -11.264    0.000
%----------------------------------------
%      ltagSF     1.721   -1.721   -0.113
%         JES     3.537   -3.537   -0.161
%       ttgen     3.578   -3.578    0.089
%      btagSF     4.827   -4.827    0.134
%       stgen     6.943   -6.932    0.384
%    partshow    11.040  -10.443    3.581
%      isrfsr    33.640  -23.941   23.631
%       Total    44.972  -29.524   33.923

%Plot("/msu/data/t3work6/jenny/SPR/rootfiles/boosttreevar_comb10_testM3n150l2500_variables10_yield2jqcd.lepton.thesis.sigonly.root", 0.636667, plotdir, kFALSE, 2);
%Plot("/msu/data/t3work6/jenny/SPR/rootfiles/boosttreevar_comb10_testM3n150l1500_variables10_yield3jqcd.lepton.thesis.sigonly.root", 0.423333, plotdir, kFALSE, 3);

 \begin{figure}[!h!tpb]
 \centering
 \includegraphics[width=0.46\textwidth]{figures/appendix/bdtvar/BDTqcd_test_cut_2jall_classiTopBJet1_mass.data.eps}
 \includegraphics[width=0.46\textwidth]{figures/appendix/bdtvar/BDTqcd_test_cut_2jall_classiUJet1_Jet_eta.data.eps}
 \includegraphics[width=0.46\textwidth]{figures/appendix/bdtvar/BDTqcd_test_cut_2jall_classiLepton_charge.data.eps}
 \includegraphics[width=0.46\textwidth]{figures/appendix/bdtvar/BDTqcd_test_cut_2jall_classiHt_AllJetsLeptonMissingEt.data.eps}
 \includegraphics[width=0.46\textwidth]{figures/appendix/bdtvar/BDTqcd_test_cut_2jall_classiDeltaEta_BJet1UJet1.data.eps}
 \includegraphics[width=0.46\textwidth]{figures/variables/Plot_Legend.eps}
\vspace{-0.5cm}
 \caption{Discriminating variables for the 2 jets selection. The figures are after the selection on the BDT classifier formed using 10 analysis variables.  The last bin contains the sum of the events in that bin or higher.  The $t$-channel single-top contribution is normalized to the observed cross-section determined using all four channels. Other top refers to the $s$-channel and $Wt$ single-top contributions.}
 \label{fig:Plot_2ABDT}
 \end{figure}

\begin{figure}[!h!tpb]
 \centering
  \includegraphics[width=0.46\textwidth]{figures/appendix/bdtvar/BDTqcd_test_cut_2jall_classiInvariantMass_AllJets.data.eps}
 \includegraphics[width=0.46\textwidth]{figures/appendix/bdtvar/BDTqcd_test_cut_2jall_classiDeltaEta_BJet1Lepton.data.eps}
\includegraphics[width=0.46\textwidth]{figures/appendix/bdtvar/BDTqcd_test_cut_2jall_classiCos_Lepton_UJet1_RBJet1Top.data.eps}
 \includegraphics[width=0.46\textwidth]{figures/appendix/bdtvar/BDTqcd_test_cut_2jall_classiLepton_eta.data.eps}
 \includegraphics[width=0.46\textwidth]{figures/appendix/bdtvar/BDTqcd_test_cut_2jall_classiW_TransverseMass.data.eps}
 \includegraphics[width=0.46\textwidth]{figures/variables/Plot_Legend.eps}
\vspace{-0.5cm}
 \caption{Discriminating variables for the 2 jets selection. The figures are after the selection on the BDT classifier formed using 10 analysis variables.  The last bin contains the sum of the events in that bin or higher.  The $t$-channel single-top contribution is normalized to the observed cross-section determined using all four channels. Other top refers to the $s$-channel and $Wt$ single-top contributions.}
 \label{fig:Plot_2ABDT2}
 \end{figure}

 \begin{figure}[!h!tpb]
 \centering
 \includegraphics[width=0.46\textwidth]{figures/appendix/bdtvar/BDTqcd_test_cut_3jall_classiTopBJet1_mass.data.eps}
 \includegraphics[width=0.46\textwidth]{figures/appendix/bdtvar/BDTqcd_test_cut_3jall_classiUJet1_Jet_eta.data.eps}
 \includegraphics[width=0.46\textwidth]{figures/appendix/bdtvar/BDTqcd_test_cut_3jall_classiLepton_charge.data.eps}
 \includegraphics[width=0.46\textwidth]{figures/appendix/bdtvar/BDTqcd_test_cut_3jall_classiHt_AllJetsLeptonMissingEt.data.eps}
 \includegraphics[width=0.46\textwidth]{figures/appendix/bdtvar/BDTqcd_test_cut_3jall_classiDeltaEta_BJet1UJet1.data.eps}
 \includegraphics[width=0.46\textwidth]{figures/variables/Plot_Legend.eps}
\vspace{-0.5cm}
 \caption{Discriminating variables for the 3 jets selection. The figures are after the selection on the BDT classifier formed using 10 analysis variables.  The last bin contains the sum of the events in that bin or higher.  The $t$-channel single-top contribution is normalized to the observed cross-section determined using all four channels. Other top refers to the $s$-channel and $Wt$ single-top contributions.}
 \label{fig:Plot_3ABDT}
 \end{figure}

\begin{figure}[!h!tpb]
 \centering
  \includegraphics[width=0.46\textwidth]{figures/appendix/bdtvar/BDTqcd_test_cut_3jall_classiInvariantMass_AllJets.data.eps}
 \includegraphics[width=0.46\textwidth]{figures/appendix/bdtvar/BDTqcd_test_cut_3jall_classiDeltaEta_BJet1Lepton.data.eps}
\includegraphics[width=0.46\textwidth]{figures/appendix/bdtvar/BDTqcd_test_cut_3jall_classiCos_Lepton_UJet1_RBJet1Top.data.eps}
 \includegraphics[width=0.46\textwidth]{figures/appendix/bdtvar/BDTqcd_test_cut_3jall_classiLepton_eta.data.eps}
 \includegraphics[width=0.46\textwidth]{figures/appendix/bdtvar/BDTqcd_test_cut_3jall_classiW_TransverseMass.data.eps}
 \includegraphics[width=0.46\textwidth]{figures/variables/Plot_Legend.eps}
\vspace{-0.5cm}
 \caption{Discriminating variables for the 3 jets selection. The figures are after the selection on the BDT classifier formed using 10 analysis variables.  The last bin contains the sum of the events in that bin or higher.  The $t$-channel single-top contribution is normalized to the observed cross-section determined using all four channels. Other top refers to the $s$-channel and $Wt$ single-top contributions.}
 \label{fig:Plot_3ABDT2}
 \end{figure}