%% Before beginning to type your dissertation, read the formatting guide, 
%% which can be found at http://grad.msu.edu/etd/docs/formattingguide.pdf
\documentclass{msuphddissertation}
\usepackage{amsmath,amssymb,amsthm,paralist}
\usepackage{modatlasstyle}
\usepackage{mathrsfs}
\usepackage{graphicx}
\usepackage{lineno}
%% Include other packages you wish to use except setspace. 
%% That package is loaded automatically.
%% IMPORTANT: Load only those packages you know you will use.
%% Some packages can cause conflicts resulting in improper formatting.
\author{Jenny Lyn Holzbauer} %% Put your name in full as it is officially recognized by Michigan State University here.
\title{Measurement of Single-top t-channel Production using ATLAS Data} %% Put the title of your dissertation here.
\field{Physics} %% Put the name of the field of your degree (NOT department or division, or college) here.
%% For example Dynamical Systems, Psychology, String Theory, etc.

%% Put additional preamble items here.

%%%%%%%%%%%%%%%%%%%%%%%%%%%%
%%%%%%%%  NOTE   %%%%%%%%%%%%%%
%% PREPARING A DISSERTATION WITH THIS CLASS FILE DOES NOT %%%
%% GUARANTEE THAT THE GRADUATE SCHOOL WILL APPROVE IT %%%
%%%%%%%%%%%%%%%%%%%%%%%%%%%%%%%

%%%%%%%%%%%%%%%%%%%%%%%%%%%%%%%%%%
%%%%%%%%%%%% WARNING %%%%%%%%%%%%%%%
%% The Graduate School requires that all text, including superscripts %%
%% and subscripts at all levels, as well as that in imported %%
%% graphics files be in 12 point. For that reason it's recommended %%
%% that no text be part of any imported files. %%

%% Once your document has been filed with the Graduate School,
%% if you wish to produce a version of it whose subscripts and superscripts
%% are in traditional smaller proportion, remove the "%" sign 
%% in front of following command. 
%\DeclareMathSizes{12}{12}{10}{8}
%% If your document has footnotes, remove the "%" sign 
%% in front of following command. 
%\renewcommand{\footnotesize}{\small}

\begin{document}

\maketitlepage %%This command will produce the title page of your thesis.
\begin{abstract}
This document reports the measurement of the single-top t-channel cross-section using data from the ATLAS detector, located at the Large Hadron Collider on the border of France and Switzerland.  The data used were collected during the first half of 2011, from proton-proton collisions with a 7 TeV center-of-mass collision energy.  Single-top is electroweak top-quark production and t-channel is one of the standard model production modes.  To isolate this production, selections are applied to find events with a similar final state.  A cut-based analysis is used to further isolate the signal using a series of selections in several orthogonal kinematic regions.  Finally, a statistical analysis is performed to determine the measured cross-section and the CKM matrix element $|V_{tb}|$.  The cross-section for top and anti-top production is considered separately and the resulting cross-sections are $\sigma_{t^{+}}= 59^{+18}_{-16}$~pb for the plus charge channel and $\sigma_{t^{-}}= 33^{+13}_{-12}$~pb for the minus charge channel.  The total measured single-top t-channel cross-section using all kinematic channels in this analysis is  $92^{+29}_{-26}$~pb with an expected cross-section of $\sigma_{t}= 65^{+22}_{-20}$~pb.

%% Type your abstract here. An abstract is REQUIRED and limited to two pages.
%% The abstract must not include any figures.
\end{abstract}

%% If you wish to have a copyright page, remove the "%" in front of  \begin{copyrt}
%% and remove the "%" in front of \end{copyrt}.
%% The mandatory form of the Copyright will be generated automatically. 
%% A copyright statement is optional.

\begin{copyrt}
\end{copyrt}

%% If you wish to have a dedication, remove the "%" in front of
%% \begin{dedication}
%% and remove the "%" in front of 
%% \end{dedication}
%% A dedication must be single-spaced and 
%% centered on the page.  Both will be done automatically. 

%\begin{dedication} 
%% Type your dedication here. A dedication is optional.
%\end{dedication}

%% If you wish to have an acknowledgment, remove the "%" in front of  \begin{acknowledgment}
%% and remove the "%" in front of  \end{acknowledgment}  
%\begin{acknowledgment}
%% Type your acknowledgment here. An acknowledgment is optional.
%\end{acknowledgment}

%% If you wish to have a preface, remove the "%" in front of  \begin{preface}
%% and remove the "%" in front of  \end{preface}. The formatting of
%% a preface isn't specified. 
%\begin{preface}
%% Type your preface here. A preface is optional.
%\end{preface}

\tableofcontents

%% If your document contains tables, remove the "%" in front of  \listoftables 
%\listoftables 

%% If your document contains figures
%% remove the "%" in front of  \listoffigures
%\listoffigures
%% If any of your figures contain color, you must
%% include the following disclaimer in the caption of your first figure.
%% For interpretation of the references to color in this and all other figures, 
%% the reader is referred to the electronic version of this dissertation.

%%%% LIST OF SYMBOLS AND ABBREVIATIONS %%%%
%definitions in modatlassty file
%% Such a list is possible using the environment
%% abbreviationskey
%% at the place in the document where you wish the list to appear.
%% The list will be included in the TOC as KEY TO SYMBOLS AND ABBREVIATIONS
%%%%%%%%%%


\newpage
\pagenumbering{arabic}
%% Put the body of your dissertation here.
\linenumbers
\chapter{Introduction}
Science never rests. It constantly drives the boundaries of knowledge to new and unexpected realms. Through human history we have seen this knowledge progress from a practical, intuitive, and frequently incorrect understanding of the world to more rigorous models with greater predictive power than our ancestors could have ever dreamed. 
One of the themes seen throughout the history of science is the push to understand the basic building blocks of the universe. Ancient models posited four or five basic elements, made up of the most common materials found. In the 19th century, atomic theory was developed, which drove the smallest objects down to the atomic level, and then later even further when scientists discovered that atoms were made of protons, neutrons, and electrons. In the mid 20th century, scientists discovered that protons and neutrons were made of even smaller particles, which were named quarks~\cite{physicshistory}. Through the scientific process we probe the smallest scales, trying to understand the list of particles that we now consider fundamental.

Investigating these particles can be difficult, as the proton is tightly bound and high energies are required to break it apart. Even more energy is necessary to create the most massive particles we have discovered. To reach these massive energies an accelerator 24 kilometers in length, the Large Hadron Collider (LHC), has been constructed. At the LHC the proton is broken apart by accelerating two sets of protons to near the speed of light and colliding them. These collisions can create new particles, the products of which are detected at massive detectors built around the collision points. Through these collisions we study the properties of the known particles and, if we are lucky, discover new ones.

This dissertation will detail the search for a special kind of production of the most massive fundamental particle known, the top quark. This kind of production is known as the \Wtchan. In the following pages the workings of the LHC and the ATLAS detector will be discussed. From there I will explain the efforts required to go from a set of raw observations to a complete picture of the results of a collision. I will discuss how systematic uncertainties impact our measurement, and the steps we take to reduce them. Finally, the experimental and statistical methodology used to extract the measurements made will be detailed and the results will be shown.


\chapter{Single-top Production and the Standard Model}
%\chapter{Why Study Single Top-Quark Production?}

High energy physics deals with the very fundamental parts of our universe, the fundamental particles and forces.  Our present understanding is that there are four forces: gravity, electromagnetism, strong and weak.  As energies increase, it is predicted that these forces can be united into one force, starting with the electromagnetic and weak forces, which form the electroweak force.  Each force has a mediating particle, a force carrier, which governs interactions of various particles.  These particles are discussed in the next section.

Single-top production is the process where a top quark is created in an electroweak interaction.  As stated previously, the top quark is the most massive of the elemetary particles.  Only one top quark is produced in an electroweak interaction.  There is another version of top quark production using the strong force and involving a top and anti-top (\ttbar).  This was the process detected in 1995 to claim discovery~\cite{ttbardiscovery1, ttbardiscovery2}, meaning a likelihood of less than 0.0000006 of background events imitating the signal.  Single-top itself was only recently discovered~\cite{singletopdiscovery:group, singletopdiscovery:D0, singletopdiscovery:CDF} and the particular channel (t-channel) discussed in this document was separately observed in 2011~\cite{tchanneldiscovery} by the D0 collaboration at the Tevatron.  Shortly afterwards, measurements of t-channel single-top production were reported by the CMS experiment~\cite{CMStchannel} and the ATLAS experiment~\cite{ATLAS-CONF-2011-088,ATLAS-CONF-2011-101} at the LHC (see the next chapter for more details on the LHC and ATLAS).  

This single-top t-channel process is still very new and, as such, has not been fully studied.  It is possible that deviations could be found in its various fundamental properties, which could indicate new particles or anomalous parts of the standard model.  In this document, we will measure the cross-section of this process and consider its kinematics.  But first, before performing a new measurement, we must understand what is already known.

\section{The Standard Model Particles}
The standard model is the basic theory of particle physics~\cite{QuarkModelReview, PDGSummary, Griffiths, Halzen} and was formulated in the 1960's and 1970's.  It describes and predicts various particles and their properties based on symmetry relations.  The model divides the fundamental particles into several categories and subcategories, pictured in Figure~\ref{fig:SM}.  There are three major particle catagories: leptons, quarks and bosons (force carriers).  There are also three generations of the quarks and leptons, where each generation is designated by a different shade in the figure.
  
\begin{figure}[!htpb]
  \centering
    \includegraphics[width=0.50\textwidth]{figures/theory/QuarkTable.eps}
    \label{fig:SM} 
\caption{The known standard model particles.  Different generations of quarks and leptons are indicated by different shades.}
\end{figure}

\subsection{Leptons}
One category of particles are the leptons.  Leptons typically include electrons, muons, taus, and their corresponding neutrinos.  However, we will use this term in this document to generally refer to electrons, muons and/or taus, while neutrinos are considered as a separate set of particles.  The electron is a stable lepton.  The leptons do not interact via the strong force and so are not involved in hadronic bound states like the proton.  Electrons are involved in the structure of the atom.  Muons and taus are heavier than the electron and decay to other particles.  This is particularly true for the tau.  Unlike the tau, the muon survives long enough to escape our detector which is important for particle identification, but has a short decay time relative to the electron.  For each lepton there is a corresponding anti-lepton with opposite charge.  

There is also a corresponding neutrino flavor for each lepton flavor, or type (electron neutrino, muon neutrino, and tau neutrino).  Each lepton and its corresponding neutrino form a set of particles where each lepton type is considered a separate generation (there are three generations).  Neutrinos are the lightest of the known particles and have no charge.  There are no known right handed neutrinos or left handed anti-neutrinos, where right handed indicates spin and momenta are in the same direction and left handed indicates the opposite.  This difference, rather than a difference in charge, distinguishes the neutrino and anti-neutrino.

Because neutrinos are nearly massless and neutral, they are very difficult to detect.  Neutrinos usually pass right through detectors without interacting.  This makes neutrino astrophysics possible, because neutrinos from distant sources will travel from the source without interacting and scattering off clouds of matter between the source and Earth.  However, this is problematic for collider physics.  Particles from collisions need to interact with matter and deposit their energy into the detector to be detected.  While this may not always happen, it should happen nearly 100\% of the time to prevent uncertainties on the measurements from getting large.  To handle neutrinos, we don't build a dedicated neutrino detector but instead make use of event kinematics to account for the neutrino via missing energy in the event.  This will be discussed further in Section~\ref{sec:Neutrinos}.  

\subsection{Quarks}
%1/lambdaQCD = 1/200MeV ~ 10^-24 s 
% decay width 1.3 GeV/c^2 >> lambdaQCD~200 MeV
Quarks are arranged like the leptons into generations, as seen in Figure~\ref{fig:SM}.  They are different from the leptons because they can interact via the strong force and form bound states, like the proton.  There are three generations and each contains two particles, making six different flavors in total (u, d, s, c, t, b).  The first generation contains lighter quarks, up (u) and down (d), the only stable quarks.  The second generation contains strange (s) and charm (c) quarks, and the third contains the heaviest quarks, bottom (b) and top (t), which are sometimes also called beauty and truth.  The first three (u, d, s) are typically called light quarks, and the charm quark is sometimes included in this category as well, but for the purposes of this document will either be considered separately or considered to be a heavy quark.  The bottom quark is considered heavy, but the top quark though is by far the heaviest and this is a distinguishing characteristic of this quark.  It is also special because it will decay to other particles before it hadronizes (unlike the other quarks) preserving ``bare quark'' information in its decay products.  This is because its decay time is ~$0.5\times 10^{-24}~s$~\cite{PDGSummary}, which is shorter than the hadronization time scale.  This scale, $\Lambda^{-1}_{QCD}$, corresponds roughly to $10^{-23}~s$~\cite{TopHadronization}.  Other quarks survive longer than this scale and will hadronize instead of decay, which means they produce a bound state of mesons or baryons.  Mesons are combinations of quarks and anti-quarks, while baryons are combinations of three quarks.

\subsection{Force Carriers}
The other major particle category contains the force carriers, or gauge bosons. One of these is the gluon (g), particularly involved in strong interactions and can also self-interact.  Photons ($\gamma$) are the force carriers of the electromagnetic interaction, but are not usually involved in the single-top interactions.  The other mediators are bosons associated with electroweak interactions, the Z and W.  Both are relatively heavy (80 to 90 GeV) compared to the other particles at this scale, and are about half as heavy as the top quark.  Additionally, it has been postulated that there is a Higgs boson and a Higgs field which gives mass to the particles in the standard model.  However, at the time of publication this has not been observed, so we will not go into detail about it here.

\section{Particle Properties and $|V_{tb}|$}\label{sec:vtb}
The standard model particles have very different characteristics, including a variety of charges and masses.  Anti-particles are designated with a bar over the top of their symbol and have the negative of the normal particle's charge.  A particle's charge is given as a fraction of the elementary charge, $e=1.6\times 10^{-19}$ Coulomb.  The down, strange and bottom quark all have -1/3 charge while the up, charm and top quarks have +2/3 charge.  The electrons, muons, and taus have -1 charge while the neutrinos, photon, and Z boson have 0 charge.  The W boson has $\pm1$ charge.  Additionally, particles also have a flavor, as discussed previously, and quarks have a color charge.  The color charge is much like the electric charge but related to the strong interaction (hence its relation to quarks).  The allowed meson and baryon bound states are determined by the color charge.

The particle masses vary over several orders of magnitude.  The range of quark and lepton masses (neutrinos are not pictured), are displayed in Figure~\ref{fig:QuarkMass}.  Notice that there are three quarks with masses of 1 GeV or larger, the c, b, and t quarks.  The mass of the top quark is of particular interest in this document and we use the value 172.5 GeV, which is consistent with the current Particle Data Group value~\cite{PDGSummary}.

\begin{figure}[!htpb]
  \centering
    \includegraphics[width=0.45\textwidth]{figures/theory/LeptonMasses.eps}
    \includegraphics[width=0.45\textwidth]{figures/theory/QuarkMasses.eps}
    \label{fig:QuarkMass} 
\caption{The standard model leptons (left) and quarks (right), by mass.}
\end{figure}

%mention Z, photon?
The gluon and W boson have the special properties that they can effectively change the color and flavor of a quark, respectively.  A gluon for instance may form a vertex with top and anti-top but not two tops.  Gluons are special as well because they may self-interact and form more gluons.  The t-channel single-top process involves the W boson and thus flavor exchange.  For example, a vertex involving a W$^{+}$ may include a top and anti-bottom quark, but not two top quarks.  The probability that a W vertex could involve a top and a down or strange quark is nearly zero according to the standard model, while the probability of top and bottom is nearly one.  This is displayed in the Cabibbo-Kobayashi-Maskawa (CKM) matrix~\cite{CKM1, CKM2}, which is nearly, but not quite, a unit matrix.  

The CKM matrix describes how likely it is for a quark to change to a quark of another flavor.  Specifically, the probability is the relevant matrix entry squared.  These are traditionally called $V_{qq'}$, where q is a quark and q' is the quark in another flavor.  More information about the values may be found in the PDG~\cite{PDGSummary}.  The matrix element we will be particularly interested in, $V_{tb}$, may be indirectly measured with \ttbar~production, but directly observed with single-top production.  The standard model value is 1.

%descibe what gamma mu really is, etc
The single-top cross-section, related to the number of single-top events produced in the collider, is derived from the square of the matrix element, M.  The matrix element varies as follows in the standard model, where $P_L$ is $1/2(1-\gamma^5)$:
\begin{equation} M \propto \bar{b}\gamma^{\mu}V_{tb} P_L t\end{equation}
Thus, the cross-section is proportional to $|V_{tb}|^2$ in the standard model.  If we allow anomolous couplings in this term above some new physics scale, the term $V_{tb}$ may be rewritten as $V_{L}$, where $V_{L}$ is just $V_{tb}$ plus a factor that depends on the new physics scale.  

The Lagrangian, allowing anomolous coupling terms, may be written as follows~\cite{Vtbtheory2, Vtbtheory}.  Here $P_R$ is $1/2(1+\gamma^5)$, $M_W$ is the W boson mass, the $\gamma$ and $\sigma$ terms are constant (Dirac or Pauli) matrices, g is a coupling constant, $q_{\nu}$ is the W boson momentum four-vector, and $\bar{b}$, t, and $W^{-}_{\mu}$ are field terms for the bottom quark, top quark, and W boson respectively:
\begin{equation}
L_{Wtb} = \frac{g}{\sqrt{2}}\bar{b}\gamma^{\mu}(V_L P_L + V_R P_R )tW^{-}_{\mu} - \frac{g}{\sqrt{2}}\bar{b} \frac{i\sigma^{\mu\nu}q_{\nu}}{M_W} (g_L P_L + g_R P_R )tW^{-}_{\mu} + ... 
\end{equation}
We will assume the anomolous couplings $V_R$, $g_L$ and $g_R$ are 0 in this document and will measure the value $|V_L|$ to see if it deviates from the standard model expectation.

Because the matrix element squared is proportional to the cross-section, by using both expected and observed cross-section and $|V_{L}|$ values, one may write:
\begin{equation} |V_{L,obs}|^2=\frac{\sigma_{t,obs}}{\sigma_{t,sm}}|V_{L,sm}|^2
\label{eq:vtbxs}
\end{equation}
where $\sigma$ is the cross-section, obs refers to the observed value and sm refers to the standard model.  In this way, one may directly find the $|V_{L}|$ value from a single-top observation.  The standard model expectation for $V_{L,sm} = V_{tb}$ is 1, so a value greater than 1 for $|V_{L}|$ would indicate non-standard model couplings.

%Need to talk about PDFs somewhere
\section{Overview of Physics Processes}
Unfortunately, the LHC collisions do not just produce single-top events, nor are single-top events extremely distinct or more common than the many other processes that are produced.  In this section we overview the single-top processes, other physics processes, and some of the characteristics that will be considered in order to distinguish them.

\subsection{Single-top and Other Processes}
 \label{sec:Feynman}
%Mandelstam variables = s, t  include this?
Feynman diagrams are a common way to visualize particle physics interactions and also the equations that describe them.  In these diagrams, time flows from left to right.  The leftmost particles are the initial particles (initial state) and the rightmost particles are the final particles (final state).  Figure~\ref{fig:Feynman_singletop} shows the Feynman diagrams for single-top processes.  There are three different production modes: $t$-channel, $Wt$, and $s$-channel.  $Wt$ is also known as associated production and $Wt$-channel.  The $t$-channel production is a scattering interaction while the $s$-channel production is an annihilation interaction.  These are standard high energy physics terms.  In this dissertation, the signal channel is the $t$-channel mode and the other two are considered to be backgrounds.  

Notice in each case there is exactly one top quark, the characteristic of single-top production.  The top quark comes from an interaction mediated by a W boson except in the case of Wt production, where it is produced along with a W boson.  The $t$-channel in particular has two quarks in the initial state, a b-quark (or a gluon producing a b-quark, as is shown) and generic q-quark.  This q-quark is usually a valence quark, while the b-quark may come from the sea of quarks in the proton or from a gluon.  The final state involves the lone top quark and another generic quark in the opposite flavor of the initial q-quark, which is often energetic and forward (close to the beam line).  It is also possible to have an extra jet in the final state from a gluon in the initial state.  Incidentally, in the previous section we noticed the mass of the W boson is smaller than the top quark, but it is still possible for a W to produce a top quark if it is a virtual W, in the s-channel diagram for example.

Although it is not shown in these diagrams, the top quark decays to a W and b-quark.  The W further decays to either a lepton and neutrino or two quarks.  For this analysis, we will focus on the lepton decay case.%  This decay chain is pictured in Figure~\ref{}.
 
\begin{figure}[!h!tpb]
 \centering
 \includegraphics[width=0.30\textwidth]{figures/theory/t_channelNLO.eps}
 \includegraphics[width=0.30\textwidth]{figures/theory/wt1.eps}
 \includegraphics[width=0.30\textwidth]{figures/theory/s_channel.eps}
%\vspace{-0.5cm}
 \caption{Feynman diagrams for single-top production.  The signal is the diagram on the left, $t$-channel single-top production.  It is also possible to have a version of this diagram without the incoming gluon and outgoing $\bar{b}$.  The central diagram is $Wt$ production and the diagram on the right is $s$-channel single-top production.}
 \label{fig:Feynman_singletop}
 \end{figure}

Figure~\ref{fig:Feynman_background} shows the diagrams for the other backgrounds for our single-top $t$-channel signal.  These include multijets (also called QCD), W+jets, Z+jets, \ttbar, and diboson (includes WW, WZ, and ZZ), where jets are streams of particle decays and interactions stemming from quarks that have hadronized.  Of these, only \ttbar~contains top quarks and it contains two of them, rather than the one top quark that single-top t-channel contains.  Nevertheless, it is difficult to distinguish single-top t-channel from its backgrounds.  This is partly because the final states can appear to be quite similar, especially given that the detector is not perfect at particle identification, and partly because of the smaller number of expected signal events relative to background events.
 
\begin{figure}[!h!tpb]
 \centering
 \includegraphics[width=0.30\textwidth]{figures/theory/ttbar.eps}
 \includegraphics[width=0.30\textwidth]{figures/theory/wjet.eps}
 \includegraphics[width=0.30\textwidth]{figures/theory/qcd.eps}

\vspace{8.00mm}

 \includegraphics[width=0.30\textwidth]{figures/theory/zjet.eps}
 \includegraphics[width=0.30\textwidth]{figures/theory/diboson.eps}
%\vspace{-0.5cm}
 \caption{Feynman diagrams for backgrounds to single-top production.  The \ttbar~is the diagram on the top left, $W$+jets is the top central diagram and multijet production is the top right diagram.  The final two diagrams are the smallest backgrounds, $Z$+jets on the bottom left and diboson on the bottom right.}
 \label{fig:Feynman_background}
 \end{figure}

Although the diagrams in Figures~\ref{fig:Feynman_background} and~\ref{fig:Feynman_singletop} are basic, straight-forward diagrams, it is possible in more complex diagrams with extra gluons in the initial or final state, or loops of particle production (such as a gluon making two gluons which in turn recombine into one gluon).  These extra possibilities can be described in separate diagrams, and it is possible to have diagrams from two different processes which give the same final state.  In this case, the diagrams are said to interfere and it is important to consider such things when generating Monte Carlo (MC).  For the signals and backgrounds considered in this analysis, there are two cases of interference in particular.  The first is single-top Wt production and \ttbar~production.  Wt is already very similar to \ttbar~except for a b-quark (a top quark decays to a W and a b-quark).  If a gluon in the Wt initial state produces the incoming b, it will also produce an outgoing $\bar{b}$, making the final state look like \ttbar.  This does not have a large contribution to the analysis however, as this $\bar{b}$ is generally low in \pt.  The \pt~requirement for jets (see Section~\ref{sec:quarks}) generally removes this from consideration.  However, the scenario is considered when MC is generated.

Another possibility involves the t-channel single-top signal.  If the incoming quark is from a gluon instead of a valence quark, the gluon produces the incoming quark plus an extra quark for the final state.  The new final state contains two light quarks, t, and a possible b.  This is the same final state discussed in the last paragraph, if the W decays to two light quarks instead of a lepton and neutrino in the case of Wt and for one of the top quarks in the case of \ttbar.  However, this sort of initial state is very uncommon.  Also, the extra particles from the gluons often having low \pt (thus not satisfying the jet definition), so this scenario does not impact the analysis.

\subsection{Cross-section}
 \label{sec:ExpectedCrosssections}
The cross-section for a process reflects how often we expect a collision to produce that particular process.  It is often useful to think of it in terms of the number of events produced:
\begin{equation} N = \sigma L \label{equ:numberofevents} \end{equation}
where N is the number of events, $\sigma$ is the process cross-section and L is the integrated luminosity (explained in Section~\ref{sec:LHC}), which represents the number of collisions.

The cross-section may include additional factors like k-factors or branching ratios.  The k-factors are corrective fractions which change a cross-section from a leading order to next-to-leading order value, for example.  A leading order (LO) cross-section is a theory calculation involving just basic diagrams without loops or extra vertices.  Next-to-leading order (NLO) includes an additional level of complexity of loops and vertices, making it a longer, more difficult calculation.  With each step up in completeness, the calculation becomes more technically difficult, so we do not have exact theoretical cross-sections for our processes.  

Branching ratios (BR) are just fractions to change the total cross-section for a complete process to a partial cross-section.  For instance, in the $t$-channel diagram, the final state involves a top quark, which decays to a b quark and a W.  This W may decay either to two more quarks, like up and down quarks, or to a lepton and neutrino.  For reasons discussed in Section~\ref{sec:EventSelection}, we require exactly one lepton in our selection and only generate Monte Carlo for final states including a lepton.  Thus, the cross-section we actually normalize the MC to is the fraction of the total cross-section that involves a single lepton in the final state.  The probability for the W to decay to a lepton and neutrino is a branching ratio that is multiplied with the cross-section.

The cross-sections used in this analysis for the signals and backgrounds are listed in Table~\ref{TABLE-MCSAMPLES} for 7 TeV collisions, including k-factors (if applicable) and branching ratios.  They are given in the standard units of picobarns ($pb$), where a barn is $10^{-28}~m^{2}$.  The W+jets is divided here by flavor in some cases.  A special procedure is applied to separate the W+jets into light and heavy flavors later in the analysis based on the truth-level hadron type, in a way that avoids double-counting events.  Truth level refers to monte carlo information about the particle generated before applying detector effects (see Chapter~\ref{chap:partreco}).  The division is into light (u,d,s), c, and heavy ($c\bar{c}$ and $b\bar{b}$) for the jets (the processes may also have additional light jets).  The k-factors used here are 1.20 for W+jets in general (except W+cjets, which uses 1.52), 1.25 for Z+jets and 1.12 for \ttbar.

Measuring $t$-channel single-top production is the focus of this analysis but we would like to compare the measurement with an expected cross-section value.  For this we use the cross-sections given in Table~\ref{TABLE-STOP}~\cite{Kidonakis:2010tc,Kidonakis:2010ux, Kidonakis:2011wy}, and the branching ratios from the PDG, with values of 10.75\% for $W \rightarrow e \nu_{e}$, 10.57\% for $W \rightarrow \mu \nu_{\mu}$, and 11.25\% for $W \rightarrow \tau \nu_{\tau}$~\cite{PDGSummary}.  The cross-sections contain both top and anti-top contributions, and we expect these particle and anti-particle contributions to be different for processes that have valence quarks in the initial state.  The LHC collides two protons, each of which contains two up and one down quarks, leading to an excess of positively charged quarks.  For the $t$-channel, which usually has a valence quark in the initial state, the standard model cross-section contains $41.9$~pb due to events containing $t$ quarks and $22.7$~pb from events containing $\bar{t}$ quarks.
\begin{table}[htdp]
\begin{center}
\begin{tabular}{l|r}
 \hline
 Process        & Cross-section [pb] \\
\hline
 $t$-channel & 64.57 + 2.71 - 2.01 pb\\
 $Wt$ &  15.74 + 1.06 - 1.08 pb \\
 $s$ channel &  4.63 + 0.19 - 0.17 pb \\ 
\hline
\end{tabular}
\caption{ (N)NLO cross-sections for single-top processes~\cite{Kidonakis:2010tc,Kidonakis:2010ux, Kidonakis:2011wy}}
\label{TABLE-STOP}
\end{center}
\end{table}

It is also interesting to note that the difference between the signal cross-section and background cross-sections are different by many orders of magnitude, as seen in Table~\ref{TABLE-MCSAMPLES}.  This means that many, many events have to be identified correctly and rejected in order to try to pick out our needle in this immense haystack.

\begin{table}[htdp]
\begin{center}
\begin{tabular}{l|r}
\hline
Process        & Cross-section [pb] \\
\hline\hline
$t$-channel $\to e\nu_{e}$ & 6.9 \\
$t$-channel $\to \mu\nu_{\mu}$ & 6.8 \\
$t$-channel $\to \tau\nu_{\tau}$ & 7.3 \\
\hline
\ttbar~(non-hadronic) & 90 \\
$Wt$ & 16 \\
$s$-channel $\to e\nu_{e}$ & 0.50 \\
$s$-channel $\to \mu\nu_{\mu}$ & 0.49 \\
$s$-channel $\to \tau\nu_{\tau}$ & 0.52 \\
\hline 
$Z$ + 0 jet   & 835  \\
$Z$ + 1 jets  & 168 \\
$Z$ + 2 jets  & 51 \\
$Z$ + 3 jets  & 14 \\
$Z$ + 4 jets  & 4  \\
$Z$ + 5 jets  & 1  \\
\hline
$W$ + 0 jet   & 8,300 \\
$W$ + 1 jets  & 1,600 \\ 
$W$ + 2 jets  &   460  \\
$W$ + 3 jets  &   120 \\
$W$ + 4 jets  &    31 \\
$W$ + 5 jets  &    8  \\
\hline
$W+b\bar{b}$ + 0 jet   & 57 \\
$W+b\bar{b}$ + 1 jets  & 43 \\
$W+b\bar{b}$ + 2 jets  & 21 \\
$W+b\bar{b}$ + 3 jets  & 8  \\
\hline
$W+c\bar{c}$ + 0 jet   & 153 \\
$W+c\bar{c}$ + 1 jets  & 126 \\
$W+c\bar{c}$ + 2 jets  & 62 \\
$W+c\bar{c}$ + 3 jets  & 20 \\
\hline
$W+c$ + 0 jet   & 980 \\
$W+c$ + 1 jets  & 312 \\
$W+c$ + 2 jets  & 77  \\
$W+c$ + 3 jets  & 17  \\
$W+c$ + 4 jets  & 4  \\ 
\hline
$WW$                         & 17 \\
$WZ$                         & 6 \\
$ZZ$                         & 1 \\
\hline\hline
\end{tabular}
\caption{ Cross-sections, including branching ratios and k-factors. Shown for one lepton decay (ex. electron) in the case of Z+jets and W+jets processes.  Single-top $s$-channel and $t$-channel list different lepton decays for the W separately to show the branching ratios used.}
\label{TABLE-MCSAMPLES}
\end{center}
\end{table}

Most of these cross-sections listed are fairly well known, partly because they are so much larger (relatively), and large statistics samples have been available for some time at long running experiments (like the Tevatron experiments) with relatively low systematic uncertainties.  However, lower cross-section processes such as our signal have only recently been observed and the cross-sections are not necessarily well measured.  The goal of this analysis is to provide a cross-section measurement of the $t$-channel process and to see if it agrees with the standard model prediction.

\section{New Physics Possibilities}
In recent years, there have been several indications that the standard model does not explain everything.  Although the standard model has been very successful, observations in astronomy have indicated the presence of so called dark matter~\cite{DarkMatterReview} and dark energy~\cite{DarkEnergySN} which are not predicted by the standard model and in proportions larger than the standard model matter we know of~\cite{WMAP}.  Several theories have been proposed to account for this
%~\cite{SUSY, ExtraDimensions}
, but none have been shown to exist.  It is importation to check the standard model with detailed experimental measurements, to confirm the standard model and perhaps gain information about new physics if deviations are discovered.

Single-top $t$-channel production, a standard model process, is interesting because it is still new and not fully examined.  Although it is now known to exist, we are only now accumulating enough events to do precision measurements of the cross-section.  If this cross-section is not consistent with the standard model, it may indicate new physics.  It is possible that there could be a flavor changing (like the W) neutral current (neutral like the Z) in the process for instance, that would change the Wtb vertex.  It is also possible there could be a fourth generation of quarks, which would again cause the CKM matrix $V_{tb}$ value to deviate the standard model value.  Detailed measurements of the single-top production allow direct evidence of these phenomena.

In this document, we take the first step, which is to measure the $t$-channel single-top cross-section and compare it to the standard model value.  We do this by applying a small number of kinematic requirements to the events, to provide a straight-forward measurement.  This is the first cut-based analysis with this level of precision on the single-top t-channel cross-section.  It is also possible to use more sophisticated statistical methods to do a measurement of this signal, and the usefulness of this approach is explored in Appendix~\ref{app:MultivariateApproach}.


\chapter{ATLAS and the LHC}
In order to study the single-top $t$-channel cross-section, we must first collect information from these rare events.  The top quark is very massive and requires a large amount of energy to produce events containing it.  To do this, we generate high energy particles in beams and collide them together in an underground ring.  Here, we can collect most of the information about the particle tracks and energies and also produce a large number of these collisions.  This last point is crucial for low cross-section processes like the signal in this analysis.

\section{The Large Hadron Collider, a Short Overview}\label{sec:LHC} 
The Large Hadron Collider~\cite{LHCoverview}, or LHC, is the particle collider in question, a proton-proton collider located on the border of Switzerland and France, near Geneva, Switzerland.  It is 26.7 km in circumference, or 5.3 miles in diameter, and the beams collide with a center-of-mass energy of 7 TeV during normal data taking.  This is half of the design center-of-mass energy (14 TeV) and is what is used for the data in this analysis.  The first 7 TeV collisions occurred in March of 2010, and this document considers data taken in the first half of 2011.

The LHC is the main ring, which reuses the former LEP tunnel, and there are several other rings that boost the beam up to its injection energy of 450 GeV.  First, though there is the proton source, where hydrogen gas is separated into protons and electrons via a magnetic field.  The protons are then sent into the first part of the accelerator complex, a linear accelerator called the LINAC2.  It is also possible to collide lead in the LHC and in this case a different source is used, but for our purposes we will focus on the standard proton-proton collisions.  After reaching the LINAC2, the protons go through circular accelerators to boost the beam energy, the Proton Synchrotron Booster (PSB), Proton Synchrotron (PS), and Super Proton Synchrotron (SPS) systems, before being injected into the LHC.

The protons are formed into bunches and trains of bunches before they are collided, incidentally allowing some spacing for one collision's particles to decay or leave the detector before the next set of particles collide.  Within one bunch there are about 100 billion protons, not all of which actually collide or collide to produce interesting events.  Each bunch is spaced apart by 50 ns and the number of bunches in the ring has been increasing steadily as data taking has progressed, up to about 1000.  
%shouldn't bunch train spacing get smaller with more bunches...?
%bunch train spacing?  Spacing between trains depends on the number of bunches in the train.  About 225ns for 36 bunches, about 1000ns for 144 bunches used in June (need source).  http://blog.vixra.org/2011/05/29/new-luminosity-record-for-lhc/  MC has 225ns spacing.
%Because the ring is 26.7 km around, this means each bunch is spaced apart by only about XXX m near the end of the data set we will consider in this analysis.

Of course, the reason these bunches are put so close together and contain so many particles is related to getting enough data to find the single-top t-channel production we are looking for.  The instantaneous luminosity~\cite{Luminosityoverview}, which reflects how many events are produced, is determined by various accelerator settings:

\begin{equation} 
L = \frac{f_{r}n_{1}n_{2}n_{b}\gamma_{r}F(\theta,\sigma)}{4\pi \epsilon_{n}\beta^{*}}
\end{equation}

Here $f_{r}$ is the frequency the protons go around the main LHC ring (approximately the speed of light, c,  divided by 27 km), $n_{1}$ and $n_{2}$ are the number of protons per bunch, $n_{b}$ is the number of bunches in each beam, $\epsilon_{n}$ is the normalized emittance (related to the deviation of particles from the ideal beam and thus also beam lifetime), and $\beta^{*}$ is related to the beam focus at the interaction point.  The combination $\epsilon_{n}\beta^{*}$ is the overall beam size at the collision point.  Here, the $\beta^{*}$ is 1.5 m and the emittances are on the order of $\mu m$, $4 \times 10^{-6}~m$~\cite{LHCdetail}.  The $\gamma_{r}$ is the relativistic $\gamma$, which is just the beam energy (3.5 TeV per beam) divided by the proton mass (about 1 $GeV/c^2$), about $3.5\times 10^{3}$.  Finally, the $F(\theta,\sigma)$ is a geometrical luminosity reduction factor 
% involving the Piwinki parameter, 
related to the beam size and crossing angle, and is about 0.84~\cite{LHC2}.  The peak instantaneous luminosity varies day by day (it will fall off as a data collection run goes on), but is approximately $1 \times 10^{33}cm^{-2}s^{-1}$ for the time period in question.  Following the equation, and putting in approximate values, we can get a similar number:
\begin{equation} 
L = \frac{10^{4}s^{-1} \cdot 10^{11}\cdot 10^{11}\cdot 10^{3}\cdot 3.5\times10^{3}\cdot 0.84}{4 \pi \cdot 4\times 10^{-4} cm \cdot 1.5\times 10^{2} cm} = 4 \times 10^{32}cm^{-2}s^{-1}
\end{equation}

With such tight beam focus and so many particles, it is possible to have more than one collision per bunch crossing.  On average, for the data we consider here, there are about six interactions per crossing.  The impact of the change in $\beta^{*}$ for the data used for this analysis and the following data can be seen in Figure~\ref{fig:betastar}.  The decrease in $\beta^{*}$ approximately doubled the number of events per crossing in later data sets.  The lower number of interactions per crossing is an advantage of the data set used in this document.  Most of these extra interactions are not interesting, but it is possible that the events could mix in a way that confuses the event identification.  Studies are done to check that the analysis is not biased by these ``pileup'' effects.

\begin{figure}[!htpb]
  \centering
    \includegraphics[width=0.75\textwidth]{figures/detector/interactionspercrossing2011.eps}
    \label{fig:betastar} 
\caption{Average number of interactions per crossing for the 2011 ATLAS data set for different values of $\beta^{*}$ used by the LHC.  The $\beta^{*} = 1.5~m$ data set was used for the analysis in this document~\cite{betastar}. ATLAS Experiment \copyright 2011 CERN.}
\end{figure}

What is typically quoted is not the instantaneous luminosity but the integrated luminosity (luminosity for a given period of time).  This is usually expressed in units like $pb^{-1}$ (a barn is $10^{-28} m^{2}$), which means this can be easily multiplied by a cross-section in $pb$ to determine the number of events expected, as seen in Section~\ref{sec:ExpectedCrosssections}.  For this analysis, we are considering 1035.27 $pb^{-1}$, or 1.04 $fb^{-1}$.

However, even after all of these events are produced, nothing can be measured without a detector to collect the relevant information about the collision.  The information provided is not a snapshot of the interaction we are interested in like the Feynman diagrams in Section~\ref{sec:Feynman}, but rather the final, relatively stable particles that come out of it which actually reach the detector.  There are four different detectors located around the LHC ring at different points where the beams cross to produce collisions, and this analysis uses data from the ATLAS detector.

\section{The ATLAS Detector}
The ATLAS (A Toroidal LHC ApparatuS) detector~\cite{ATLASoverview}, shown in Figure~\ref{fig:ATLAS} is a multipurpose detector designed to detect many different processes.  It is a very large detector, the largest constructed by volume, and is about 25 meters (or 82 feet) high.  It consists of several different detector components designed to detect the various particles that travel through it.  In general, these include b-quarks, lighter quarks, electrons, and muons (as well as photons, but these don't appear in our final state).  The quarks hadronize to form ``jets'' of particles which are actually detected in the detector.
\begin{figure}[!htpb]
  \centering
    \includegraphics[width=1.00\textwidth]{figures/detector/ATLAS_cutaway.eps}
    \label{fig:ATLAS} 
\caption{Cut away view of the ATLAS detector~\cite{ATLASoverview}, ATLAS Experiment \copyright 2008 CERN.}
\end{figure}

\subsection{Detector Variables and Geometry}
There is certain information that is determined in the detector itself: energy, timing and particle track information.  The layout of the detector is with the z-axis along the beamline.  The y-axis points up vertically from the detector and the x-axis is the remaining direction, pointing towards the center of the LHC ring.  The $\phi$ direction is the angle measured in the x and y plane, starting from the positive x axis, and the $\theta$ direction is measured in the y and z plane, starting from the positive z axis.  Figure~\ref{fig:ATLASGEO} shows the orientation of the axes with respect to the surrounding area.

\begin{figure}[!htpb]
  \centering
    \includegraphics[width=1.00\textwidth]{figures/detector/LHCexperiment_axis1.eps}
    \label{fig:ATLASGEO} 
\caption{View of ATLAS, the LHC and other experiments~\cite{ATLASaxes}, ATLAS Experiment \copyright 1999 CERN.  The black axis lines are added for reference.}
\end{figure}

The $\theta$ angle is generally not used as such but transformed into a quantity called pseudo-rapidity ($\eta$):

\begin{equation} \eta = -ln(tan(\theta/2)) \end{equation}

%rapidity vs pseudo rapidity?
This quantity is 0 if the particle heads out of the interaction perpendicular to the beam ($\theta = 90^{\circ}$) and is about 4.5 close to the beamline ($\theta= 1^{\circ}$), at the limit of the detector.  Figure~\ref{fig:EtaTheta} shows absolute values of $\eta$ for various values of $\theta$.

\begin{figure}[!htpb]
  \centering
    \includegraphics[width=0.60\textwidth]{figures/detector/EtaTheta.eps}%was 0.45 for size
    \label{fig:EtaTheta} 
\caption{Graphic showing absolute values of $\eta$ for various values of $\theta$.}
\end{figure}

It is also common to use the quantity $\Delta R$ as a measure of separation.  We define this as:
\begin{equation} \Delta R^2 = \Delta \eta^2 + \Delta \phi^2 \end{equation}
Additionally, the term transverse in this document (denoted by a subscript T) means the combination of the X and Y directions, which is perpendicular to the beam direction along the z axis.  For instance, \pt is the transverse particle momenta, $\rm \sqrt{p_{X}^2 + p_{Y}^2}$.


\subsection{The Inner Detector}
%give pixel physical size? 
The inner detector has the finest resolution of the various sub-detectors.  The fine resolution is particularly important for identifying and reconstructing hadronized b-quarks, or b-jets, which will be discussed in more detail in Section~\ref{sec:Btag}.  The primary purpose of the inner detector is tracking.  This is also the closest detector to the beam pipe in the central region and covers a region of $|\eta| < 2.5$.

There are three major sections: Pixel, SemiConductor Tracker (SCT), and Transition Radiation Tracker (TRT).  The Pixel is the innermost section and has an initial layer called the B-layer.  The closeness of this layer to the interaction is limited by the beam pipe itself, which is about 6 cm in diameter.  This section of the detector is composed of small squares of silicon (pixels) and offers very good postitional resolution, 10 $\mu m$ in R-$\phi$ space and 115 $\mu m$ in Z.  As charged particles hit the silicon, ionization electrons flow to anodes and a signal is created.  There are three pixel layers circling the barrel region and an additional three layers on each side.  The next section is the SCT.  It is very similar to the pixel section but has microstrips of silicon about 6 cm long rather than pixels.  It has four layers of back-to-back strips giving a possible 8 hits per track.  The resolution is still good although not quite as precise as the pixel region particularly in the Z direction (17 $\mu m$ in R-$\phi$ space and 580 $\mu m$ in Z).

Finally there is the TRT which is basically a two part detector.  It consists of ``straw tubes'' which are tubes filled with Xenon gas and a wire down the middle. Each tube is 4 mm in diameter and 37 cm long in the endcap region or 144 cm long in the barrel region.   Around these tubes are various materials with different dielectric constants.  When particles, especially very high energy, low-mass particles like electrons pass through these different materials, transition radiation is emitted.  These photons hit the Xenon-filled tubes and create ions which, because of a potential different between the tube and the wire in the center, drift towards the wire and cause a signal.  This is particularly useful in helping with electron identification, especially for $|\eta| < 2.0$.  The position resolution in this section isn't as good as the pixel or SCT detectors, but there are still about 300,000 straws over a large area, and particles will have more ``hits'' in the TRT straws than the previous detector sections, assisting with particle track reconstruction.  The TRT only provides R-$\phi$ information and can resolve to 130 $\mu m$ per straw.  However, each track has approximately 36 hits in this region, compared to 3 or 8 in the other two inner detector regions.

\subsection{The EM Calorimeter}
%kapton electrodes?
The electro-magnetic (EM) calorimeter is particularly intended to pick out the tracks and energy of electrons and photons, which tend to stop in this region.  It is composed of layers of lead with steel and liquid argon (LAr), starting with an initial LAr layer called the presampler which gathers information about showers that may have occurred in previous detector material.  Through the rest of the calorimeter, the electrons will interact with layers of lead.  There are three major layers, and most electrons of high enough energy for physics analyses like this one are deposited in the central layer.  This layer has 0.025x0.025 resolution in $\eta - \phi$ space.  The first layer helps with rejection of photons or pions and the last layer helps collect energy from very energetic electrons.  More energetic electrons will make showers in more of the lead layers.  The showers themselves are detected via creation of ions in the LAr.  Photons are also detected in this region and are distinguished from electrons by the lack of a track in the inner detector.  There are two levels of coverage in this detector. The central region, $|\eta| < 1.5$ contains slightly more layers and better resolution than the two-wheel endcap region, $1.4 < |\eta| < 3.2$.  The resolution is worst in the forward region of this detector, $2.5 < |\eta| < 3.2$, and this analysis will not consider electrons from this region.

It should be noted that there is one particular region of the detector between the barrel and endcap in the EM calorimeter, $1.37 < | \eta | < 1.52$, where there is excessive extra material between the inner detector and the EM calorimeter~\cite{Electron}.  This makes it difficult to properly reconstruct the energy of electrons that are detected, and of course they may deposit most of their energy in this region and never make it into the rest of the detector at all.  This is sometimes referred to as the "crack" region and electrons from this region are not considered in the analysis.

\subsection{The Hadronic and Forward Calorimeters}
The hadronic calorimeter is where the hadronic showers from hadronizing quarks tend to reach and eventually stop.  Here we complete the track and energy information for jets, the shower of particles from a hadronizing quark.  The portion of the jets that hit the calorimeter are actually composed of various light particles, commonly including particles such as pions and kaons (which have masses of about 140 MeV and 500 MeV, respectively). This part of the detector is special because it contains not only a central and barrel region, but also a forward region which is next to the beam pipe (as is the inner detector).  Each of the regions have some overlap with each other to avoid lining up too many detector transition regions with each other (where resolution extra material is present and resolution is not as good).  Extra material can cause extra interactions that may not be well modeled and particles could be missed, so it is important to minimize this.

The central region ($|\eta| < 1.7$) contains scintillating tiles and steel, and is known as the tile calorimeter.  The hadrons interact with the layers of steel and the showering particles create photons when they hit the scintillating tiles.  These photons are then collected by photomultipliers, which turn the photons into an electrical signal.  The barrel region ($1.5 < |\eta| < 3.2$) contains the hadronic end-cap calorimeters (HEC), which uses LAr and is essentially an extension of the EM calorimeter but with copper plates.  There are three layers in the barrel and four in the endcap, with a resolution of about 0.1x0.1 in $\eta - \phi$ space for $| \eta | < 2.5$ and 0.2x0.2 otherwise in the endcap region.

The forward region ($3.1 < |\eta| < 4.9$) has special forward calorimeters right next to the beam pipe and thus has a different configuration to handle the larger amounts of radiation.  Here, copper has tube-shaped holes formed in it with each hole containing a tungsten rod and LAr between the two.  The particles shower in the copper and the ions form in the LAr and travel towards the rod.  This region is especially important for the $t$-channel single-top searches as an energetic forward jet is a distinguishing characteristic between it and its backgrounds.

\subsection{The Muon Spectrometer}
%http://arxiv.org/abs/1101.3276
%Muon systems at the LHC are described in detail in Ref. [6]. The muon differs
%from the electron only by its mass, which is around a factor 200 larger. As a consequence,
%the critical energy Ec (the energy for which in a given material the rates
%of energy loss through ionization and bremsstrahlung are equal) is much larger for
%muons: it is around 400 GeV for muons on copper, while for electrons on copper2
%it is only around 20MeV. As a consequence, muons do in general not produce
%electromagnetic showers and can thus be identified easily by their presence in the
%outermost detectors, as all other charged particles are absorbed in the calorimeter system.
%
%try to convert some resolution planes to others somehow?
Finally there is the muon system, primarily intended to detect muons, which tend to travel farther through the detector than other particles (except neutrinos, which interact so weakly that it is difficult to detect them).  This is related to the mass of the muon, which is about 106 MeV (much larger than the electron, at 0.5 MeV), and its decay time, which is much longer than the particles like pions and kaons in jets.  The longer decay time allows it to reach the outer regions of the detector ($c\tau$ is 659 m) and its larger mass prevents it from showering too much earlier in the detector.  Thus, we can have a special detector for muons, in the outermost portion of the detector, to determine information about the direction and momenta of the muons.  

There are four major components of the muon system.  Two components are dedicated to detecting the muon track and the other two are dedicated to reporting the presence of a muon (triggering) and giving additional position information.  In each case, one component is in the barrel region and the other in the endcap region.  The Monitored Drift Tubes (MDT) are primarily responsible for track determination over the full $|\eta| < 2.7$ region, except for the inner section of the muon detector forward region ($2.0 < |\eta| < 2.7$), where Cathode Strip Chambers (CSC) are used.  The general principle is similar in both cases.  There is a gas filling drift tubes or between plates, and a charged particle creates ions which drift towards a wire.  The resolution of the MDT is about 35 $\mu$m in the Z direction, while the CSC has a resolution of about 40$\mu$m in the plane orthogonal to $\phi$ and 5 mm in the $phi$ (non-bending) direction.  The triggering portions are the Resistive Plate Chambers (RPC) and Thin Gap Chambers (TGC) in the central ($|\eta| < 1.05$) and endcap ($1.05 < |\eta| < 2.4$) regions, respectively.  The first is composed of sets of plates (no wires) that the ionized particles travel between.  The second contains many wires between plates, like the CSC, but the wires are arranged differently to favor a faster response time.  These extra triggering systems are needed because the response time of the main systems is too long to allow triggering of a high \pt~ muon associated with some events, and also to provide information about the muon track in an additional ($\phi$) direction.  The RPC has a resolution of 10 mm in both the Z and $\phi$ directions while the TGC has a resolution of 2 to 6 mm in Z and 3 to 7 mm in $\phi$.

\subsection{Magnets}
It should be mentioned that one of the primary methods of measuring the momenta of charged particles is by measuring the curvature of their tracks in a magnetic field.  Magentic fields also help to distinguish charged and neutral particles (whose tracks have no curvature due to a magnetic field), aiding in particle identification.  Magnetic fields are created by two different sets of magnets in the ATLAS detector.  The first set is a 2 Tesla solenoid magnet system located between the inner detector and the EM calorimeter which provides a magnetic field for the inner detector.  In addition to providing a strong field the magnent coil and related structure must not be too thick or dense, as the particles are intended to pass through this magnet layer relatively unimpeded.  The second set consists of large toroid magnets (about 0.5 to 1 T within the muon detector) surrounding the muon system, in both the barrel and endcap regions. The tendency of a particle to curve in a magnetic field indicates that it is charged, but the degree of curvature also gives information about the momentum of the particle.  This can be seen from the equating the Lorentz and Centripetal force equations, giving (assuming non-relativistic conditions for the moment):
\begin{equation} F = Bqv = mv^2 /r \rightarrow mv = p = qBr \rightarrow p \propto r\end{equation}
where B is the magnetic field, q is the particle charge, v is particle velocity, r is the radius of curvature and p is the momentum.  From this we can see that particles with more momentum have a larger radius of curvature, meaning that the tracks will curve less (be straighter) in the detector.  Particles that are not charged will not curve due to the magnetic field.  Thus, these magnets are essential for particle identification and measurements.

\subsection{The Trigger and Data Collection}
%talk about deadtime?  detector performance ~95%?
%How large are the ROIs?
%which triggers in data streams
Finally, there is the trigger and data collection system (DAQ)~\cite{Trigger2010}.  Although not strictly part of the ATLAS detector itself, per se, this system is essential to data analyses.  The LHC produces collisions at such a high rate that it is impossible to store all of the collected data for analysis.  Most of the data, however, are glancing or low energy collisions that are not the events we are looking for in studies of processes such as single-top.  It is possible to reject many of these events immediately, using hardware triggers.  There are then two other trigger levels which spend increasing amounts of time determining if an event is worth saving or not before the data are finally recorded for use in analyses.

The three different trigger levels are called level 1 (L1), level 2 (L2) and event filter (EF).  At each level, more information is considered to determine if an event should be kept or rejected.  This is important as computer storage space would rapidly run out if all events were kept.  Most events are ``common'' events involving low energy jets.  We want to be sure that we collect enough of the less common high energy events (like the single-top events we are looking for) so we reject many of these less interesting events.

The L1 trigger is hardware only and rejects events very quickly (less than 2 $\mu s$ per event) and in large number, with a maximum rate of 75 kHz~\cite{Trigger2010, Electron} although in practice the rate may be half this value.  The other two triggers are software based.  The L1 trigger essentially just looks for high transverse energy objects in the event, but the L2 trigger considers the regions of interest (RoIs) containing these objects and can consider full detector information in these regions.  The rate after the L2 decision is about 3 kHz and it takes about 50 ms per event to make a decision to keep or reject the event at this level.  Finally the EF is the last level which looks at the whole detector and uses standard analysis reconstruction software to find the event information and make a decision.  After this stage, the event is permanently stored and disseminated to analysers.  The overall event rate at this level is about 200 to 600 Hz, much smaller than before.  It takes longer to determine whether to keep events at this level, about 4 CPU seconds per event by design (as low as 0.4 CPU seconds per event during data taking), but this is still quite fast.

There are many different types of triggers.  In this analysis, we use single lepton triggers, corresponding to the single leptons expected in the t-channel single-top final state.  When the data are processed, a low threshold trigger is initially applied, and higher threshold triggers are applied later at the analysis level.  This application of the low threshold triggers divides the data into different analysis streams.  In this document we use the Muon and Egamma (electron) streams for the main analysis.  There is also another main physics stream, the JetTauEtMiss stream, which is used in this analysis for the multijet background estimate.

\subsection{Data Quality}\label{sec:dataquality}
In some cases the detector may have a component temporarily fail or go offline, and it may not be possible to reconstruct certain particles well.  In this case, events taken during these times are rejected due to data quality issues.  This rejection is done ``offline'', meaning it is performed after the initial low threshold triggers are applied, and removes data events from the analysis by applying a ``good runs list'' (GRL) selection as the first selection on the data sample.  This is because some analyses do not use the full detector, so even if some of the data for a muon analysis for instance are not collected correctly because of technical problems with the muon spectrometer, an analysis only using the inner detector information can still use the data.  On the other hand, an analysis such as this one, which uses nearly the full detector range, would not be able to use such data.

One exception to this GRL selection is the so-called ``LAr hole'' issue, which was a problem with the front end electronics for the LAr calorimeter that created a ``hole'' in the detector data collection.  This problem persisted for a few months before being fixed and was present for all but the first 165 $pb^-1$ of the data set used in this analysis, meaning about 85\% of the data has the potential to be affected.  In this case an additional event selection is applied to the data to account for this issue, removing only events where the particle reconstruction is affected by this hole (rather than removing all of the events, which would have been the standard GRL selection procedure for an analysis like this one).  In the end, only about 10\% of data events are actually removed from the analysis due to this issue.


\chapter{Particle Reconstruction}\label{chap:partreco}
In order to do any sort of analysis, we must be able to compare the data from the experiment to the theory.  The data collected by the ATLAS detector are not particularly easy to interpret at first glance.  The data begin as energy deposits and tracks while the theory consists of simulated hadronized quarks, leptons and neutrinos (this is referred to as truth level MC).  Truth level MC information is largely unused in this analysis.  Instead, the two sets, data and MC, are processed to reconstruct the event and, in the case of MC, include detector information like extra material or overlapping tracks affecting particle reconstruction.  Event reconstruction is applied to form particles and reconstruct the event, so at this level there are quantities such as \met~(missing transverse energy) rather than neutrinos and jets rather than quarks.  We will call this stage of processing (the final stage before analysis) the reconstruction or detector level.  At this point, the two sets, data and MC, should be equivalent (for instance, \met~distributions should be the same between the two if the data are perfectly modeled by the MC). In the following sections, we will give the definitions for a muon, electron, neutrino, or jet at the reconstruction level. We will also include criteria that require particles to be separated and well reconstructed.

\section{Electrons}
Electrons appear to be narrow curving cones of energy in the detector.  The narrow curving track, with a shorter trail of energy depositions through the detector than a muon, is its primary distinguishing feature.  There are electrons that can occur in the detector from sources other than being directly produced in the main collision however, including electrons inside of jets and electrons from photon interactions.  It is also possible to mis-identify narrow jets as electrons, or photons as electrons (such as from bremsstrahlung radiation, produced as the electron interacts in the detector material).  In this analysis, we apply several criteria when identifying if a certain energy deposit and associated tracks are really an electron from the primary collision.

In the ATLAS experiment, there are three different initial electron selections which can be used in different analyses.  These are referred to as loose, medium, and tight, where medium includes loose as well as extra medium requirements, and tight includes both medium and loose as well as extra tight requirements~\cite{Electron}.  The more selections that are applied, the more confidence one has that the particle identified as an electron is really an electron, although you will also remove some real electrons which happen to fail these requirements (making it less efficient).  For this analysis, we prefer to be sure that the particle is what we have identified it as (high purity), so we require the tight selection. Overall, this selection includes requirements to ensure that the energy deposits are narrow and where we expect them to be for an electron (the EM calorimeter), to reduce jets in particular being mis-identified as electrons, and that a track is well matched to this deposit and inner detector deposits, to reduce photon conversions being mis-identified as electrons. 

The requirements for tight electrons are given elsewhere~\cite{Electron}, but we repeat them here for completeness. The loose selection requires the electron $|\eta|<2.47$, low leakage of energy depositions into the hadronic calorimeter, and includes a requirement on energy deposits in the middle of the EM calorimeter, where most electron energy deposits would be expected to be.  The shower width is examined in this layer as well.  The medium selection has additional criteria related to the shower width using the first EM calorimeter layer and the deviation in the energies of the largest and second largest deposits in this layer.  There are requirements related to the track, that there is at least 1 hit in the pixel portion of the inner detector, at least 7 hits from both the pixel and SCT, and that the track's transverse impact parameter, $|d_0|$, is less than 5 mm.  The final medium requriement is related to track and cluster matching, requiring the distance in $|\eta|$ between the cluster in the initial EM layer and the determined track to be less than 0.01.  The final set of selections to make the electron tight include additional cluster and track matching requirements: that the distance in $|\phi|$ between the cluster in the middle EM layer and the determined track be less than 0.02, a requirement on the cluster energy divided by the track momentum, and tightens the $|\eta|$ distance requirement applied for medium electrons from 0.01 to 0.005.  The $|d_0|$ requirement is also tightened to be less than 1 mm.  The TRT portion of the inner detector is used, introducing requirements on the total number of TRT hits and considering the ratio of high threshold hits to total hits in the TRT.  Finally, there are requirements to reduce photon conversions.  The number of B-layer hits (the first pixel detector layer) must be at least one and electron candidates that are matched to reconstructed photon conversions are rejected.

We further require electrons to have a transverse momentum (\pt) of at least 20 GeV.  Electrons must also be isolated, meaning they are not near other particles.  The isolation requirement is specifically optimized for single-top analyses and requires $\rm etcone30/Et < 0.15$ and $\rm ptcone30/Et < 0.10$.  Etcone30 and ptcone30 refer to the amount of transverse energy deposited or track momentum in a cone around the electron track(s) with a $R$ of 0.3, where $\Delta R = \sqrt{\Delta \eta^2 + \Delta \phi^2} < 0.3$.  Electrons must also have $|\eta| < 2.47$, and exclude the region $1.37 < |\eta| < 1.52$ due to detector limitations.  Additionally, if electrons fall within the LAr hole, then they are not considered to be electrons.

\section{Muons}
Muons are primarily distinguished by their relatively long lifetimes and long, curved tracks which reach into the muon calorimeter section of the detector.  Muons are required to satisfy several strict quality requirements.  As with electrons, the muons have several categories for an initial identification definition.  In this case, the categories refer to different muon reconstruction algorithms.  The one used here is the ``combined muon''~\cite{Muon1, Muon2Wjet} algorithm, which considers both inner detector and muon spectrometer tracks, which are reconstructed separately.  A combined fit is performed on the tracks from the two detectors to form a final muon track.  If a combined track cannot be formed, the particle is not considered to be a muon.  Of the different algorithms, this is the one that has the highest purity.

%https://twiki.cern.ch/twiki/bin/view/AtlasProtected/BTaggingJetTagNtuple
%xpectbla[ntracks]:true if track crossed an active pixel module on b-layer 

%https://twiki.cern.ch/twiki/bin/view/AtlasPublic/InDetTrackingPerformanceApprovedPlots
%A hole is defined as a missing measurement when it is expected i.e. a track crosses an operational silicon modules and have hits both in silicon modules proceeding and following it. Since tracking runs with a requirement of no holes in the Pixel Detector, only SCT holes can be counted. 

There are a few track quality requirements used to define a muon which are related to inner detector information, including at least one B-layer hit, at least two pixel hits, and SCT and TRT hit and quality requirements.  These are a bit detailed, and are given here for completeness.  We require the flag expectBLayerHit to be false or the number of BLayer hits $> 0$, meaning there must be a hit in the B-layer unless the track passes through a dead area of the detector.  A muon must have the number of pixel hits plus the number of crossed dead pixel sensors $\ge 2$, the number of SCT hits plus the number of crossed dead SCT sensors $\ge 6$, and the number of pixel holes plus the number of SCT holes $\le 2$.  Holes are where a module did not respond as expected, even though modules elsewhere along the track did.  Finally, there is a complex requirement on the number of TRT hits divided by the number of outliers related to the quality of the track fit, where outliers are hits that deviate from the track.  We require, where $n$ is the number of TRT hits plus the number of TRT outliers, $n\ge 6$ and the number of TRT outliers divided by $n$ to be $<0.9$ for $|\eta|<1.9$.  Then we also require the number of TRT outliers divided by $n$ to be $<0.9$ if $n\ge 6$ for $|\eta|\ge 1.9$.  In this last case, if $n< 6$, the event will pass, unlike the first case.

The isolation requirement is the same as the electron isolation requirement, namely that $\rm etcone30/Et < 0.15$ and $ptcone30/Et < 0.10$.  The muons we select are specifically not allowed to overlap in position with jets, meaning any muon candidate within $\Delta R$ of 0.4 of a jet is not considered.  For this purpose, we consider all jets with \pt~above 20 GeV and include jets that overlap with electrons.  Additionally muons must have $\rm \pt > 20$ GeV and $|\eta| < 2.5$.

\section{Quarks}\label{sec:quarks}
%not sure I understand collinear safety when two jets overlap.  is it because they are so parallel there isn't a boundry split issue?  soft particles combined with hard particles before these are combined with each other to make some jet???  minimum jet distance so jets are well separated???
Perhaps the most complex reconstructed objects in the detector are jets.  Jets are hadronized quarks, showers of many particles that tend to be absorbed in the hadronic calorimeter.  Because they are basically sprays of particles, it is possible for them to overlap and be in odd shapes.  In order to work with these, we need to understand which energy deposits correspond to which jets.

The method used to form jets in this analysis is an algorithm called the anti-$k_{t}$ algorithm~\cite{AntiKt}.  There are two major jet algorithm types, cone and clustering algorithms, where anti-$k_{t}$ is a clustering algorithm that forms jets that happen to have very cone-like shapes.  A clustering algorithm is a bottom-up algorithm that combines individual tracks together to form a jet, while a cone algorithm is a top-down algorithm which forms a cone for the jet and considers deposits within that cone part of the jet.  For this algorithm, the area in $\eta - \phi$ space is $\pi R^{2}$.  This is the area containing energy deposits associated to one particular jet, assuming that there are no other high $p_{T}$ (hard) objects within a distance $2R$.  For this analysis we use a cone size $R$ of 0.4.  If there is another hard jet, the harder jet will have a cone shape and the lower $p_{T}$ (soft) jet will have a crescent shape.  The anti-$k_{t}$ algorithm is used because not only is it reasonably fast, but the jets are grouped using the highest $p_{T}$ energy deposits first and then looking at surrounding objects, meaning that random low energy deposits will not change the jet shape.  This means it is infrared safe because it avoids potential divergences from an infinite number of very soft, low energy jets.  Also, each deposit and track is assigned to some jet.  There is no splitting and merging of overlapping jets, so it doesn't matter if a jet is split into two parallel (collinear) particles, making it collinear safe.

The jet energies are corrected from what is actually measured in the detector to what we have in the MC simulation.  The first correction is to the ``EM scale'', the main correction to the detected energy, based on test beam and cosmic ray studies.  The second correction is a jet energy scale (JES) correction~\cite{JESnew,JES}, an additional calibration based on the jet \pt~and $\eta$. It includes corrections for losses due to dead material or leakage energy from particles depositing energy outside the hadronic calorimeter.  The jets are thus called ``EM-JES'' calibrated jets~\cite{JER}.  Specifically, we use jets called AntiKt4TopoEMJets, where Topo refers to topological clusters~\cite{JESnew, Topo}.  This is an algorithm that clusters energy deposits together for the jet, by starting with an energetic deposit with a signal to noise ratio greater than 4, and adding neighboring deposits that have a signal to noise ratio greater than 2.

We have additional quality requirements.  We remove any jets that have been reconstructed with corrected energy that is negative and thus are not physical (this is a very small effect).  Any jet candidates that overlap with electrons with $\Delta R < 0.2$ are not considered.  Further, we require jets to have $\pt > 25$ GeV and $|\eta| < 4.5$.  Notice that this is a much more forward requirement than that of the leptons.  The calorimeters allow information this far forward in the detector and it is particularly important information for our analysis.

\subsection{b-tagging}
 \label{sec:Btag}
There are two subsets of jets that are used frequently in this analysis, tagged and untagged.  Jets that are b-tagged (tagged) are required to have a high probability of being a hadronized b-quark.  The rest remaining jets are referred to as untagged.  Additionally, jets must have $|\eta| < 2.5$ to be b-tagged because of the inner detector range.  This means all jets with $|\eta| > 2.5$ are considered untagged.

\subsubsection{How b-quarks are Identified}\label{sec:Bdist}
%compare distance to c or u quarks?
The bottom quark is an unusual particle.  It is heavy, as mentioned above, and it also travels quite a long distance, relatively, from its creation before it forms a jet (b-jet).  The lifetime of the b-quark is about $\tau = 1 \times 10^{-12}$ seconds.  We can determine the distance it should travel, on average, by assuming an energy of about 40 GeV.  Because $E=\gamma mc^{2}$ and $d = \gamma c \tau$, where $\tau$ is the lifetime, m is the mass (about 4 GeV/c$^2$) and E is the energy, we can write, where c is the speed of light $c=3\times 10^3~m/s$):
\begin{equation} d = \frac{E}{mc^{2}} c \tau \approx \frac{40}{4}\cdot 3 \times 10^{8}m/s\cdot 1 \times 10^{-12}s = 0.003~m \end{equation}
This means that the b's travel about 3 mm from the main interaction point before forming a jet.

Tracks from the inner detector can be reconstructed and traced back inside the beam pipe (where there is, of course, no detector).  These tracks will then intersect within the beam pipe, which has a diameter of about 6 cm.  Most intersect in a primary vertex, the place the proton-proton collision occurred.  However, some may intersect in other places, secondary vertices, where a b hadron has formed a jet (see Figure~\ref{fig:Display_bjet}).  Of course, as there are multiple proton-proton collisions producing pileup events, and just more particles in general, it can be difficult to really distinguish which tracks go where, and to which vertex.  This is why the inner detector resolution is so very important and also the reconstruction algorithms to determine these vertices.  Because of the importance of the inner detector, b-tagged jets are only defined within its range, $|\eta|<2.5$.

\begin{figure}[!h!tpb]
 \centering
 \includegraphics[width=1.00\textwidth]{figures/detector/bjet.eps}
\vspace{-0.5cm}
 \caption{Event display for a b-jet, with the secondary vertex shown in the dashed box and primary vertex shown as a round ball~\cite{EventDisplayBTag}, ATLAS Experiment \copyright 2011 CERN}
 \label{fig:Display_bjet}
 \end{figure}

The chosen b-tagger is just a distribution related to the likelihood of a jet coming from a b, and a jet is b-tagged based on whether the b-tagger value for that jet is above or below a certain threshold, called an operating point.  A jet is considered to be mis-tagged if the jet was not really from a b-hadron but was still b-tagged.  Different operating points have different levels of performance.  There are two major ways to determine performance, the b-tagging efficiency and the mis-tagging efficiency.  These two measures are inversely proportional, so a high b-tagging efficiency sample will have low mis-tagging efficiency.  This means that if the b-tagging efficiency is high, most of the jets that are really b-hadrons will be b-tagged, but there will also be a relatively high proportion of jets that were not really b-hadrons that were nevertheless b-tagged as well.

In our case, our final state has both a b-jet and a (typically) light quark jet.  Our large backgrounds before b-tagging are backgrounds with many light jets in them.  Therefore, while we remove some of our signal by having a lower b-tagging efficiency, we prefer to remove proportionally more of our background by choosing an operating point with a high mis-tagging efficiency.  Even though fewer jets are b-tagged, we have more confidence that the ones we do b-tag are really b-hadrons than if we had chosen a higher b-tagging efficiency.

The b-tagger used in this analysis is the JetFitterCombNN b-tagger~\cite{BTag:CONF}.  This is a combination or two b-taggers called JetFitter and IP3D.  The JetFitter algorithm uses a Kalman filter~\cite{Kalman} to determine the path along which b and c hadrons (from decays inside the b-jet) and the primary vertex lie, and this determines a track for the b-jet.  Additional discrimination based on the secondary vertex and its uncertainty is done using a likelihood method.  The IP3D b-tagger uses the impact parameter information in all three dimensions with a likelihood technique to discriminate between the b-jets and lighter jets.  An impact parameter is the shortest distance from the primary vertex (interaction point) to a track.  The transverse impact parameter is a common quantity known as $d_{0}$, and IP3D uses both transverse and longitudinal ($z_{0}$) impact parameter information.  The JetFitterCombNN actually uses both JetFitter and IP3D and forms a neural network based using information from these two algorithms.  The output of the neural network forms the JetFitterCombNN b-tagger.  For more information, please see~\cite{BTag:CONF}.


\subsubsection{Impact of Different Operation Points on the Analysis}
For this analysis, we choose the JetFitterCombNN b-tagger with a threshold of 2.4, so if the JetFitterCombNN value is $> 2.4$ the jet is b-tagged.  This gives a 57\% b-tagging efficiency, but a very high light quark rejection of about 1000~\cite{BTag:CONF} (or a mis-tagging rate of about 0.1\%).  This is the lowest b-tagging efficiency (and highest mis-tagging efficiency) operating point approved for use.  In Figure~\ref{fig:OPThreshold_bjet}, the effect on the yields of using different b-tagging operating points can be seen, where the chosen operating point is shown with a vertical line.  The yields for each process are given for a particular threshold for the JetFitterCombNN b-tagging variable, as well as scaled versions of the signal divided by the background or the square root of the background.  Both of these are rough indications of signal separation and analysis performance (but note that they don't include systematic uncertainties).  In general it is clear that while the $t$-channel yields go up for a looser operating point, the backgrounds also increase, at a greater rate.  The separation appears to be better for higher thresholds.  Although we lose some of our overall event yield, the background is reduced at a greater rate than the signal is reduced.  For this analysis, we use the highest threshold available, 2.4.

%discuss b-tagging scale factors here?  data based determination, aka pt rel?
\begin{figure}[!h!tpb]
 \centering
 \includegraphics[width=1.0\textwidth]{figures/btag/2j1b_btaglog_nostack.eps}
 \includegraphics[width=1.0\textwidth]{figures/btag/3j1b_btaglog_nostack.eps}
\vspace{-0.5cm}
 \caption{Distribution of the yields for the signal (t-channel) and its backgrounds, given a selection on the JetFitterCombNN b-tagging variable at the given x-axis value.  Selections at higher x-axis values have lower b-tagging efficiencies but higher mis-tagging efficiencies.  The black vertical line shows the threshold used in the analysis.}
 \label{fig:OPThreshold_bjet}
 \end{figure}

\section{Taus}
Thus far, tau leptons have not really been discussed.  This is because we do not specifically reconstruct or select for taus in this analysis, although of course this is a lepton that could be involved in the W decay from the top quark.  The reason for this is the short tau decay time and the nature of its decay particles.  Unlike the other two leptons, electrons and muons, taus do not travel very far into the detector before decaying (it has a lifetime of about $3\times 10^{-13} s$ and a mass of about 1.8 GeV~\cite{PDGSummary}).  Taus may decay into quarks, at which point they look like a jet (it is theoretically possible to reconstruct a tau like a b-jet, but that is not currently done).  It is also possible for the tau to decay to the one of the other two lepton types, plus neutrinos (this happens about 40\% of the time).  In this case, we incidentally select for these when we select for a muon or electron in our event.  However, we don't specifically select or reconstruct a tau.  The MC and data both have taus in them and the same selection (and lack of special reconstruction) is applied in both cases, so this is consistent.

\section{Neutrinos and Missing Energy}\label{sec:Neutrinos} 
%mention lepton z from E and m?
Although there is a neutrino in the t-channel single-top final state, it has not been heavily discussed.  This is because neutrinos interact weakly (and not very often).  Thus, we do not try to detect the neutrinos but instead use energy conservation to determine the missing transverse energy, or \met, which corresponds to the neutrino's \pt~(or sum of the neutrino \pt~values).  If there is more than one neutrino, of course, there is still just one \met~value.  Information about the momentum in the z direction is not preserved as there is no information collected inside of the beam pipe.  We can't determine how much energy is missing in that direction from particles that may have traveled only in the z direction and missed the detector.  We do have information everywhere else though, so we are able to determine the missing energy in the x and y directions.

The \met for ATLAS~\cite{MET} is calculated from the sum of the calorimeter energy deposits (includes jets, electrons, photons, and $|tau$ decays as well as other deposits) and the energy from reconstructed muons in the x and y directions.  The portion of the calorimeter energy deposits in the ``other'' catagory are called cell out deposits and jets with \pt below 20 GeV but above 7 GeV (too low to be considered jets in this analysis) have energy deposits classified as soft jet deposits, rather than standard jet deposits.  Cell out deposits are often low energy deposits too low to be classified as soft jets or other physics objects.  When all of these contributions are summed, because the inital collision is in the z direction with no energy in the x or y directions, we expect the sum to be 0.  The deviation from 0 gives the \met.

For this analysis, we will want to use the momentum information in the longitudinal or z direcation, so it is reconstructed from information about the W boson.  The single lepton and \met~(which we expect comes from one neutrino in our selected events) correspond to a single W boson.  The mass of the W is well known, and we take it to be 80.42 GeV, consistent with the current Particle Data Group value~\cite{PDGSummary}.  From this information, we can reconstruct the neutrino momentum in the z direction, using the following equation and taking the neutrino and lepton masses to be zero:
\begin{equation} P^{\mu}P_{\mu} = M^{2}_{W} \end{equation}
Here, P is the four-momentum of the lepton and neutrino combination, and $M_{W}$ is the W mass.

After some manipulation, the equation can be rewritten as:
\begin{equation} l_e^2\nu_{pT}^2 - (A+B)^2 + (l_e^2-l_{pz}^2)\nu^2_{pz} + 2(A+B)l_{pz}\nu_{pz} = 0\end{equation}
where $A= \frac{M^{2}_{W}}{2}$ and $B=l_{px}\nu_{px} + l_{py}\nu_{py}$.  Here, $\nu$ is the neutrino and $l$ is the lepton where $l_e$ is the lepton energy, $l_{pz}$ is the lepton longitudinal momentum and $l_{pT}^2 = l_{px}^2 + l_{py}^2$ is the lepton transverse momentum.  The quantity $\nu_{pT}$ is just \met.  This is a quadratic equation which can be solved in the usual way for $\nu_{pz}$.  In the solution, it is possible to have two results, and in this case we take the smaller of the two values.  It is also possible to have a negative discriminant ($1+(\nu_{pT}^2 l_{pz}^2) < \nu_{pT}^2 l_e^2$) and in that case a 0 value is taken.  The \met~is primarily used in the analysis until the final cut-based selections are applied, but the reconstructed neutrino $p_z$ is used for quantities such as the reconstructed top mass.
%1.0E0 + pow(nu_pt,2) * (pow(l_pz,2)-pow(l_e,2))

\chapter{Monte Carlo Simulation and Corrections}\label{chap:MC}
The Monte Carlo (MC) which simulates events produced by the proton-proton collisions does not simply appear from equations produced by high energy physics theorists, but instead must be produced and corrected using computer algorithms.  There are two steps to this process, a generator that produces the general event with quarks, leptons and neutrinos, and then a showering algorithm that adds extra jets and gluons to these ``simple'' processes made by the generator so they resemble more closely what we see in the detector, where higher-ordered processes are produced.  Additionally, after considering the effects of the detector and producing what is called reconstruction level MC from the truth level MC (which does not include detector effects), there are still corrections to be made to the particle energies and the MC events in general to better match the data.  In this chapter we discuss the MC production and these additional MC corrections.

\section{MC Generator and Showering}
%discuss matching scheme between shower/generator, scales, pt cut offs?  single-top special??
%MC settings: https://twiki.cern.ch/twiki/bin/view/AtlasProtected/McProductionCommonParameters
%add details about versions?
%AcerMC for single-top http://indico.cern.ch/getFile.py/access?contribId=20&sessionId=1&resId=0&materialId=slides&confId=3757

The simulation of MC requires a few processing steps.  The first is a MC generator which reproduces the basic Feynman diagrams and the second is a showerer, which adds on additional particles.  In practice it isn't quite so clear-cut.  There can be some overlap between what the showerer and the generator do, and diagram overlap removal procedures are applied in this case.  There may also be some uncovered diagrams depending on the process and choice of showering algorithm or generator.

For this analysis we use {\sc Pythia}~\cite{SAMPLES-PYTHIA} for the parton showering for the single-top processes and {\sc Herwig} ~\cite{SAMPLES-HERWIG} with {\sc Jimmy}~\cite{SAMPLES-JIMMY} for the showering for all other processes.  The leading order parton density functions are from {\sc CTEQ6L}~\cite{cteq6}.  The generators are more varied.  For our signal we use {\sc AcerMC}~\cite{SAMPLES-ACER}, which in our studies uses a procedure to reproduce an extra, soft b-quark from a incoming gluon more correctly than alternative generators~\cite{SAMPLES-ACER:tchan}.  The \ttbar~process is generated using {\sc MC@NLO}~\cite{SAMPLES-MCNLO} generator, which includes more diagrams than a standard leading-order generator.  The W+jets and Z+jets processes use {\sc Alpgen}~\cite{SAMPLES-ALPGEN}, as do diboson simulations.

Generators and showering algorithms are updated frequently as they are tuned to better match the data.  There is some uncertainty related to our measurement because we know these probably don't precisely match our data.  However, the extensive studies with Tevatron data have produced generators and showering algorithms that match up well with LHC data, and the agreement will be discussed in Chapter~\ref{chap:variables}.  Additionally, the MC used in this analysis has the ATLAS tag of ``MC10b''.  It has three simulated bunch trains with 225 ns separation between the trains.  Each train has 36 filled bunches with 50 ns separation between bunches, the same separation as the data.  

Finally, it should be noted that the final simulation of the particles going through the material of the ATLAS detector is done with separate programs~\cite{ATLASSIM} based on {\sc GEANT4} software~\cite{GEANT}.  This stage of processing accounts for the specific configuration of the ATLAS detector, including regions where there may be more or less material.  This stage also introduces simulation of the resolution in different portions of the detector.  The events are then reconstructed in the same was as data collected by ATLAS.

\section{Monte Carlo Weighting and Corrections}
When the Monte Carlo (MC) is produced, no particular attention is paid to the number of events generated, other than to be sure there are enough events to allow a sufficient variety of kinematics (and make the statistical uncertainty low).  In order to compare the MC to the data sample, the MC must be weighted so that the proportions of the different processes are as we expect and the overall normalization is correct.  Here we describe how the event weighting is done and what weights and corrections are applied.

%MCatNLO weight
%b-tagging weight
%lepton SF
%discuss lepton energy smearing somewhere  scaling vs smearing
%pileup weight and redo of MC Event count
%W+jets weight- short mention, reference to next section?  Or just discuss in next section

\subsection{Theoretical cross-section and luminosity weight}
The first and perhaps most important weight is the factor that normalizes each process to its theoretical cross-section and also the whole MC sample to the number of events expected for the amount of data we have (the integrated luminosity).  The factor multiplied onto each MC event is formed as:
\begin{equation} \frac{XS \cdot BR \cdot K \cdot L}{N_{MC}} \end{equation}
where $XS \cdot BR \cdot K$ is the cross-section times branching ratio times k-factor discussed in Section~\ref{sec:ExpectedCrosssections}, $L$ is the integrated luminosity (1035.27 $pb^{-1}$) and $N_{MC}$ is the number of Monte Carlo events.  The numerator is simply the number of expected events from Equation~\ref{equ:numberofevents}, with the cross-section written out with corrections.  The values for $XS \cdot BR \cdot K$ are given in Table~\ref{TABLE-MCSAMPLES}.

The denominator is the number of MC events, as stated.  However, it is important that this include any weights that can affect the overall number of event before analysis selections.  These include the pileup weight, discussed next, and negative weights associated with MC generators.  Certain MC generators, particularly NLO top-quark process generators like {\sc MC@NLO} (\ttbar) or {\sc AcerMC} (single-top production) give some events negative weights during generation related to interference effects when including NLO processes.  This weight is applied when determining the $N_{MC}$ value, as well as later in the analysis as another event weight.  

\subsection{Pile-up weight}
The other weight applied to the $N_{MC}$ value (and all MC events) is the pileup weight, which is related to the number of primary vertices.  This is a special weight to adjust the MC to represent the events one expects under certain (data-like) pileup conditions.  The pileup conditions may change after a large sample of MC is generated, so this allows more flexibility.  The difference between the $N_{MC}$ in the full MC sample with and without this pileup weight is typically about 1\%, so the effect on $N_{MC}$ is minimal.

The weight itself ranges in value from about 0 to 5, where many events are given weight 0 because they are simulated with pileup conditions exceeding the current data conditions.  This of course has an effect on the MC statistical uncertainty, as the MC statistics are effectively reduced in this case.

\subsection{Lepton scale factor}
The lepton scale factor is a factor used to adjust the MC so that the lepton efficiencies match those found in the data.  Scale factors are discrete numbers applied to the MC which may have some dependency on \pt~or $\eta$.  There are different scale factors related to the trigger, reconstruction and identification for each lepton type.  These scale factors are all approximately 1~\cite{Electron, Muon1, Muon2} and have a minimal impact on the analysis.

\subsection{Mis-tagging and b-tagging scale factor}
Like the leptons, the b-tagging and mis-tagging efficiencies we see in data are not exactly the same as MC, so we apply a scale factor to correct the MC.  This scale factor is also typically close to 1~\cite{BTag:CONF, BTag:CONF2}.  However, the uncertainties on the b-tagging scale factor are larger than the others for this analysis, so the b-tagging scale factor has an increased level of importance.  It would be possible to eliminate this scale factor and uncertainty if b-tagging were not used in the analysis, but the signal separation is not sufficient to do this with the current data total.  For more details on b-tagging and mis-tagging efficiencies in this analysis, see Section~\ref{sec:Btag}.

\subsection{Energy corrections in the analysis}\label{sec:energyresolution} 
The energies of different particles are not necessarily quite the same in MC as they are in data.  Although some corrections are applied to the files before they reach analysers, there is some fine-tuning done at the analysis level.  We can smear or scale the particle energy, where smearing involves changing the particle energy using some distribution, like a Gaussian.  This may be done as an uncertainty on the analysis, as is the case with jets, or it may be applied to the nominal sample, as is the case for leptons.  

In the case of the leptons, the corrections are chosen to have a better match between the Z mass peak and width in the data and MC.  The electrons have two instances of energy corrections~\cite{Electron}.  The first is a scaling done in the data to correct the energy of the electrons.  This sort of correction is usually done before the analysis level, as for jets, but in this case it is done afterwards.  The second is a smearing, done in MC, to adjust the width of the Z peak to what we see in data.  For the muons, scaling and smearing are both applied to the MC only.  In this case, because tracks are used from the inner detector and muon spectrometer to form the full muon track, there are separate corrections on the tracks in each region~\cite{MuonER}.

For the jets, the jet energy scale (JES) correction was applied when the jets are formed and was discussed in Section~\ref{sec:quarks}.  This is a factor that adjusts the jet energy to account for issues such as dead material or energy leakage.  The jet energy resolution (JER) corrections are done in a separate iteration of the analysis and the difference from the nominal sample is taken as an uncertainty~\cite{JER, JER2}.  The JER adjustment affects the jet energy by adjusting the value according to a Gaussian distribution, where the Gaussian width depends on the jet \pt and $\eta$ value.  There are two major techniques to determine the jet energy resolution, a di-jet balance method and a bi-sector method.  The di-jet balance method uses a two jet event where the jets are both expected to have a similar \pt value (back to back jets).  The deviation between the jet \pt values divided by the sum for each jet can be plotted and fit with a Gaussian distribution.  The bi-sector method is slightly different.  It is based on the sum of the \pt in a 2 jet, back to back event.  This \pt vector is projected into the transverse plane (transverse to the beam axis) such that the $\eta$ coordinate in this plane bisects the difference in the $\phi$ angles of the two jets.  The deviations from nominal (where the sum of the jet \pt values is 0) for each tranverse plane angle is considered separately and can be found by fitting gaussian distributions to the \pt values for each angular component in different \pt ranges.  The function used to adjust the jet energy in this analysis is based on a combination of the results of these two techniques.




\chapter{Preselection}
 \label{sec:EventSelection}
In Section~\ref{sec:ExpectedCrosssections} the cross-sections for the signal and background processes were given.  It is clear that without selections to reduce the background events relative to the signal events, single-top $t$-channel cannot be distinguished.  The haystack is large, and our needles are buried and hard to see.  The preselection selects events that have single-top t-channel-like kinematic characteristics.

To see how these selections are chosen, we examine the t-channel final state, shown in detail in Figure~\ref{fig:Feynman_tchan}.  The figure shows that there is at least one b-tagged jet and at least two jets (the b-quark from the intial gluon is not always present or detected), with one lepton and one neutrino (\met) from the decay of the W.  The scenario where a W decays to quarks is not selected for, as the final state is then all jets and is very difficult to distinguish from the large cross-section multijet background.  The multijets cross-section is so large that even requiring one lepton, despite the low lepton fake rate in this analysis, still results in a fairly large number of multijet events selected.

\begin{figure}[!h!tpb]
 \centering
 \includegraphics[width=0.45\textwidth]{figures/theory/t_channelbigNLO.eps}
\vspace{-0.5cm}
 \caption{Feynman diagram for $t$-channel single-top production, showing the final state after the top quark decay.}
 \label{fig:Feynman_tchan}
 \end{figure}

Requiring a t-channel final state incidentally helps to reduce the backgrounds, in addition to choosing events that look like our signal.  Rejecting events with less than 2 or more than 3 jets helps to reduce W+jets and \ttbar, respectively, as these processes tend to have fewer or more jets than the signal.  

The preselection used in this analysis in detail is given below.  The selection without the b-tagged jet number requirement is called the pretag selection:

\begin{list} {$\bullet$} {}
\item The event must have a quality primary vertex (has at least 5 tracks)
\item Exactly one, triggered lepton (muon or electron), matched to a reconstructed lepton object
\item The leptons must have $\pt > 25~GeV$ and muons must also have $\pt < 150~GeV$ 
\item There must be $\met > 25~GeV$
\item Two or three jets, with a one jet selection used for a sideband region
\item The jets must not be ``bad''
\item LAr quality requirements related to the LAr hole must be met
\item Data events with LAr bursts (noise) are removed
\item Triangular cut of $\met + W_{T} > 60~GeV$
\item At least one of the jets must be b-tagged
\end{list}

The primary vertex requirement helps to reduce contamination from secondary vertices or events where an extra pileup interaction (often multijet) vertex might be confused for the one we are interested in.  The lepton requirement helps to reduce multijet events, which do not have a real lepton.  The trigger requirement specifically requires the EF\_mu18 trigger for the one muon selection and EF\_e20\_medium trigger for the one electron selection.  The trigger matching ensures that the lepton in the analysis matches with a trigger-level object.  This selection has a small effect on the analysis.   Due to an issue with the MC, the muon trigger matching was not applied for the muon channel (although the trigger itself was still applied).  The \pt~requirement for the leptons is 25 GeV partly to be sufficiently away from the trigger thresholds of 18 and 20 GeV, to reduce the related uncertainty.  The upper \pt~threshold is applied due to low statistics when determining the muon scale factors in this region (the impact on the analysis from this selection is very small).  The \met~selection has a threshold similar to the lepton to reduce multijet events faking a small amount of \met.  The lower \pt~thresholds for the particles help to reduce the multijet and W+jets backgrounds, which often have lower \pt~particles.

The analysis requires 2 or 3 jets.  This chooses a final state that looks like the signal, but also helps to reduce W+jets and \ttbar, respectively, as these processes tend to have fewer or more jets than the signal.   The events must satisfy several selections to remove events which contain bad jets.  These are jets that arise due to cosmic rays, detector problems, or beam issues and the whole event is removed if it includes a bad jet.  For completeness, the cuts are as follows~\cite{BadJet, BadJet2}.  There is a bad jet if the energy fraction in the HEC is $> 0.5$ and the fraction of energy corresponding to hadronic end-cap calorimeter (HEC) cells with a cell Q-factor (related to the energy pulse shape measured versus expected) greater than 4000 is $> 0.5$ (corresponds to HEC spikes; hardware issues).  The event is rejected if the jet's energy fraction in the electromagnetic calorimeter is $> 0.95$, the fraction of energy corresponding to LAr cells with a cell Q-factor greater than 4000 is $> 0.8$, and $|\eta| < 2.8$ for the jet (EM calorimeter noise issues).  Finally, the event is rejected if the jet timing is $> 25~ns$ (indicates out-of-time jets, from a cosmic ray for instance).  The timing is the deviation of the event time from the time of energy deposition for the detector cells related to the jet, weighted by their energy squared.

Of the remaining four selections, there are two selections related to the LAr.  The first removes events where jet reconstruction is affected by the LAr hole.  The second removes events with noise bursts related to the LAr.  Finally, we apply a triangular cut which reduces the multijet background and, last, we require at least one jet to be b-tagged.


\chapter{Modeling the Signal and the Backgrounds}\label{chap:bkgd}
In order to study single-top $t$-channel production, we need to know what the standard model version of this should look like in our data.  This is actually quite a complicated task, to simulate all of the standard model processes and also how these different particles interact with the detector.  The Monte Carlo (MC) techniques that were used in this analysis to simulate ``data'' were discussed in Chapter~\ref{chap:MC}.  Here, we discuss the data-based estimates we use for the multijets normalization and kinematic shapes, as well as how the W+jets normalization and heavy flavor fractions are obtained from data.

\section{Multijet Estimation}\label{sec:multijets} 
%from Kathrin:
%for QCD we use a JetTauEtmiss stream
%as we are looking for jets and use jet triggers
Multijets (sometimes colloquially referred to as QCD) are difficult to simulate in quantities necessary to be useful in analyses.  This is a process with a very large cross-section and a very, very small proportion of events left after cuts, relative to the starting yield.  It just isn't feasible to generate MC for this background.

We do, however, have a lot of multijets in our rejected data sample, in our off-signal region.  We can make use of this to select a relatively pure sample of multijet events which is used for both kinematic shapes and to determine the normalization (i.e. how many multijet events are actually in the preselection sample).  There are several ways to form such a region.  For instance, one could require the leptons to not be tight or not be isolated, keep the other selections and cuts the same, and end up with an orthogonal multijets sample.  This particular method, however, suffers from too much contamination from W+jets events.

The method chosen for this analysis is the ``jet-electron'' method.  In this method, the usual electron trigger is replaced by a jet trigger.  Correspondingly the data stream is replaced by the JetTauEtmiss stream (the main analysis uses muon and electron streams).  The triggered jet must also have a high EM fraction, so most of the energy is deposited in the EM calorimeter, and at least 4 tracks, to avoid including photon conversions.  All other selections and cuts are unchanged.  This sample is used to determine the kinematic shapes.  Because of the low statistics due to increasing trigger thresholds as the data taking has progressed, the shapes before the b-tagging selection are used for distributions after b-tagging as well.  Checks have been performed which show that the shapes are indeed similar.

The overall normalization is found by fitting to a kinematic distribution.  The \met~distribution is usually used, although the transverse W mass has been used as a cross-check and to help determine the uncertainty on our multijets estimate, which is 50\%.  The yields are given in Table~\ref{tab:QCD}.

\begin{table}[!h]
  \begin{center}
    \begin{tabular}{lcccc}
      \hline
      \hline
      & \multicolumn{2}{c}{Pretag events} & \multicolumn{2}{c}{Tagged events} \\
      Jet bin & e channel & $\mu$ channel & e channel & $\mu$ channel \\
      \hline
      1-jet & $ 24000 \pm 12000$ & $ 12000 \pm 6000 $   & $ 320 \pm 160  $ & $ 290 \pm 145$ \\
      2-jet & $ 15000  \pm 7500$ & $ 6800 \pm 3400 $   & $ 710 \pm 355  $ & $ 440 \pm 220$ \\
      3-jet & $ 6000  \pm 3000$ & $ 1700 \pm 850 $    & $ 580 \pm 290  $ & $ 270  \pm 135 $ \\
      \hline
      \hline
    \end{tabular}
    \caption{\label{tab:QCD} Estimate of multijet yields for the pretag and preselection samples for different number of jet selections, separated by lepton type.}
  \end{center}
\end{table}

\section{W+jets Estimation}\label{sec:wjets} 
%CITE W+jets flavour not well understood..?
The W+jets process is a large background for this analysis after preselection and is not especially well understood in the MC, particularly the heavy flavor fractions.  For this reason, we use the data to determine the overall W+jets normalization as well as the normalization of the separate flavor W+jet productions. These are W+light jets, W+cjets, W+c$\bar{c}$jets and W+b$\bar{b}$jets.  The last two are combined together for the purposes of this normalization.  

The method used here is called the ``cut and count'' method.  This method was first developed during the ATLAS \ttbar~redisovery~\cite{ttbarPaper}, although not used due to low statistics, and has been used in each data-based single-top note~\cite{ATLAS-CONF-2011-027,ATLAS-CONF-2011-088,ATLAS-CONF-2011-101}.  The general idea is to form a series of equations involving off-signal regions which can then be solved for the scale factors of interest (to scale each W+jets MC sample).  The scaling is done based on the number of jets and the W+jets MC flavor, which is based on what type of quark the W+jets event is associated with, light (lq), c, $c\bar{c}$ or  $b\bar{b}$.

First, the overall W+jets normalization is determined as a function of the number of jets in the event using the sample before b-tagging is applied, the pretag sample.  The scale factor is determined as:

\begin{equation} \frac{N_{W+jets, data}}{N_{W+jets, MC}} = \frac{N_{data} - N_{multijets} - N_{Non-W+jets, MC}}{N_{W+jets, MC}} \end{equation}

The overall normalization scale factors are given in Table~\ref{wjetoverallnorm}.

\begin{table}[htdp]
\begin{center}
\begin{tabular}{ccc}
\hline
W+1jet & W+2jets & W+3jets \\
\hline
$0.966\pm0.001$ & $0.914\pm0.002$ & $0.879\pm0.004$ \\
\hline
\end{tabular}
\caption{Scale factors for the overall normalization factor used to normalize MC to data for W+jets.  The uncertainties are statistical only.}
\label{wjetoverallnorm}
\end{center}
\end{table}

The normalization of the individual flavor scale factors involves additional equations.  We use three different off-signal regions for this: 2 jet pretag, 1 jet 1 b-tag, and 2 jets 1 b-tag.  This last region contains part of the final analysis signal region, so that this portion is subtracted off before doing the estimate of the W+jets scale factor.  It is also possible to use the 1 jet pretag bin and include a different combination of regions, so that this equation will also be shown, although it is not used.

We will solve a series of equations for the flavor fractions, $F_{b\bar{b}2},~F_{c2},$~and~$F_{lq2}$, where $F_{c\bar{c}2}$ and $F_{b\bar{b}2}$ are assumed to be the same (one scale factor for both processes).  These can then be converted into different bins using MC assumptions.  These flavor fractions we solve for are all from the 2 jet pretag selections and will be propagated later on into other regions.  $F_{c2}$ for example is, where W+jets refers to all W+jets flavors combined:
\begin{equation} F_{c2} = \frac{N_{c2}^{pretag}}{N_{W+jets 2}^{pretag}} \end{equation}
Here, N is the number of events, 2 indicates two jets, and the letters refer to the different flavors (lq is light quarks).  W+jets refers to all W+jets MC events.

The set of four equations are written as follows, where in this analysis we make use of the last three.  The equations state that the total data minus background (i.e. non-W+jets) is the same as the sum of the MC W+jet samples separated by flavor.  In these equations, the superscripts p and t mean pretag and b-tagged samples.  All other quantities use MC pretag values except the b-tagging probabilities P, which use both MC tag and pretag information, and N's are from the data minus the non W+jets MC yields and multijets estimate.  Specific definitions of quantities follow these equations:

\begin{eqnarray*}
N^{p}_1 &=& N^{p}_1 \cdot (k_{b\bar{b}2to1} \cdot F_{b\bar{b}2} +  k_{c\bar{c}tob\bar{b}} \cdot k_{b\bar{b}2to1} \cdot F_{b\bar{b}2}  + k_{c2to1} \cdot F_{c2} + k_{lq2to1} \cdot F_{lq2})\\
N^{p}_2 &=& N^{p}_2 \cdot (F_{b\bar{b}2} +  k_{c\bar{c}tob\bar{b}} \cdot F_{b\bar{b}2}  + F_{c2}  + F_{lq2}) \\
N^{t}_1 &=& N^{p}_1 \cdot (P_{b\bar{b}1}\cdot k_{b\bar{b}2to1} \cdot F_{b\bar{b}2} +  k_{c\bar{c}tob\bar{b}} \cdot P_{b\bar{b}1} \cdot k_{b\bar{b}2to1} \cdot F_{b\bar{b}2}  \\
&&+ P_{c1} \cdot  k_{c2to1} \cdot F_{c2} + P_{lq1}  \cdot k_{lq2to1} \cdot F_{lq2}) \\
N^{t}_2 &=& N^{p}_2 \cdot ( P_{b\bar{b}2}\cdot F_{b\bar{b}2} +  k_{c\bar{c}tob\bar{b}} \cdot P_{b\bar{b}2} \cdot F_{b\bar{b}2}  + P_{c2} \cdot F_{c2} + P_{lq2} \cdot F_{lq2})
\end{eqnarray*}

In the equations above, the P's are the b-tagging probability where the number of jets and jet flavor are specified by the subscripts.  They are used to convert the b-tagged (tag) sample flavor fractions (F) to the pretag versions we are solving for.  For instance,
\begin{equation}P_{b\bar{b},1} = \frac{N_{wb\bar{b}, 1jet, tag}}{N_{wb\bar{b}, 1jet, pretag}} \end{equation}
which means that
\begin{equation}N^{tag}_{c2} = N^{pretag}_{c2} \cdot P_{c2} \cdot F_{c2} \end{equation}

The k's in the equations are the ratio of yields in different number-of-jet bins or, in one case($k_{c\bar{c}tob\bar{b}}$), the flavor, and are always determined using the pretag sample.  They are conversion factors. For example,
%below is confirmed, 1 jet pretag was used for HCP Paper analysis for conversion factors
\begin{equation}k_{b\bar{b}2to1} = \frac{N_{wb\bar{b}, 1jet, pretag}}{N_{wb\bar{b},2jet, pretag}} ~~~\textrm{and} ~~~k_{b\bar{b}toc\bar{c}} = \frac{N_{wc\bar{c}, 1jet, pretag}}{N_{wb\bar{b},1jet, pretag}}\end{equation}
Finally, the N's in the equation represent the number of data minus background (where background is non-W+jets MC and multijets) events for the given bin and sample specified by p1 for pretag, 1 jet bin; p2 for pretag 2 jet bin; t1 for 1 b-tag (preselection) 1 jet bin; and t2 for 1 b-tag (preselection) 2 jet bin.

Note that the N's are the only data based quantities in the flavor fraction determination.  The P's and k's are taken from the values in Monte Carlo.  Thus, there are three (or four) equations and three unknowns, meaning a solution may be found with simple algebra.  These F's are then propagated into other number of jet bins.  When these values are combined with the overall W+jets normalization factors discussed earlier, the final W+jets scale factors for this analysis are obtained (WSF).

The equation used to form the scale factors for the two jet bin (which doesn't involve extra propagation) is given below in Equation~\ref{eq:propwsf}.  The F/F portion is the flavor fraction scaling and the N/N portion at the end of the equation is the overall W+jets normalization.  N's are data minus the non W+jets MC yields and multijets estimate as before, unless specified to be the MC W+jets estimate.  All quantities are pretag:
\begin{equation}\label{eq:propwsf}
 WSF_{c, 2} = \frac{N_{c, 2}}{N_{c, 2}^{MC}} =  \frac{F_{c, 2} \cdot N_{Wjets, 2}}{F^{MC}_{c, 2} \cdot N_{Wjets, 2}^{MC}}
\end{equation}
To find the scale factors in other bins, the three jet bin in particular, we use the following formula (Equation~\ref{eq:propwsf2}), shown here for the 3 jet c scale factor, where all quantities are pretag:
\begin{equation}\label{eq:propwsf2}
 WSF_{c, 3} = \frac{F_{c, 2} \cdot N_{Wjets, 3}}{F^{MC}_{c, 2} \cdot (N_{b\bar{b}, 3}^{MC} +N_{c\bar{c}, 3}^{MC} +N_{c, 3}^{MC}\cdot WSF_{c, 2}  +N_{lq, 3}^{MC} )}
\end{equation}

The final scale factor (WSF) values are shown in Table~\ref{tab:KFactor_comb_final} for the various number of jet bins and W+jets flavor types.  These are the values used to adjust the W+jets normalization in the analysis.

\begin{table}[!h!tpb]
  \begin{center}
    \begin{tabular}{lccc}
      \hline \hline
        Jet Bin &   $WSF_{b\bar{b}}$ &  $WSF_{light}$    &  $WSF_{c}$ \\\hline
        $W$+1jet   &1.361$\pm$0.090$\pm$1.066  &0.908$\pm$0.004$\pm$0.270& 1.273$\pm$0.040$\pm$0.449\\
        $W$+2jet   &1.252$\pm$0.090$\pm$0.864  &0.835$\pm$0.004$\pm$0.230& 1.172$\pm$0.004$\pm$0.302\\
        $W$+3jet   &1.182$\pm$0.090$\pm$0.854  &0.788$\pm$0.004$\pm$0.369& 1.106$\pm$0.004$\pm$0.443\\
      \hline
      \hline
    \end{tabular}
  \caption{Correction factor WSF for each $W$+jets flavor for the muon and electron samples combined, with statistical (first) and systematic (second) uncertainties.
  \label{tab:KFactor_comb_final} }
  \end{center}
\end{table}


\chapter{Event Yields and Discriminating Variables}~\label{chap:variables}
%Add more variables?  NJets, PDGID, Wt, met, etc?
In the full data set, it is impossible to distinguish the signal from the immense background.  To make an accurate measurement of the cross-section, we need to apply more selections that will reduce the background and isolate the single-top $t$-channel signal.  In this chapter we discuss the yields after the preselection and the effect of the b-tagging preselection requirement.  We will also outline the variables considered to achieve additional signal discrimination and demonstrate the agreement between data and MC using the preselection with and without the b-tagging requirement.

\section{Event Yields}
After applying the preselection, data-based normalization and models for multijets and W+jets, as well as additional event corrections, we obtain the the inital analysis yields which may be compared to data.  The pretag yields by process are as given in Table~\ref{tab:pretag_eventyields} and the yields after preselection (i.e. including b-tagging) are given in Table~\ref{tab:presel_eventyields}.  The signal divided by background (S/B) is only about 0.1 after the preselection, but improved by about a factor of 10 from the pretag yields, showing the importance of the b-tagging selection.  The yields are given in the different analysis channels which will be considered in Section~\ref{sec:channels}, based on the number of jets and charge of the lepton.

\begin{table}[!h!tpb]
  \begin{center}
     \begin{tabular}{lrr|rr}
    \hline \hline
        &\multicolumn{2}{c|}{2 Jets} &\multicolumn{2}{c} {3 Jets}  \\
        & Lepton + & Lepton -  & Lepton + & Lepton -  \\

    \hline \hline
    $t$-channel             & 1230	& 678	& 816	& 455  \\
    \hline                                                                       
    $t\bar t$, Other top    & 1730	& 1680	& 3510	& 3510  \\
    $W$+light jets          & 103000	& 64800	& 26600	& 16000  \\
    $W$+heavy flavor jets  & 35400	& 30800	& 10800	& 8920  \\
    $Z$+jets, Diboson       & 10200	& 9580	& 3560	& 3500  \\
    Multijets (e)           & 6960      & 7930  & 3160   & 2830 \\
    Multijets ($\mu$)       & 3300      & 3500  & 800    & 860 \\
    \hline    
    TOTAL Exp               & 162000	& 119000 & 49200 & 36100  \\
    S/B                     & 0.01	& 0.01	 & 0.02	 & 0.01  \\
    \hline \hline
    DATA                    & 162148	& 117010 & 46830 & 34925  \\
    \hline \hline
    \end{tabular}
 \caption{Event yields for the two-jets and three-jets tag positive and negative lepton-charge channels after the 
preselection, except for the b-tagging selection. The multijets and $W$+jets backgrounds are normalized to the data, all other samples are normalized to theory cross-sections.  Lepton types (muon and electron) are combined unless otherwise noted.
\label{tab:pretag_eventyields}}
  \end{center}
\end{table}
%mu 2j +,-: 1.446  1.489
%mu 3j +,-:5.49 5.68
%el 2j +,-:1.361 1.453
%el 3j +,-:5.57 5.27

\begin{table}[!h!tpb]
  \begin{center}
     \begin{tabular}{lrr|rr}
    \hline \hline
        &\multicolumn{2}{c|}{2 Jets} &\multicolumn{2}{c} {3 Jets}  \\
        & Lepton + & Lepton -  & Lepton + & Lepton -  \\

    \hline \hline
    $t$-channel             & 611    & 327     & 399     & 221        \\   
    \hline                                                                       
    $t\bar t$, Other top    & 805     & 781     & 1720    & 1730 \\
    $W$+light jets          & 544     & 308     & 183     & 154 \\
    $W$+heavy flavor jets   & 3100    & 2630    & 1350   & 1020 \\
    $Z$+jets, Diboson       & 175     & 150     & 92      & 83 \\
    Multijets (e)           & 365     & 342     & 279     & 295 \\
    Multijets ($\mu$)       & 221     & 215     & 139     & 135 \\
    \hline    
    TOTAL Exp               & 5820    & 4750	& 4160	  & 3630 \\
    S/B                     & 0.12    & 0.07	& 0.11	  & 0.06 \\
    \hline \hline
    DATA                    & 5912    & 4701    & 4016    & 3491 \\
    \hline \hline
    \end{tabular}
 \caption{Event yields for the two-jets and three-jets tag positive and negative lepton-charge channels after the 
preselection. The multijets and $W$+jets backgrounds are normalized to the data, all other samples are normalized
to theory cross-sections.  Lepton types (muon and electron) are combined unless otherwise noted.
\label{tab:presel_eventyields}}
  \end{center}
\end{table}

%    $t$-channel             & 611.583    & 327.256     & 398.966     & 220.826        \\   
%    \hline                                                                       
%    $t\bar t$, Other top    & 804.597     & 781.424     & 1719     & 1725.17 \\
%    $W$+light jets          & 544.46     & 307.552     & 183.111     & 154.058 \\
%    $W$+heavy flavor jets  & 3098.38     & 2630.49     & 1351.56     & 1016.22 \\
%    $Z$+jets, Diboson       & 174.985     & 149.737     & 92.1761     & 82.8742 \\
%    Multijets (e)           & 221.546     & 215.149     & 139.373     & 134.711 \\
%    Multijets ($\mu$)       & 365.511     & 342.367     & 279.477    &  295.386 \\
%    \hline    
%   TOTAL Exp              & 5821.06	& 4753.98	& 4163.66	& 3629.25 \\

%name     2jet+     2jet-        3jet+        3jet-
%QCDmu     365.511     342.367     279.477     295.386
%QCDel     221.546     215.149     139.373     134.711
%Zjets,WW,WZ     10195.7     9583.26     3556.53     3493.58
%WjetLQ     102526     64813.6     26582.6     16009.1
%WjetC,BB,CC     35368.9     30838.5     10835.7     8917.16
%top     1727.35     1676.37     3510.67     3512.38
%t-channel     1230.11     678.089     815.872     455.067
%data,default     162148     117010     46830     34925

\section{Discriminating Variables}
%Add more variables?  NJets, PDGID, Wt, met, etc?
Because the signal divided by the background is only 0.1 after applying the preselection, we would like to apply more selections that will reduce the background and isolate the single-top $t$-channel signal.  There are many variables considered in this analysis, about 80 in total.  We consider the \pt~and $\eta$ of all of the particles, as well as differences in the $\eta$, $\phi$ and $R$ quantities.  For example, we consider the $|\Delta\eta(\rm b, j_{u})|$, the $\Delta R$ between the leading untagged jet and the b-tagged jet.  We also consider the angles between various particles, as well as the invariant mass of several particle combinations, including all of the jets in the event ($M(\mathrm{All Jets})$) and the b-quark, lepton and neutrino ($M_{top}(\rm l\nu b)$).  This last is the reconstructed top quark mass (note that we use the reconstructed neutrino Z momentum, see Section~\ref{sec:Neutrinos}).  We also consider the transverse mass of the W and the sum of the \pt~of all the particles in the event (using \met~for the neutrino), called $H_{T}$.  Finally, we use the number of jets, the number of b-tagged jets, and the charge of the lepton to define analysis channels.  The lepton type, muon or electron, is not used for this purpose, so analysis channels contain a combination of the lepton types.

The variables used in this analysis are shown in the following figures.  The first set of figures shows the distributions using the preselection requirements except b-tagging (pretag) for 2 jets (Figure~\ref{fig:Plot_2NoTag}) and for 3 jets (Figure~\ref{fig:Plot_3NoTag}).  This first set includes a band showing the jet energy scale uncertainty.
In these figures and others, ``other top'' refers to the $s$-channel and $Wt$ single-top contributions.  The $W$+jets heavy flavor includes $Wc$, $Wc\bar c$, and $Wb\bar b$ jets

 \begin{figure}[!h!tpb]
 \centering
 \includegraphics[width=0.49\textwidth]{figures/variables/PaperFinal_MCtchannorm_2jetpretag__wflavorTopJet1_mass.eps}
 \includegraphics[width=0.49\textwidth]{figures/variables/PaperFinal_MCtchannorm_2jetpretag__wflavorJet2_Jet_eta.eps}
 \includegraphics[width=0.49\textwidth]{figures/variables/PaperFinal_MCtchannorm_2jetpretag__wflavorHt_Jet1Jet2LeptonMissingEt.eps}
 \includegraphics[width=0.49\textwidth]{figures/variables/PaperFinal_MCtchannorm_2jetpretag__wflavorDeltaEta_Jet1Jet2.eps}
 \includegraphics[width=0.49\textwidth]{figures/variables/Plot_Legend_JES.eps}
\vspace{-0.49cm}
 \caption{Discriminating variables for the pretag sample for two-jets events. Hatched bands show the jet energy scale uncertainty. The last bin contains the sum of the events in that bin or higher. Other top refers to the $s$-channel and $Wt$ single-top contributions.}
 \label{fig:Plot_2NoTag}
 \end{figure}


 \begin{figure}[!h!tpb]
 \centering
 \includegraphics[width=0.49\textwidth]{figures/variables/PaperFinal_MCtchannorm_3jetpretag__wflavorTopJet1_mass.eps}
 \includegraphics[width=0.49\textwidth]{figures/variables/PaperFinal_MCtchannorm_3jetpretag__wflavorJet2_Jet_eta.eps}
 \includegraphics[width=0.49\textwidth]{figures/variables/PaperFinal_MCtchannorm_3jetpretag__wflavorHt_AllJetsLeptonMissingEt.eps}
 \includegraphics[width=0.49\textwidth]{figures/variables/PaperFinal_MCtchannorm_3jetpretag__wflavorInvariantMass_AllJets.eps}
 \includegraphics[width=0.49\textwidth]{figures/variables/Plot_Legend_JES.eps}
\vspace{-0.49cm}
 \caption{Discriminating variables for the pretag sample for three-jets events. Hatched bands show the jet energy scale uncertainty. The last bin contains the sum of the events in that bin or higher. Other top refers to the $s$-channel and $Wt$ single-top contributions.}
 \label{fig:Plot_3NoTag}
 \end{figure}

Figures~\ref{fig:Plot_2Tag} and ~\ref{fig:Plot_3Tag} show the same distributions
after the requirement of exactly one $b$-tagged jet in the event for 2 or 3 jet samples.

 \begin{figure}[!h!tpb]
 \centering
 \includegraphics[width=0.49\textwidth]{figures/variables/PaperFinal_MCtchannorm_2jet1tag__wflavorTopBJet1_mass.eps}
 \includegraphics[width=0.49\textwidth]{figures/variables/PaperFinal_MCtchannorm_2jet1tag__wflavorUJet1_Jet_eta.eps}
 \includegraphics[width=0.49\textwidth]{figures/variables/PaperFinal_MCtchannorm_2jet1tag__wflavorHt_Jet1Jet2LeptonMissingEt.eps}
 \includegraphics[width=0.49\textwidth]{figures/variables/PaperFinal_MCtchannorm_2jet1tag__wflavorDeltaEta_BJet1UJet1.eps}
 \includegraphics[width=0.49\textwidth]{figures/variables/Plot_Legend.eps}
\vspace{-0.49cm}
 \caption{Discriminating variables for the preselection sample for two-jets events. The last bin contains the sum of the events in that bin or higher. Other top refers to the $s$-channel and $Wt$ single-top contributions.}
 \label{fig:Plot_2Tag}
 \end{figure}

\begin{figure}[!h!tpb]
 \centering
 \includegraphics[width=0.49\textwidth]{figures/variables/PaperFinal_MCtchannorm_3jet1tag__wflavorTopBJet1_mass.eps}
 \includegraphics[width=0.49\textwidth]{figures/variables/PaperFinal_MCtchannorm_3jet1tag__wflavorUJet1_Jet_eta.eps}
 \includegraphics[width=0.49\textwidth]{figures/variables/PaperFinal_MCtchannorm_3jet1tag__wflavorHt_AllJetsLeptonMissingEt.eps}
 \includegraphics[width=0.49\textwidth]{figures/variables/PaperFinal_MCtchannorm_3jet1tag__wflavorInvariantMass_AllJets.eps}
 \includegraphics[width=0.49\textwidth]{figures/variables/Plot_Legend.eps}
\vspace{-0.49cm}
 \caption{Discriminating variables for the preselection sample for three-jets events.  The last bin contains the sum of the events in that bin or higher. Other top refers to the $s$-channel and $Wt$ single-top contributions.}
 \label{fig:Plot_3Tag}
 \end{figure}


\chapter{The Cut-Based Analysis}
The separation of the $t$-channel single-top signal from its backgrounds has been performed with a cut-based analysis.  This analysis type typically requires a limited number of selections using a limited number of variables, so the selections that are made are strongly discriminating.  One advantage of a cut-based analysis is that it is relatively easy to interpret.  In this chapter, we discuss the kinematic regions (channels) chosen for this analysis as well as the selections and how they were determined.

\section{Analysis Channels}\label{sec:channels}
The analysis channels are chosen to be orthogonal (non-overlapping) kinematic regions.  We choose quantities for this that are discrete, specifically the jet number and lepton charge.  The background composition is closely related tot he number of jets in the evnet.  The $W$+jets and multijet backgrounds tend to have lower numbers of jets in the event while the \ttbar~background usually has four jets (although of course can also have less or more based on the $W$ decay, and jet reconstruction and \pt).  The "standard" t-channel single-top diagram has 2 or possibly 3 jets, so it is natural to look in these channels.  

We also consider the lepton charge when creating analysis channels.  The LHC collides protons with protons and because protons are composed of two up and one down valence quarks, there is an excess of positively charged up valence quarks.  This translates into an excess of positively charged leptons in the case of the $t$-channel single-top diagram, which usually has a valence quark in the initial state.  Processes like $W$+jets also have some charge asymmetry, but others like \ttbar~form primarily from gluons in the initial state and do not.  Thus, this sort of channel separation helps to reduce the background in the positively charged lepton channel in particular and changes the background composition.

\section{Analysis Method}
Performing the cut-based analysis includes determining the choice of selections to be applied to each analysis channel.  Each channel has its selections optimized separately, although sometimes the final selections are the same for certain channels.

\subsection{Selection Optimization}\label{sec:optimization}
The optimization to determine the analysis selections for a given channel uses a significance criterion (this has not been used to determine a significance of the result, only to optimize the selections).  The analysis itself is a cross-section measurement analysis, so one might expect that a cross-section criterion would be used in the optimization.  However, expected cross-sections were calculated for several cut sequences and the ones with the lowest cross-section uncertainties also tended to have the lowest significances.

The significance used includes the background uncertainties, and the calculation is very fast, which is important given the number of  variables and selection thresholds that are considered.  The method is a binomial significance method, also called Zbi, and is documented elsewhere~\cite{Zbi}.  This method is chosen over other common criteria, such as $\frac{S}{\sqrt{B}}$, because it is a real significance and includes systematic uncertainties.  The way it is implemented in this analysis is as suggested in the Zbi documentation~\cite{Zbi}, where $\sigma_b$ is the background systematic uncertainties, $N_b$ is the background yield and $N_{on}$ is the signal plus background yield.  These three parameters are the only inputs, so signal yield uncertainties are not included.  The value $p_{bi}$ is the probability, written in the form of ``the'' incomplete beta function~\cite{BetaIncomp} ($B_{incomp}$), as used in the analysis.  The significance is $Z_{bi}$ and written in terms of the error functions $Eff$:
% and then in terms of more standard incomplete beta ($B_i$) and beta fuctions ($B$)
\begin{eqnarray*}
\tau &=& \frac{N_b}{\sigma_b^2} \\
N_{off} &=& \tau*N_b \\
p_{bi} &=& B_{incomp}(\frac{1}{1+\tau},~N_{on},~N_{off}+1) \\
%       &=& B_i(\frac{1}{1+\tau}, ~N_{on} , ~N_{off}+1)/B (N_{on} , ~N_{off}+1) \\
Z_{bi} &=& \sqrt{2}Eff^{-1}(1-2p_{bi}) \\
\end{eqnarray*}
%
%\begin{verbatim} 
%	  sigmab = TMath::Sqrt(backgroundUncertainties)
%	  tau = bkgdYield/(sigmab*sigmab)
%	  noff = tau*bkgdYield
%	  non = signalYield + bkgdYield
%	  
%	  pbi = TMath::BetaIncomplete(1. / (1. + tau), non, noff + 1)
%	  zbi = TMath::Sqrt(2)*TMath::ErfInverse(1-2*pbi)
%\end{verbatim} 
%where background is the sum of the squares of the uncertainties on the background, bkgdYield is the total background MC yield, and signalYield is the total signal MC yield.  These three parameters are the only inputs, so signal uncertainties are not included.  
Not all of the background uncertainties are included for the purposes of the optimization, but several important ones are.  Included systematic uncertainties, discussed in Section~\ref{Sys}, are jet energy scale, b-tagging scale factor, mis-tagging scale factor, MC statistical, multijet background normalization, $W$+jets background normalization and flavor composition, and theoretical cross-section uncertainties.

The optimization of the selections themselves is done in an iterative way.  For a particular variable, the significance is evaluated for about 300 different possible thresholds over the variable range.  For each histogram bin, an integral is taken in both the left and right directions, the equivalent of a selection that is less than or greater than the threshold.  The two significance options (less than and greater than) are stored in two corresponding histograms at that bin location. When all bins for a variable have been evaluated, the maximum significance for each case for the variable in question is reported.  After all the variables we are interested in are considered, the variable with the largest significance (and its associated threshold) is chosen.  This selection is applied to the sample and the process is repeated to choose successive selections.  Figure~\ref{fig:OPThreshold_cut} shows these histograms for an example variable (the reconstructed top mass).  The curve is relatively smooth and the choice of threshold is noted.  The threshold peak is relatively broad, so small changes to the MC do not significantly impact the cut selection.  

\begin{figure}[!h!tpb]
 \centering
 \includegraphics[width=0.55\textwidth]{figures/sigopt/SigOpt_2jetbothcharge_ueta20_lines2.eps}
\vspace{-0.5cm}
 \caption{Distribution of the significance (y-axis) for the reconstructed top mass, for the 2 jet channel after preselection.  The vertical lines show the optimal cut thresholds for the two selections shown (less than and greater than some reconstructed top mass value) and the arrows indicate the region that is kept after the selection is applied.}
 \label{fig:OPThreshold_cut}
 \end{figure}

Additionally, selection sequences are considered that include the second best variable as the first selection, or other high significance variables for this or other selections.  This is because it is possible that the best selection and threshold from the first round may be a very harsh selection. After this selection, the statistics might be too low for further selections to improve the significance (considering the impact on the statistical uncertainties).  On the other hand, a different sequence, starting with a weaker cut but involving two other cuts, could, as a sequence, give a better uncertainty than the first sequence (one selection) did.  Still, it should be noted that even with this variation, not all possible cut sequences are tested and the method is biased towards selection sequences that start with strongly discriminating variables and cut thresholds.

Because the method includes uncertainties (including MC statistical uncertainty) and involves integrals from a given bin to the end of a range, it is relatively insensitive to random fluctuations.  Additionally, the thresholds are rounded.  There is no particular reason that a selection on the reconstructed top mass of greater than 192.75 GeV, for example, should be much better than a selection at 190 GeV.  This then acts as a check on the selections reported by the automated method and gives a more realistic view of the detector resolution.

\subsection{B-tagging Threshold and Cut-Based Selections}
As discussed in Section~\ref{sec:Btag}, the b-tagging threshold choice can have a large impact on the analysis.  Although the yields were shown in that section, we can also evaluate the impact later in the analysis.  Here, the selection optimization is repeated for three different b-tagging operating points.  The best significance (for some associated threshold) for each variable is given, where each y-axis entry corresponds to some variable i.  Figure~\ref{fig:OPThreshold_bjet_cuts} shows this for the 3 jets channel with positively charged leptons, preselection only, and then preselection plus one of two strong selections on the reconstructed top mass or untagged jet $\eta$.  In all three cases, the higher operating point is favored.  This was not necessarily expected; it could have been that a looser operating point might have been paired with a tighter threshold for some variable to give a higher significance than a tighter operating point.  However, we can see that this is not the case.

\begin{figure}[!h!tpb]
 \centering
 \includegraphics[width=0.49\textwidth]{figures/sigopt/SigOP_nocut_3jetplus.eps}
 \includegraphics[width=0.49\textwidth]{figures/sigopt/SigOP_topmass210_3jetplus.eps}
 \includegraphics[width=0.49\textwidth]{figures/sigopt/SigOP_ueta20_3jetplus.eps}
\vspace{-0.5cm}
 \caption{Distribution of the significance (x-axis) for various variables (each y-axis entry is a separate variable), given a JetFitterCombNN b-tagging operating point, denoted by different marker shapes.  The plots are all for the 3 jets, positively charged lepton channel. The top left plot is preselection only, the top right is preselection plus a requirement that the reconstructed top mass be less than 210 GeV, and the bottom plot is preselection plus a requirement that the $|\eta|$ of the highest \pt~ untagged jet be greater than 2.0.  The 2.4 operating point is used in the analysis.}
 \label{fig:OPThreshold_bjet_cuts}
 \end{figure}

\section{Selection Choices}\label{sec:selectionchoices}
For this analysis, the optimal variables and selection thresholds consist of four different selections, where the last selection is different between the channels depending on jet number.  This is due to the much larger \ttbar~background in the 3 jet bin, which is better rejected by a different selection.  There is no difference in selection for this analysis based on the lepton charge, although it is not unreasonable that some difference in selection could happen based on lepton charge in a future analysis, because of the different background composition.

The selections in common for all channels are: $|\eta({\rm j_{u}})|>2.0$, $150\GeV<M_{top}(\rm l\nu b)<190\GeV$ and , $H_{\mathrm{T}}>210\GeV$.  The 2 jet selection also requires $|\Delta\eta(\rm b, j_{u})| > 1.0$~ and the 3 jet selection requires $M(\mathrm{All Jets}) > 450\GeV$.  In the case of the three jet bin, the untagged light jet is taken to be the highest $p_{T}$ untagged jet in the final state.

These selections have some physical justification.  The first selection makes use of the untagged jet.  Because the t-channel initial state usually contains a valence quark, the untagged jet in the final state is often energetic and close to the beam line, much more often than for the background processes.  Thus, we require the untagged jet to be forward.  The second selection simply requires the reconstructed top mass to be close to the measured value.  The single-top $t$-channel process only has one top quark so the decay products are the reconstructed W and the b-tagged jet (assuming we have identified this correctly).  In the case of the backgrounds, there either is no top quark, or there are too many and the correct decay products may not be matched together during the top reconstruction.  Thus, this selection also is a powerful discriminator.  The third common selection requires the sum of the transverse momenta of the final state particles to be large, which helps to reject lower energy $W$+jets or multijet events.

The final selection is different for the different jet bins.  In the 2 jet bin, we require the b-tagged jet (associated with the top) and the untagged jet to be separated in $|\eta|$.  This  helps to reject backgrounds where the two jets may have come from a gluon or both from a top quark decay, and are more likely to be close together.  In the 3 jet bin, we require the invariant mass of the three jets to be large.  This is a particularly good discriminator against \ttbar~events, where these three jets may have come from a top quark, for instance.  In the t-channel single-top signal, we expect the untagged jet to be energetic and separated from the b-jet, leading to a potentially large invariant mass.

The individual channel compositions after all cuts are shown in Figure~\ref{fig:tch_cut_pdgid} and distributions after all selections except the one on the variable pictured are shown in Figure~\ref{fig:Plot_2TagCuts} for the 2 jet selection and Figure~\ref{fig:Plot_3TagCuts} for the 3 jet selection.  In all three cases, the t-channel cross-section is normalized to the observed result formed using all four channels, discussed in Section~\ref{sec:fourchanresult}.
%Include plots for plus and minus channels also, normalized to top and anti-top?

\begin{figure}[!h!tpb]
 \centering
%\hspace{-0.3cm} 
\includegraphics[width=0.46\textwidth]{figures/variables/PaperFinal_Datatchannorm_LeptonCharge_Channels__wflavorLepton_pdgId.eps}
 \includegraphics[width=0.46\textwidth]{figures/variables/Plot_Legend.eps}
 \caption{Distribution of the lepton charge after the full cut-based selection for two jets and three jets.  These are the four primary analysis channels. The $t$-channel single-top contribution is normalized to the observed cross-section determined using all four channels. Other top refers to the $s$-channel and $Wt$ single-top contributions.}
 \label{fig:tch_cut_pdgid}
 \end{figure}

 \begin{figure}[!h!tpb]
 \centering
 \includegraphics[width=0.46\textwidth]{figures/variables/PaperFinal_Datatchannorm_2jet1tag_cut__wflavorTopBJet1_mass.eps}
 \includegraphics[width=0.46\textwidth]{figures/variables/PaperFinal_Datatchannorm_2jet1tag_cut__wflavorUJet1_Jet_eta.eps}

 \includegraphics[width=0.46\textwidth]{figures/variables/PaperFinal_Datatchannorm_2jet1tag_cut__wflavorHt_Jet1Jet2LeptonMissingEt.eps}
 \includegraphics[width=0.46\textwidth]{figures/variables/PaperFinal_Datatchannorm_2jet1tag_cut__wflavorDeltaEta_BJet1UJet1.eps}
 \includegraphics[width=0.46\textwidth]{figures/variables/Plot_Legend.eps}
\vspace{-0.5cm}
 \caption{Discriminating variables for the tagged sample for two-jets events after applying all cut based cuts except for the cut on the plotted variable.  The $t$-channel single-top contribution is normalized to the observed cross-section determined using all four channels. Other top refers to the $s$-channel and $Wt$ single-top contributions.  The last bin contains the sum of the events in that bin or higher. }
 \label{fig:Plot_2TagCuts}
 \end{figure}


 \begin{figure}[!h!tpb]
 \centering
 \includegraphics[width=0.46\textwidth]{figures/variables/PaperFinal_Datatchannorm_3jet1tag_cut__wflavorTopBJet1_mass.eps}
 \includegraphics[width=0.46\textwidth]{figures/variables/PaperFinal_Datatchannorm_3jet1tag_cut__wflavorUJet1_Jet_eta.eps}
 \includegraphics[width=0.46\textwidth]{figures/variables/PaperFinal_Datatchannorm_3jet1tag_cut__wflavorHt_AllJetsLeptonMissingEt.eps}
 \includegraphics[width=0.46\textwidth]{figures/variables/PaperFinal_Datatchannorm_3jet1tag_cut__wflavorInvariantMass_AllJets.eps}
 \includegraphics[width=0.46\textwidth]{figures/variables/Plot_Legend.eps}
\vspace{-0.5cm}
 \caption{Discriminating variables for the tagged sample for three-jets events after applying all cut based cuts except for the cut on the plotted variable. The $t$-channel single-top contribution is normalized to the observed cross-section determined using all four channels. Other top refers to the $s$-channel and $Wt$ single-top contributions.  The last bin contains the sum of the events in that bin or higher. }
 \label{fig:Plot_3TagCuts}
 \end{figure}

Table~\ref{tab:tch_eventyields} shows the number of events after all cut-based analysis selections for the positive and negative lepton charge and number of jet channels.  The t-channel yield is the standard model expectation in this table.  All analysis systematic uncertainties are included in the reported yields.  The individual uncertainty contributions are discussed in Section~\ref{Sys}.

Although we do not split the events by the lepton type when making analysis channels and determining the result, it is possible to investigate what the proportion of the different leptons is in this analysis.  There is no particular dependency on the lepton type inherent this analysis and we would expect the number of electrons and muons to be roughly equal.  To determine these numbers we use all of the analysis channels combined (plus and minus charge, two and three jets) after the cut-based selections, the $t$-channel single-top contribution is normalized to the observed cross-section determined using all four channels, and the multijets and W+jets contributions determined using the data-based normalizations.  The expected number of of events with muons is 204 and there are 182 corresponding data events observed.  For the electron selection, there are 181 events expected and 204 events observed in data.  These numbers are very similar and demonstrate the roughly one to one ratio of muons and electrons in this analysis.  The deviation of the electron yield from muon yield is about 10\%, which is well within the uncertainties for this analysis.  The systematic uncertainty on the total expected yield by channel is given in Table~\ref{tab:tch_eventyields} and is about 15 to 20\%, while the data statistical uncertainty is about 7 to 8\%.

\begin{table}[!h!tpb]
  \begin{center}
     \begin{tabular}{lrr|rr}
    \hline \hline
        &\multicolumn{2}{c|}{Cut-based 2 Jets} &\multicolumn{2}{c} {Cut-based 3 Jets}  \\
        & Lepton + & Lepton -  & Lepton + & Lepton -  \\

    \hline \hline
    $t$-channel            & $ 85.2 \pm 28.6 $ & $ 39.4 \pm 12.8 $ & $ 33.6 \pm 7.0$ & $ 14.6 \pm 6.2 $ \\
    \hline                                                                       
    $t\bar t$, Other top   & $ 14.0 \pm 6.4 $ & $ 12.8 \pm 4.2 $ & $ 10.5 \pm 4.2 $ & $ 10.7 \pm 7.9 $ \\
    $W$+light jets         & $ 3.3 \pm 1.9 $   & $ 2.0 \pm 1.2 $ & $ 0.8 \pm 1.3 $ & $ 0.3 \pm 0.3 $ \\
    $W$+heavy flavour jets & $ 39.1 \pm 10.6 $  & $ 27.1 \pm 7.5 $ & $ 8.7 \pm 6.0 $ & $ 3.4 \pm 3.1 $ \\
    $Z$+jets, Diboson      & $ 1.1 \pm 0.8 $  & $ 1.0 \pm 0.8 $ & $ 0.3 \pm 0.2 $ & $ 0.2 \pm 0.3 $ \\
    Multijets              & $ 0.2 \pm 0.2 $ & $ 0.3 \pm 0.3 $ & $ 1.5 \pm 1.1 $ & $ 3.1 \pm 2.0 $ \\
    \hline    
    TOTAL Exp              & $ 142.9 \pm 31.2 $ & $ 82.6 \pm 15.5 $ & $ 55.5 \pm 10.2 $ & $ 32.2 \pm 10.68 $ \\
    S/B                    &  1.5  & 0.9 &  1.6 &   1.0  \\
    \hline \hline
    DATA                   &   193  &   101   &   53  &   39   \\
    \hline \hline
    \end{tabular}
 \caption{Event yield for the two-jets and three-jets tag positive and negative lepton-charge channels after the 
cut-based selection. The multijets and $W$+jets backgrounds are normalized to the data, all other samples are normalized
to theory cross-sections (including single-top $t$-channel).  Uncertainties shown are systematic uncertainties. Other top refers to the $s$-channel and $Wt$ single-top contributions.
\label{tab:tch_eventyields}}
  \end{center}
\end{table}


\chapter{The Measurements}
The purpose of this dissertation is to measure the single-top t-channel cross-section.  In previous chapters we have spent much time reducing the backgrounds, first in an initial preselection and then again using cut-based selections.  In this chapter, we evaluate the signal cross-section after applying these selections.  We also estimate the value of the CKM matrix element $|V_{tb}|$.

\section{Systematic Uncertainties}\label{Sys}
Before we can determine the cross-section, we need to evaluate the uncertainties on the quantities which go into the calulation.  The cross-section is related to the number of events that are observed for some given amount of proton-proton interactions, as stated in Section~\ref{sec:ExpectedCrosssections}.  If more events are observed than expected, the cross-section is higher than the expected value.  Uncertainties on the measurement are important here, as deviations from the expected cross-section may well be due to systematic uncertainties.  In this section, we discuss the systematic uncertainties on the measurement.

There are several systematic uncertainties in this analysis and we overview them by category and then give some information about their impact on the signal and background yields.  Most of the uncertainties are related to the MC.  For additional information on most of the scale factors, corrections, and MC itself, see Chapter~\ref{chap:MC}.

\textbf{b-tagging}:
There is an uncertainty associated with the b-tagging and mis-tagging scale factors, which relate the efficiencies measured in data to that in MC.  The b-tagging scale factor uncertainty in particular can be large in this analysis.  There is also a c-tagging efficiency uncertainty, which for this analysis is assumed to be twice that of the b-tagging efficiency uncertainty.  This is considered fully correlated with the uncertainty in b-tagging efficiency and is included in the reported b-tagging efficiency uncertainty (which may also be called the heavy flavor b-tagging scale factor uncertainty).  This is a large uncertainty for this analysis, with variations of around $10\%$ on the signal and background yields.

\textbf{Leptons}:
There are uncertainties on the lepton scale factors, which relate the trigger, ID, and reconstruction scale factors in data to MC, and also uncertainties on the lepton energy scale and resolution, which are uncertainties related to smearing the lepton energy, described in Section~\ref{sec:energyresolution}.  Also, for this study there was an issue with the MC related to the muon trigger matching.  This caused us not to apply trigger matching for the muon channel (although the trigger itself was still applied).  An uncertainty of 1.5\% was added to account for this.  These uncertainties are typically $<5\%$ for the different analysis processes.

\textbf{Jets}:
%apparently no bjes citations avaliable
There are three uncertainties associated with jets regarding the jet energy scale (JES), jet energy resolution (JER) and jet reconstruction (jetreco).  The JES uncertainty~\cite{JESnew, JES} is related to the energy calibration.  For example, we may not have perfectly simulated the dead material or leakage when adjusting the energy and there is some uncertainty related to this.  There may be noise or uncertainties related to the MC as well. JES includes a few different components, including a pileup and b-JES contribution.  The pileup is a special correction to account for pileup conditions during 2011 data taking and the effect on jet energies.  The b-JES factor is a separate corrections for jets which have a truth b-quark assignment.  It considers b-quark fragmentation, material and calorimeter response separately for these jets.  There is also some consideration of flavor composition uncertainty (gluon fraction distribution), which has a different distribution for each of the top samples and a flat distribution for other processes. The distance to other jets is also considered and jets that are close to one another have a different uncertainty.  Overall, this JES uncertainty (after including W+jets scale factor correlations) is largest for the light quark W+jets events, which are removed effectively by analysis cuts.  The impact is around $10\%$ for \ttbar and a few percent for the signal in the largest signal channel, 2 jets with a positive lepton charge.
%check that close-by jet corrections are on!!

The other two jet related uncertainties generally have an impact of a few percent for the different processes.  The  JER uncertainty is related to the jet energy value and is evaluated by smearing the jet energy (this is not done in the nominal sample, unlike for leptons, as discussed in Section~\ref{sec:energyresolution}).  The jet reconstruction evaluates how sensitive the analysis is to a missed jet.  This is done by randomly dropping jets from the event based on jet kinematics.

\textbf{Theoretical cross-section}:
There are several processes for which we do not have data-based normalization estimates.  In most cases, the contribution of these processes to the final yield is small.  In the case of \ttbar, we have performed a cross-check (see Appendix~\ref{app:ttbar}) and found the estimated normalization is consistent with the theoretical value, which has a smaller uncertainty.  We use a 10\% uncertainty for the single-top s-channel and Wt processes, 5\% for diboson processes, 60\% uncertainty for $Z$+jets and the cross-section variation is taken to be $164.57_{-15.7}^{+11.4}~pb$ for the \ttbar~process.  In this analysis we combine certain processes together when reporting yields and results.  When this is done, uncertainties such as the theoretical cross-section uncertainty are based on the proportion of each process in the combined sample (rather than taking the largest uncertainty, for instance).

\textbf{Multijets}:
There is an uncertainty on the multijets normalization, discussed in Section~\ref{sec:multijets}.  This is determined by re-doing the fit, which determines the normalization, using a different variable (W transverse mass).  We use 50\% for this uncertainty.

\textbf{$W$+jets}:
There are uncertainties on the $W$+jets scale factors discussed in Section~\ref{sec:wjets}.  These include b-tagging scale factor, mis-tagging scale factor, JES, theoretical cross-section, and data statistical uncertainties.  Many of these uncertainties are correlated with the uncertainties in the t-channel single-top cross-section measurement.  This means that the behavior of the JES uncertainty for the $W$+jets scale factor estimate and W+jets yield JES uncertainty are related to each other.  

To properly include these correlations, we re-estimate the $W$+jets scale factors for each uncertainty scenario and then apply the appropriate scale factor when estimating the W+jets yield uncertainty.  We assume that the JES upward shift scenario, for example, is the ``real'' scenario and do all of the estimations such as we would for the nominal sample, using JES upwardly shifted numbers instead of nominal numbers.  Then, to find the total JES upwardly shifted uncertainty, we compare the final yield (with the JES shifted scale factors applied to the JES shifted sample) to the nominal sample (with the nominal scale factors applied to the nominal sample).  

The JES, b-tagging scale factor and mis-tagging scale factor uncertainties quoted in this document always include these correlation effects.  The theoretical cross-section uncertainties and multijet normalization uncertainties for the $W$+jets scale factors are also correlated, but because they are not correlated with $W$+jets yield uncertainties, they are listed separately when uncertainties are given by processes (and are given in this way to the statistical tool).  The correlations are included in the final cross-section measurement.  Finally, the statistical uncertainties are considered separately and are called the $W$+jets normalization uncertainties.  These are $\le 5\%$.

There is another uncertainty associated with the $W$+jets normalization, related to the propagation of the scale factors from the 2 jet bin to other bins.  This is a 25\% uncertainty for a movement to the 3 jet bin, which is the other primary analysis bin.  This is referred to as the $W$+3 jet bin normalization.

One additional uncertainty related to the $W$+jets is an uncertainty on the simulated shape.  To evaluate this, two {\sc Alpgen}~\cite{SAMPLES-ALPGEN, ALPGENFAQ} parameters are varied and the uncertainties from these two variations are added in quadrature.  These parameters are the minimum {\sc Alpgen} \pt~to make a parton a hard (high \pt) parton (ptjmin) and the function which gives the factorization scale for the pdf (iqopt).

\textbf{MC statistical}:
There is an uncertainty associated with the number of simulated MC events.  If not enough events are generated, there may not be a sufficient range of kinematics to accurately represent the data.  The uncertainty is evaluated as the square root of the sum of the squares of the event weights and can range as high as $98\%$ after all cut-based analysis selections.

\textbf{LAr hole}:
There is some uncertainty on the removal of events affected by the LAr hole, discussed in Section~\ref{sec:dataquality}.  The uncertainty is a $\pm1$ sigma variation of the hole size and typically has a $<5\%$ effect on the signal and background yields.

\textbf{Missing $E_{T}$}:  There are two \met~related uncertainties.  The first is due to pileup effects (\met~pileup uncertainty) and the second is due to energy scale and energy resolution effects (\met~uncertainty), including cell out contribution uncertainties (energy deposits not associated with jets, electrons, $\tau$'s or photons) and soft jet uncertainties (related to objects that have a \pt too low to be considered a jet).  The pileup uncertainty portion is a 10\% variation.  Both uncertainties typically range from $1-10\%$.

\textbf{ISR/FSR}:
There is some uncertainty on the MC simulation of the initial and final state radiation.  These are extra particles perhaps formed by gluons producing extra radiation (jets) in the initial or final state portion of the Feynman diagram.  Extra jets, if the \pt~is high enough, could move events from the 2 jet channel to the 3 jet channel and thus affect the analysis.  This uncertainty is evaluated by changing certain parameters when producing the MC, and is evaluated separately for the \ttbar~and single-top processes.  Special {\sc AcerMC} samples showered with {\sc Pythia} are used for all the top processes.  For this analysis, we vary the ISR and FSR simultaneously (which produces a larger variation than varying them separately for the largest signal channel, 2 jets with positive leptons).  This is one of the largest uncertainties, with variations of around $10-30\%$ depending on the process.

%https://twiki.cern.ch/twiki/bin/view/AtlasProtected/TopMC2010#MC10_Common_Conventions
%The variations technically are PARP (67)=0.5, PARP (64)=4*D, PARP (72)=0.5*D, PARJ(82)=2*D for ISR/FSR down and PARP (67)=6, PARP (64)=1/4*D, PARP (72)=2*D, PARJ(82)=0.5*D for ISR/FSR up, where D is the default value (PARP (67) is 4, PARP (64) is 1, PARP(72) is 0.192 GeV, PARJ(82) is 1 GeV).
%
%    PARP (67) : controls suppression of ISR branchings above the coherence scale,
%    PARP (64) : multiplies ISR alpha_strong evolution scale, the effect is \propto 1/(lambda_ISR^2),
%    PARP(72) : lambda FSR,
%    PARJ(82) : low-pt cutoff of the FSR branchings (i.e. it sets the lower pt at which hadronization takes over from parton shower). 
%
%Values of the parameters in the central sample are (in both AMBT1 and MC09 tunes): PARP (67) : 4, PARP (64) : 1, PARP(72) : 0.192 GeV, PARJ(82) : 1 GeV.
%The parameter variations are done around the default[D] values listed above for the samples as follows (the terms up and down refer to more and less Parton Shower activity respectively):
%    117259 : ISR down and FSR down : PARP (67)=0.5, PARP (64)=4*D, PARP (72)=0.5*D, PARJ(82)=2*D,
%    117260 : ISR up and FSR up : PARP (67)=6, PARP (64)=1/4*D, PARP (72)=2*D, PARJ(82)=0.5*D. 

\textbf{PDF}:
The parton distribution function may also not be well modelled.  The uncertainty is evaluated by finding the variation from changing the PDF in the preselection sample from the one used in this analysis,{\sc CTEQ6L}, to {\sc CTEQ66}~\cite{cteq6}, {\sc NNPDF20}~\cite{NNPDF1, NNPDF2}, or {\sc MSTW2008}nnlo68cl~\cite{MSTW1, MSTW2}.  This uncertainty ranges from $1\%$ to $8\%$ depending on the process.
%taken in consideration are: MSTW2008nnlo68cl, cteq66 and NNPDF20.We reweight the signal and background MC samples according to
%each of the PDF uncertainty eigenvectors and take the largest variation as the uncertainty

\textbf{Generator and Shower}:
The MC generator or showering programs may not exactly match the data.  To evaluate these uncertainties, an alternative generator or showering program is used and the deviation determined.  This is done for the \ttbar~and the single-top processes.  For the t-channel single-top process, {\sc MCFM}~\cite{MCFM} is used to determine a deviation with the preselection sample of $7\%$.  For the other processes, the generator uncertainties are determined after cut-based selections as usual, using \textsc{MC@NLO} versus \textsc{Herwig} for \ttbar~and {\textsc AcerMc} versus \textsc{MC@NLO} for the s-channel and Wt single-top processes.  The shower uncertainties for the single-top processes are determined using {\textsc AcerMc} plus \textsc{Pythia} versus \textsc AcerMc plus \textsc{Herwig}.  The \ttbar~shower uncertainties are found by comparing yields from {\textsc Powheg}~\cite{SAMPLES-POWHEG, SAMPLES-POWHEG-1} plus \textsc{Pythia} and \textsc Powheg plus \textsc{Herwig}.  These uncertainties are all symmetrized, so the deviation between the nominal and alternate program is divided in half.  One half is taken as the up shift, and the other is taken as the down shift.  These are some of the larger uncertainties in the analysis, with variations around $10-15\%$ depending on the process.

\textbf{\eta~reweighting}:
The shape of the \eta~distribution of the forward jet is not especially well-modeled.  We renormalize the MC to the data in a pretag sample and then evaluate the difference between using this and using the nominal sample after all of the analysis selections.  This uncertainty is a one-sided uncertainty (there is only a positive shift, no negative shift).  The uncertainty is about $5-10\%$ depending on the process.

\textbf{Luminosity}:
The luminosity estimate has some uncertainty associated with it.  The luminosity estimate is done with dedicated luminosity estimate runs.  The uncertainty is 3.7\%~\cite{Luminosityoverview} for the data used in this analysis.

The individual uncertainties that are used to find the total cross-section uncertainties are given in Table~\ref{tab:uncertainty-exp1}, Table~\ref{tab:uncertainty-exp2}, Table~\ref{tab:uncertainty-exp3}, and Table~\ref{tab:uncertainty-exp4} by process, where each table gives the values for a different analysis channel.  These are the values which are used in the statistical tool (see Section~\ref{sec:xs}) to determine the cross-section.  In certain cases, processes have very high MC statistical uncertainties after all cut-based selections, especially in the 3 jet channels.  This can cause some large estimates for other uncertainties as well.  Although the actual uncertainties may not be as high as we estimate, we keep the large values to be conservative.

\begin{table}[htdp]
\begin{center}
   \begin{tabular}{l|cccccc}
    \hline
Uncertainties(\%) & $t$-channel & \ttbar, Wt, s & $W$+light  &$W$+heavy & $Z$,Dib.& Multijets \\ 
\hline
Jet energy scale  & -3 & -11 & -19 & -1 & 33 & - \\ 
 & -1 & 7 & 28 & -9 & -9 & - \\ 
\hline
Jet energy resolution & $\pm$4 & $\pm$1 & - & - & $\pm$4 & - \\ 
\hline
Jet reconstruction & $<1$ & $<1$ & $\pm$2 & $<1$ & $\pm$1 & - \\ 
\hline
$b$ tagging scale factor  & 12 & 9 & 7 & -3 & 15 & - \\ 
 & -12 & -9 & -10 & 3 & -15 & - \\ 
\hline
Mistag scale factor  & $<1$ & $<1$ & 24 & -4 & 5 & - \\ 
 & $<1$ & $<1$ & -23 & 4 & -5 & - \\ 
\hline
Lepton scale factor & $\pm$3 & $\pm$3 & - & - & $\pm$2 & - \\ 
\hline
Lepton efficiencies & $\pm$1 & $\pm$1 & - & - & $\pm$4 & - \\ 
\hline
Generator single-top & $\pm$7 & $\pm$1 & - & - & - & - \\ 
\hline
Generator \ttbar & - & $\pm$1 & - & - & - & - \\ 
\hline
Shower & $\pm$11 & $\pm$12 & - & - & - & - \\ 
\hline
ISR/FSR  & -15 & 32 & - & - & - & - \\ 
 & 27 & 39 & - & - & - & - \\ 
\hline
PDF & $\pm$3 & $\pm$8 & - & - & $\pm$1 & - \\ 
\hline
Luminosity & $\pm$4 & $\pm$4 & - & - & $\pm$4 & - \\ 
\hline
\met  & -1 & -1 & -5 & -5 & -1 & - \\ 
 & $<1$ & 1 & -16 & $<1$ & $<1$ & - \\ 
\hline
\met~pileup  & -2 & -1 & -5 & -5 & -1 & - \\ 
 & $<1$ & 1 & -16 & -2 & $<1$ & - \\ 
\hline
LAr  & 1 & 1 & $<1$ & $<1$ & $<1$ & - \\ 
 & -1 & -1 & $<1$ & -1 & $<1$ & - \\ 
\hline
$\eta$ reweighting & 5 & 2 & $<1$ & 4 & 9 & - \\ 
\hline
$W$ shape & - & - & $<1$ & $<1$ & - & - \\ 
\hline
$Wjj$ norm & - & - & $<1$ & - & - & - \\ 
\hline
$Wc,cc,bb$ norm & - & - & - & $\pm$5 & - & - \\ 
\hline
$W$ 3 jet norm & - & - & - & - & - & - \\ 
\hline
Multijets & - & - & $\pm$3 & $\pm$9 & - & $\pm$50 \\ 
\hline
\ttbar~XS & - & $\pm$5 & $\pm$1 & $\pm$5 & - & - \\ 
\hline
single-top XS & - & $\pm$2 & $\pm$4 & $\pm$17 & - & - \\ 
\hline
Z+jets XS & - & - & $\pm$6 & $\pm$2 & $\pm$41 & - \\ 
\hline
Diboson XS & - & - & $<1$ & $<1$ & $\pm$2 & - \\ 
\hline
MC Statistics & $\pm$4 & $\pm$6 & $\pm$34 & $\pm$12 & $\pm$47 & $\pm$100 \\ 
\hline

   \end{tabular}
\caption{Percent systematic uncertainties for the 2 jet plus channel.  Here, XS means cross-section, $Z$ means $Z$+jets, and Dib. means diboson.  Norm refers to normalization, s indicates single-top $s$-channel.  If two values are given, the top value is the upshift and the bottom value is the downshift.}
\label{tab:uncertainty-exp1}
\end{center}
\end{table}

\begin{table}[htdp]
\begin{center}
   \begin{tabular}{l|cccccc}
    \hline
Uncertainties(\%) & $t$-channel & \ttbar, Wt, s & $W$+light  &$W$+heavy & $Z$, Dib. & Multijets \\ 
\hline
Jet energy scale  & $<1$ & -7 & -22 & 3 & -1 & - \\ 
 & -4 & 9 & -19 & -11 & -20 & - \\ 
\hline
Jet energy resolution & $\pm$3 & $\pm$1 & - & - & $\pm$30 & - \\ 
\hline
Jet reconstruction & $<1$ & $<1$ & $\pm$1 & $\pm$2 & $\pm$2 & - \\ 
\hline
$b$ tagging scale factor  & 12 & 9 & 9 & 2 & 4 & - \\ 
 & -12 & -9 & -12 & -3 & -4 & - \\ 
\hline
Mistag scale factor  & $<1$ & $<1$ & 23 & -3 & 21 & - \\ 
 & $<1$ & $<1$ & -22 & 3 & -21 & - \\ 
\hline
Lepton scale factor & $\pm$3 & $\pm$3 & - & - & $\pm$3 & - \\ 
\hline
Lepton efficiencies & $\pm$2 & $<1$ & - & - & $\pm$5 & - \\ 
\hline
Generator single-top & $\pm$7 & $<1$ & - & - & - & - \\ 
\hline
Generator \ttbar & - & $\pm$9 & - & - & - & - \\ 
\hline
Shower & $\pm$14 & $\pm$5 & - & - & - & - \\ 
\hline
ISR/FSR  & -14 & -16 & - & - & - & - \\ 
 & 24 & 25 & - & - & - & - \\ 
\hline
PDF & $\pm$3 & $\pm$8 & - & - & $\pm$1 & - \\ 
\hline
Luminosity & $\pm$4 & $\pm$4 & - & - & $\pm$4 & - \\ 
\hline
\met  & -4 & $<1$ & -7 & -13 & $<1$ & - \\ 
 & $<1$ & 2 & $<1$ & 2 & 1 & - \\ 
\hline
\met~pileup  & -3 & $<1$ & -7 & -11 & $<1$ & - \\ 
 & $<1$ & 2 & $<1$ & 2 & $<1$ & - \\ 
\hline
LAr  & $<1$ & 1 & $<1$ & $<1$ & $<1$ & - \\ 
 & -1 & -1 & -1 & -3 & $<1$ & - \\ 
\hline
$\eta$ reweighting & 4 & 2 & 4 & 4 & 4 & - \\ 
\hline
$W$ shape & - & - & $\pm$3 & $\pm$2 & - & - \\ 
\hline
$Wjj$ norm & - & - & $<1$ & - & - & - \\ 
\hline
$Wc,cc,bb$ norm & - & - & - & $\pm$4 & - & - \\ 
\hline
$W$ 3 jet norm & - & - & - & - & - & - \\ 
\hline
Multijets & - & - & $\pm$3 & $\pm$6 & - & $\pm$50 \\ 
\hline
\ttbar~XS & - & $\pm$6 & $\pm$1 & $\pm$1 & - & - \\ 
\hline
single-top XS & - & $\pm$2 & $\pm$4 & $\pm$5 & - & - \\ 
\hline
Z+jets XS & - & - & $\pm$6 & $\pm$2 & $\pm$40 & - \\ 
\hline
Diboson XS & - & - & $<1$ & $<1$ & $\pm$2 & - \\ 
\hline
MC Statistics & $\pm$6 & $\pm$6 & $\pm$45 & $\pm$15 & $\pm$55 & $\pm$100 \\ 
\hline

   \end{tabular}
\caption{Percent systematic uncertainties by process for the 2 jet minus channel.  Here, XS means cross-section, $Z$ means $Z$+jets, and Dib. means diboson.   Norm refers to normalization, s indicates single-top $s$-channel.  If two values are given, the top value is the upshift and the bottom value is the downshift. }
\label{tab:uncertainty-exp2}
\end{center}
\end{table}

\begin{table}[htdp]
\begin{center}
   \begin{tabular}{l|cccccc}
    \hline
Uncertainties(\%) & $t$-channel & \ttbar, Wt, s & $W$+light  &$W$+heavy & $Z$, Dib. & Multijets \\ 
\hline
Jet energy scale  & 4 & -8 & 18 & 37 & -32 & - \\ 
 & -10 & 16 & -17 & 17 & -40 & - \\ 
\hline
Jet energy resolution & $<1$ & $<1$ & - & - & $\pm$10 & - \\ 
\hline
Jet reconstruction & $\pm$1 & $<1$ & $\pm$2 & $\pm$2 & $\pm$3 & - \\ 
\hline
$b$ tagging scale factor  & 9 & 7 & 2 & -7 & 11 & - \\ 
 & -9 & -8 & -4 & 9 & -11 & - \\ 
\hline
Mistag scale factor  & $<1$ & $<1$ & 33 & -4 & 1 & - \\ 
 & $<1$ & $<1$ & -32 & 3 & -1 & - \\ 
\hline
Lepton scale factor & $\pm$3 & $\pm$3 & - & - & $\pm$2 & - \\ 
\hline
Lepton efficiencies & $\pm$1 & $\pm$1 & - & - & $\pm$9 & - \\ 
\hline
Generator single-top & $\pm$7 & $<1$ & - & - & - & - \\ 
\hline
Generator \ttbar & - & $\pm$22 & - & - & - & - \\ 
\hline
Shower & $\pm$7 & $\pm$8 & - & - & - & - \\ 
\hline
ISR/FSR  & -5 & 4 & - & - & - & - \\ 
 & -1 & 22 & - & - & - & - \\ 
\hline
PDF & $\pm$3 & $\pm$8 & - & - & $\pm$1 & - \\ 
\hline
Luminosity & $\pm$4 & $\pm$4 & - & - & $\pm$4 & - \\ 
\hline
\met  & -1 & -1 & -98 & 5 & $<1$ & - \\ 
 & 3 & -2 & $<1$ & -3 & $<1$ & - \\ 
\hline
\met~pileup  & 1 & $<1$ & -98 & 5 & $<1$ & - \\ 
 & 1 & -2 & $<1$ & -5 & $<1$ & - \\ 
\hline
LAr  & $<1$ & 2 & $<1$ & $<1$ & $<1$ & - \\ 
 & -2 & -1 & -2 & $<1$ & $<1$ & - \\ 
\hline
$\eta$ reweighting & 6 & 4 & 4 & 7 & 5 & - \\ 
\hline
$W$ shape & - & - & $\pm$2 & $\pm$1 & - & - \\ 
\hline
$Wjj$ norm & - & - & $\pm$1 & - & - & - \\ 
\hline
$Wc,cc,bb$ norm & - & - & - & $\pm$5 & - & - \\ 
\hline
$W$ 3 jet norm & - & - & $\pm$25 & $\pm$25 & - & - \\ 
\hline
Multijets & - & - & $\pm$6 & $\pm$15 & - & $\pm$50 \\ 
\hline
\ttbar~XS & - & $\pm$7 & $\pm$3 & $\pm$14 & - & - \\ 
\hline
single-top XS & - & $<1$ & $\pm$2 & $\pm$38 & - & - \\ 
\hline
Z+jets XS & - & - & $\pm$7 & $\pm$3 & $\pm$40 & - \\ 
\hline
Diboson XS & - & - & $<1$ & $<1$ & $\pm$2 & - \\ 
\hline
MC Statistics & $\pm$7 & $\pm$6 & $\pm$70 & $\pm$23 & $\pm$67 & $\pm$48 \\ 
\hline

   \end{tabular}
\caption{Percent systematic uncertainties by process for the 3 jet plus channel.  Here, XS means cross-section, $Z$ means $Z$+jets, and Dib. means diboson.    Norm refers to normalization, s indicates single-top $s$-channel.  If two values are given, the top value is the upshift and the bottom value is the downshift.}
\label{tab:uncertainty-exp3}
\end{center}
\end{table}

\begin{table}[htdp]
\begin{center}
   \begin{tabular}{l|cccccc}
    \hline
Uncertainties(\%) & $t$-channel & \ttbar, Wt, s & $W$+light  &$W$+heavy & $Z$, Dib. & Multijets \\ 
\hline
Jet energy scale  & 7 & -6 & -39 & 73 & $<1$ & - \\ 
 & -21 & 14 & -26 & 24 & 55 & - \\ 
\hline
Jet energy resolution & $\pm$1 & $\pm$6 & - & - & $\pm$12 & - \\ 
\hline
Jet reconstruction & $<1$ & $\pm$1 & $\pm$2 & $\pm$1 & $\pm$10 & - \\ 
\hline
$b$ tagging scale factor  & 9 & 6 & 6 & -3 & 18 & - \\ 
 & -9 & -8 & -7 & 3 & -18 & - \\ 
\hline
Mistag scale factor  & $<1$ & $<1$ & 16 & -4 & $<1$ & - \\ 
 & $<1$ & $<1$ & -16 & 4 & $<1$ & - \\ 
\hline
Lepton scale factor & $\pm$3 & $\pm$3 & - & - & $\pm$98 & - \\ 
\hline
Lepton efficiencies & $\pm$3 & $\pm$2 & - & - & $\pm$17 & - \\ 
\hline
Generator single-top & $\pm$7 & $<1$ & - & - & - & - \\ 
\hline
Generator \ttbar & - & $\pm$30 & - & - & - & - \\ 
\hline
Shower & $\pm$10 & $\pm$50 & - & - & - & - \\ 
\hline
ISR/FSR  & 29 & 40 & - & - & - & - \\ 
 & 30 & 22 & - & - & - & - \\ 
\hline
PDF & $\pm$3 & $\pm$8 & - & - & $<1$ & - \\ 
\hline
Luminosity & $\pm$4 & $\pm$4 & - & - & $\pm$4 & - \\ 
\hline
\met  & -4 & $<1$ & $<1$ & $<1$ & $<1$ & - \\ 
 & -4 & $<1$ & $<1$ & $<1$ & $<1$ & - \\ 
\hline
\met~pileup  & -4 & -1 & $<1$ & -1 & $<1$ & - \\ 
 & -3 & -1 & $<1$ & $<1$ & $<1$ & - \\ 
\hline
LAr  & $<1$ & 1 & $<1$ & $<1$ & $<1$ & - \\ 
 & -1 & -2 & $<1$ & $<1$ & $<1$ & - \\ 
\hline
$\eta$ reweighting & 5 & 5 & 6 & 4 & 5 & - \\ 
\hline
$W$ shape & - & - & $\pm$3 & $\pm$1 & - & - \\ 
\hline
$Wjj$ norm & - & - & $\pm$1 & - & - & - \\ 
\hline
$Wc,cc,bb$ norm & - & - & - & $\pm$5 & - & - \\ 
\hline
$W$ 3 jet norm & - & - & $\pm$25 & $\pm$25 & - & - \\ 
\hline
Multijets & - & - & $\pm$6 & $\pm$12 & - & $\pm$50 \\ 
\hline
\ttbar~XS & - & $\pm$6 & $\pm$3 & $\pm$10 & - & - \\ 
\hline
single-top XS & - & $<1$ & $\pm$2 & $\pm$26 & - & - \\ 
\hline
Z+jets XS & - & - & $\pm$7 & $\pm$3 & $\pm$60 & - \\ 
\hline
Diboson XS & - & - & $<1$ & $<1$ & $<1$ & - \\ 
\hline
MC Statistics & $\pm$11 & $\pm$6 & $\pm$66 & $\pm$35 & $\pm$98 & $\pm$39 \\ 
\hline

   \end{tabular}
\caption{Percent systematic uncertainties by process for the 3 jet minus channel.  In this table, XS means cross-section, $Z$ means $Z$+jets, and  Dib. means diboson.   Norm refers to normalization, s indicates single-top $s$-channel.  If two values are given, the top value is the upshift and the bottom value is the downshift. }
\label{tab:uncertainty-exp4}
\end{center}
\end{table}

\subsection{Effect of Pileup}
%mention num track requirement for Nvertices?
For this study, there are on average about 6 interactions per crossing (primary vertices), and it is possible that extra events could cause problems at  the reconstruction level when identifying the primary vertex or reconstructing jets.  To determine the impact of pileup on this analysis, the MC was divided into two samples based on the number of primary vertices in the event, where high pileup is considered to be $\geq 6$ primary vertices and low pileup is considered to be $< 6$ primary vertices.  The sample is divided before any selections and then normalized to the expected yields in both cases.  The analysis is repeated using each sample, and we find that the cross-section shifts by 6\% versus nominal when using the high pileup sample and 4\% versus nominal when using the low pileup sample.  This is within the statistical uncertainty of the analysis and also within the MC statistical uncertainty, which increases when the sample is halved.  Based on this study, we consider the analysis to be insensitive to pileup effects.

\section{Results}
In this section we discuss the technique used to determine the observed cross-section and the $|V_{tb}|$ value.  We discuss five different results, involving different combinations of the four channels considered based on the number of jets and lepton charge: 2 jets with a positively charged lepton, 2 jets with a negatively charged lepton, 3 jets with a positively charged lepton, 3 jets with a negatively charged lepton.  These combinations are 2 jets, 3 jets, plus (positively charged lepton), minus (negatively charged lepton), and all four channels combined.  The measurement from the combination of the four channels leads to the primary analysis result.

\subsection{Cross-section Calculation and Measurements}\label{sec:xs}
As mentioned earlier, the cross-section is related to the number of observed events.  However, multiple analysis channels and a variety of uncertainties make the calculation more complicated that simply subtracting the expected background yield from the data and finding the deviation of this value from the expected signal yield.  The cross-section calculation is performed using a statistical tool called BILL (Binned Log Likelihood Fitter)~\cite{Wolfgang}, used previously for a neural network single-top analysis~\cite{ATLAS-CONF-2011-101}.  

The cross-section is determined via a maximum likelihood fit of the MC to the data, allowing different yields to float by different amounts within a range related to a Gaussian constraint term.  Scale factors ($\beta$) are determined for each process, where these scale factors are the ones that give the best fit to the data, for all channels considered.  The data-based $W$+jets and multijet estimates are not allowed to vary at all, while the other non-signal processes may float within their theoretical cross-section uncertainties.  The signal yield has no restrictions.  The fit is based on a product of Poisson likelihoods for each channel which is multiplied by the product of the Gaussian constraints for all the backgrounds.  The Gaussian distributions account for our prior knowledge of the backgrounds, and have a mean of 1 and a width of the theoretical uncertainty variation (~0 for data-based backgrounds, the theoretical uncertainty for other backgrounds).  

Because this analysis is a cut-and-count type of analysis, each channel has a distribution which is just one bin, each measurement uses 2 or 4 channels, and the fit itself is very straight-forward.  The results of the fit are given in Table~\ref{tab:beta}, where these values are scale factors to be multiplied onto the MC to get the observed yield.  These factors are the output from the BILL tool.  Because the data-based backgrounds have a $\beta$ value defined to be 1, and the other backgrounds have low theoretical uncertainties, the only $\beta$ values that are not approximately 1 are those for the signal.  The t-channel factor is multiplied by the expected cross-section to obtain the observed cross-section.

\begin{table}[htdp]
\begin{center}
   \begin{tabular}{l|cccccc}
    \hline
Channels & $t$-channel & \ttbar, Other top & $W$+light  &$W$+heavy & $Z$+jets, Diboson & Multijets \\
    \hline
All Channels & 1.41843 & 0.99361 & 1.000000 & 1.000000 & 1.00834 & 1.00000 \\
2 Jets & 1.55434 & 0.99848 & 1.000000 & 1.000000 & 0.99740 & 1.00000 \\
3 Jets & 1.05444 & 1.00790 & 1.000000 & 1.000000 & 1.00147 & 1.00000 \\
Plus Charge & 1.40058 & 0.99168 & 1.000000 & 1.000000 & 1.00654 & 1.00000 \\
Minus Charge & 1.46533 & 1.00001 & 1.000000 & 1.000000 & 1.00005 & 1.00000 \\
   \hline
   \end{tabular}
\caption{The fit values by process and channel.  The 2 or 3 jet channels include both lepton charges, and the lepton charge channels include both 2 and 3 jet events.  All channels is the combination of plus and minus lepton charge events, with 2 or 3 jets.}
\label{tab:beta}
\end{center}
\end{table}

This tool uses a frequentist method to determine the cross-section uncertainties, meaning many (100,000) different pseudo-experiments are generated based on the input yield and uncertainties.  In this way, all the various possibilities within uncertainties are explored and a distribution reflecting the probability of all possible outcomes is created, where the RMS reflects the overall combined uncertainty of the measurement.  The number of events in each pseudo-experiment are determined via a Poisson distribution with a mean of the expected yield and the uncertainties varied by Gaussian distributed random numbers.  There is also a factor related specifically to the theoretical uncertainties of the backgrounds, as was the case for the fit to determine the $\beta$ values, but again this has a small impact on the result.

The results of all of these repetitions are displayed in a distribution like in Figure~\ref{fig:betafit}, where the total cross-section uncertainty is derived from the mean and the RMS of the distribution.  For the observed uncertainty, the yields are scaled by the fit values, so $\beta$ is now 1.  The deviation of the mean from 1 is the bias (representing the asymmetry of the uncertainties), and this added in quadrature with the RMS gives one side of the uncertainty, while the RMS alone (0 bias assumption) gives the other uncertainty shift.  In other words, the uncertainty is the $\sqrt{RMS^2}$ or $\sqrt{ (1- mean)^2 + RMS^2 }$.  For the example in Figure~\ref{fig:betafit}, which uses observed yields for all four channels and includes all of the uncertainties, the RMS is 0.284 and the mean is 1.133, giving uncertainties of +31\% and -28\%.

 \begin{figure}[!h!tpb]
 \centering
 \includegraphics[width=0.75\textwidth]{figures/variables/PaperFinal_Beta_allsysstat_allchan_atlaswip.eps}
 \caption{Pseudo-experiment distribution used for the final cross-section uncertainty determination.  This distribution is for the observed cross-section uncertainty, for all channels combined.  The $\beta$ value is the fit for a given pseudo-experiment with yields scaled by the values in Table~\ref{tab:beta}, and the uncertainty is determined from the distribution RMS and deviation of the mean from 1.}
 \label{fig:betafit}
 \end{figure}


\subsubsection{Two and Three jet Single Top Quark t-channel Production}
We can combine our four channels into sets of 2 jets and 3 jets (lepton charges are combined).  When this is done we find a cross-section of  $\sigma_{t}= 100 ^{+9}_{-9} \mathrm{(stat)} ^{+32}_{-31} \mathrm{(syst)} = 100^{+33}_{-32}$~pb for 2 jets, where the expected cross-section is $\sigma_{t}= 65^{+23}_{-23}$~pb, and $\sigma_{t}= 68 ^{+13}_{-13} \mathrm{(stat)} ^{+28}_{-22} \mathrm{(syst)} = 68^{+30}_{-25}$~pb for 3 jets, where the expected cross-section is $\sigma_{t}= 65^{+30}_{-24}$~pb.  Both results are consistent with the standard model value within two standard deviations and consistent with each other within uncertainties.
%CHECK THIS LAST STATEMENT

\subsubsection{Positively and Negatively Charged Single Top Quark t-channel Production}
One can also combine the four channels into a positive and negative lepton charge sample.  Because the top quark decays to a W and b (and then the W decays to a lepton on neutrino) without hadronizing, the charge information from the top quark is preserved in the lepton.  Therefore, the positively charged lepton channel measurement corresponds to a measurement of the positively charged top quark portion of the t-channel single-top cross-section.  There is a separate theoretical prediction for the top and anti-top portions of the cross-section, given in Section~\ref{sec:ExpectedCrosssections}.  The results of this measurement are $\sigma_{t^{+}}= 59 ^{+6}_{-6} \mathrm{(stat)} ^{+17}_{-16} \mathrm{(syst)} = 59^{+18}_{-16}$~pb for top (positive lepton charge), where the expected cross-section is $\sigma_{t^{+}}= 42^{+14}_{-13}$~pb.  The measurement is $\sigma_{t^{-}}= 33 ^{+5}_{-5} \mathrm{(stat)} ^{+12}_{-11} \mathrm{(syst)} = 33^{+13}_{-12}$~pb for anti-top (negative lepton charge), where the expected cross-section is $\sigma_{t^{-}}= 23^{+10}_{-10}$~pb.

\subsubsection{Combined t-channel Production Cross-section Result}\label{sec:fourchanresult}
Finally, all four channels can be combined, and this is the final reported total cross-section result for this study.  The observed t-channel single-top cross-section is $\sigma_{t}= 92 ^{+7}_{-7} \mathrm{(stat)} ^{+28}_{-25} \mathrm{(syst)} = 92^{+29}_{-26}$~pb, where $\sigma_{t}= 65^{+22}_{-20}$~pb is expected.  This is consistent with the standard model and within two standard deviations of the theoretical single-top t-channel cross-section.  
%Maybe include a plot or table of how this matches up with other measurements from other experiments, the paper?

Table~\ref{tab:xs-uncertainty-exp} shows a breakdown of the systematic uncertainties and their contribution to the expected cross-section measurement for the combination of all four channels and Table~\ref{tab:xs-uncertainty-obs} shows the same but for the observed data.  The data statistical uncertainty is much lower than the systematic uncertainties, meaning that this cross-section measurement is dominated by systematic uncertainties.  The largest uncertainties for this analysis are ISR/FSR, shower/generator, and b-tagging uncertainties.  

The ISR/FSR uncertainty may decrease in future analyses as this is studied further and the level of variation required is better understood.  The b-tagging uncertainty will also likely improve in future analyses as more data are collected and the b-tagging efficiencies and scale factors are better estimated.  The shower/generator uncertainty is unlikely to change very much until shower/generator programs are updated.  On the other hand, the MC statistical uncertainty will become more of an issue in future analyses.  As the data statistics increase, the number of MC events that must be generated increases.  This means that the MC statistical uncertainty will increase in future analyses unless they are altered to use looser selections or faster MC generation methods.

\begin{table}[htdp]
\begin{center}
   \begin{tabular}{l|c}
    \hline
%    Source & \multicolumn{2}{c}{$\Delta\sigma/\sigma$ [\] }\\
%           & cut-based (2jet)  & cut-based (3jet)  \\
    Source & $\Delta\sigma/\sigma$ (\%) \\
    \hline \hline
    Data statistics              & +10/-10  \\
    MC statistics                & +6/-6  \\
    \hline  
    $b$ tagging scale factor     & +13/-13  \\
    Mistag scale factor          & +1/-1  \\
    Lepton scale factor          & +3/-3  \\
    Lepton efficiencies          & +1/-1  \\
    Jet energy scale             & +2/-3  \\
    Jet energy resolution        & +2/-2  \\ %or 3 3
    Jet reconstruction           & +1/-1  \\
    $W$ shape                    & +1/-1  \\
    $Wjj$ normalization          & +1/-1   \\
    $Wc,cc,bb$ normalization     & +2/-2   \\
    $W$ 3 jet normalization      & +2/-2  \\
    $\eta$ reweighting           & +8/-5  \\
    \met                         & +1/-2  \\
    \met~pileup                  & +1/-2  \\
    LAr                          & +1/-1  \\
    PDF                          & +5/-5  \\
%    Generator, $t\bar{t}$        & +3/-3  \\
%    Generator, single-top        & +7/-7  \\
    Generator                    & +8/-8  \\
    Shower                       & +12/-11  \\ %or 12 12
    ISR/FSR                      & +21/-19  \\
    Theory cross-section         & +7/-7  \\
%aka QCD
    Multijets                    & +3/-3  \\
    Luminosity                   & +5/-5  \\
    \hline  
    Total Systematics            & +33/-29  \\
    Total                        & +34/-31  \\
    \hline
   \end{tabular}
\caption{Systematic uncertainties for the expected $t$-channel cross-section measurement, where the final line includes all systematic uncertainties and the data statistical uncertainty.}
\label{tab:xs-uncertainty-exp}
\end{center}
\end{table}


\begin{table}[htdp]
\begin{center}
   \begin{tabular}{l|c}
    \hline
%    Source & \multicolumn{2}{c}{$\Delta\sigma/\sigma$ [\] }\\
%           & cut-based (2jet)  & cut-based (3jet)  \\
    Source & $\Delta\sigma/\sigma$ (\%) \\
    \hline \hline
    Data statistics              & +8/-8  \\
    MC statistics                & +4/-4  \\
    \hline  
    $b$ tagging scale factor     & +12/-12  \\
    Mistag scale factor          & +1/-1  \\
    Lepton scale factor          & +3/-3  \\
    Lepton efficiencies          & +2/-2  \\
    Jet energy scale             & +2/-3  \\
    Jet energy resolution        & +2/-2  \\
    Jet reconstruction           & +1/-1  \\
    $W$ shape                    & +1/-1  \\
    $Wjj$ normalization          & +1/-1   \\
    $Wc,cc,bb$ normalization     & +2/-2   \\
    $W$ 3 jet normalization      & +2/-2  \\
    $\eta$ reweighting           & +7/-5  \\
    \met                         & +1/-2  \\
    \met~pileup                  & +1/-1  \\
    LAr                          & +1/-1  \\
    PDF                          & +4/-4  \\
%    Generator, $t\bar{t}$        & +3/-3  \\
%    Generator, single-top        & +7/-7  \\
    Generator                    & +7/-7  \\
    Shower                       & +11/-11  \\
    ISR/FSR                      & +19/-18  \\
    Theory cross-section         & +5/-5  \\
%aka QCD
    Multijets                    & +2/-2  \\
    Luminosity                   & +4/-4  \\
    \hline  
    Total Systematics            & +30/-27  \\
    Total                        & +31/-28  \\
    \hline
   \end{tabular}
\caption{Systematic uncertainties for the observed $t$-channel cross-section measurement, where the final line includes all systematic uncertainties and the data statistical uncertainty.}
\label{tab:xs-uncertainty-obs}
\end{center}
\end{table}


\subsection{Estimate of $|V_{L}|$}
As discussed in Section~\ref{sec:vtb}, the CKM matrix element $|V_{L}|$ can be directly estimated from t-channel single-top production using the ratio of the observed and standard model cross-section.  We may write (based on Equation~\ref{eq:vtbxs}), where $\sigma$ is the cross-section, obs refers to the observed value and sm refers to the standard model:
\begin{equation} |V_{L, obs}|~ = ~\sqrt{~\frac{~\sigma_{obs}}{~\sigma_{sm}}} ~|V_{L, sm}|
\end{equation}
or, with $|V_{L, sm}| = |V_{tb}| = 1$ from the standard model we obtain,
\begin{equation} |V_{L, obs}|~ = ~\sqrt{~\frac{~\sigma_{obs}}{~\sigma_{sm}}}
\end{equation}
Performing the calculation to propagate the uncertainties gives
\begin{equation} \delta_{VL,obs} ~= ~\frac{V_{L, obs}}{2}~\sqrt{(\frac{~\delta_{obs}}{\sigma_{obs}})^2~+~(\frac{~\delta_{sm}}{\sigma_{sm}})^2}
\end{equation}
where $\delta$ refers to the uncertainty.  Thus, we obtain a value of $|V_{L, obs}|~ =~1.19^{+0.20}_{-0.18}$ for the main, four channel combination.  In this case we used 10\% for the theoretical cross-section uncertainty, as was done for the other single-top processes during the cross-section determination.  This result is consistent with the standard model value of 1.0 (and thus being simply the standard model $|V_{tb}|$) within two standard deviations.

It is also possible to determine a lower 95\% confidence level limit on the $|V_{tb}|$ value, assuming a standard model upper value of 1.  We form a Gaussian with a mean of 1.42 and use the uncertainty given for the combined result.  We integrate from 1 towards 0, taking the limit to be the point where 95\% of the curve has been integrated. With this standard model assumption, we find $|V_{tb}|> 0.67$ observed.

\subsection{Comment on Significance}
It is fairly straightforward to determine a significance using a frequentist tool like BILL.  One simply determines the likelihood of a background-only hypothesis fluctuating to imitate the signal hypothesis by generating many, many pseudo-experiments and seeing how likely this is to happen.  Pseudo-experiments are generated as described in Section~\ref{sec:xs}.  The calculation is done by determining the value $-2ln(Q)$, the test statistic (also known as the log-likelihood ratio or LLR).  A fit is done for a given ensemble to determine how likely it is that the ensemble satisfies the background only ($H_{B}$) or signal plus background ($H_{SB}$) hypotheses.  The ratio of the probabilities is the Q in the LLR value:
 \begin{equation} Q = \frac{p(H_{SB})}{p(H_{B})}
\end{equation}

The significance is determined by seeing how many background-only pseudo-experiments have a LLR value that is greater than the mean of the pseudo-experiments which assume a signal plus background (standard model) hypothesis.  This is done by finding the mean of the signal plus background LLR distribution and finding how many background only ensembles have LLR values above this mean, compared to the total number of background only ensembles.  In this way, the probability of the background fluctuating to look like the signal, and thus the significance, can be determined.

The difficulty with this method is that around and especially above the 5 sigma significance level (the level at which observation is typically claimed in high energy physics), the number of pseudo-experiments needed can become very, very large (10 million to 100 million or more).  For this dissertation, the result has been shown to be above 5 sigma previously~\cite{ATLAS-CONF-2011-088, ATLAS-CONF-2011-101}  with less data and larger systematic uncertainties, so we do not repeat this for the main result.  We do demonstrate the individual expected results for two channels below 5 sigma, the 3 jet and negative charge channels.  In high energy physics, discovery requires a 5 sigma significance, or a p-value (probability) of the background fluctuating to look like signal of 0.0000006.  Evidence requires 3 sigma significance, or a p-value of 0.003.  

The LLR distributions for the three jet and negative charge channels are shown in Figure~\ref{fig:significance}.  About 800000 pseudo-data sets were created in each case.  For the three jet channel, the expected significance is 3.8 sigma with a p-value of background fluctuating to look like a standard model signal of $8\times 10^{-5}$.  For the minus charge channel, the expected significance is 4.1 sigma with a p-value of background fluctuating to look like a standard model signal of $2\times 10^{-5}$.

 \begin{figure}[!h!tpb]
 \centering
 \includegraphics[width=0.49\textwidth]{figures/result/pvalue_3jetsnew3.eps}
 \includegraphics[width=0.49\textwidth]{figures/result/pvalue_minusnew3.eps}
 \includegraphics[width=0.49\textwidth]{figures/result/SigLegend.eps}
 \caption{Distribution used to determine the expected significance for the 3 jet channel (all lepton charges are allowed) and negative charge channel (2 and 3 jets allowed).  The two curves are ensembles with and without the assumption of a standard model signal.  The vertical line shows the mean of the standard model signal and background distribution.}
 \label{fig:significance}
 \end{figure}




\chapter{Conclusion}
We have analyzed \LUMI\ of data collected with the ATLAS detector. In our search for the \Wtchan\ we have seen a statistically significant excess of 3.3$\sigma$. This is sufficient to claim evidence, and although this does not meet the $>5\sigma$ criteria to claim observation, it is a significant step to verifying the Standard Model prediction. The estimated cross-section is also extracted from the data, giving a result of $\sigma(pp\rightarrow Wt + X) = 16.8 ^{+2.9}_{-2.9} \mathrm{(stat)} ^{+4.9}_{-4.9} \mathrm{(syst)}~pb$. This analysis also allowed us to make measurements of other Standard Model parameters. The CKM matrix element $V_{tb}$ is measured to be $|V_{tb}| = 1.03^{+0.16}_{-0.19}$. The width of the top quark is measured at $\Gamma_{t}^{obs.} = 1.4\pm 0.5~\rm GeV$ (Note the increase in the percent uncertainty due to the $|V_{tb}|^2$ dependence), giving a lifetime of $\tau_{t}=(4.7^{+1.2}_{-1.2})\times 10^{-25}~s$. These measurements are all consistent with theoretical Standard Model predictions and other experimental measurements. This analysis is published in Physics Letters B~\cite{WTEVIDENCE}.

In this analysis I implemented the BDT used, which includes the variable selection and testing, the training procedure, and the parameter optimization. I implemented the ATLAS and top group recommendations for the object definitions, event selection, and studied most of the systematics (the jet energy scale, jet reconstruction, jet ID, lepton ID, lepton resolution, \MET, and pile-up uncertainties). The data-driven \Ztt\ normalization is estimated by me. I prepared the plots of the BDT and plots of the variables used. During the preparation of the paper and the associated note, I gave many single top working group talks and the approval talk to the top working group. I also collaborated with Huaqiao Zhang to perform many cross-checks while going through review.

With time the systematic uncertainties will be better understood and in the future this analysis will be repeated with more data. However, there is ample room for improvement in the analysis procedure itself. Note that the BDT optimization is done using only the nominal \MC. A look at the uncertainty composition of the final cross-section measurement will reveal that this analysis is quite systematically limited. A BDT optimization using information from the systematically shifted datasets could bring significant improvement to the result as a whole. This is not a trivial undertaking, as the existing toolsets are not equipped to do this kind of optimization out of the box, however implementing a systematics-sensitive optimization has the potential to greatly increase the significance.

This evidence for the existence of the \Wtchan\ was also confirmed independently by the CMS collaboration~\cite{CMSEVIDENCE}. Both the CMS and ATLAS collaborations will continue to update these analyses with better analysis techniques, a better understanding of the systematic uncertainties, and more data. The discovery of the \Wtchan\ is not the end, of course. Precision measurements of $V_{tb}$ and the top quark properties and searches for new physics in the \Wtchan\ signal region are all exciting new analyses waiting to be explored.

The LHC era is already showing its promise, giving exciting results like the recent Higgs discovery~\cite{HiggsATLAS,HiggsCMS} and confirming the predictions of the Standard Model. Even with the Higgs boson discovered, there remains much discovery ahead. The LHC will be running for years, pushing our understanding forward. With each collision we strive for a better understanding of our universe, and with time and hard work, these efforts will be rewarded.


\bibliographystyle{atlasnote}
\bibliography{bibmain}

\newpage
\appendix
\part*{Appendices}
\addcontentsline{toc}{part}{Appendices}
%\clearpage
\chapter{Data Based Cross-check of \ttbar~ Background}\label{app:ttbar}
In the main text we have discussed the data-based estimation of the multijets and W+jets background processes.  However, there is another large background, \ttbar, where the theoretical cross-section was used for the normalization.  It is also possible to do a data-based estimate of this background.  In this section, we review one straight-forward way to do this estimate.  The \ttbar~estimate discussed here is not used in the analysis described in the main text, but instead is intended as a cross-check of the value used (1.0) and its uncertainty.

To determine the \ttbar~background, we define orthogonal off-signal regions (as we did for the W+jets estimate).  All preselection requirements except for the number of jets and b-tagged jets selections are applied.  We also require the total t-channel yield to be $\rm <6\%$ of the total, and apply as few selections as possible beyond the preselection (with different numbers of jets and b-tagged jets).  Two \ttbar~dominated selections are defined as the number of jets equal to four or more with 1 b-tagged jet, and the 2 b-tagged jets selection with at least 2 jets.  Both of these regions are also discussed as potential signal regions in Section~\ref{alternativechannels}, so if this is done, the regions used for signal determination would need to be removed, just as the 2 jet signal kinematic region was removed from the W+jets estimate.  Additionally, the 3 jet region is considered but this region has a relatively large (~8\%) portion of t-channel single-top events (hence its use as a signal channel) relative to the other \ttbar~determination regions.  In this case, an additional selection must be introduced to control the amount of t-channel and to exclude the signal region.  Here, we choose to require the reconstructed top mass to be $> 210$ GeV.  This removes single-top events, which are more likely to have the b-quark, lepton, and missing energy associated with a top quark all correctly identified and thus an invariant mass of these particles closer to the top quark mass value.

We also select channels based on the event having a muon or electron as the selected lepton.  Thus we have 6 different channels in total, which are each of the following selections with a muon selection or electron selection:

\begin{list} {*} {}
\item 2 b-tagged jets, at least 2 jets
\item 1 b-tagged jet, at least 4 jets
\item 1 b-tagged jet, exactly 3  jets, $M_{top}(\rm l\nu b)>210\GeV$
\end{list}

These channels are all orthogonal to each other.  We can consider each result from these six channels as a separate experiment and combine them.  First, we calculate a scale factor.  This is defined as:
\begin{equation} \rm SF_{\ttbar} = \frac{Data - MC(not~\ttbar)}{MC(\ttbar)} 
\end{equation}
If the data and the Monte Carlo were to agree exactly, then the \ttbar~SF woult be 1.0.  To find the combined statistical uncertainty for the channels, we follow the method discussed by Lyons~\cite{Lyons, Lyonspaper}.  The statistical uncertainty is written as $\sigma$, where the square of this value is the variance ($\sigma^2$).  This $\sigma$ should not be confused with the cross-section.  The combination is found as follows, using i for the different channels:
\begin{equation} \rm \frac{1}{\sigma_{tot}^2} = \sum \frac{1}{\sigma_i^2} 
\label{eq:uncert}
\end{equation}

The scale factors themselves are also combined for the different channels as discussed by Lyons.  We weight each channel by the inverse of the statistical uncertainty squared.  The combination of scale factors is done as follows, where SF is the scale factor.  Notice the denominator is just $\rm \frac{1}{\sigma_{tot}}$ from Equation~\ref{eq:uncert}:
\begin{equation} \rm SF_{tot} = \frac{\sum \frac{SF_i}{\sigma_i^2}}{\sum \frac{1}{\sigma_i^2}}
\label{eq:ttbarSF}
\end{equation}

To determine the systematic uncertainties for each channel, the SF is estimated using MC values shifted due to a given uncertainty.  For a given systematic uncertainty scenario, the systematic-shifted SF are combined for each of the channels as in the nominal sample, again using the same statistical uncertainty as a weight as in Equation~\ref{eq:ttbarSF}.  The deviation between the combined nominal SF and the combined uncertainty-shifted SF is then the uncertainty for the combination due to the systematic in question.  By determining the combined systematic uncertainties this way, correlations between the channels can be properly included.  Finally, all of the systematic uncertainties and the data statistical uncertainty are added in quadrature to obtain the overall uncertainty.  The systematic uncertainties which are considered are the b-tagging scale factor, mis-tagging scale factor, and jet energy scale.  Other systematic uncertainties are neglected for the purposes of this cross-check.  

%This method assumes that the uncertainties between channels are uncorrelated.  This is true for the data statistical uncertainty, but uncertainties like the jet energy scale and b-tagging scale factor uncertainty are correlated between the channels.  However, the purpose of this cross-check is to show the \ttbar~estimate and uncertainty used in the analysis is consistent with what we see in data, and this only requires us to show that the scale factor is consistent with one, with uncertainties larger than those in the analysis.  In this analysis, the deviations for a given uncertainty have the same sign (or nearly so) between channels, with minor exceptions, implying that including correlation effects will not reduce the overall uncertainty significantly (and would likely increase it).  In general, it is possible correlations between channels could reduce an uncertainty compared to its value without correlation effects, but this occurs when deviations in different channels have an opposite sign.  This is related to the total deviation of completely correlated uncertainties being the sum of the individual deviations, rather than being the individual deviations added in quadrature.

The multijet estimate for these channels is done using the values from Section~\ref{sec:multijets} and multiplying by the fraction of multijets in the 3 jet bin and the desired bin to get the multijet total for the desired bin.  The multijet estimate from the 3 jet bin is thus propogated into the other regions considered for the \ttbar~scale factor estimate by assuming the multijets number of jets distribution is correct.  For the 3 jet \ttbar~region, which makes a selection on the reconstructed top mass, this same selection is simply applied to the 3 jets multijet sample.  No new fits to the \met~distribution are performed for the \ttbar~scale factor estimtae.  These values are approximate and the uncertainty related to propagating the multijet yield into different bins is neglected for this cross-check.  The values are given in Table~\ref{tab:QCDttbar}.
\begin{table}[!h]
  \begin{center}
    \begin{tabular}{lcc}
      \hline
      \hline
      & \multicolumn{2}{c}{Tagged events} \\
      Selection & e channel & $\mu$ channel \\
      \hline
      2 b-tag jets, $\ge$ 2 jets & $ 36.6 $ & $ 18.7 $ \\
      1 b-tag jet, $\ge$ 4 jets &  $ 302.6 $ & $ 144.3 $ \\
      1 b-tag jet, 3  jets, $M_{top}(\rm l\nu b)>210\GeV$& $ 78.4 $ & $ 35.7 $ \\
      \hline
      \hline
    \end{tabular}
    \caption{\label{tab:QCDttbar} Estimate of multijet yields for different selections, separated by lepton type.}
  \end{center}
\end{table}

The W+jets estimate for these channels uses the data-based 3 jet bin heavy flavor fractions from Section~\ref{sec:wjets} for the 1 b-tagged jet, exactly 3  jets, $M_{top}(\rm l\nu b)>210\GeV$ channel.  This is because there is a selection used for this channel which impacts processes differently.  For the other two channels, we use the W+jets data based normalizations appropriate for the number of jets in question, including a normalization of 0.75 for the 4 jets channel.  The heavy flavor fractions do not have an impact on the scale factor determination for these two channels, as there are no selections beyond preselection applied.

The scale factors and their uncertainties for each of the six channels plus the combination of channels by lepton type, and then all channels, can be seen in Figure~\ref{fig:ttbarsf}.  The statistical uncertainties are quite small, and if they are not smaller than the marker size, they are given by colored portions of the vertical uncertainty line.  Again, only the b-tagging scale factor, mis-tagging scale factor, jet energy scale and data statistical uncertainties are included in this cross-check.  The electron and muon channel combinations give very similar values, $SF_{e} = 1.10\pm0.02 \mathrm{(stat)} ^{+0.23}_{-0.14} \mathrm{(syst)} = 1.10^{+0.23}_{-0.14}$ and $SF_{\mu} = 1.13\pm0.02 \mathrm{(stat)} ^{+0.23}_{-0.15} \mathrm{(syst)} = 1.13^{+0.23}_{-0.15}$ respectively.  The final result from the combination of all six channels is $SF = 1.12\pm0.01 \mathrm{(stat)} ^{+0.23}_{-0.15} \mathrm{(syst)} = 1.12^{+0.23}_{-0.15}$.  This is consistent with a scale factor of 1.0 and the uncertainty on the result is larger than the \ttbar~theoretical cross-section uncertainty used in the main text (approximately 10\%, see Section~\ref{Sys}).  From this cross-check, it is clear that the \ttbar~estimate used in the t-channel single-top analysis is consistent with the data.

Future analyses will want to include additional uncertainties such as ISR/FSR, shower and generator uncertainties, as well as others used in the single-top t-channel cross-section estimate.  However, including these uncertainties in the current study would not change the conclusion.  As more data is taken, the uncertainties will likely become better understood and have lower values than in this study.  The data statistical uncertainty is already quite low in this study, but the b-tagging scale factor uncertainty in particular is 14\% for the six channel combination, the dominant systematic uncertainty.  If this uncertainty could be reduced, it would have a significant impact on the precision of this scale factor estimate.

 \begin{figure}[!h!tpb]
 \centering
 \includegraphics[width=1.00\textwidth]{figures/appendix/ttbarSF.eps}
 \includegraphics[width=0.50\textwidth]{figures/appendix/ttbarsfLegend.eps}
 \caption{Scale factors for \ttbar~production using six separate channels, the combination of electron channels, the combination of muon channels, and the combination of all six channels.  Statistical uncertainties are given by colored portions of the black lines unless the statistical uncertainties are so small as to be covered by the marker itself.  The black line shows the data statistical, b-tagging scale factor, mis-tagging scale factor, and jet energy scale uncertainties combined.  Other uncertainties are neglected in this cross-check.}
 \label{fig:ttbarsf}
 \end{figure}

\chapter{Multivariate Analysis}\label{app:MultivariateApproach}
Although this document has focused on a simple, cut-based analysis approach, there are more complicated analysis options.  These include multivariate techniques such as boosted decision trees.  Multivariate techniques use computer algorithms to determine several dimensions of selection sequences, making use of events which both pass and fail individual selections.  The result is in an output related to the probability of an event being signal or background.  In this chapter we review the boosted decision tree technique and then suggest options for future analyses using the variables from the main document and an additional set of variables.  As this section is intended as suggestions for future work and another view point of the t-channel single-top analysis, only the large uncertainties from the cut-based analysis are considered.

%citation for SPR and the different classifier algorithms

\section{Boosted Decision Tree Overview}\label{app:bdtoverview}
The boosted decision tree (BDT)~\cite{BDT2} method has traditionally been used in single-top analyses~\cite{singletopdiscovery:D0, singletopdiscovery:CDF, CMStchannel} and this method is used here, as provided in a statistical package called StatPatternRecognition~\cite{SPR}.  A boosted decision tree is based on a collection of decision trees, and an example of one is pictured in Figure~\ref{fig:bdtex}.  A decision tree is a cut-flow diagram, where selections are applied that eventually result in sets of mostly signal or mostly background events.  These final sets of events are the leaves or terminal nodes of the tree (each selection has an associated node that is not necessarily terminal).  When a decision tree is applied to a new event, if this event passes background-like selections it is probably background, whereas if it passes signal-like selections it is probably signal.  The ultimate output of the multivariate classifier indicates how likely it is that an event is background or signal.  One can then take a simple cut on the classifier output, to select for high signal probability, or do a fitting technique to determine how much signal is present in the data.

 \begin{figure}[!h!tpb]
 \centering
 \includegraphics[width=0.60\textwidth]{figures/appendix/BDTXY.eps}%bdt2 has green background.  image looks ok in acrobat reader
 \caption{A pictoral representation of a single decision tree, where A and C are variables values, X and Y are selection thresholds.  The node is designated as S for signal and B for background.  The S and B circles are the final nodes, or leaves, in the tree.}
 \label{fig:bdtex}
 \end{figure}

The tree itself is formed using an optimization criterion to determine which variable to use for each selection (or node split) and what threshold to take.  The goal of such a criterion is to optimize the signal and background separation.  For this study we use the default optimization criterion, which is the Gini index.  This is the purity times 1 minus the purity, p(1-p).  The purity is the signal events divided by all of the events considered in that node, so if a node would have only signal events, the purity would be 1, giving a Gini index of 0.  If there are only background events, the purity is 0, and the Gini index is again 0.  The goal of the splitting is to obtain nodes that are background or signal dominated, so nodes are optimized to obtain a Gini index of 0 (or close to 0).  Unlike the cut-based analysis in Section~\ref{sec:optimization}, systematic uncertainties cannot be considered when determining each individual selection of a tree.

The ``boosted'' portion of the name boosted decision tree refers to an algorithm which reweights events based on whether or not they were mis-classified as signal and background in the previous tree.  These weights then affect how the performance is evaluated in the training of the next tree.  The boosting algorithm used in this study is called $\epsilon$-boost~\cite{SPR, BDT2}.  This particular algorithm increases the weights of incorrectly classified events by a factor of $e^{2\epsilon}$.  The $\epsilon$ value may be set to different values, but here we use the default value of 0.01.  In the end, the various trees are all averaged to give a final boosted tree.  The classifiers for each tree, the functions which are averaged, are a formed by minimizing what is called a quadratic loss criterion.  This is the average of the square of the difference between the true and predicted classifications for all events.  

\subsection{Classifier Formation and Parameter Optimization}
When forming a classifier, there are three different MC samples considered.  These are formed by taking the modulus of the event number and are called training, validation, and yield.  The yield sample is only used for the final analysis evaluation to ensure an unbiased result.  The training sample is used to form various classifiers and the validation sample is used to evaluate if some particular classifier is the one we desire.  The one exception to this sample division in this study is the multijets process. The statistics are quite low for this sample, so it is divided in half to form a training and yield sample, again using the modulus of the event number.  Additionally, because of limitations in the statistical package, negatively weighted events cannot be used during the training phase when the classifier is generated.  There are such events in most of the MC samples for the various top-quark processes.  However, the proportion of negatively weighted events is low (about 7\% in the training sample for the signal).  Even if this did have some effect, it would simply result in a less than optimally trained classifier, not a biased result, because the sample used for the result includes both negative and positively weighted events.

It is possible to train different combinations of channels: each channel separately, the number of jet channels separately (but with both lepton charges allowed), and all four channels combined.  For this study we use two BDT's for the final result, each with a different number of jets (2 or 3).  When samples are split in this way, the classifiers can take advantage of the different kinematics in each number of jets channel.  However, the MC statistics will be lower after splitting the sample, potentially causing the kinematics and events to be unevenly distributed.  A multivariate classifier can be particularly sensitive to this, especially if a tight cut on the classifier is taken, as we do here.  This is one reason why we do not further divide this sample in additional kinematic regions and combine several classifiers for the final result.

There are several different classifier settings to choose from when forming a BDT.  In this study, we vary the number of decision trees the BDT uses and the minimum number of events in the leaves for each tree.  We use the default settings for other parameters, including the type of boost ($\epsilon$-boost with $\epsilon$ of 0.01), the per event loss (quadratic), and the optimization criterion (Gini index), discussed at the beginning of Section~\ref{app:bdtoverview}.  The number of variables considered is another parameter of the BDT and we consider several different combinations of variables.

Many different trees are generated using the training sample with a variety of classifier settings.  We choose the trained BDT classifier and cut threshold for that classifier by using a criteria that includes systematic uncertainties, as we do in the main text in Section~\ref{sec:optimization}.  In this case, we found a few classifiers that had consistent distributions in the training and validation samples and were continuous. We then determine which of these would give the best expected result, based on the significance calculated using validation sample information.  Multijets are not considered during the significance calculation.

It is possible to overtrain a BDT during the generation of the classifier, which means it is too tuned to the particular MC sample's kinematics subtleties, like being trained on noise.  Overtraining results in a BDT that is sub-optimal, which we would like to avoid, but doesn't invalidate the analysis.  The $\chi^2$~\cite{Lyons} and Kolmogorov-Smirnov (KS)~\cite{KS} tests are used to check the training and validation sample agreements.  Classifiers are chosen which have good agreement ($>5\%$).  Additionally, because the validation sample is used to determine the BDT settings and threshold, it is also possible to be sensitive to this sample's particular distributions.  We save a yield sample for the cross-section calculation to ensure that such a sensitivity won't impact the final result.

\subsection{Cut-based Analysis Variables}\label{app:sbdt}
Because this dissertation focuses on a cut-based analysis, it is interesting to consider what would happen if we train BDTs using only the variables from the cut based analysis.  For this, we use all four variables considered in Section~\ref{sec:selectionchoices} for each number of jet channel, as well as lepton charge for a total of five variables in each channel.  The variables used for both channels are: sum of the transverse momenta of all jets, lepton, and \met; leading untagged jet $\eta$; top quark mass reconstructed using the b-tagged jet, lepton, and reconstructed neutrino; and lepton charge.  Additionally, $\Delta\eta$ between the b-tagged jet and leading untagged jet is used for the 2 jet selection and the invariant mass of all jets is used for the 3 jet selection.  

For the 2 jet selection, the classifier parameters and cut threshold are: 250 trees, 1500 events minimum per leaf, and 0.74 cut threshold.  For the 3 jet selection, these are: 150 trees, 1250 events minimum per leaf, and 0.41 cut threshold.  The BDT classifier distribution before and after the selection for each channel is shown in Figure~\ref{fig:Plot_SBDT}, normalized to the observed t-channel cross-section.  The variable distributions after this cut threshold for each channel are given in Figures~\ref{fig:Plot_2SBDT} and~\ref{fig:Plot_3SBDT} for the 2 and 3 jet selections respectively, also normalized to the observed t-channel cross-section.  Notice that after the selections, the kinematic regions chosen for the distributions look similar to those in Figures~\ref{fig:Plot_2TagCuts} and~\ref{fig:Plot_3TagCuts}, particularly the reconstructed top mass, leading untagged jet $\eta$ and the invariant mass of all of the jets.  Overall the agreement is fairly good between data and MC in these plots, keeping in mind the lower MC statistics from splitting the MC into thirds and also the somewhat large systematic uncertainties.

 \begin{figure}[!h!tpb]
 \centering
% \includegraphics[width=0.46\textwidth]{figures/appendix/bdtvar/BDTqcd_test_nocut_2jshort_NoSigScale_classiAdaBoost.log.data.eps}
%\includegraphics[width=0.46\textwidth]{figures/appendix/bdtvar/BDTqcd_test_nocut_3jshort_NoSigScale_classiAdaBoost.log.data.eps}
 \includegraphics[width=0.46\textwidth]{figures/appendix/bdtvar/BDTqcd_test_nocut_2jshort_classiAdaBoost.log.data.eps}
 \includegraphics[width=0.46\textwidth]{figures/appendix/bdtvar/BDTqcd_test_cut_2jshort_classiAdaBoost.data.eps}
\includegraphics[width=0.46\textwidth]{figures/appendix/bdtvar/BDTqcd_test_nocut_3jshort_classiAdaBoost.log.data.eps}
  \includegraphics[width=0.46\textwidth]{figures/appendix/bdtvar/BDTqcd_test_cut_3jshort_classiAdaBoost.data.eps}
 \includegraphics[width=0.46\textwidth]{figures/variables/Plot_Legend.eps}
\vspace{-0.5cm}
 \caption{BDT classifier distributions for the 2 jet (top) and 3 jet (bottom) selections, formed using cut-based analysis variables.  The left column is before the selection on the BDT classifier in a log scale, and the right column is after.  The $t$-channel single-top contribution is normalized to the observed cross-section determined using all four channels. Other top refers to the $s$-channel and $Wt$ single-top contributions.}
 \label{fig:Plot_SBDT}
 \end{figure}

The yields after the selections on the BDT thresholds are given in Table~\ref{tab:tch_eventyields_sbdt}.  Overall the yields a little lower than the cut-based analysis (see Section~\ref{sec:selectionchoices}).  The signal to background ratios are much higher in the 2 jet channel but a little lower or the same in the 3 jet channel.  This indicates that the BDT for the 2 jet selection in particular has better separating power between signal and background than the cut-based analysis cuts.
\begin{table}[!h!tpb]
  \begin{center}
     \begin{tabular}{lrr|rr}
    \hline \hline
        &\multicolumn{2}{c|}{BDT 5 Variables 2 Jets} &\multicolumn{2}{c} { BDT 5 Variables 3 Jets}  \\
        & Lepton + & Lepton -  & Lepton + & Lepton -  \\

    \hline \hline
    $t$-channel            & $ 27.7 $ & $ 7.6 $ & $ 56.9$ & $ 12.8 $ \\
    \hline                                                                       
    $t\bar t$, Other top   & $ 2.2 $ & $ 1.0 $ & $ 25.9$ & $ 6.9$ \\
    $W$+light jets         & $ 0.8 $   & $ < 0.1$ & $ 4.4 $ & $ < 0.1 $ \\
    $W$+heavy flavour jets & $ 7.5$  & $ 2.0$ & $ 23.6 $ & $ 4.5 $ \\
    $Z$+jets, Diboson      & $ < 0.1$  & $ < 0.1 $ & $ 1.1 $ & $ < 0.1$ \\
    Multijets              & $ 0.3 $ & $ < 0.1 $ & $ 7.8  $ & $ < 0.1 $ \\
    \hline    
    TOTAL Exp              & $ 38.5 $ & $ 10.7$ & $ 119.7$ & $ 24.2 $ \\
    S/B                    &  2.6  & 2.4 &  0.9 &   1.1  \\
    \hline \hline
    DATA                   &  60  &   16   &   115  &   24   \\
     \hline \hline
    \end{tabular}
 \caption{Event yield for the two-jets and three-jets tag positive and negative lepton-charge channels after the selection on the BDT formed using the cut-based analysis variables. The multijets and $W$+jets backgrounds are normalized to the data; all other samples are normalized to theory cross-sections.   The $t$-channel single-top contribution is normalized to the observed cross-section determined using all four channels. Other top refers to the $s$-channel and $Wt$ single-top contributions.
\label{tab:tch_eventyields_sbdt}}
  \end{center}
\end{table}

The expected cross-section was also calculated for the BDT distributions.  The 2 and 3 jet channels were split into negative and positive lepton charge channels, after selecting the desired region of the BDT classifier, and the combination was calculated using all four channels.  The expected cross-section calculation used the systematic shifts and statistical methods from the cut-based analysis (Section~\ref{Sys}) except in the cases of the largest systematic uncertainties, which were estimated using the shifts in the yields of the BDT classifier distributions after the selection on it.  These systematic uncertainties that were re-estimated are the statistical, b-tagging scale factor, mis-tagging scale factor, jet energy scale, generator, parton shower, and ISR/FSR uncertainties.  The MC statistical uncertainty is not changed here, but we might expect it to be about 1.7 times as large as in the cut-based analysis, if the proportion of events from the different processes is relatively unchanged.  This is because the MC event weights increase by a factor of 3 and there are 1/3 as many events, giving a factor of $\sqrt{3}$ multiplied with the square root of the sum of the squares of the weights of the events (the MC statistical uncertainty).  The uncertainties which are re-estimated and the total uncertainties using both re-estimated and cut-based analysis values are given in Table~\ref{tab:xs-uncertainty-exp-sbdt}.  

The uncertainties are generally comparable with the cut-based analysis except for the ISR/FSR uncertainty, which is much larger.  This may be due to the ISR/FSR uncertainty not being considered during the optimization of the classifier with the validation sample, leading to the selection of events that happen to have a larger uncertainty.  This is something that could be added to the classifier optimization in a future study.  Additionally, the jet energy scale uncertainty is higher and the b-tagging scale factor uncertainty is lower versus the cut-based analysis, reflecting some differences in the selected events in the BDT versus the cut-based analysis.  If we use the cut-based ISR/FSR uncertainty value, assuming the BDT analysis could be improved to reduce this uncertainty, the expected cross-section uncertainty would be $\sigma_{t}= 65^{+21}_{-19}$~pb, compared to $\sigma_{t}= 65^{+22}_{-20}$~pb from the cut-based analysis in Section~\ref{sec:fourchanresult}.  The expected cross-section uncertainty with the re-estimated ISR/FSR is $\sigma_{t}= 65^{+34}_{-26}$~pb.  The observed cross-section value is $\sigma_{t}= 82.9^{+36}_{-28}$~pb, which is consistent with the cut-based analysis result within uncertainties.

\begin{table}[htdp]
\begin{center}
   \begin{tabular}{l|c}
    \hline
    Source & $\Delta\sigma/\sigma$ (\%) \\
    \hline \hline
    Expected statistics              & +13/-13  \\
     \hline  
    $b$ tagging scale factor     & +6/-6  \\
    Mistag scale factor          & +1/-1  \\
    Jet energy scale             & +7/-8  \\
    Generator                    & +7/-7  \\
    Shower                       & +14/-14  \\ 
    ISR/FSR                      & +40/-29  \\
    \hline  
    Total Systematics            & +50/-38  \\
    Total                        & +52/-40  \\
    \hline
   \end{tabular}
\caption{Systematic uncertainties for the expected $t$-channel cross-section measurement for the BDT formed using cut-based analysis variables, where the final line includes all systematic uncertainties and the statistical uncertainty of the data.  Uncertainties that were re-estimated versus the cut-based analysis (Section~\ref{sec:xs}) are listed individually.  Others are not listed but are included in the totals.}
\label{tab:xs-uncertainty-exp-sbdt}
\end{center}
\end{table}

%     SysName     up[%]  down[%]  bias[%]
%  Data stat.    13.044  -13.044    0.001
%----------------------------------------      
%      btagSF     5.926   -5.926   -0.181
%         JES     6.775   -7.666   -3.587
%    partshow    13.635  -13.635   -0.127
%      isrfsr    40.222  -29.300   27.555
%       Total sys   50.337  -38.143   32.847
% 
%     ttgen     4.618   -4.618   -0.019
%      stgen     6.207   -6.207   -0.071
%
%       Total sys   31.129  -27.431   14.717
%       Total    33  30

 \begin{figure}[!h!tpb]
 \centering
 \includegraphics[width=0.46\textwidth]{figures/appendix/bdtvar/BDTqcd_test_cut_2jshort_classiTopBJet1_mass.data.eps}
 \includegraphics[width=0.46\textwidth]{figures/appendix/bdtvar/BDTqcd_test_cut_2jshort_classiUJet1_Jet_eta.data.eps}
 \includegraphics[width=0.46\textwidth]{figures/appendix/bdtvar/BDTqcd_test_cut_2jshort_classiLepton_charge.data.eps}
 \includegraphics[width=0.46\textwidth]{figures/appendix/bdtvar/BDTqcd_test_cut_2jshort_classiHt_AllJetsLeptonMissingEt.data.eps}
 \includegraphics[width=0.46\textwidth]{figures/appendix/bdtvar/BDTqcd_test_cut_2jshort_classiDeltaEta_BJet1UJet1.data.eps}
 \includegraphics[width=0.46\textwidth]{figures/variables/Plot_Legend.eps}
\vspace{-0.5cm}
 \caption{Discriminating variables for the 2 jets selection after a selection on the BDT classifier formed using cut-based analysis variables.  The last bin contains the sum of the events in that bin or higher.  The $t$-channel single-top contribution is normalized to the observed cross-section determined using all four channels. Other top refers to the $s$-channel and $Wt$ single-top contributions.}
 \label{fig:Plot_2SBDT}
 \end{figure}

\begin{figure}[!h!tpb]
 \centering
 \includegraphics[width=0.46\textwidth]{figures/appendix/bdtvar/BDTqcd_test_cut_3jshort_classiTopBJet1_mass.data.eps}
 \includegraphics[width=0.46\textwidth]{figures/appendix/bdtvar/BDTqcd_test_cut_3jshort_classiUJet1_Jet_eta.data.eps}
 \includegraphics[width=0.46\textwidth]{figures/appendix/bdtvar/BDTqcd_test_cut_3jshort_classiLepton_charge.data.eps}
 \includegraphics[width=0.46\textwidth]{figures/appendix/bdtvar/BDTqcd_test_cut_3jshort_classiHt_AllJetsLeptonMissingEt.data.eps}
 \includegraphics[width=0.46\textwidth]{figures/appendix/bdtvar/BDTqcd_test_cut_3jshort_classiInvariantMass_AllJets.data.eps}
 \includegraphics[width=0.46\textwidth]{figures/variables/Plot_Legend.eps}
\vspace{-0.5cm}
 \caption{Discriminating variables for the 3 jets selection after a selection on the BDT classifier formed using cut-based analysis variables.  The last bin contains the sum of the events in that bin or higher.  The $t$-channel single-top contribution is normalized to the observed cross-section determined using all four channels. Other top refers to the $s$-channel and $Wt$ single-top contributions.}
 \label{fig:Plot_3SBDT}
 \end{figure}

%There were some studies done before the detector running that looked at what performance one might expect with a BDT.  In these studies a selection of variables was proposed, and we again consider this list and.  Ideally, you would form many different BDT's using various combinations of these variables and find which BDT gives the best separation.   Adding more variables to the truly optimized BDT would give a similar performance as the optimal one.  Extra variables beyond this point won't hurt performance, but they are not necessary.  This can be seen in a distribution of the number of variables versus significance using the validation sample, for various classifiers that were trained.  

\subsection{Additional Variables}
Of course, there is no reason to choose only the variables used for the cut based analysis when generating the BDTs.  Starting from these variables, we considered many additional variable combinations, using variables that were considered for the cut-based analysis but not used (see Section~\ref{chap:variables}).  After many options were considered, BDT classifiers were chosen for the 2 jet and 3 jet selections which happen to use the same variables.  These BDT classifiers were chosen to have a large significance in the validation sample, a relatively low number of variables, and good agreement between the training and validation BDT distributions.

The best classifiers had 10 variables, including the lepton charge variable.  In addition to the 6 variables considered in this analysis (listed in Section~\ref{app:sbdt}), the following were used for both jet number selection channels: $\eta$ of the lepton; cosine of the angle between the lepton and the untagged jet, both in the rest frame of the top quark reconstructed using the leading b-tagged jet; W transverse mass; and $\Delta\eta$ between the b-tagged jet and the lepton.  Note that the $\Delta\eta$ between the b-tagged jet and leading untagged jet and the invariant mass of all jets are now used for both the 2 and 3 jet selections, unlike in Section~\ref{app:sbdt}.  The distributions of the additional variables with the preselection applied for the full MC set are shown in Figure~\ref{fig:preselbdt2} for the 2 jet selection and Figure~\ref{fig:preselbdt3} for the 3 jet selection, showing good agreement between the data and the MC.

%The cosine variable is also known as the spin correlation variable for single-top.  The spin for the top is almost 100% polarized with the forward light jet in the standard model and this cosine variable is directly related to the spin~\ref{spin}.  We look a the angle between the decay products and this forward jet with this variable, and single-top tends to prefer a region where the cosine 

\begin{figure}[!h!tpb]
 \centering
 \includegraphics[width=0.46\textwidth]{figures/appendix/bdtvar/PaperFinal_MCtchannorm_2jet1tag__wflavorDeltaEta_BJet1Lepton.eps}
 \includegraphics[width=0.46\textwidth]{figures/appendix/bdtvar/PaperFinal_MCtchannorm_2jet1tag__wflavorCos_Lepton_UJet1_RBJet1Top.eps}
\includegraphics[width=0.46\textwidth]{figures/appendix/bdtvar/PaperFinal_MCtchannorm_2jet1tag__wflavorLepton_eta.eps}
\includegraphics[width=0.46\textwidth]{figures/appendix/bdtvar/PaperFinal_MCtchannorm_2jet1tag__wflavorW_TransverseMass.eps}
 \includegraphics[width=0.46\textwidth]{figures/appendix/bdtvar/PaperFinal_MCtchannorm_2jet1tag__wflavorInvariantMass_AllJets.eps}
 \includegraphics[width=0.46\textwidth]{figures/variables/Plot_Legend.eps}
\vspace{-0.5cm}
 \caption{Discriminating variables for the 2 jets selection before any BDT classifier selection.  The last bin contains the sum of the events in that bin or higher. Other top refers to the $s$-channel and $Wt$ single-top contributions.}
 \label{fig:preselbdt2}
 \end{figure}

\begin{figure}[!h!tpb]
 \centering
 \includegraphics[width=0.46\textwidth]{figures/appendix/bdtvar/PaperFinal_MCtchannorm_3jet1tag__wflavorDeltaEta_BJet1Lepton.eps}
 \includegraphics[width=0.46\textwidth]{figures/appendix/bdtvar/PaperFinal_MCtchannorm_3jet1tag__wflavorCos_Lepton_UJet1_RBJet1Top.eps}
\includegraphics[width=0.46\textwidth]{figures/appendix/bdtvar/PaperFinal_MCtchannorm_3jet1tag__wflavorLepton_eta.eps}
\includegraphics[width=0.46\textwidth]{figures/appendix/bdtvar/PaperFinal_MCtchannorm_3jet1tag__wflavorW_TransverseMass.eps}
 \includegraphics[width=0.46\textwidth]{figures/appendix/bdtvar/PaperFinal_MCtchannorm_3jet1tag__wflavorDeltaEta_BJet1UJet1.eps}
 \includegraphics[width=0.46\textwidth]{figures/variables/Plot_Legend.eps}
\vspace{-0.5cm}
 \caption{Discriminating variables for the 3 jets selection before any BDT classifier selection.  The last bin contains the sum of the events in that bin or higher. Other top refers to the $s$-channel and $Wt$ single-top contributions.}
 \label{fig:preselbdt3}
 \end{figure}

For the 2 jet selection, the classifier parameters and cut threshold are: 150 trees, 2500 events minimum per leaf, and 0.64 cut threshold.  For the 3 jet selection, these are: 150 trees, 1500 events minimum per leaf, and 0.42 cut threshold.  The BDT classifier distributions before and after the selection for each channel are shown in Figure~\ref{fig:Plot_ABDT}, normalized to the observed t-channel cross-section.  The variable distributions after this cut threshold for each channel are given in Figures~\ref{fig:Plot_2ABDT} and~\ref{fig:Plot_2ABDT2} for the 2 jet selection, and Figures~\ref{fig:Plot_3ABDT} and~\ref{fig:Plot_3ABDT2} for the 3 jet selection, with all normalized to the observed t-channel cross-section.  Again, notice that after the selections, the kinematic regions selected in the distributions look similar to those in Figures~\ref{fig:Plot_2TagCuts} and Figures~\ref{fig:Plot_3TagCuts}, particularly the reconstructed top mass, invariant mass and leading untagged jet $\eta$ distributions.  

 \begin{figure}[!h!tpb]
 \centering
 \includegraphics[width=0.46\textwidth]{figures/appendix/bdtvar/BDTqcd_test_nocut_2jall_classiAdaBoost.log.data.eps}
 \includegraphics[width=0.46\textwidth]{figures/appendix/bdtvar/BDTqcd_test_cut_2jall_classiAdaBoost.data.eps}
 \includegraphics[width=0.46\textwidth]{figures/appendix/bdtvar/BDTqcd_test_nocut_3jall_classiAdaBoost.log.data.eps}
 \includegraphics[width=0.46\textwidth]{figures/appendix/bdtvar/BDTqcd_test_cut_3jall_classiAdaBoost.data.eps}
 \includegraphics[width=0.46\textwidth]{figures/variables/Plot_Legend.eps}
\vspace{-0.5cm}
 \caption{BDT classifier distributions for the 2 jet selection on the top line and the 3 jet selection on the next line, for the BDT formed using 10 analysis variables.  The left figures are before the selection on the BDT classifier, the right figures are after.  Note that the BDT distributions before selections are in a log scale.   The $t$-channel single-top contribution is normalized to the observed cross-section determined using all four channels. Other top refers to the $s$-channel and $Wt$ single-top contributions.}
 \label{fig:Plot_ABDT}
 \end{figure}

The yields after the selections on the BDT thresholds are given in Table~\ref{tab:tch_eventyields_abdt}.  The signal to background ratios here are about the same as those from Section~\ref{app:sbdt} for the 3 jet bin but are much improved for the 2 jet selection.  This indicates that the extra variables have particularly helped the signal versus background discrimination in the 2 jet bin.
\begin{table}[!h!tpb]
  \begin{center}
     \begin{tabular}{lrr|rr}
    \hline \hline
        &\multicolumn{2}{c|}{BDT 10 Variables 2 Jets} &\multicolumn{2}{c} {BDT 10 Variables 3 Jets}  \\
        & Lepton + & Lepton -  & Lepton + & Lepton -  \\

    \hline \hline
    $t$-channel            & $ 45.9 $ & $ 19.7 $ & $ 46.0$ & $ 16.7 $ \\
    \hline                                                                       
    $t\bar t$, Other top   & $ 2.2 $ & $ 2.1 $ & $ 16.6$ & $ 7.5 $ \\
    $W$+light jets         & $ 0.8 $ & $ < 0.1$ & $ 2.4 $ & $ < 0.1 $ \\
    $W$+heavy flavour jets & $ 9.7$  & $ 6.5 $ & $ 25.4 $ & $ 8.3 $ \\
    $Z$+jets, Diboson      & $ < 0.1$ & $ < 0.1 $ & $ 0.3 $ & $ 0.6$ \\
    Multijets              & $ 1.9 $ & $ 1.6 $ & $ 3.6  $ & $ < 0.1 $ \\
    \hline    
    TOTAL Exp              & $ 60.5 $ & $ 29.9$ & $ 94.3$ & $ 33.1 $ \\
    S/B                    &  3.1  & 1.9 &  0.9 &   1.0  \\
    \hline \hline
    DATA                   &  94  &   33   &   82  &   38   \\
     \hline \hline
    \end{tabular}
 \caption{Event yield for the two-jets and three-jets tag positive and negative lepton-charge channels after the selection on the BDT formed using 10 analysis variables. The multijets and $W$+jets backgrounds are normalized to the data; all other samples are normalized to theory cross-sections.   The $t$-channel single-top contribution is normalized to the observed cross-section determined using all four channels. Other top refers to the $s$-channel and $Wt$ single-top contributions.
\label{tab:tch_eventyields_abdt}}
  \end{center}
\end{table}

The additional variables are used to try to improve the uncertainty on the cross-section measurement versus the BDT formed with cut-based variables only.  The extra information should improve the signal and background separation, and may also help to improve the selection of low uncertainty kinematic regions.  As in Section~\ref{app:sbdt}, the 2 and 3 jet channels were split into negative and positive lepton charge channels, after selecting the desired region of the BDT classifier, and the combination was calculated using all four channels.  The expected cross-section uncertainties for the combination is given in Table~\ref{tab:xs-uncertainty-exp-abdt}.  Also, as in Section~\ref{app:sbdt}, only the statistical, b-tagging scale factor, mis-tagging scale factor, jet energy scale, generator, parton shower, and ISR/FSR uncertainties are re-estimated.  Again, the ISR/FSR uncertainty is very high here, and including this uncertainty in the BDT optimization might improve the result in future studies.  The other uncertainties are generally similar to the cut-based result, although we again see that the b-tagging scale factor uncertainty is lower than it was in the cut-based analysis from Section~\ref{sec:xs}.  The overall uncertainty is lower than the BDT cross-section expectation using only cut-based analysis variables, indicating the usefulness of the additional variables.  The expected cross-section is $\sigma_{t}= 65^{+30}_{-21}$~pb, while combined result from the BDTs using only cut-based variables had an expected cross-section of $\sigma_{t}= 65^{+34}_{-26}$~pb.  The observed cross-section value is $\sigma_{t}= 83.8^{+34}_{-25}$~pb, which has a lower uncertainty than, and is consistent with, the value found from the BDTs using only cut-based analysis variables.  In particular the central value is very similar.  This cross-section is also consistent with the cut-based analyis value.

\begin{table}[htdp]
\begin{center}
   \begin{tabular}{l|c}
    \hline
    Source & $\Delta\sigma/\sigma$ (\%) \\
    \hline \hline
    Expected statistics              & +11/-11  \\
     \hline  
    $b$ tagging scale factor     & +5/-5  \\
    Mistag scale factor          & +2/-2  \\
    Jet energy scale             & +4/-4  \\
    Generator                    & +8/-8  \\
    Shower                       & +11/-10  \\ 
    ISR/FSR                      & +34/-24  \\
    \hline  
    Total Systematics            & +45/-30  \\
    Total                        & +46/-32  \\
    \hline
   \end{tabular}
\caption{Systematic uncertainties for the expected $t$-channel cross-section measurement determined using the BDT created with 10 analysis variables, where the final line includes all systematic uncertainties and the statistical uncertainty of the data.  Uncertainties that were re-estimated versus the cut-based analysis (Section~\ref{sec:xs}) are listed individually.  Others are not listed but are included in the totals.}
\label{tab:xs-uncertainty-exp-abdt}
\end{center}
\end{table}
%Process: tchan
%     SysName     up[%]  down[%]  bias[%]
%  Data stat.    11.264  -11.264    0.000
%----------------------------------------
%      ltagSF     1.721   -1.721   -0.113
%         JES     3.537   -3.537   -0.161
%       ttgen     3.578   -3.578    0.089
%      btagSF     4.827   -4.827    0.134
%       stgen     6.943   -6.932    0.384
%    partshow    11.040  -10.443    3.581
%      isrfsr    33.640  -23.941   23.631
%       Total    44.972  -29.524   33.923

%Plot("/msu/data/t3work6/jenny/SPR/rootfiles/boosttreevar_comb10_testM3n150l2500_variables10_yield2jqcd.lepton.thesis.sigonly.root", 0.636667, plotdir, kFALSE, 2);
%Plot("/msu/data/t3work6/jenny/SPR/rootfiles/boosttreevar_comb10_testM3n150l1500_variables10_yield3jqcd.lepton.thesis.sigonly.root", 0.423333, plotdir, kFALSE, 3);

 \begin{figure}[!h!tpb]
 \centering
 \includegraphics[width=0.46\textwidth]{figures/appendix/bdtvar/BDTqcd_test_cut_2jall_classiTopBJet1_mass.data.eps}
 \includegraphics[width=0.46\textwidth]{figures/appendix/bdtvar/BDTqcd_test_cut_2jall_classiUJet1_Jet_eta.data.eps}
 \includegraphics[width=0.46\textwidth]{figures/appendix/bdtvar/BDTqcd_test_cut_2jall_classiLepton_charge.data.eps}
 \includegraphics[width=0.46\textwidth]{figures/appendix/bdtvar/BDTqcd_test_cut_2jall_classiHt_AllJetsLeptonMissingEt.data.eps}
 \includegraphics[width=0.46\textwidth]{figures/appendix/bdtvar/BDTqcd_test_cut_2jall_classiDeltaEta_BJet1UJet1.data.eps}
 \includegraphics[width=0.46\textwidth]{figures/variables/Plot_Legend.eps}
\vspace{-0.5cm}
 \caption{Discriminating variables for the 2 jets selection. The figures are after the selection on the BDT classifier formed using 10 analysis variables.  The last bin contains the sum of the events in that bin or higher.  The $t$-channel single-top contribution is normalized to the observed cross-section determined using all four channels. Other top refers to the $s$-channel and $Wt$ single-top contributions.}
 \label{fig:Plot_2ABDT}
 \end{figure}

\begin{figure}[!h!tpb]
 \centering
  \includegraphics[width=0.46\textwidth]{figures/appendix/bdtvar/BDTqcd_test_cut_2jall_classiInvariantMass_AllJets.data.eps}
 \includegraphics[width=0.46\textwidth]{figures/appendix/bdtvar/BDTqcd_test_cut_2jall_classiDeltaEta_BJet1Lepton.data.eps}
\includegraphics[width=0.46\textwidth]{figures/appendix/bdtvar/BDTqcd_test_cut_2jall_classiCos_Lepton_UJet1_RBJet1Top.data.eps}
 \includegraphics[width=0.46\textwidth]{figures/appendix/bdtvar/BDTqcd_test_cut_2jall_classiLepton_eta.data.eps}
 \includegraphics[width=0.46\textwidth]{figures/appendix/bdtvar/BDTqcd_test_cut_2jall_classiW_TransverseMass.data.eps}
 \includegraphics[width=0.46\textwidth]{figures/variables/Plot_Legend.eps}
\vspace{-0.5cm}
 \caption{Discriminating variables for the 2 jets selection. The figures are after the selection on the BDT classifier formed using 10 analysis variables.  The last bin contains the sum of the events in that bin or higher.  The $t$-channel single-top contribution is normalized to the observed cross-section determined using all four channels. Other top refers to the $s$-channel and $Wt$ single-top contributions.}
 \label{fig:Plot_2ABDT2}
 \end{figure}

 \begin{figure}[!h!tpb]
 \centering
 \includegraphics[width=0.46\textwidth]{figures/appendix/bdtvar/BDTqcd_test_cut_3jall_classiTopBJet1_mass.data.eps}
 \includegraphics[width=0.46\textwidth]{figures/appendix/bdtvar/BDTqcd_test_cut_3jall_classiUJet1_Jet_eta.data.eps}
 \includegraphics[width=0.46\textwidth]{figures/appendix/bdtvar/BDTqcd_test_cut_3jall_classiLepton_charge.data.eps}
 \includegraphics[width=0.46\textwidth]{figures/appendix/bdtvar/BDTqcd_test_cut_3jall_classiHt_AllJetsLeptonMissingEt.data.eps}
 \includegraphics[width=0.46\textwidth]{figures/appendix/bdtvar/BDTqcd_test_cut_3jall_classiDeltaEta_BJet1UJet1.data.eps}
 \includegraphics[width=0.46\textwidth]{figures/variables/Plot_Legend.eps}
\vspace{-0.5cm}
 \caption{Discriminating variables for the 3 jets selection. The figures are after the selection on the BDT classifier formed using 10 analysis variables.  The last bin contains the sum of the events in that bin or higher.  The $t$-channel single-top contribution is normalized to the observed cross-section determined using all four channels. Other top refers to the $s$-channel and $Wt$ single-top contributions.}
 \label{fig:Plot_3ABDT}
 \end{figure}

\begin{figure}[!h!tpb]
 \centering
  \includegraphics[width=0.46\textwidth]{figures/appendix/bdtvar/BDTqcd_test_cut_3jall_classiInvariantMass_AllJets.data.eps}
 \includegraphics[width=0.46\textwidth]{figures/appendix/bdtvar/BDTqcd_test_cut_3jall_classiDeltaEta_BJet1Lepton.data.eps}
\includegraphics[width=0.46\textwidth]{figures/appendix/bdtvar/BDTqcd_test_cut_3jall_classiCos_Lepton_UJet1_RBJet1Top.data.eps}
 \includegraphics[width=0.46\textwidth]{figures/appendix/bdtvar/BDTqcd_test_cut_3jall_classiLepton_eta.data.eps}
 \includegraphics[width=0.46\textwidth]{figures/appendix/bdtvar/BDTqcd_test_cut_3jall_classiW_TransverseMass.data.eps}
 \includegraphics[width=0.46\textwidth]{figures/variables/Plot_Legend.eps}
\vspace{-0.5cm}
 \caption{Discriminating variables for the 3 jets selection. The figures are after the selection on the BDT classifier formed using 10 analysis variables.  The last bin contains the sum of the events in that bin or higher.  The $t$-channel single-top contribution is normalized to the observed cross-section determined using all four channels. Other top refers to the $s$-channel and $Wt$ single-top contributions.}
 \label{fig:Plot_3ABDT2}
 \end{figure}
\chapter{Alternative Analysis Channels}
\label{alternativechannels}
The main analysis in this document made use of the 2 and 3 jet channels, split into positive or negative lepton charge (see Section~\ref{sec:channels}).  However, there are other possibilities.  For instance, the four jet bin, although the ``natural'' bin for \ttbar~production, could still be a useful channel if the \ttbar~can be successfully removed.  Similarly the 3 jet bin, where 2 jets are b-tagged (unlike 1 b-tagged jet in the main analysis) is also dominated by \ttbar~production.  If \ttbar~can be removed from this bin, we may be able to see our single-top signal.  The event yields for these two bins (split into lepton charge) after all preselection cuts including either 1 or 2 b-tagged jets, are shown in Table~\ref{tab:tag_eventyields_extra}.  The signal divided by background (S/B) value is also shown.  No W+jets normalization scale factors or multijet estimates are included in any of the tables in this discussion.  However, as the yields are dominated by top processes, neither exclusion is expected to have a large effect on the conclusions of this study.

\begin{table}[!h!tpb]
  \begin{center}
     \begin{tabular}{lrr|rr}
    \hline \hline
        &\multicolumn{2}{c|}{4 Jets, 1 b-tagged} &\multicolumn{2}{c} {3 Jets, 2 b-tagged}  \\
        & Lepton + & Lepton -  & Lepton + & Lepton -  \\

    \hline \hline
    $t$-channel      & 200	& 100	& 72	& 46  \\       
    \hline                                                                       
    $t\bar t$, Other top     & 1800	& 1800	& 560	& 550  \\
    $W$+light jets          & 99	& 43	& 0.2	& 0.2  \\
    $W$+heavy flavour jets  & 400	& 310	& 73	& 45  \\
    $Z$+jets, Diboson       & 50	& 43	& 4.9	& 2.7  \\
    \hline    
    S/B                     & 0.08	& 0.05	& 0.11	& 0.08  \\
    \hline \hline
    \end{tabular}
 \caption{Event yields for the four jets, one b-tag and three jets, two b-tags with positive and negative lepton-charge channels after the preselection. The multijets are neglected and $W$+jets backgrounds are normalized to the MC expectation, all other samples are also normalized to theory cross-sections.
\label{tab:tag_eventyields_extra}}
  \end{center}
\end{table}

The first two selections in the main analysis (Section~\ref{sec:selectionchoices}) make use of differences between the t-channel single-top production and its backgrounds, and we apply both of these selections here with a slight variation on the reconstructed top mass selection.  The reasoning is the same as before.  The t-channel single-top production often has an energetic forward non-b jet, whereas \ttbar~tends to have more central jets.  Also, the signal only has one top quark whereas \ttbar~has two.  Due to reconstruction difficulties, the b associated with the top decaying to leptons is not always correctly assigned when determining the reconstructed top mass for \ttbar~prodcution, so a selection requiring a top mass near the expected value is useful.  Similarly, requiring the highest \pt~untagged jet to be forward is also helpful, just as in the main analysis.  The yields after requiring $|\eta({\rm j_{u}})|>2.0$ and then after also requiring $M_{top}(\rm l\nu b)<190\GeV$ are shown in Tables~\ref{tab:tag_eventyields_extra_ueta} and~\ref{tab:tag_eventyields_extra_uetatm} respectively.  We use only the upper top mass selection from the main analysis here because of the low multijet expectation (which is ignored for this discussion) in these bins.

\begin{table}[!h!tpb]
  \begin{center}
     \begin{tabular}{lrr|rr}
    \hline \hline
        &\multicolumn{2}{c|}{4 Jets, 1 b-tagged} &\multicolumn{2}{c} {3 Jets, 2 b-tagged}  \\
        & Lepton + & Lepton -  & Lepton + & Lepton -  \\

    \hline \hline
    $t$-channel            & 74	& 33	& 38	& 21  \\ 
    \hline                                                                       
    $t\bar t$, Other top    & 240	& 230	& 82	& 81  \\
    $W$+light jets          & 24	& 5.4	& 0.2	& $<$0.1  \\
    $W$+heavy flavour jets  & 65	& 56	& 14	& 6.3  \\
    $Z$+jets, Diboson       & 7.1	& 7.6	& 0.5	& 0.5  \\
    \hline    
    S/B                     & 0.22	& 0.11	& 0.40	& 0.24  \\
    \hline \hline
    \end{tabular}
 \caption{Event yields for the four jets, one b-tag and three jets, two b-tags with positive and negative lepton-charge channels after the preselection and $|\eta({\rm j_{u}})|>2.0$. The multijets are neglected and $W$+jets backgrounds are normalized to the MC expectation, all other samples are also normalized to theory cross-sections.
\label{tab:tag_eventyields_extra_ueta}}
  \end{center}
\end{table}

\begin{table}[!h!tpb]
  \begin{center}
     \begin{tabular}{lrr|rr}
    \hline \hline
        &\multicolumn{2}{c|}{4 Jets, 1 b-tagged} &\multicolumn{2}{c} {3 Jets, 2 b-tagged}  \\
        & Lepton + & Lepton -  & Lepton + & Lepton -  \\

    \hline \hline
    $t$-channel           & 50	& 22	& 19	& 12   \\
    \hline                                                                       
    $t\bar t$, Other top    & 120	& 120   & 23.0	& 23.0 \\
    $W$+light jets          & 16	& 2.4	& 0.2	& $<$0.1 \\
    $W$+heavy flavour jets  & 32	& 27	& 6.6	& 3.2 \\
    $Z$+jets, Diboson       & 3.9	& 3.6	& 0.4  & $<$0.1	 \\
    \hline    
    S/B                     & 0.29	& 0.14	& 0.64	& 0.45 \\
    \hline \hline
    \end{tabular}
 \caption{Event yields for the four jets, one b-tag and three jets, two b-tags with positive and negative lepton-charge channels after the preselection, $|\eta({\rm j_{u}})|>2.0$, and $M_{top}(\rm l\nu b)<190\GeV$. The multijets are neglected and $W$+jets backgrounds are normalized to the MC expectation, all other samples are also normalized to theory cross-sections.
\label{tab:tag_eventyields_extra_uetatm}}
  \end{center}
\end{table}


Finally, we can also make use of the invariant mass of the jets.  In the main analysis (Section~\ref{sec:selectionchoices}), the 3 jet bin had a large \ttbar~component, and we used a selection involving the invariant mass of the jets to remove \ttbar~events.  For the 4 jet bin, we propose requiring the invariant mass of all jets except the best jet to be greater than 450 GeV.  The best jet is the jet which, plus the lepton and \met, best produces the expected standard model top mass.  Thus, we are looking at the invariant mass of what should be the other top quark in the case of \ttbar~production, and the invariant mass should be close to that value.  Single-top t-channel can have a very energetic jet separated from the top quark decay products, leading to a potentially very high invariant mass.  These two effects give some separation, and we remove events where the invariant mass is lower.  Similarly for the 3 jet, 2-btagged jet selection, we can look at this same invariant mass of all jets minus the best jet greater than 250 GeV.  This should give us the invariant mass of most of the jets from the other top quark in the case of \ttbar~production and so we again remove events where this invariant mass is lower to reduce \ttbar~contanimation.  Specifically we require $M(\mathrm{All Jets Minus Best Jet}) > 450\GeV$ for the four jet channel and $M(\mathrm{All Jets Minus Best Jet}) > 250\GeV$ for the three jet channel.

The yields after these two invariant mass selections are given in Table~\ref{tab:tag_eventyields_extra_uetatminvmass}.  This table shows that the \ttbar~events are removed at a larger rate than the signal, improving the S/B value from the previous tables.  Most S/B values are over 0.5 and one is over 1.5.  Of course, the yields themselves are low, leading to larger statistical uncertainties.  However, these tables are normalized to the integrated luminosity used in the main analysis, 1.04 fb$^{-1}$.  Future analysis will use data sets of 5, 10, or more times this value, making these sorts of tight selections more useful for those analyses.

\begin{table}[!h!tpb]
  \begin{center}
     \begin{tabular}{lrr|rr}
    \hline \hline
        &\multicolumn{2}{c|}{4 Jets, 1 b-tagged} &\multicolumn{2}{c} {3 Jets, 2 b-tagged}  \\
        & Lepton + & Lepton -  & Lepton + & Lepton -  \\

    \hline \hline
    $t$-channel         & 27	& 12	& 10	& 4.0 \\    
    \hline                                                                       
    $t\bar t$, Other top    & 29	& 31	& 4.5	& 4.3 \\
    $W$+light jets          & 1.8	& 0.6	& $<$0.1   & $<$0.1 \\
    $W$+heavy flavour jets  & 6.2	& 5.9	& 0.7	& 1.2 \\
    $Z$+jets, Diboson       & 1.2	& 2.0	& 0.2   & $<$0.1 \\	
    \hline    
    S/B                     & 0.71	& 0.30	& 1.89	& 0.74 \\
    \hline \hline
    \end{tabular}
 \caption{Event yields for the four jets, one b-tag and three jets, two b-tags with positive and negative lepton-charge channels after the preselection, $|\eta({\rm j_{u}})|>2.0$, $M_{top}(\rm l\nu b)<190\GeV$, and either $M(\mathrm{All Jets Minus Best Jet}) > 450\GeV$ for the four jet channel or $M(\mathrm{All Jets Minus Best Jet}) > 250\GeV$ for the three jet channel. The multijets are neglected and $W$+jets backgrounds are normalized to the MC expectation, all other samples are also normalized to theory cross-sections.
\label{tab:tag_eventyields_extra_uetatminvmass}}
  \end{center}
\end{table}


\end{document}
