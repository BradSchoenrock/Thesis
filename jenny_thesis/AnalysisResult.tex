\chapter{The Measurements}
The purpose of this dissertation is to measure the single-top t-channel cross-section.  In previous chapters we have spent much time reducing the backgrounds, first in an initial preselection and then again using cut-based selections.  In this chapter, we evaluate the signal cross-section after applying these selections.  We also estimate the value of the CKM matrix element $|V_{tb}|$.

\section{Systematic Uncertainties}\label{Sys}
Before we can determine the cross-section, we need to evaluate the uncertainties on the quantities which go into the calulation.  The cross-section is related to the number of events that are observed for some given amount of proton-proton interactions, as stated in Section~\ref{sec:ExpectedCrosssections}.  If more events are observed than expected, the cross-section is higher than the expected value.  Uncertainties on the measurement are important here, as deviations from the expected cross-section may well be due to systematic uncertainties.  In this section, we discuss the systematic uncertainties on the measurement.

There are several systematic uncertainties in this analysis and we overview them by category and then give some information about their impact on the signal and background yields.  Most of the uncertainties are related to the MC.  For additional information on most of the scale factors, corrections, and MC itself, see Chapter~\ref{chap:MC}.

\textbf{b-tagging}:
There is an uncertainty associated with the b-tagging and mis-tagging scale factors, which relate the efficiencies measured in data to that in MC.  The b-tagging scale factor uncertainty in particular can be large in this analysis.  There is also a c-tagging efficiency uncertainty, which for this analysis is assumed to be twice that of the b-tagging efficiency uncertainty.  This is considered fully correlated with the uncertainty in b-tagging efficiency and is included in the reported b-tagging efficiency uncertainty (which may also be called the heavy flavor b-tagging scale factor uncertainty).  This is a large uncertainty for this analysis, with variations of around $10\%$ on the signal and background yields.

\textbf{Leptons}:
There are uncertainties on the lepton scale factors, which relate the trigger, ID, and reconstruction scale factors in data to MC, and also uncertainties on the lepton energy scale and resolution, which are uncertainties related to smearing the lepton energy, described in Section~\ref{sec:energyresolution}.  Also, for this study there was an issue with the MC related to the muon trigger matching.  This caused us not to apply trigger matching for the muon channel (although the trigger itself was still applied).  An uncertainty of 1.5\% was added to account for this.  These uncertainties are typically $<5\%$ for the different analysis processes.

\textbf{Jets}:
%apparently no bjes citations avaliable
There are three uncertainties associated with jets regarding the jet energy scale (JES), jet energy resolution (JER) and jet reconstruction (jetreco).  The JES uncertainty~\cite{JESnew, JES} is related to the energy calibration.  For example, we may not have perfectly simulated the dead material or leakage when adjusting the energy and there is some uncertainty related to this.  There may be noise or uncertainties related to the MC as well. JES includes a few different components, including a pileup and b-JES contribution.  The pileup is a special correction to account for pileup conditions during 2011 data taking and the effect on jet energies.  The b-JES factor is a separate corrections for jets which have a truth b-quark assignment.  It considers b-quark fragmentation, material and calorimeter response separately for these jets.  There is also some consideration of flavor composition uncertainty (gluon fraction distribution), which has a different distribution for each of the top samples and a flat distribution for other processes. The distance to other jets is also considered and jets that are close to one another have a different uncertainty.  Overall, this JES uncertainty (after including W+jets scale factor correlations) is largest for the light quark W+jets events, which are removed effectively by analysis cuts.  The impact is around $10\%$ for \ttbar and a few percent for the signal in the largest signal channel, 2 jets with a positive lepton charge.
%check that close-by jet corrections are on!!

The other two jet related uncertainties generally have an impact of a few percent for the different processes.  The  JER uncertainty is related to the jet energy value and is evaluated by smearing the jet energy (this is not done in the nominal sample, unlike for leptons, as discussed in Section~\ref{sec:energyresolution}).  The jet reconstruction evaluates how sensitive the analysis is to a missed jet.  This is done by randomly dropping jets from the event based on jet kinematics.

\textbf{Theoretical cross-section}:
There are several processes for which we do not have data-based normalization estimates.  In most cases, the contribution of these processes to the final yield is small.  In the case of \ttbar, we have performed a cross-check (see Appendix~\ref{app:ttbar}) and found the estimated normalization is consistent with the theoretical value, which has a smaller uncertainty.  We use a 10\% uncertainty for the single-top s-channel and Wt processes, 5\% for diboson processes, 60\% uncertainty for $Z$+jets and the cross-section variation is taken to be $164.57_{-15.7}^{+11.4}~pb$ for the \ttbar~process.  In this analysis we combine certain processes together when reporting yields and results.  When this is done, uncertainties such as the theoretical cross-section uncertainty are based on the proportion of each process in the combined sample (rather than taking the largest uncertainty, for instance).

\textbf{Multijets}:
There is an uncertainty on the multijets normalization, discussed in Section~\ref{sec:multijets}.  This is determined by re-doing the fit, which determines the normalization, using a different variable (W transverse mass).  We use 50\% for this uncertainty.

\textbf{$W$+jets}:
There are uncertainties on the $W$+jets scale factors discussed in Section~\ref{sec:wjets}.  These include b-tagging scale factor, mis-tagging scale factor, JES, theoretical cross-section, and data statistical uncertainties.  Many of these uncertainties are correlated with the uncertainties in the t-channel single-top cross-section measurement.  This means that the behavior of the JES uncertainty for the $W$+jets scale factor estimate and W+jets yield JES uncertainty are related to each other.  

To properly include these correlations, we re-estimate the $W$+jets scale factors for each uncertainty scenario and then apply the appropriate scale factor when estimating the W+jets yield uncertainty.  We assume that the JES upward shift scenario, for example, is the ``real'' scenario and do all of the estimations such as we would for the nominal sample, using JES upwardly shifted numbers instead of nominal numbers.  Then, to find the total JES upwardly shifted uncertainty, we compare the final yield (with the JES shifted scale factors applied to the JES shifted sample) to the nominal sample (with the nominal scale factors applied to the nominal sample).  

The JES, b-tagging scale factor and mis-tagging scale factor uncertainties quoted in this document always include these correlation effects.  The theoretical cross-section uncertainties and multijet normalization uncertainties for the $W$+jets scale factors are also correlated, but because they are not correlated with $W$+jets yield uncertainties, they are listed separately when uncertainties are given by processes (and are given in this way to the statistical tool).  The correlations are included in the final cross-section measurement.  Finally, the statistical uncertainties are considered separately and are called the $W$+jets normalization uncertainties.  These are $\le 5\%$.

There is another uncertainty associated with the $W$+jets normalization, related to the propagation of the scale factors from the 2 jet bin to other bins.  This is a 25\% uncertainty for a movement to the 3 jet bin, which is the other primary analysis bin.  This is referred to as the $W$+3 jet bin normalization.

One additional uncertainty related to the $W$+jets is an uncertainty on the simulated shape.  To evaluate this, two {\sc Alpgen}~\cite{SAMPLES-ALPGEN, ALPGENFAQ} parameters are varied and the uncertainties from these two variations are added in quadrature.  These parameters are the minimum {\sc Alpgen} \pt~to make a parton a hard (high \pt) parton (ptjmin) and the function which gives the factorization scale for the pdf (iqopt).

\textbf{MC statistical}:
There is an uncertainty associated with the number of simulated MC events.  If not enough events are generated, there may not be a sufficient range of kinematics to accurately represent the data.  The uncertainty is evaluated as the square root of the sum of the squares of the event weights and can range as high as $98\%$ after all cut-based analysis selections.

\textbf{LAr hole}:
There is some uncertainty on the removal of events affected by the LAr hole, discussed in Section~\ref{sec:dataquality}.  The uncertainty is a $\pm1$ sigma variation of the hole size and typically has a $<5\%$ effect on the signal and background yields.

\textbf{Missing $E_{T}$}:  There are two \met~related uncertainties.  The first is due to pileup effects (\met~pileup uncertainty) and the second is due to energy scale and energy resolution effects (\met~uncertainty), including cell out contribution uncertainties (energy deposits not associated with jets, electrons, $\tau$'s or photons) and soft jet uncertainties (related to objects that have a \pt too low to be considered a jet).  The pileup uncertainty portion is a 10\% variation.  Both uncertainties typically range from $1-10\%$.

\textbf{ISR/FSR}:
There is some uncertainty on the MC simulation of the initial and final state radiation.  These are extra particles perhaps formed by gluons producing extra radiation (jets) in the initial or final state portion of the Feynman diagram.  Extra jets, if the \pt~is high enough, could move events from the 2 jet channel to the 3 jet channel and thus affect the analysis.  This uncertainty is evaluated by changing certain parameters when producing the MC, and is evaluated separately for the \ttbar~and single-top processes.  Special {\sc AcerMC} samples showered with {\sc Pythia} are used for all the top processes.  For this analysis, we vary the ISR and FSR simultaneously (which produces a larger variation than varying them separately for the largest signal channel, 2 jets with positive leptons).  This is one of the largest uncertainties, with variations of around $10-30\%$ depending on the process.

%https://twiki.cern.ch/twiki/bin/view/AtlasProtected/TopMC2010#MC10_Common_Conventions
%The variations technically are PARP (67)=0.5, PARP (64)=4*D, PARP (72)=0.5*D, PARJ(82)=2*D for ISR/FSR down and PARP (67)=6, PARP (64)=1/4*D, PARP (72)=2*D, PARJ(82)=0.5*D for ISR/FSR up, where D is the default value (PARP (67) is 4, PARP (64) is 1, PARP(72) is 0.192 GeV, PARJ(82) is 1 GeV).
%
%    PARP (67) : controls suppression of ISR branchings above the coherence scale,
%    PARP (64) : multiplies ISR alpha_strong evolution scale, the effect is \propto 1/(lambda_ISR^2),
%    PARP(72) : lambda FSR,
%    PARJ(82) : low-pt cutoff of the FSR branchings (i.e. it sets the lower pt at which hadronization takes over from parton shower). 
%
%Values of the parameters in the central sample are (in both AMBT1 and MC09 tunes): PARP (67) : 4, PARP (64) : 1, PARP(72) : 0.192 GeV, PARJ(82) : 1 GeV.
%The parameter variations are done around the default[D] values listed above for the samples as follows (the terms up and down refer to more and less Parton Shower activity respectively):
%    117259 : ISR down and FSR down : PARP (67)=0.5, PARP (64)=4*D, PARP (72)=0.5*D, PARJ(82)=2*D,
%    117260 : ISR up and FSR up : PARP (67)=6, PARP (64)=1/4*D, PARP (72)=2*D, PARJ(82)=0.5*D. 

\textbf{PDF}:
The parton distribution function may also not be well modelled.  The uncertainty is evaluated by finding the variation from changing the PDF in the preselection sample from the one used in this analysis,{\sc CTEQ6L}, to {\sc CTEQ66}~\cite{cteq6}, {\sc NNPDF20}~\cite{NNPDF1, NNPDF2}, or {\sc MSTW2008}nnlo68cl~\cite{MSTW1, MSTW2}.  This uncertainty ranges from $1\%$ to $8\%$ depending on the process.
%taken in consideration are: MSTW2008nnlo68cl, cteq66 and NNPDF20.We reweight the signal and background MC samples according to
%each of the PDF uncertainty eigenvectors and take the largest variation as the uncertainty

\textbf{Generator and Shower}:
The MC generator or showering programs may not exactly match the data.  To evaluate these uncertainties, an alternative generator or showering program is used and the deviation determined.  This is done for the \ttbar~and the single-top processes.  For the t-channel single-top process, {\sc MCFM}~\cite{MCFM} is used to determine a deviation with the preselection sample of $7\%$.  For the other processes, the generator uncertainties are determined after cut-based selections as usual, using \textsc{MC@NLO} versus \textsc{Herwig} for \ttbar~and {\textsc AcerMc} versus \textsc{MC@NLO} for the s-channel and Wt single-top processes.  The shower uncertainties for the single-top processes are determined using {\textsc AcerMc} plus \textsc{Pythia} versus \textsc AcerMc plus \textsc{Herwig}.  The \ttbar~shower uncertainties are found by comparing yields from {\textsc Powheg}~\cite{SAMPLES-POWHEG, SAMPLES-POWHEG-1} plus \textsc{Pythia} and \textsc Powheg plus \textsc{Herwig}.  These uncertainties are all symmetrized, so the deviation between the nominal and alternate program is divided in half.  One half is taken as the up shift, and the other is taken as the down shift.  These are some of the larger uncertainties in the analysis, with variations around $10-15\%$ depending on the process.

\textbf{\eta~reweighting}:
The shape of the \eta~distribution of the forward jet is not especially well-modeled.  We renormalize the MC to the data in a pretag sample and then evaluate the difference between using this and using the nominal sample after all of the analysis selections.  This uncertainty is a one-sided uncertainty (there is only a positive shift, no negative shift).  The uncertainty is about $5-10\%$ depending on the process.

\textbf{Luminosity}:
The luminosity estimate has some uncertainty associated with it.  The luminosity estimate is done with dedicated luminosity estimate runs.  The uncertainty is 3.7\%~\cite{Luminosityoverview} for the data used in this analysis.

The individual uncertainties that are used to find the total cross-section uncertainties are given in Table~\ref{tab:uncertainty-exp1}, Table~\ref{tab:uncertainty-exp2}, Table~\ref{tab:uncertainty-exp3}, and Table~\ref{tab:uncertainty-exp4} by process, where each table gives the values for a different analysis channel.  These are the values which are used in the statistical tool (see Section~\ref{sec:xs}) to determine the cross-section.  In certain cases, processes have very high MC statistical uncertainties after all cut-based selections, especially in the 3 jet channels.  This can cause some large estimates for other uncertainties as well.  Although the actual uncertainties may not be as high as we estimate, we keep the large values to be conservative.

\begin{table}[htdp]
\begin{center}
   \begin{tabular}{l|cccccc}
    \hline
Uncertainties(\%) & $t$-channel & \ttbar, Wt, s & $W$+light  &$W$+heavy & $Z$,Dib.& Multijets \\ 
\hline
Jet energy scale  & -3 & -11 & -19 & -1 & 33 & - \\ 
 & -1 & 7 & 28 & -9 & -9 & - \\ 
\hline
Jet energy resolution & $\pm$4 & $\pm$1 & - & - & $\pm$4 & - \\ 
\hline
Jet reconstruction & $<1$ & $<1$ & $\pm$2 & $<1$ & $\pm$1 & - \\ 
\hline
$b$ tagging scale factor  & 12 & 9 & 7 & -3 & 15 & - \\ 
 & -12 & -9 & -10 & 3 & -15 & - \\ 
\hline
Mistag scale factor  & $<1$ & $<1$ & 24 & -4 & 5 & - \\ 
 & $<1$ & $<1$ & -23 & 4 & -5 & - \\ 
\hline
Lepton scale factor & $\pm$3 & $\pm$3 & - & - & $\pm$2 & - \\ 
\hline
Lepton efficiencies & $\pm$1 & $\pm$1 & - & - & $\pm$4 & - \\ 
\hline
Generator single-top & $\pm$7 & $\pm$1 & - & - & - & - \\ 
\hline
Generator \ttbar & - & $\pm$1 & - & - & - & - \\ 
\hline
Shower & $\pm$11 & $\pm$12 & - & - & - & - \\ 
\hline
ISR/FSR  & -15 & 32 & - & - & - & - \\ 
 & 27 & 39 & - & - & - & - \\ 
\hline
PDF & $\pm$3 & $\pm$8 & - & - & $\pm$1 & - \\ 
\hline
Luminosity & $\pm$4 & $\pm$4 & - & - & $\pm$4 & - \\ 
\hline
\met  & -1 & -1 & -5 & -5 & -1 & - \\ 
 & $<1$ & 1 & -16 & $<1$ & $<1$ & - \\ 
\hline
\met~pileup  & -2 & -1 & -5 & -5 & -1 & - \\ 
 & $<1$ & 1 & -16 & -2 & $<1$ & - \\ 
\hline
LAr  & 1 & 1 & $<1$ & $<1$ & $<1$ & - \\ 
 & -1 & -1 & $<1$ & -1 & $<1$ & - \\ 
\hline
$\eta$ reweighting & 5 & 2 & $<1$ & 4 & 9 & - \\ 
\hline
$W$ shape & - & - & $<1$ & $<1$ & - & - \\ 
\hline
$Wjj$ norm & - & - & $<1$ & - & - & - \\ 
\hline
$Wc,cc,bb$ norm & - & - & - & $\pm$5 & - & - \\ 
\hline
$W$ 3 jet norm & - & - & - & - & - & - \\ 
\hline
Multijets & - & - & $\pm$3 & $\pm$9 & - & $\pm$50 \\ 
\hline
\ttbar~XS & - & $\pm$5 & $\pm$1 & $\pm$5 & - & - \\ 
\hline
single-top XS & - & $\pm$2 & $\pm$4 & $\pm$17 & - & - \\ 
\hline
Z+jets XS & - & - & $\pm$6 & $\pm$2 & $\pm$41 & - \\ 
\hline
Diboson XS & - & - & $<1$ & $<1$ & $\pm$2 & - \\ 
\hline
MC Statistics & $\pm$4 & $\pm$6 & $\pm$34 & $\pm$12 & $\pm$47 & $\pm$100 \\ 
\hline

   \end{tabular}
\caption{Percent systematic uncertainties for the 2 jet plus channel.  Here, XS means cross-section, $Z$ means $Z$+jets, and Dib. means diboson.  Norm refers to normalization, s indicates single-top $s$-channel.  If two values are given, the top value is the upshift and the bottom value is the downshift.}
\label{tab:uncertainty-exp1}
\end{center}
\end{table}

\begin{table}[htdp]
\begin{center}
   \begin{tabular}{l|cccccc}
    \hline
Uncertainties(\%) & $t$-channel & \ttbar, Wt, s & $W$+light  &$W$+heavy & $Z$, Dib. & Multijets \\ 
\hline
Jet energy scale  & $<1$ & -7 & -22 & 3 & -1 & - \\ 
 & -4 & 9 & -19 & -11 & -20 & - \\ 
\hline
Jet energy resolution & $\pm$3 & $\pm$1 & - & - & $\pm$30 & - \\ 
\hline
Jet reconstruction & $<1$ & $<1$ & $\pm$1 & $\pm$2 & $\pm$2 & - \\ 
\hline
$b$ tagging scale factor  & 12 & 9 & 9 & 2 & 4 & - \\ 
 & -12 & -9 & -12 & -3 & -4 & - \\ 
\hline
Mistag scale factor  & $<1$ & $<1$ & 23 & -3 & 21 & - \\ 
 & $<1$ & $<1$ & -22 & 3 & -21 & - \\ 
\hline
Lepton scale factor & $\pm$3 & $\pm$3 & - & - & $\pm$3 & - \\ 
\hline
Lepton efficiencies & $\pm$2 & $<1$ & - & - & $\pm$5 & - \\ 
\hline
Generator single-top & $\pm$7 & $<1$ & - & - & - & - \\ 
\hline
Generator \ttbar & - & $\pm$9 & - & - & - & - \\ 
\hline
Shower & $\pm$14 & $\pm$5 & - & - & - & - \\ 
\hline
ISR/FSR  & -14 & -16 & - & - & - & - \\ 
 & 24 & 25 & - & - & - & - \\ 
\hline
PDF & $\pm$3 & $\pm$8 & - & - & $\pm$1 & - \\ 
\hline
Luminosity & $\pm$4 & $\pm$4 & - & - & $\pm$4 & - \\ 
\hline
\met  & -4 & $<1$ & -7 & -13 & $<1$ & - \\ 
 & $<1$ & 2 & $<1$ & 2 & 1 & - \\ 
\hline
\met~pileup  & -3 & $<1$ & -7 & -11 & $<1$ & - \\ 
 & $<1$ & 2 & $<1$ & 2 & $<1$ & - \\ 
\hline
LAr  & $<1$ & 1 & $<1$ & $<1$ & $<1$ & - \\ 
 & -1 & -1 & -1 & -3 & $<1$ & - \\ 
\hline
$\eta$ reweighting & 4 & 2 & 4 & 4 & 4 & - \\ 
\hline
$W$ shape & - & - & $\pm$3 & $\pm$2 & - & - \\ 
\hline
$Wjj$ norm & - & - & $<1$ & - & - & - \\ 
\hline
$Wc,cc,bb$ norm & - & - & - & $\pm$4 & - & - \\ 
\hline
$W$ 3 jet norm & - & - & - & - & - & - \\ 
\hline
Multijets & - & - & $\pm$3 & $\pm$6 & - & $\pm$50 \\ 
\hline
\ttbar~XS & - & $\pm$6 & $\pm$1 & $\pm$1 & - & - \\ 
\hline
single-top XS & - & $\pm$2 & $\pm$4 & $\pm$5 & - & - \\ 
\hline
Z+jets XS & - & - & $\pm$6 & $\pm$2 & $\pm$40 & - \\ 
\hline
Diboson XS & - & - & $<1$ & $<1$ & $\pm$2 & - \\ 
\hline
MC Statistics & $\pm$6 & $\pm$6 & $\pm$45 & $\pm$15 & $\pm$55 & $\pm$100 \\ 
\hline

   \end{tabular}
\caption{Percent systematic uncertainties by process for the 2 jet minus channel.  Here, XS means cross-section, $Z$ means $Z$+jets, and Dib. means diboson.   Norm refers to normalization, s indicates single-top $s$-channel.  If two values are given, the top value is the upshift and the bottom value is the downshift. }
\label{tab:uncertainty-exp2}
\end{center}
\end{table}

\begin{table}[htdp]
\begin{center}
   \begin{tabular}{l|cccccc}
    \hline
Uncertainties(\%) & $t$-channel & \ttbar, Wt, s & $W$+light  &$W$+heavy & $Z$, Dib. & Multijets \\ 
\hline
Jet energy scale  & 4 & -8 & 18 & 37 & -32 & - \\ 
 & -10 & 16 & -17 & 17 & -40 & - \\ 
\hline
Jet energy resolution & $<1$ & $<1$ & - & - & $\pm$10 & - \\ 
\hline
Jet reconstruction & $\pm$1 & $<1$ & $\pm$2 & $\pm$2 & $\pm$3 & - \\ 
\hline
$b$ tagging scale factor  & 9 & 7 & 2 & -7 & 11 & - \\ 
 & -9 & -8 & -4 & 9 & -11 & - \\ 
\hline
Mistag scale factor  & $<1$ & $<1$ & 33 & -4 & 1 & - \\ 
 & $<1$ & $<1$ & -32 & 3 & -1 & - \\ 
\hline
Lepton scale factor & $\pm$3 & $\pm$3 & - & - & $\pm$2 & - \\ 
\hline
Lepton efficiencies & $\pm$1 & $\pm$1 & - & - & $\pm$9 & - \\ 
\hline
Generator single-top & $\pm$7 & $<1$ & - & - & - & - \\ 
\hline
Generator \ttbar & - & $\pm$22 & - & - & - & - \\ 
\hline
Shower & $\pm$7 & $\pm$8 & - & - & - & - \\ 
\hline
ISR/FSR  & -5 & 4 & - & - & - & - \\ 
 & -1 & 22 & - & - & - & - \\ 
\hline
PDF & $\pm$3 & $\pm$8 & - & - & $\pm$1 & - \\ 
\hline
Luminosity & $\pm$4 & $\pm$4 & - & - & $\pm$4 & - \\ 
\hline
\met  & -1 & -1 & -98 & 5 & $<1$ & - \\ 
 & 3 & -2 & $<1$ & -3 & $<1$ & - \\ 
\hline
\met~pileup  & 1 & $<1$ & -98 & 5 & $<1$ & - \\ 
 & 1 & -2 & $<1$ & -5 & $<1$ & - \\ 
\hline
LAr  & $<1$ & 2 & $<1$ & $<1$ & $<1$ & - \\ 
 & -2 & -1 & -2 & $<1$ & $<1$ & - \\ 
\hline
$\eta$ reweighting & 6 & 4 & 4 & 7 & 5 & - \\ 
\hline
$W$ shape & - & - & $\pm$2 & $\pm$1 & - & - \\ 
\hline
$Wjj$ norm & - & - & $\pm$1 & - & - & - \\ 
\hline
$Wc,cc,bb$ norm & - & - & - & $\pm$5 & - & - \\ 
\hline
$W$ 3 jet norm & - & - & $\pm$25 & $\pm$25 & - & - \\ 
\hline
Multijets & - & - & $\pm$6 & $\pm$15 & - & $\pm$50 \\ 
\hline
\ttbar~XS & - & $\pm$7 & $\pm$3 & $\pm$14 & - & - \\ 
\hline
single-top XS & - & $<1$ & $\pm$2 & $\pm$38 & - & - \\ 
\hline
Z+jets XS & - & - & $\pm$7 & $\pm$3 & $\pm$40 & - \\ 
\hline
Diboson XS & - & - & $<1$ & $<1$ & $\pm$2 & - \\ 
\hline
MC Statistics & $\pm$7 & $\pm$6 & $\pm$70 & $\pm$23 & $\pm$67 & $\pm$48 \\ 
\hline

   \end{tabular}
\caption{Percent systematic uncertainties by process for the 3 jet plus channel.  Here, XS means cross-section, $Z$ means $Z$+jets, and Dib. means diboson.    Norm refers to normalization, s indicates single-top $s$-channel.  If two values are given, the top value is the upshift and the bottom value is the downshift.}
\label{tab:uncertainty-exp3}
\end{center}
\end{table}

\begin{table}[htdp]
\begin{center}
   \begin{tabular}{l|cccccc}
    \hline
Uncertainties(\%) & $t$-channel & \ttbar, Wt, s & $W$+light  &$W$+heavy & $Z$, Dib. & Multijets \\ 
\hline
Jet energy scale  & 7 & -6 & -39 & 73 & $<1$ & - \\ 
 & -21 & 14 & -26 & 24 & 55 & - \\ 
\hline
Jet energy resolution & $\pm$1 & $\pm$6 & - & - & $\pm$12 & - \\ 
\hline
Jet reconstruction & $<1$ & $\pm$1 & $\pm$2 & $\pm$1 & $\pm$10 & - \\ 
\hline
$b$ tagging scale factor  & 9 & 6 & 6 & -3 & 18 & - \\ 
 & -9 & -8 & -7 & 3 & -18 & - \\ 
\hline
Mistag scale factor  & $<1$ & $<1$ & 16 & -4 & $<1$ & - \\ 
 & $<1$ & $<1$ & -16 & 4 & $<1$ & - \\ 
\hline
Lepton scale factor & $\pm$3 & $\pm$3 & - & - & $\pm$98 & - \\ 
\hline
Lepton efficiencies & $\pm$3 & $\pm$2 & - & - & $\pm$17 & - \\ 
\hline
Generator single-top & $\pm$7 & $<1$ & - & - & - & - \\ 
\hline
Generator \ttbar & - & $\pm$30 & - & - & - & - \\ 
\hline
Shower & $\pm$10 & $\pm$50 & - & - & - & - \\ 
\hline
ISR/FSR  & 29 & 40 & - & - & - & - \\ 
 & 30 & 22 & - & - & - & - \\ 
\hline
PDF & $\pm$3 & $\pm$8 & - & - & $<1$ & - \\ 
\hline
Luminosity & $\pm$4 & $\pm$4 & - & - & $\pm$4 & - \\ 
\hline
\met  & -4 & $<1$ & $<1$ & $<1$ & $<1$ & - \\ 
 & -4 & $<1$ & $<1$ & $<1$ & $<1$ & - \\ 
\hline
\met~pileup  & -4 & -1 & $<1$ & -1 & $<1$ & - \\ 
 & -3 & -1 & $<1$ & $<1$ & $<1$ & - \\ 
\hline
LAr  & $<1$ & 1 & $<1$ & $<1$ & $<1$ & - \\ 
 & -1 & -2 & $<1$ & $<1$ & $<1$ & - \\ 
\hline
$\eta$ reweighting & 5 & 5 & 6 & 4 & 5 & - \\ 
\hline
$W$ shape & - & - & $\pm$3 & $\pm$1 & - & - \\ 
\hline
$Wjj$ norm & - & - & $\pm$1 & - & - & - \\ 
\hline
$Wc,cc,bb$ norm & - & - & - & $\pm$5 & - & - \\ 
\hline
$W$ 3 jet norm & - & - & $\pm$25 & $\pm$25 & - & - \\ 
\hline
Multijets & - & - & $\pm$6 & $\pm$12 & - & $\pm$50 \\ 
\hline
\ttbar~XS & - & $\pm$6 & $\pm$3 & $\pm$10 & - & - \\ 
\hline
single-top XS & - & $<1$ & $\pm$2 & $\pm$26 & - & - \\ 
\hline
Z+jets XS & - & - & $\pm$7 & $\pm$3 & $\pm$60 & - \\ 
\hline
Diboson XS & - & - & $<1$ & $<1$ & $<1$ & - \\ 
\hline
MC Statistics & $\pm$11 & $\pm$6 & $\pm$66 & $\pm$35 & $\pm$98 & $\pm$39 \\ 
\hline

   \end{tabular}
\caption{Percent systematic uncertainties by process for the 3 jet minus channel.  In this table, XS means cross-section, $Z$ means $Z$+jets, and  Dib. means diboson.   Norm refers to normalization, s indicates single-top $s$-channel.  If two values are given, the top value is the upshift and the bottom value is the downshift. }
\label{tab:uncertainty-exp4}
\end{center}
\end{table}

\subsection{Effect of Pileup}
%mention num track requirement for Nvertices?
For this study, there are on average about 6 interactions per crossing (primary vertices), and it is possible that extra events could cause problems at  the reconstruction level when identifying the primary vertex or reconstructing jets.  To determine the impact of pileup on this analysis, the MC was divided into two samples based on the number of primary vertices in the event, where high pileup is considered to be $\geq 6$ primary vertices and low pileup is considered to be $< 6$ primary vertices.  The sample is divided before any selections and then normalized to the expected yields in both cases.  The analysis is repeated using each sample, and we find that the cross-section shifts by 6\% versus nominal when using the high pileup sample and 4\% versus nominal when using the low pileup sample.  This is within the statistical uncertainty of the analysis and also within the MC statistical uncertainty, which increases when the sample is halved.  Based on this study, we consider the analysis to be insensitive to pileup effects.

\section{Results}
In this section we discuss the technique used to determine the observed cross-section and the $|V_{tb}|$ value.  We discuss five different results, involving different combinations of the four channels considered based on the number of jets and lepton charge: 2 jets with a positively charged lepton, 2 jets with a negatively charged lepton, 3 jets with a positively charged lepton, 3 jets with a negatively charged lepton.  These combinations are 2 jets, 3 jets, plus (positively charged lepton), minus (negatively charged lepton), and all four channels combined.  The measurement from the combination of the four channels leads to the primary analysis result.

\subsection{Cross-section Calculation and Measurements}\label{sec:xs}
As mentioned earlier, the cross-section is related to the number of observed events.  However, multiple analysis channels and a variety of uncertainties make the calculation more complicated that simply subtracting the expected background yield from the data and finding the deviation of this value from the expected signal yield.  The cross-section calculation is performed using a statistical tool called BILL (Binned Log Likelihood Fitter)~\cite{Wolfgang}, used previously for a neural network single-top analysis~\cite{ATLAS-CONF-2011-101}.  

The cross-section is determined via a maximum likelihood fit of the MC to the data, allowing different yields to float by different amounts within a range related to a Gaussian constraint term.  Scale factors ($\beta$) are determined for each process, where these scale factors are the ones that give the best fit to the data, for all channels considered.  The data-based $W$+jets and multijet estimates are not allowed to vary at all, while the other non-signal processes may float within their theoretical cross-section uncertainties.  The signal yield has no restrictions.  The fit is based on a product of Poisson likelihoods for each channel which is multiplied by the product of the Gaussian constraints for all the backgrounds.  The Gaussian distributions account for our prior knowledge of the backgrounds, and have a mean of 1 and a width of the theoretical uncertainty variation (~0 for data-based backgrounds, the theoretical uncertainty for other backgrounds).  

Because this analysis is a cut-and-count type of analysis, each channel has a distribution which is just one bin, each measurement uses 2 or 4 channels, and the fit itself is very straight-forward.  The results of the fit are given in Table~\ref{tab:beta}, where these values are scale factors to be multiplied onto the MC to get the observed yield.  These factors are the output from the BILL tool.  Because the data-based backgrounds have a $\beta$ value defined to be 1, and the other backgrounds have low theoretical uncertainties, the only $\beta$ values that are not approximately 1 are those for the signal.  The t-channel factor is multiplied by the expected cross-section to obtain the observed cross-section.

\begin{table}[htdp]
\begin{center}
   \begin{tabular}{l|cccccc}
    \hline
Channels & $t$-channel & \ttbar, Other top & $W$+light  &$W$+heavy & $Z$+jets, Diboson & Multijets \\
    \hline
All Channels & 1.41843 & 0.99361 & 1.000000 & 1.000000 & 1.00834 & 1.00000 \\
2 Jets & 1.55434 & 0.99848 & 1.000000 & 1.000000 & 0.99740 & 1.00000 \\
3 Jets & 1.05444 & 1.00790 & 1.000000 & 1.000000 & 1.00147 & 1.00000 \\
Plus Charge & 1.40058 & 0.99168 & 1.000000 & 1.000000 & 1.00654 & 1.00000 \\
Minus Charge & 1.46533 & 1.00001 & 1.000000 & 1.000000 & 1.00005 & 1.00000 \\
   \hline
   \end{tabular}
\caption{The fit values by process and channel.  The 2 or 3 jet channels include both lepton charges, and the lepton charge channels include both 2 and 3 jet events.  All channels is the combination of plus and minus lepton charge events, with 2 or 3 jets.}
\label{tab:beta}
\end{center}
\end{table}

This tool uses a frequentist method to determine the cross-section uncertainties, meaning many (100,000) different pseudo-experiments are generated based on the input yield and uncertainties.  In this way, all the various possibilities within uncertainties are explored and a distribution reflecting the probability of all possible outcomes is created, where the RMS reflects the overall combined uncertainty of the measurement.  The number of events in each pseudo-experiment are determined via a Poisson distribution with a mean of the expected yield and the uncertainties varied by Gaussian distributed random numbers.  There is also a factor related specifically to the theoretical uncertainties of the backgrounds, as was the case for the fit to determine the $\beta$ values, but again this has a small impact on the result.

The results of all of these repetitions are displayed in a distribution like in Figure~\ref{fig:betafit}, where the total cross-section uncertainty is derived from the mean and the RMS of the distribution.  For the observed uncertainty, the yields are scaled by the fit values, so $\beta$ is now 1.  The deviation of the mean from 1 is the bias (representing the asymmetry of the uncertainties), and this added in quadrature with the RMS gives one side of the uncertainty, while the RMS alone (0 bias assumption) gives the other uncertainty shift.  In other words, the uncertainty is the $\sqrt{RMS^2}$ or $\sqrt{ (1- mean)^2 + RMS^2 }$.  For the example in Figure~\ref{fig:betafit}, which uses observed yields for all four channels and includes all of the uncertainties, the RMS is 0.284 and the mean is 1.133, giving uncertainties of +31\% and -28\%.

 \begin{figure}[!h!tpb]
 \centering
 \includegraphics[width=0.75\textwidth]{figures/variables/PaperFinal_Beta_allsysstat_allchan_atlaswip.eps}
 \caption{Pseudo-experiment distribution used for the final cross-section uncertainty determination.  This distribution is for the observed cross-section uncertainty, for all channels combined.  The $\beta$ value is the fit for a given pseudo-experiment with yields scaled by the values in Table~\ref{tab:beta}, and the uncertainty is determined from the distribution RMS and deviation of the mean from 1.}
 \label{fig:betafit}
 \end{figure}


\subsubsection{Two and Three jet Single Top Quark t-channel Production}
We can combine our four channels into sets of 2 jets and 3 jets (lepton charges are combined).  When this is done we find a cross-section of  $\sigma_{t}= 100 ^{+9}_{-9} \mathrm{(stat)} ^{+32}_{-31} \mathrm{(syst)} = 100^{+33}_{-32}$~pb for 2 jets, where the expected cross-section is $\sigma_{t}= 65^{+23}_{-23}$~pb, and $\sigma_{t}= 68 ^{+13}_{-13} \mathrm{(stat)} ^{+28}_{-22} \mathrm{(syst)} = 68^{+30}_{-25}$~pb for 3 jets, where the expected cross-section is $\sigma_{t}= 65^{+30}_{-24}$~pb.  Both results are consistent with the standard model value within two standard deviations and consistent with each other within uncertainties.
%CHECK THIS LAST STATEMENT

\subsubsection{Positively and Negatively Charged Single Top Quark t-channel Production}
One can also combine the four channels into a positive and negative lepton charge sample.  Because the top quark decays to a W and b (and then the W decays to a lepton on neutrino) without hadronizing, the charge information from the top quark is preserved in the lepton.  Therefore, the positively charged lepton channel measurement corresponds to a measurement of the positively charged top quark portion of the t-channel single-top cross-section.  There is a separate theoretical prediction for the top and anti-top portions of the cross-section, given in Section~\ref{sec:ExpectedCrosssections}.  The results of this measurement are $\sigma_{t^{+}}= 59 ^{+6}_{-6} \mathrm{(stat)} ^{+17}_{-16} \mathrm{(syst)} = 59^{+18}_{-16}$~pb for top (positive lepton charge), where the expected cross-section is $\sigma_{t^{+}}= 42^{+14}_{-13}$~pb.  The measurement is $\sigma_{t^{-}}= 33 ^{+5}_{-5} \mathrm{(stat)} ^{+12}_{-11} \mathrm{(syst)} = 33^{+13}_{-12}$~pb for anti-top (negative lepton charge), where the expected cross-section is $\sigma_{t^{-}}= 23^{+10}_{-10}$~pb.

\subsubsection{Combined t-channel Production Cross-section Result}\label{sec:fourchanresult}
Finally, all four channels can be combined, and this is the final reported total cross-section result for this study.  The observed t-channel single-top cross-section is $\sigma_{t}= 92 ^{+7}_{-7} \mathrm{(stat)} ^{+28}_{-25} \mathrm{(syst)} = 92^{+29}_{-26}$~pb, where $\sigma_{t}= 65^{+22}_{-20}$~pb is expected.  This is consistent with the standard model and within two standard deviations of the theoretical single-top t-channel cross-section.  
%Maybe include a plot or table of how this matches up with other measurements from other experiments, the paper?

Table~\ref{tab:xs-uncertainty-exp} shows a breakdown of the systematic uncertainties and their contribution to the expected cross-section measurement for the combination of all four channels and Table~\ref{tab:xs-uncertainty-obs} shows the same but for the observed data.  The data statistical uncertainty is much lower than the systematic uncertainties, meaning that this cross-section measurement is dominated by systematic uncertainties.  The largest uncertainties for this analysis are ISR/FSR, shower/generator, and b-tagging uncertainties.  

The ISR/FSR uncertainty may decrease in future analyses as this is studied further and the level of variation required is better understood.  The b-tagging uncertainty will also likely improve in future analyses as more data are collected and the b-tagging efficiencies and scale factors are better estimated.  The shower/generator uncertainty is unlikely to change very much until shower/generator programs are updated.  On the other hand, the MC statistical uncertainty will become more of an issue in future analyses.  As the data statistics increase, the number of MC events that must be generated increases.  This means that the MC statistical uncertainty will increase in future analyses unless they are altered to use looser selections or faster MC generation methods.

\begin{table}[htdp]
\begin{center}
   \begin{tabular}{l|c}
    \hline
%    Source & \multicolumn{2}{c}{$\Delta\sigma/\sigma$ [\] }\\
%           & cut-based (2jet)  & cut-based (3jet)  \\
    Source & $\Delta\sigma/\sigma$ (\%) \\
    \hline \hline
    Data statistics              & +10/-10  \\
    MC statistics                & +6/-6  \\
    \hline  
    $b$ tagging scale factor     & +13/-13  \\
    Mistag scale factor          & +1/-1  \\
    Lepton scale factor          & +3/-3  \\
    Lepton efficiencies          & +1/-1  \\
    Jet energy scale             & +2/-3  \\
    Jet energy resolution        & +2/-2  \\ %or 3 3
    Jet reconstruction           & +1/-1  \\
    $W$ shape                    & +1/-1  \\
    $Wjj$ normalization          & +1/-1   \\
    $Wc,cc,bb$ normalization     & +2/-2   \\
    $W$ 3 jet normalization      & +2/-2  \\
    $\eta$ reweighting           & +8/-5  \\
    \met                         & +1/-2  \\
    \met~pileup                  & +1/-2  \\
    LAr                          & +1/-1  \\
    PDF                          & +5/-5  \\
%    Generator, $t\bar{t}$        & +3/-3  \\
%    Generator, single-top        & +7/-7  \\
    Generator                    & +8/-8  \\
    Shower                       & +12/-11  \\ %or 12 12
    ISR/FSR                      & +21/-19  \\
    Theory cross-section         & +7/-7  \\
%aka QCD
    Multijets                    & +3/-3  \\
    Luminosity                   & +5/-5  \\
    \hline  
    Total Systematics            & +33/-29  \\
    Total                        & +34/-31  \\
    \hline
   \end{tabular}
\caption{Systematic uncertainties for the expected $t$-channel cross-section measurement, where the final line includes all systematic uncertainties and the data statistical uncertainty.}
\label{tab:xs-uncertainty-exp}
\end{center}
\end{table}


\begin{table}[htdp]
\begin{center}
   \begin{tabular}{l|c}
    \hline
%    Source & \multicolumn{2}{c}{$\Delta\sigma/\sigma$ [\] }\\
%           & cut-based (2jet)  & cut-based (3jet)  \\
    Source & $\Delta\sigma/\sigma$ (\%) \\
    \hline \hline
    Data statistics              & +8/-8  \\
    MC statistics                & +4/-4  \\
    \hline  
    $b$ tagging scale factor     & +12/-12  \\
    Mistag scale factor          & +1/-1  \\
    Lepton scale factor          & +3/-3  \\
    Lepton efficiencies          & +2/-2  \\
    Jet energy scale             & +2/-3  \\
    Jet energy resolution        & +2/-2  \\
    Jet reconstruction           & +1/-1  \\
    $W$ shape                    & +1/-1  \\
    $Wjj$ normalization          & +1/-1   \\
    $Wc,cc,bb$ normalization     & +2/-2   \\
    $W$ 3 jet normalization      & +2/-2  \\
    $\eta$ reweighting           & +7/-5  \\
    \met                         & +1/-2  \\
    \met~pileup                  & +1/-1  \\
    LAr                          & +1/-1  \\
    PDF                          & +4/-4  \\
%    Generator, $t\bar{t}$        & +3/-3  \\
%    Generator, single-top        & +7/-7  \\
    Generator                    & +7/-7  \\
    Shower                       & +11/-11  \\
    ISR/FSR                      & +19/-18  \\
    Theory cross-section         & +5/-5  \\
%aka QCD
    Multijets                    & +2/-2  \\
    Luminosity                   & +4/-4  \\
    \hline  
    Total Systematics            & +30/-27  \\
    Total                        & +31/-28  \\
    \hline
   \end{tabular}
\caption{Systematic uncertainties for the observed $t$-channel cross-section measurement, where the final line includes all systematic uncertainties and the data statistical uncertainty.}
\label{tab:xs-uncertainty-obs}
\end{center}
\end{table}


\subsection{Estimate of $|V_{L}|$}
As discussed in Section~\ref{sec:vtb}, the CKM matrix element $|V_{L}|$ can be directly estimated from t-channel single-top production using the ratio of the observed and standard model cross-section.  We may write (based on Equation~\ref{eq:vtbxs}), where $\sigma$ is the cross-section, obs refers to the observed value and sm refers to the standard model:
\begin{equation} |V_{L, obs}|~ = ~\sqrt{~\frac{~\sigma_{obs}}{~\sigma_{sm}}} ~|V_{L, sm}|
\end{equation}
or, with $|V_{L, sm}| = |V_{tb}| = 1$ from the standard model we obtain,
\begin{equation} |V_{L, obs}|~ = ~\sqrt{~\frac{~\sigma_{obs}}{~\sigma_{sm}}}
\end{equation}
Performing the calculation to propagate the uncertainties gives
\begin{equation} \delta_{VL,obs} ~= ~\frac{V_{L, obs}}{2}~\sqrt{(\frac{~\delta_{obs}}{\sigma_{obs}})^2~+~(\frac{~\delta_{sm}}{\sigma_{sm}})^2}
\end{equation}
where $\delta$ refers to the uncertainty.  Thus, we obtain a value of $|V_{L, obs}|~ =~1.19^{+0.20}_{-0.18}$ for the main, four channel combination.  In this case we used 10\% for the theoretical cross-section uncertainty, as was done for the other single-top processes during the cross-section determination.  This result is consistent with the standard model value of 1.0 (and thus being simply the standard model $|V_{tb}|$) within two standard deviations.

It is also possible to determine a lower 95\% confidence level limit on the $|V_{tb}|$ value, assuming a standard model upper value of 1.  We form a Gaussian with a mean of 1.42 and use the uncertainty given for the combined result.  We integrate from 1 towards 0, taking the limit to be the point where 95\% of the curve has been integrated. With this standard model assumption, we find $|V_{tb}|> 0.67$ observed.

\subsection{Comment on Significance}
It is fairly straightforward to determine a significance using a frequentist tool like BILL.  One simply determines the likelihood of a background-only hypothesis fluctuating to imitate the signal hypothesis by generating many, many pseudo-experiments and seeing how likely this is to happen.  Pseudo-experiments are generated as described in Section~\ref{sec:xs}.  The calculation is done by determining the value $-2ln(Q)$, the test statistic (also known as the log-likelihood ratio or LLR).  A fit is done for a given ensemble to determine how likely it is that the ensemble satisfies the background only ($H_{B}$) or signal plus background ($H_{SB}$) hypotheses.  The ratio of the probabilities is the Q in the LLR value:
 \begin{equation} Q = \frac{p(H_{SB})}{p(H_{B})}
\end{equation}

The significance is determined by seeing how many background-only pseudo-experiments have a LLR value that is greater than the mean of the pseudo-experiments which assume a signal plus background (standard model) hypothesis.  This is done by finding the mean of the signal plus background LLR distribution and finding how many background only ensembles have LLR values above this mean, compared to the total number of background only ensembles.  In this way, the probability of the background fluctuating to look like the signal, and thus the significance, can be determined.

The difficulty with this method is that around and especially above the 5 sigma significance level (the level at which observation is typically claimed in high energy physics), the number of pseudo-experiments needed can become very, very large (10 million to 100 million or more).  For this dissertation, the result has been shown to be above 5 sigma previously~\cite{ATLAS-CONF-2011-088, ATLAS-CONF-2011-101}  with less data and larger systematic uncertainties, so we do not repeat this for the main result.  We do demonstrate the individual expected results for two channels below 5 sigma, the 3 jet and negative charge channels.  In high energy physics, discovery requires a 5 sigma significance, or a p-value (probability) of the background fluctuating to look like signal of 0.0000006.  Evidence requires 3 sigma significance, or a p-value of 0.003.  

The LLR distributions for the three jet and negative charge channels are shown in Figure~\ref{fig:significance}.  About 800000 pseudo-data sets were created in each case.  For the three jet channel, the expected significance is 3.8 sigma with a p-value of background fluctuating to look like a standard model signal of $8\times 10^{-5}$.  For the minus charge channel, the expected significance is 4.1 sigma with a p-value of background fluctuating to look like a standard model signal of $2\times 10^{-5}$.

 \begin{figure}[!h!tpb]
 \centering
 \includegraphics[width=0.49\textwidth]{figures/result/pvalue_3jetsnew3.eps}
 \includegraphics[width=0.49\textwidth]{figures/result/pvalue_minusnew3.eps}
 \includegraphics[width=0.49\textwidth]{figures/result/SigLegend.eps}
 \caption{Distribution used to determine the expected significance for the 3 jet channel (all lepton charges are allowed) and negative charge channel (2 and 3 jets allowed).  The two curves are ensembles with and without the assumption of a standard model signal.  The vertical line shows the mean of the standard model signal and background distribution.}
 \label{fig:significance}
 \end{figure}


