\chapter{Data Based Cross-check of \ttbar~ Background}\label{app:ttbar}
In the main text we have discussed the data-based estimation of the multijets and W+jets background processes.  However, there is another large background, \ttbar, where the theoretical cross-section was used for the normalization.  It is also possible to do a data-based estimate of this background.  In this section, we review one straight-forward way to do this estimate.  The \ttbar~estimate discussed here is not used in the analysis described in the main text, but instead is intended as a cross-check of the value used (1.0) and its uncertainty.

To determine the \ttbar~background, we define orthogonal off-signal regions (as we did for the W+jets estimate).  All preselection requirements except for the number of jets and b-tagged jets selections are applied.  We also require the total t-channel yield to be $\rm <6\%$ of the total, and apply as few selections as possible beyond the preselection (with different numbers of jets and b-tagged jets).  Two \ttbar~dominated selections are defined as the number of jets equal to four or more with 1 b-tagged jet, and the 2 b-tagged jets selection with at least 2 jets.  Both of these regions are also discussed as potential signal regions in Section~\ref{alternativechannels}, so if this is done, the regions used for signal determination would need to be removed, just as the 2 jet signal kinematic region was removed from the W+jets estimate.  Additionally, the 3 jet region is considered but this region has a relatively large (~8\%) portion of t-channel single-top events (hence its use as a signal channel) relative to the other \ttbar~determination regions.  In this case, an additional selection must be introduced to control the amount of t-channel and to exclude the signal region.  Here, we choose to require the reconstructed top mass to be $> 210$ GeV.  This removes single-top events, which are more likely to have the b-quark, lepton, and missing energy associated with a top quark all correctly identified and thus an invariant mass of these particles closer to the top quark mass value.

We also select channels based on the event having a muon or electron as the selected lepton.  Thus we have 6 different channels in total, which are each of the following selections with a muon selection or electron selection:

\begin{list} {*} {}
\item 2 b-tagged jets, at least 2 jets
\item 1 b-tagged jet, at least 4 jets
\item 1 b-tagged jet, exactly 3  jets, $M_{top}(\rm l\nu b)>210\GeV$
\end{list}

These channels are all orthogonal to each other.  We can consider each result from these six channels as a separate experiment and combine them.  First, we calculate a scale factor.  This is defined as:
\begin{equation} \rm SF_{\ttbar} = \frac{Data - MC(not~\ttbar)}{MC(\ttbar)} 
\end{equation}
If the data and the Monte Carlo were to agree exactly, then the \ttbar~SF woult be 1.0.  To find the combined statistical uncertainty for the channels, we follow the method discussed by Lyons~\cite{Lyons, Lyonspaper}.  The statistical uncertainty is written as $\sigma$, where the square of this value is the variance ($\sigma^2$).  This $\sigma$ should not be confused with the cross-section.  The combination is found as follows, using i for the different channels:
\begin{equation} \rm \frac{1}{\sigma_{tot}^2} = \sum \frac{1}{\sigma_i^2} 
\label{eq:uncert}
\end{equation}

The scale factors themselves are also combined for the different channels as discussed by Lyons.  We weight each channel by the inverse of the statistical uncertainty squared.  The combination of scale factors is done as follows, where SF is the scale factor.  Notice the denominator is just $\rm \frac{1}{\sigma_{tot}}$ from Equation~\ref{eq:uncert}:
\begin{equation} \rm SF_{tot} = \frac{\sum \frac{SF_i}{\sigma_i^2}}{\sum \frac{1}{\sigma_i^2}}
\label{eq:ttbarSF}
\end{equation}

To determine the systematic uncertainties for each channel, the SF is estimated using MC values shifted due to a given uncertainty.  For a given systematic uncertainty scenario, the systematic-shifted SF are combined for each of the channels as in the nominal sample, again using the same statistical uncertainty as a weight as in Equation~\ref{eq:ttbarSF}.  The deviation between the combined nominal SF and the combined uncertainty-shifted SF is then the uncertainty for the combination due to the systematic in question.  By determining the combined systematic uncertainties this way, correlations between the channels can be properly included.  Finally, all of the systematic uncertainties and the data statistical uncertainty are added in quadrature to obtain the overall uncertainty.  The systematic uncertainties which are considered are the b-tagging scale factor, mis-tagging scale factor, and jet energy scale.  Other systematic uncertainties are neglected for the purposes of this cross-check.  

%This method assumes that the uncertainties between channels are uncorrelated.  This is true for the data statistical uncertainty, but uncertainties like the jet energy scale and b-tagging scale factor uncertainty are correlated between the channels.  However, the purpose of this cross-check is to show the \ttbar~estimate and uncertainty used in the analysis is consistent with what we see in data, and this only requires us to show that the scale factor is consistent with one, with uncertainties larger than those in the analysis.  In this analysis, the deviations for a given uncertainty have the same sign (or nearly so) between channels, with minor exceptions, implying that including correlation effects will not reduce the overall uncertainty significantly (and would likely increase it).  In general, it is possible correlations between channels could reduce an uncertainty compared to its value without correlation effects, but this occurs when deviations in different channels have an opposite sign.  This is related to the total deviation of completely correlated uncertainties being the sum of the individual deviations, rather than being the individual deviations added in quadrature.

The multijet estimate for these channels is done using the values from Section~\ref{sec:multijets} and multiplying by the fraction of multijets in the 3 jet bin and the desired bin to get the multijet total for the desired bin.  The multijet estimate from the 3 jet bin is thus propogated into the other regions considered for the \ttbar~scale factor estimate by assuming the multijets number of jets distribution is correct.  For the 3 jet \ttbar~region, which makes a selection on the reconstructed top mass, this same selection is simply applied to the 3 jets multijet sample.  No new fits to the \met~distribution are performed for the \ttbar~scale factor estimtae.  These values are approximate and the uncertainty related to propagating the multijet yield into different bins is neglected for this cross-check.  The values are given in Table~\ref{tab:QCDttbar}.
\begin{table}[!h]
  \begin{center}
    \begin{tabular}{lcc}
      \hline
      \hline
      & \multicolumn{2}{c}{Tagged events} \\
      Selection & e channel & $\mu$ channel \\
      \hline
      2 b-tag jets, $\ge$ 2 jets & $ 36.6 $ & $ 18.7 $ \\
      1 b-tag jet, $\ge$ 4 jets &  $ 302.6 $ & $ 144.3 $ \\
      1 b-tag jet, 3  jets, $M_{top}(\rm l\nu b)>210\GeV$& $ 78.4 $ & $ 35.7 $ \\
      \hline
      \hline
    \end{tabular}
    \caption{\label{tab:QCDttbar} Estimate of multijet yields for different selections, separated by lepton type.}
  \end{center}
\end{table}

The W+jets estimate for these channels uses the data-based 3 jet bin heavy flavor fractions from Section~\ref{sec:wjets} for the 1 b-tagged jet, exactly 3  jets, $M_{top}(\rm l\nu b)>210\GeV$ channel.  This is because there is a selection used for this channel which impacts processes differently.  For the other two channels, we use the W+jets data based normalizations appropriate for the number of jets in question, including a normalization of 0.75 for the 4 jets channel.  The heavy flavor fractions do not have an impact on the scale factor determination for these two channels, as there are no selections beyond preselection applied.

The scale factors and their uncertainties for each of the six channels plus the combination of channels by lepton type, and then all channels, can be seen in Figure~\ref{fig:ttbarsf}.  The statistical uncertainties are quite small, and if they are not smaller than the marker size, they are given by colored portions of the vertical uncertainty line.  Again, only the b-tagging scale factor, mis-tagging scale factor, jet energy scale and data statistical uncertainties are included in this cross-check.  The electron and muon channel combinations give very similar values, $SF_{e} = 1.10\pm0.02 \mathrm{(stat)} ^{+0.23}_{-0.14} \mathrm{(syst)} = 1.10^{+0.23}_{-0.14}$ and $SF_{\mu} = 1.13\pm0.02 \mathrm{(stat)} ^{+0.23}_{-0.15} \mathrm{(syst)} = 1.13^{+0.23}_{-0.15}$ respectively.  The final result from the combination of all six channels is $SF = 1.12\pm0.01 \mathrm{(stat)} ^{+0.23}_{-0.15} \mathrm{(syst)} = 1.12^{+0.23}_{-0.15}$.  This is consistent with a scale factor of 1.0 and the uncertainty on the result is larger than the \ttbar~theoretical cross-section uncertainty used in the main text (approximately 10\%, see Section~\ref{Sys}).  From this cross-check, it is clear that the \ttbar~estimate used in the t-channel single-top analysis is consistent with the data.

Future analyses will want to include additional uncertainties such as ISR/FSR, shower and generator uncertainties, as well as others used in the single-top t-channel cross-section estimate.  However, including these uncertainties in the current study would not change the conclusion.  As more data is taken, the uncertainties will likely become better understood and have lower values than in this study.  The data statistical uncertainty is already quite low in this study, but the b-tagging scale factor uncertainty in particular is 14\% for the six channel combination, the dominant systematic uncertainty.  If this uncertainty could be reduced, it would have a significant impact on the precision of this scale factor estimate.

 \begin{figure}[!h!tpb]
 \centering
 \includegraphics[width=1.00\textwidth]{figures/appendix/ttbarSF.eps}
 \includegraphics[width=0.50\textwidth]{figures/appendix/ttbarsfLegend.eps}
 \caption{Scale factors for \ttbar~production using six separate channels, the combination of electron channels, the combination of muon channels, and the combination of all six channels.  Statistical uncertainties are given by colored portions of the black lines unless the statistical uncertainties are so small as to be covered by the marker itself.  The black line shows the data statistical, b-tagging scale factor, mis-tagging scale factor, and jet energy scale uncertainties combined.  Other uncertainties are neglected in this cross-check.}
 \label{fig:ttbarsf}
 \end{figure}
