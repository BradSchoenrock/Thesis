\chapter{Preselection}
 \label{sec:EventSelection}
In Section~\ref{sec:ExpectedCrosssections} the cross-sections for the signal and background processes were given.  It is clear that without selections to reduce the background events relative to the signal events, single-top $t$-channel cannot be distinguished.  The haystack is large, and our needles are buried and hard to see.  The preselection selects events that have single-top t-channel-like kinematic characteristics.

To see how these selections are chosen, we examine the t-channel final state, shown in detail in Figure~\ref{fig:Feynman_tchan}.  The figure shows that there is at least one b-tagged jet and at least two jets (the b-quark from the intial gluon is not always present or detected), with one lepton and one neutrino (\met) from the decay of the W.  The scenario where a W decays to quarks is not selected for, as the final state is then all jets and is very difficult to distinguish from the large cross-section multijet background.  The multijets cross-section is so large that even requiring one lepton, despite the low lepton fake rate in this analysis, still results in a fairly large number of multijet events selected.

\begin{figure}[!h!tpb]
 \centering
 \includegraphics[width=0.45\textwidth]{figures/theory/t_channelbigNLO.eps}
\vspace{-0.5cm}
 \caption{Feynman diagram for $t$-channel single-top production, showing the final state after the top quark decay.}
 \label{fig:Feynman_tchan}
 \end{figure}

Requiring a t-channel final state incidentally helps to reduce the backgrounds, in addition to choosing events that look like our signal.  Rejecting events with less than 2 or more than 3 jets helps to reduce W+jets and \ttbar, respectively, as these processes tend to have fewer or more jets than the signal.  

The preselection used in this analysis in detail is given below.  The selection without the b-tagged jet number requirement is called the pretag selection:

\begin{list} {$\bullet$} {}
\item The event must have a quality primary vertex (has at least 5 tracks)
\item Exactly one, triggered lepton (muon or electron), matched to a reconstructed lepton object
\item The leptons must have $\pt > 25~GeV$ and muons must also have $\pt < 150~GeV$ 
\item There must be $\met > 25~GeV$
\item Two or three jets, with a one jet selection used for a sideband region
\item The jets must not be ``bad''
\item LAr quality requirements related to the LAr hole must be met
\item Data events with LAr bursts (noise) are removed
\item Triangular cut of $\met + W_{T} > 60~GeV$
\item At least one of the jets must be b-tagged
\end{list}

The primary vertex requirement helps to reduce contamination from secondary vertices or events where an extra pileup interaction (often multijet) vertex might be confused for the one we are interested in.  The lepton requirement helps to reduce multijet events, which do not have a real lepton.  The trigger requirement specifically requires the EF\_mu18 trigger for the one muon selection and EF\_e20\_medium trigger for the one electron selection.  The trigger matching ensures that the lepton in the analysis matches with a trigger-level object.  This selection has a small effect on the analysis.   Due to an issue with the MC, the muon trigger matching was not applied for the muon channel (although the trigger itself was still applied).  The \pt~requirement for the leptons is 25 GeV partly to be sufficiently away from the trigger thresholds of 18 and 20 GeV, to reduce the related uncertainty.  The upper \pt~threshold is applied due to low statistics when determining the muon scale factors in this region (the impact on the analysis from this selection is very small).  The \met~selection has a threshold similar to the lepton to reduce multijet events faking a small amount of \met.  The lower \pt~thresholds for the particles help to reduce the multijet and W+jets backgrounds, which often have lower \pt~particles.

The analysis requires 2 or 3 jets.  This chooses a final state that looks like the signal, but also helps to reduce W+jets and \ttbar, respectively, as these processes tend to have fewer or more jets than the signal.   The events must satisfy several selections to remove events which contain bad jets.  These are jets that arise due to cosmic rays, detector problems, or beam issues and the whole event is removed if it includes a bad jet.  For completeness, the cuts are as follows~\cite{BadJet, BadJet2}.  There is a bad jet if the energy fraction in the HEC is $> 0.5$ and the fraction of energy corresponding to hadronic end-cap calorimeter (HEC) cells with a cell Q-factor (related to the energy pulse shape measured versus expected) greater than 4000 is $> 0.5$ (corresponds to HEC spikes; hardware issues).  The event is rejected if the jet's energy fraction in the electromagnetic calorimeter is $> 0.95$, the fraction of energy corresponding to LAr cells with a cell Q-factor greater than 4000 is $> 0.8$, and $|\eta| < 2.8$ for the jet (EM calorimeter noise issues).  Finally, the event is rejected if the jet timing is $> 25~ns$ (indicates out-of-time jets, from a cosmic ray for instance).  The timing is the deviation of the event time from the time of energy deposition for the detector cells related to the jet, weighted by their energy squared.

Of the remaining four selections, there are two selections related to the LAr.  The first removes events where jet reconstruction is affected by the LAr hole.  The second removes events with noise bursts related to the LAr.  Finally, we apply a triangular cut which reduces the multijet background and, last, we require at least one jet to be b-tagged.
