\chapter{Single-top Production and the Standard Model}
%\chapter{Why Study Single Top-Quark Production?}

High energy physics deals with the very fundamental parts of our universe, the fundamental particles and forces.  Our present understanding is that there are four forces: gravity, electromagnetism, strong and weak.  As energies increase, it is predicted that these forces can be united into one force, starting with the electromagnetic and weak forces, which form the electroweak force.  Each force has a mediating particle, a force carrier, which governs interactions of various particles.  These particles are discussed in the next section.

Single-top production is the process where a top quark is created in an electroweak interaction.  As stated previously, the top quark is the most massive of the elemetary particles.  Only one top quark is produced in an electroweak interaction.  There is another version of top quark production using the strong force and involving a top and anti-top (\ttbar).  This was the process detected in 1995 to claim discovery~\cite{ttbardiscovery1, ttbardiscovery2}, meaning a likelihood of less than 0.0000006 of background events imitating the signal.  Single-top itself was only recently discovered~\cite{singletopdiscovery:group, singletopdiscovery:D0, singletopdiscovery:CDF} and the particular channel (t-channel) discussed in this document was separately observed in 2011~\cite{tchanneldiscovery} by the D0 collaboration at the Tevatron.  Shortly afterwards, measurements of t-channel single-top production were reported by the CMS experiment~\cite{CMStchannel} and the ATLAS experiment~\cite{ATLAS-CONF-2011-088,ATLAS-CONF-2011-101} at the LHC (see the next chapter for more details on the LHC and ATLAS).  

This single-top t-channel process is still very new and, as such, has not been fully studied.  It is possible that deviations could be found in its various fundamental properties, which could indicate new particles or anomalous parts of the standard model.  In this document, we will measure the cross-section of this process and consider its kinematics.  But first, before performing a new measurement, we must understand what is already known.

\section{The Standard Model Particles}
The standard model is the basic theory of particle physics~\cite{QuarkModelReview, PDGSummary, Griffiths, Halzen} and was formulated in the 1960's and 1970's.  It describes and predicts various particles and their properties based on symmetry relations.  The model divides the fundamental particles into several categories and subcategories, pictured in Figure~\ref{fig:SM}.  There are three major particle catagories: leptons, quarks and bosons (force carriers).  There are also three generations of the quarks and leptons, where each generation is designated by a different shade in the figure.
  
\begin{figure}[!htpb]
  \centering
    \includegraphics[width=0.50\textwidth]{figures/theory/QuarkTable.eps}
    \label{fig:SM} 
\caption{The known standard model particles.  Different generations of quarks and leptons are indicated by different shades.}
\end{figure}

\subsection{Leptons}
One category of particles are the leptons.  Leptons typically include electrons, muons, taus, and their corresponding neutrinos.  However, we will use this term in this document to generally refer to electrons, muons and/or taus, while neutrinos are considered as a separate set of particles.  The electron is a stable lepton.  The leptons do not interact via the strong force and so are not involved in hadronic bound states like the proton.  Electrons are involved in the structure of the atom.  Muons and taus are heavier than the electron and decay to other particles.  This is particularly true for the tau.  Unlike the tau, the muon survives long enough to escape our detector which is important for particle identification, but has a short decay time relative to the electron.  For each lepton there is a corresponding anti-lepton with opposite charge.  

There is also a corresponding neutrino flavor for each lepton flavor, or type (electron neutrino, muon neutrino, and tau neutrino).  Each lepton and its corresponding neutrino form a set of particles where each lepton type is considered a separate generation (there are three generations).  Neutrinos are the lightest of the known particles and have no charge.  There are no known right handed neutrinos or left handed anti-neutrinos, where right handed indicates spin and momenta are in the same direction and left handed indicates the opposite.  This difference, rather than a difference in charge, distinguishes the neutrino and anti-neutrino.

Because neutrinos are nearly massless and neutral, they are very difficult to detect.  Neutrinos usually pass right through detectors without interacting.  This makes neutrino astrophysics possible, because neutrinos from distant sources will travel from the source without interacting and scattering off clouds of matter between the source and Earth.  However, this is problematic for collider physics.  Particles from collisions need to interact with matter and deposit their energy into the detector to be detected.  While this may not always happen, it should happen nearly 100\% of the time to prevent uncertainties on the measurements from getting large.  To handle neutrinos, we don't build a dedicated neutrino detector but instead make use of event kinematics to account for the neutrino via missing energy in the event.  This will be discussed further in Section~\ref{sec:Neutrinos}.  

\subsection{Quarks}
%1/lambdaQCD = 1/200MeV ~ 10^-24 s 
% decay width 1.3 GeV/c^2 >> lambdaQCD~200 MeV
Quarks are arranged like the leptons into generations, as seen in Figure~\ref{fig:SM}.  They are different from the leptons because they can interact via the strong force and form bound states, like the proton.  There are three generations and each contains two particles, making six different flavors in total (u, d, s, c, t, b).  The first generation contains lighter quarks, up (u) and down (d), the only stable quarks.  The second generation contains strange (s) and charm (c) quarks, and the third contains the heaviest quarks, bottom (b) and top (t), which are sometimes also called beauty and truth.  The first three (u, d, s) are typically called light quarks, and the charm quark is sometimes included in this category as well, but for the purposes of this document will either be considered separately or considered to be a heavy quark.  The bottom quark is considered heavy, but the top quark though is by far the heaviest and this is a distinguishing characteristic of this quark.  It is also special because it will decay to other particles before it hadronizes (unlike the other quarks) preserving ``bare quark'' information in its decay products.  This is because its decay time is ~$0.5\times 10^{-24}~s$~\cite{PDGSummary}, which is shorter than the hadronization time scale.  This scale, $\Lambda^{-1}_{QCD}$, corresponds roughly to $10^{-23}~s$~\cite{TopHadronization}.  Other quarks survive longer than this scale and will hadronize instead of decay, which means they produce a bound state of mesons or baryons.  Mesons are combinations of quarks and anti-quarks, while baryons are combinations of three quarks.

\subsection{Force Carriers}
The other major particle category contains the force carriers, or gauge bosons. One of these is the gluon (g), particularly involved in strong interactions and can also self-interact.  Photons ($\gamma$) are the force carriers of the electromagnetic interaction, but are not usually involved in the single-top interactions.  The other mediators are bosons associated with electroweak interactions, the Z and W.  Both are relatively heavy (80 to 90 GeV) compared to the other particles at this scale, and are about half as heavy as the top quark.  Additionally, it has been postulated that there is a Higgs boson and a Higgs field which gives mass to the particles in the standard model.  However, at the time of publication this has not been observed, so we will not go into detail about it here.

\section{Particle Properties and $|V_{tb}|$}\label{sec:vtb}
The standard model particles have very different characteristics, including a variety of charges and masses.  Anti-particles are designated with a bar over the top of their symbol and have the negative of the normal particle's charge.  A particle's charge is given as a fraction of the elementary charge, $e=1.6\times 10^{-19}$ Coulomb.  The down, strange and bottom quark all have -1/3 charge while the up, charm and top quarks have +2/3 charge.  The electrons, muons, and taus have -1 charge while the neutrinos, photon, and Z boson have 0 charge.  The W boson has $\pm1$ charge.  Additionally, particles also have a flavor, as discussed previously, and quarks have a color charge.  The color charge is much like the electric charge but related to the strong interaction (hence its relation to quarks).  The allowed meson and baryon bound states are determined by the color charge.

The particle masses vary over several orders of magnitude.  The range of quark and lepton masses (neutrinos are not pictured), are displayed in Figure~\ref{fig:QuarkMass}.  Notice that there are three quarks with masses of 1 GeV or larger, the c, b, and t quarks.  The mass of the top quark is of particular interest in this document and we use the value 172.5 GeV, which is consistent with the current Particle Data Group value~\cite{PDGSummary}.

\begin{figure}[!htpb]
  \centering
    \includegraphics[width=0.45\textwidth]{figures/theory/LeptonMasses.eps}
    \includegraphics[width=0.45\textwidth]{figures/theory/QuarkMasses.eps}
    \label{fig:QuarkMass} 
\caption{The standard model leptons (left) and quarks (right), by mass.}
\end{figure}

%mention Z, photon?
The gluon and W boson have the special properties that they can effectively change the color and flavor of a quark, respectively.  A gluon for instance may form a vertex with top and anti-top but not two tops.  Gluons are special as well because they may self-interact and form more gluons.  The t-channel single-top process involves the W boson and thus flavor exchange.  For example, a vertex involving a W$^{+}$ may include a top and anti-bottom quark, but not two top quarks.  The probability that a W vertex could involve a top and a down or strange quark is nearly zero according to the standard model, while the probability of top and bottom is nearly one.  This is displayed in the Cabibbo-Kobayashi-Maskawa (CKM) matrix~\cite{CKM1, CKM2}, which is nearly, but not quite, a unit matrix.  

The CKM matrix describes how likely it is for a quark to change to a quark of another flavor.  Specifically, the probability is the relevant matrix entry squared.  These are traditionally called $V_{qq'}$, where q is a quark and q' is the quark in another flavor.  More information about the values may be found in the PDG~\cite{PDGSummary}.  The matrix element we will be particularly interested in, $V_{tb}$, may be indirectly measured with \ttbar~production, but directly observed with single-top production.  The standard model value is 1.

%descibe what gamma mu really is, etc
The single-top cross-section, related to the number of single-top events produced in the collider, is derived from the square of the matrix element, M.  The matrix element varies as follows in the standard model, where $P_L$ is $1/2(1-\gamma^5)$:
\begin{equation} M \propto \bar{b}\gamma^{\mu}V_{tb} P_L t\end{equation}
Thus, the cross-section is proportional to $|V_{tb}|^2$ in the standard model.  If we allow anomolous couplings in this term above some new physics scale, the term $V_{tb}$ may be rewritten as $V_{L}$, where $V_{L}$ is just $V_{tb}$ plus a factor that depends on the new physics scale.  

The Lagrangian, allowing anomolous coupling terms, may be written as follows~\cite{Vtbtheory2, Vtbtheory}.  Here $P_R$ is $1/2(1+\gamma^5)$, $M_W$ is the W boson mass, the $\gamma$ and $\sigma$ terms are constant (Dirac or Pauli) matrices, g is a coupling constant, $q_{\nu}$ is the W boson momentum four-vector, and $\bar{b}$, t, and $W^{-}_{\mu}$ are field terms for the bottom quark, top quark, and W boson respectively:
\begin{equation}
L_{Wtb} = \frac{g}{\sqrt{2}}\bar{b}\gamma^{\mu}(V_L P_L + V_R P_R )tW^{-}_{\mu} - \frac{g}{\sqrt{2}}\bar{b} \frac{i\sigma^{\mu\nu}q_{\nu}}{M_W} (g_L P_L + g_R P_R )tW^{-}_{\mu} + ... 
\end{equation}
We will assume the anomolous couplings $V_R$, $g_L$ and $g_R$ are 0 in this document and will measure the value $|V_L|$ to see if it deviates from the standard model expectation.

Because the matrix element squared is proportional to the cross-section, by using both expected and observed cross-section and $|V_{L}|$ values, one may write:
\begin{equation} |V_{L,obs}|^2=\frac{\sigma_{t,obs}}{\sigma_{t,sm}}|V_{L,sm}|^2
\label{eq:vtbxs}
\end{equation}
where $\sigma$ is the cross-section, obs refers to the observed value and sm refers to the standard model.  In this way, one may directly find the $|V_{L}|$ value from a single-top observation.  The standard model expectation for $V_{L,sm} = V_{tb}$ is 1, so a value greater than 1 for $|V_{L}|$ would indicate non-standard model couplings.

%Need to talk about PDFs somewhere
\section{Overview of Physics Processes}
Unfortunately, the LHC collisions do not just produce single-top events, nor are single-top events extremely distinct or more common than the many other processes that are produced.  In this section we overview the single-top processes, other physics processes, and some of the characteristics that will be considered in order to distinguish them.

\subsection{Single-top and Other Processes}
 \label{sec:Feynman}
%Mandelstam variables = s, t  include this?
Feynman diagrams are a common way to visualize particle physics interactions and also the equations that describe them.  In these diagrams, time flows from left to right.  The leftmost particles are the initial particles (initial state) and the rightmost particles are the final particles (final state).  Figure~\ref{fig:Feynman_singletop} shows the Feynman diagrams for single-top processes.  There are three different production modes: $t$-channel, $Wt$, and $s$-channel.  $Wt$ is also known as associated production and $Wt$-channel.  The $t$-channel production is a scattering interaction while the $s$-channel production is an annihilation interaction.  These are standard high energy physics terms.  In this dissertation, the signal channel is the $t$-channel mode and the other two are considered to be backgrounds.  

Notice in each case there is exactly one top quark, the characteristic of single-top production.  The top quark comes from an interaction mediated by a W boson except in the case of Wt production, where it is produced along with a W boson.  The $t$-channel in particular has two quarks in the initial state, a b-quark (or a gluon producing a b-quark, as is shown) and generic q-quark.  This q-quark is usually a valence quark, while the b-quark may come from the sea of quarks in the proton or from a gluon.  The final state involves the lone top quark and another generic quark in the opposite flavor of the initial q-quark, which is often energetic and forward (close to the beam line).  It is also possible to have an extra jet in the final state from a gluon in the initial state.  Incidentally, in the previous section we noticed the mass of the W boson is smaller than the top quark, but it is still possible for a W to produce a top quark if it is a virtual W, in the s-channel diagram for example.

Although it is not shown in these diagrams, the top quark decays to a W and b-quark.  The W further decays to either a lepton and neutrino or two quarks.  For this analysis, we will focus on the lepton decay case.%  This decay chain is pictured in Figure~\ref{}.
 
\begin{figure}[!h!tpb]
 \centering
 \includegraphics[width=0.30\textwidth]{figures/theory/t_channelNLO.eps}
 \includegraphics[width=0.30\textwidth]{figures/theory/wt1.eps}
 \includegraphics[width=0.30\textwidth]{figures/theory/s_channel.eps}
%\vspace{-0.5cm}
 \caption{Feynman diagrams for single-top production.  The signal is the diagram on the left, $t$-channel single-top production.  It is also possible to have a version of this diagram without the incoming gluon and outgoing $\bar{b}$.  The central diagram is $Wt$ production and the diagram on the right is $s$-channel single-top production.}
 \label{fig:Feynman_singletop}
 \end{figure}

Figure~\ref{fig:Feynman_background} shows the diagrams for the other backgrounds for our single-top $t$-channel signal.  These include multijets (also called QCD), W+jets, Z+jets, \ttbar, and diboson (includes WW, WZ, and ZZ), where jets are streams of particle decays and interactions stemming from quarks that have hadronized.  Of these, only \ttbar~contains top quarks and it contains two of them, rather than the one top quark that single-top t-channel contains.  Nevertheless, it is difficult to distinguish single-top t-channel from its backgrounds.  This is partly because the final states can appear to be quite similar, especially given that the detector is not perfect at particle identification, and partly because of the smaller number of expected signal events relative to background events.
 
\begin{figure}[!h!tpb]
 \centering
 \includegraphics[width=0.30\textwidth]{figures/theory/ttbar.eps}
 \includegraphics[width=0.30\textwidth]{figures/theory/wjet.eps}
 \includegraphics[width=0.30\textwidth]{figures/theory/qcd.eps}

\vspace{8.00mm}

 \includegraphics[width=0.30\textwidth]{figures/theory/zjet.eps}
 \includegraphics[width=0.30\textwidth]{figures/theory/diboson.eps}
%\vspace{-0.5cm}
 \caption{Feynman diagrams for backgrounds to single-top production.  The \ttbar~is the diagram on the top left, $W$+jets is the top central diagram and multijet production is the top right diagram.  The final two diagrams are the smallest backgrounds, $Z$+jets on the bottom left and diboson on the bottom right.}
 \label{fig:Feynman_background}
 \end{figure}

Although the diagrams in Figures~\ref{fig:Feynman_background} and~\ref{fig:Feynman_singletop} are basic, straight-forward diagrams, it is possible in more complex diagrams with extra gluons in the initial or final state, or loops of particle production (such as a gluon making two gluons which in turn recombine into one gluon).  These extra possibilities can be described in separate diagrams, and it is possible to have diagrams from two different processes which give the same final state.  In this case, the diagrams are said to interfere and it is important to consider such things when generating Monte Carlo (MC).  For the signals and backgrounds considered in this analysis, there are two cases of interference in particular.  The first is single-top Wt production and \ttbar~production.  Wt is already very similar to \ttbar~except for a b-quark (a top quark decays to a W and a b-quark).  If a gluon in the Wt initial state produces the incoming b, it will also produce an outgoing $\bar{b}$, making the final state look like \ttbar.  This does not have a large contribution to the analysis however, as this $\bar{b}$ is generally low in \pt.  The \pt~requirement for jets (see Section~\ref{sec:quarks}) generally removes this from consideration.  However, the scenario is considered when MC is generated.

Another possibility involves the t-channel single-top signal.  If the incoming quark is from a gluon instead of a valence quark, the gluon produces the incoming quark plus an extra quark for the final state.  The new final state contains two light quarks, t, and a possible b.  This is the same final state discussed in the last paragraph, if the W decays to two light quarks instead of a lepton and neutrino in the case of Wt and for one of the top quarks in the case of \ttbar.  However, this sort of initial state is very uncommon.  Also, the extra particles from the gluons often having low \pt (thus not satisfying the jet definition), so this scenario does not impact the analysis.

\subsection{Cross-section}
 \label{sec:ExpectedCrosssections}
The cross-section for a process reflects how often we expect a collision to produce that particular process.  It is often useful to think of it in terms of the number of events produced:
\begin{equation} N = \sigma L \label{equ:numberofevents} \end{equation}
where N is the number of events, $\sigma$ is the process cross-section and L is the integrated luminosity (explained in Section~\ref{sec:LHC}), which represents the number of collisions.

The cross-section may include additional factors like k-factors or branching ratios.  The k-factors are corrective fractions which change a cross-section from a leading order to next-to-leading order value, for example.  A leading order (LO) cross-section is a theory calculation involving just basic diagrams without loops or extra vertices.  Next-to-leading order (NLO) includes an additional level of complexity of loops and vertices, making it a longer, more difficult calculation.  With each step up in completeness, the calculation becomes more technically difficult, so we do not have exact theoretical cross-sections for our processes.  

Branching ratios (BR) are just fractions to change the total cross-section for a complete process to a partial cross-section.  For instance, in the $t$-channel diagram, the final state involves a top quark, which decays to a b quark and a W.  This W may decay either to two more quarks, like up and down quarks, or to a lepton and neutrino.  For reasons discussed in Section~\ref{sec:EventSelection}, we require exactly one lepton in our selection and only generate Monte Carlo for final states including a lepton.  Thus, the cross-section we actually normalize the MC to is the fraction of the total cross-section that involves a single lepton in the final state.  The probability for the W to decay to a lepton and neutrino is a branching ratio that is multiplied with the cross-section.

The cross-sections used in this analysis for the signals and backgrounds are listed in Table~\ref{TABLE-MCSAMPLES} for 7 TeV collisions, including k-factors (if applicable) and branching ratios.  They are given in the standard units of picobarns ($pb$), where a barn is $10^{-28}~m^{2}$.  The W+jets is divided here by flavor in some cases.  A special procedure is applied to separate the W+jets into light and heavy flavors later in the analysis based on the truth-level hadron type, in a way that avoids double-counting events.  Truth level refers to monte carlo information about the particle generated before applying detector effects (see Chapter~\ref{chap:partreco}).  The division is into light (u,d,s), c, and heavy ($c\bar{c}$ and $b\bar{b}$) for the jets (the processes may also have additional light jets).  The k-factors used here are 1.20 for W+jets in general (except W+cjets, which uses 1.52), 1.25 for Z+jets and 1.12 for \ttbar.

Measuring $t$-channel single-top production is the focus of this analysis but we would like to compare the measurement with an expected cross-section value.  For this we use the cross-sections given in Table~\ref{TABLE-STOP}~\cite{Kidonakis:2010tc,Kidonakis:2010ux, Kidonakis:2011wy}, and the branching ratios from the PDG, with values of 10.75\% for $W \rightarrow e \nu_{e}$, 10.57\% for $W \rightarrow \mu \nu_{\mu}$, and 11.25\% for $W \rightarrow \tau \nu_{\tau}$~\cite{PDGSummary}.  The cross-sections contain both top and anti-top contributions, and we expect these particle and anti-particle contributions to be different for processes that have valence quarks in the initial state.  The LHC collides two protons, each of which contains two up and one down quarks, leading to an excess of positively charged quarks.  For the $t$-channel, which usually has a valence quark in the initial state, the standard model cross-section contains $41.9$~pb due to events containing $t$ quarks and $22.7$~pb from events containing $\bar{t}$ quarks.
\begin{table}[htdp]
\begin{center}
\begin{tabular}{l|r}
 \hline
 Process        & Cross-section [pb] \\
\hline
 $t$-channel & 64.57 + 2.71 - 2.01 pb\\
 $Wt$ &  15.74 + 1.06 - 1.08 pb \\
 $s$ channel &  4.63 + 0.19 - 0.17 pb \\ 
\hline
\end{tabular}
\caption{ (N)NLO cross-sections for single-top processes~\cite{Kidonakis:2010tc,Kidonakis:2010ux, Kidonakis:2011wy}}
\label{TABLE-STOP}
\end{center}
\end{table}

It is also interesting to note that the difference between the signal cross-section and background cross-sections are different by many orders of magnitude, as seen in Table~\ref{TABLE-MCSAMPLES}.  This means that many, many events have to be identified correctly and rejected in order to try to pick out our needle in this immense haystack.

\begin{table}[htdp]
\begin{center}
\begin{tabular}{l|r}
\hline
Process        & Cross-section [pb] \\
\hline\hline
$t$-channel $\to e\nu_{e}$ & 6.9 \\
$t$-channel $\to \mu\nu_{\mu}$ & 6.8 \\
$t$-channel $\to \tau\nu_{\tau}$ & 7.3 \\
\hline
\ttbar~(non-hadronic) & 90 \\
$Wt$ & 16 \\
$s$-channel $\to e\nu_{e}$ & 0.50 \\
$s$-channel $\to \mu\nu_{\mu}$ & 0.49 \\
$s$-channel $\to \tau\nu_{\tau}$ & 0.52 \\
\hline 
$Z$ + 0 jet   & 835  \\
$Z$ + 1 jets  & 168 \\
$Z$ + 2 jets  & 51 \\
$Z$ + 3 jets  & 14 \\
$Z$ + 4 jets  & 4  \\
$Z$ + 5 jets  & 1  \\
\hline
$W$ + 0 jet   & 8,300 \\
$W$ + 1 jets  & 1,600 \\ 
$W$ + 2 jets  &   460  \\
$W$ + 3 jets  &   120 \\
$W$ + 4 jets  &    31 \\
$W$ + 5 jets  &    8  \\
\hline
$W+b\bar{b}$ + 0 jet   & 57 \\
$W+b\bar{b}$ + 1 jets  & 43 \\
$W+b\bar{b}$ + 2 jets  & 21 \\
$W+b\bar{b}$ + 3 jets  & 8  \\
\hline
$W+c\bar{c}$ + 0 jet   & 153 \\
$W+c\bar{c}$ + 1 jets  & 126 \\
$W+c\bar{c}$ + 2 jets  & 62 \\
$W+c\bar{c}$ + 3 jets  & 20 \\
\hline
$W+c$ + 0 jet   & 980 \\
$W+c$ + 1 jets  & 312 \\
$W+c$ + 2 jets  & 77  \\
$W+c$ + 3 jets  & 17  \\
$W+c$ + 4 jets  & 4  \\ 
\hline
$WW$                         & 17 \\
$WZ$                         & 6 \\
$ZZ$                         & 1 \\
\hline\hline
\end{tabular}
\caption{ Cross-sections, including branching ratios and k-factors. Shown for one lepton decay (ex. electron) in the case of Z+jets and W+jets processes.  Single-top $s$-channel and $t$-channel list different lepton decays for the W separately to show the branching ratios used.}
\label{TABLE-MCSAMPLES}
\end{center}
\end{table}

Most of these cross-sections listed are fairly well known, partly because they are so much larger (relatively), and large statistics samples have been available for some time at long running experiments (like the Tevatron experiments) with relatively low systematic uncertainties.  However, lower cross-section processes such as our signal have only recently been observed and the cross-sections are not necessarily well measured.  The goal of this analysis is to provide a cross-section measurement of the $t$-channel process and to see if it agrees with the standard model prediction.

\section{New Physics Possibilities}
In recent years, there have been several indications that the standard model does not explain everything.  Although the standard model has been very successful, observations in astronomy have indicated the presence of so called dark matter~\cite{DarkMatterReview} and dark energy~\cite{DarkEnergySN} which are not predicted by the standard model and in proportions larger than the standard model matter we know of~\cite{WMAP}.  Several theories have been proposed to account for this
%~\cite{SUSY, ExtraDimensions}
, but none have been shown to exist.  It is importation to check the standard model with detailed experimental measurements, to confirm the standard model and perhaps gain information about new physics if deviations are discovered.

Single-top $t$-channel production, a standard model process, is interesting because it is still new and not fully examined.  Although it is now known to exist, we are only now accumulating enough events to do precision measurements of the cross-section.  If this cross-section is not consistent with the standard model, it may indicate new physics.  It is possible that there could be a flavor changing (like the W) neutral current (neutral like the Z) in the process for instance, that would change the Wtb vertex.  It is also possible there could be a fourth generation of quarks, which would again cause the CKM matrix $V_{tb}$ value to deviate the standard model value.  Detailed measurements of the single-top production allow direct evidence of these phenomena.

In this document, we take the first step, which is to measure the $t$-channel single-top cross-section and compare it to the standard model value.  We do this by applying a small number of kinematic requirements to the events, to provide a straight-forward measurement.  This is the first cut-based analysis with this level of precision on the single-top t-channel cross-section.  It is also possible to use more sophisticated statistical methods to do a measurement of this signal, and the usefulness of this approach is explored in Appendix~\ref{app:MultivariateApproach}.
