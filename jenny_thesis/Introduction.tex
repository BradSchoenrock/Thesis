\chapter{Introduction}
A rock smashes against another rock and shatters into pieces.  Some are shiny, some are differently colored.  However, if you were to take one of these rock pieces and zoom in, divide it into pieces until you can't divide further, you would see that it is composed of the same bits of matter as any other rock.  In such ways, people have searched for the fundamental building blocks of matter, the elementary particles.  

In the modern era, we have found these blocks are very small indeed and require a lot of human ingenuity to study.  It has taken the work of many men and women over the years to reach the point where we can do the experiments we now perform.  We have found that colliding very small bits of matter composed of smaller particles (protons) at very high speeds causes the creation of a flurry of new, fundamental particles.  These particles may have unintuitive properties, such as masses larger than the two protons originally collided (made possible by the large amount of energy used to collide them).  In this dissertation we will study one such fundamental particle.

Thousands of scientists are now working at the Large Hadron Collider (LHC) on four experiments located at different points around the collider ring.  Two are general purpose machines, ATLAS and CMS.  These machines are designed to try to find not only particles and processes we know exist but also new ones.  These machines smash bits of atoms that are already quite small (protons) together at incredibly high speeds, producing new particles which decay or smash into others and eventually some particles hit the detectors.  Scientists use large computing clusters to take this information and attempt to reconstruct what happened when the original bits collided, and to sort out the collisions with particles and processes they don't want from the ones they do.  

Clearly this is a challenging thing, cutting-edge work only possible in the modern era.  But the questions we look to answer are fundamental.  What is the world really made of at the smallest level?  What are the properties of those things?  How do they interact with other things at this scale and what do those interactions produce?  

In this dissertation, we will discuss specifically the search for the t-channel single-top quark production.  The top quark is the most massive small particle yet observed and the t-channel production mode refers to a particular way is is created.  We will overview the particles known to exist and the current theory related to these.  Then we will examine the collider and detector used in this study (the ATLAS detector).  Finally we will discuss the procedure to isolate this process from so many others, as well as the measurement and kinematics of this process.

