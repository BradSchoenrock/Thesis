\chapter{Conclusions and Implications for Future Work}
In this dissertation we have discussed the main details involved in the estimate of the t-channel single-top cross-section using ATLAS data.  The ATLAS detector is a multi-purpose detector located at the LHC at CERN and 1.04 $fb^{-1}$ of data from the 2011 data taking run was used, with a 7 TeV center-of-mass energy.  The data were processed to assign energy and tracks to reconstructed particles like those in a Monte Carlo simulation.  Selections were applied to both the data and MC to reduce the signal and background ratio to a more reasonable level of about 10\%.  At this point, a cut-based analysis was performed, where four additional selections for four orthogonal channels based on jet number and lepton charge were chosen and applied.  A fit and frequentist technique was then used to determine the cross-section and its uncertainty.  Separate measurements of the top and anti-top cross-section were performed, giving observed results of $\sigma_{t^{+}}= 59^{+18}_{-16}$~pb for top (positive lepton charge) and $\sigma_{t^{-}}= 33^{+13}_{-12}$~pb for anti-top (minus lepton charge).  The final result included all four channels and was $\sigma_{t}= 92^{+29}_{-26}$~pb, where $\sigma_{t}= 65^{+22}_{-20}$~pb was expected.  

Future studies will likely benefit from additional channels beyond the four used here and perhaps tighter selection thresholds, made possible by additional data.  For instance one could examine the 4 jet bin.  It is heavily contaminated with \ttbar~events, but it may be possible to remove enough of these events to be worthwhile.  Additionally, it is possible to have a 3 jet event with 2 jets b-tagged and this is another possible kinematic region, though again heavily contaminated with \ttbar~background.  In this study, the three jet bin was used, which also suffers from a large \ttbar~background.  The invariant mass of all three jets was a very effective selection for removing a large portion of this background while still retaining a reasonable amount of signal events.  Future analyses may want to consider selections using the invariant mass of all jets except the jet that best reconstructs the top mass (using a lepton and neutrino) for the four jet bin.  The remaining three jets used in the invariant mass are likely from the decay of the second top quark.  Similarly, the 3 jet 2 btag selected channel could benefit from a selection based on the invariant mass of all three jets, or just the invariant mass of all jets except the jet that best reconstructs the top mass.  The choice here is not as obvious, as some jets in \ttbar~may have been missed or have a low \pt, but both options could be useful.  With these invariant mass selections, in addition to selections related to the reconstructed top mass and untagged jet $|\eta|$, it may be possible to perform an analysis to measure single-top $t$-channel in these regions, with S/B ratios of 0.5 or more in these channels, with these types of selections, as demonstrated in Appendix~\ref{alternativechannels}.

Although modern particle physics is more difficult than smashing stones together, it is nevertheless rewarding, as new information about the universe we live in is discerned from these studies.  We know so much about the physical universe in this modern era and yet there is still much to be done and to learn.  In this case, we have made a new measurement of a standard model process only recently observed.  Future studies will likely reduce these uncertainties and extract new information about properties of the top quark.  It is a very exciting time to contribute to the understanding of single-top production, as we just start to scratch the surface of what can be done with studies of this process.

