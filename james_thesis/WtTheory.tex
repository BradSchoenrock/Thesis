
\chapter{Theory}
\label{THEORY}
This chapter will cover the theoretical background to motivate and perform this analysis. Not only will it introduce the basics of the Standard Model of high energy physics, but it will also discuss the signal and background processes in this analysis. Here the signal is the \Wtchan process, while the backgrounds are the set of processes that can appear similar to the signal in the detector. In addition, it will communicate an understanding of where this result fits in the broader scope of the field of high energy physics.

\section{Standard Model}
\label{THEORY-STANDARDMODEL}
The Standard Model describes the fundamental particles and how they interact~\cite{PDG, Griffiths}. A listing of particles and their properties is given in Table~\ref{TABLE-THEORY-STANDARDMODEL}. These particles can be separated into two categories based on their spin: fermions and bosons. Fermions, which include leptons and quarks, have half-integer spin and no two identical fermions can occupy the same quantum mechanical state. Electrons are a common example of a fermion. Bosons have integer spin and any number of identical bosons can occupy the same state. They are often carriers of force, and the photon is the most ubiquitous example of a boson. Frequently in this document a particle name indicates both itself and its anti-particle. For example, when reference is made to the \Wtchan, this descriptor refers not only to the $W^+t$ final state, but also the $W^-\bar{t}$ final state.

There are three generations of fermions. Almost all observable matter is made up of fermions from the first generation. There are two families of fundamental fermions: leptons and quarks. Protons and neutrons are examples of composite fermions, and their quark components, up and down quarks, are also fermions. The second and third generation particles tend to have larger masses and shorter lifetimes and will quickly decay into less massive particles. The exception to this are the second and third generation neutrinos whose mass hierarchy is not known and which are stable (although they can oscillate between neutrino flavor states).

\begin{table}[!h!tbp]
\begin{center}
\begin{tabular}{|l|l|c|c|c|c|}
\hline
Family & Name & Symbol & Mass & Charge & Spin \\
\hline
\qc                          & Up               & u  & 2.4 MeV    & $\nicefrac{2}{3}$  & $\nicefrac{1}{2}$\\
\qc                          & Down             & d  & 4.8 MeV    & $-\nicefrac{1}{3}$ & $\nicefrac{1}{2}$\\
\qc                          & Charm            & c  & 1.27 GeV   & $\nicefrac{2}{3}$  & $\nicefrac{1}{2}$\\
\qc                          & Strange          & s  & 104 MeV    & $-\nicefrac{1}{3}$ & $\nicefrac{1}{2}$\\
\qc                          & Top              & t  & 172 GeV    & $\nicefrac{2}{3}$  & $\nicefrac{1}{2}$\\
\qc \multirow{-6}{*}{Quarks} & Bottom           & b  & 4.2 GeV    & $-\nicefrac{1}{3}$ & $\nicefrac{1}{2}$\\
\hline
\lc                          & Electron         & e  & 511 KeV    & -$\nicefrac{1}{2}$ & $\nicefrac{1}{2}$\\
\lc                          & Electron Neutrino& $e_\nu$      & $< $2.2 eV & 0                  & $\nicefrac{1}{2}$\\
\lc                          & Muon             & $\mu$        & 105.7 MeV  & -$\nicefrac{1}{2}$ & $\nicefrac{1}{2}$\\
\lc                          & Muon Neutrino    & $\mu_\nu$    &$< $0.17 MeV& 0                  & $\nicefrac{1}{2}$\\
\lc                          & Tau              & $\tau$       & 1.78 GeV   & -$\nicefrac{1}{2}$ & $\nicefrac{1}{2}$\\
\lc \multirow{-6}{*}{Leptons}& Tau Neutrino     & $\tau_\nu$   &$< $15.5 MeV& 0                  & $\nicefrac{1}{2}$\\
\hline 
\bc                          & Photon           & $\gamma$  & 0        & 0                  & 1\\
\bc                          & \Wboson$\pm$ Boson     & $W\pm$  & 80.4 GeV & $\pm$ 1            & 1\\
\bc                          & \Zboson\ Boson          & $Z$  & 91.2 GeV & 0                  & 1\\
\bc                          & Gluon            & $g$  & 0        & 0                  & 1\\
\bc \multirow{-5}{*}{Bosons} & Higgs*            & $H$  & 125 GeV  & 0                  & 0\\

\hline
\end{tabular}
\caption{List of particles and their properties in the Standard Model. *The Higgs described here uses the mass of the Higgs candidate discovered at the LHC.~\cite{HiggsATLAS,HiggsCMS} For interpretation of the references to color in this and all other figures, the reader is referred to the electronic version of this thesis.}
\label{TABLE-THEORY-STANDARDMODEL}
\end{center}
\end{table}

The Standard Model describes the interaction of three of the four known fundamental forces: electromagnetic, weak, and strong. The fourth, gravity, is not described by the Standard Model. The strong force is the force that holds protons, neutrons, and the atomic nucleus together. Quantum chromodynamics (QCD) models this force by describing the interactions between particles with a ``color'' charge (See section~\ref{SECTION-THEORY-QCD}). The weak force describes interactions mediated by the \Wboson\ and \Zboson\ bosons. An example of the weak interaction is beta decay, where an atomic neutron decays into a proton, releasing an electron and a neutrino from the atom. The electromagnetic force describes the interactions between electronically charged particles. 

The interactions of the Standard Model can be defined by a Lagrangian. A Lagrangian can possess different symmetries under transformations.  For example, a Lagrangian can be symmetric under changes in coordinate system. Using a different coordinate system does not change the physics of the Lagrangian. The Standard Model Lagrangian has many gauge symmetries, meaning that the Standard Model Lagrangian is invariant under classes of gauge transformations with these symmetries. The consequences of each invariance is that additional conservation laws must be respected by the interaction. For example, the symmetry of the electromagnetic force leads to conservation of electric charge. 

The language of group theory describes these symmetries. A group is defined as an abstract set of elements with a defined operator that obeys certain rules. An important concept in group theory is that of generators. A set of generators A of group G is a collection of elements such that every element in G can be formed through group operations using only elements in set A. For example, for the natural numbers under addition, ${1}$ is a complete generator set, as any natural number n can be represented as the sum of $1$s. 

%Consider a set of elements and a generic operator $\otimes$. For this to be a group, the elements and operator must fulfill the following four properties~\cite{GROUPTHEORY}:

%\begin{enumerate}
%\item A group be closed under its defined operation. For any given elements A and B in the set, %the element $A \otimes B$ must also be in the set.
%\item A group must have an identity element. There must be some element I such that for any element A, $I \otimes A = A \otimes I = A$.
%\item For every element A in the group, there exists an inverse element B such that $A \otimes B = B \otimes A = I$, with I being the identify element.
%\item For any elements A B, and C in the group, $(A \otimes B) \otimes C = A \otimes (B \otimes C)$.
%\end{enumerate}


%In the case of the Standard Model, we construct groups using unitary matrices as the elements and matrix multiplication as the operator. A matrix A is a unitary matrix if and only if $A^\dagger A = AA^\dagger=I$, where I is the identify matrix. Here $A^\dagger$ is the Hermitian conjugate of $A$. A group made up of the set of all $n \times n$ unitary matrices is referred to as the unitary group of degree n, or in shorthand, $U(n)$. If an additional requirement that all elements have determinant one is applied the group is referred to as the special unitary group of degree n, or $SU(n)$.

The Standard Model is based on a Lagrangian with many symmetries, three of which are associated with the three forces the Standard Model describes. These three symmetries are the $SU(3)\times SU(2)\times U(1)$ gauge symmetries, meaning that the Lagrangian is invariant under transformations of $SU(3)\times SU(2)\times U(1)$. The $SU(3)$ group describes the interactions of the strong force, while the $SU(2)\times U(1)$ group describes the unified electroweak force. These forces are mediated by boson force carrier particles. When two particles act on each other, a virtual force carrier particle, called a propagator, is exchanged. This propogator is what carries the momentum and energy that gets traded between the two interacting particles. 


\subsection{Feynman diagrams}
\label{SECTION-THEORY-FEYNMAN}
In collider physics the concept of a cross-section is critical to making predictions. The cross-section represents the likelihood of a process given some initial conditions, measured in units of area. A barn (b) is the accepted unit for cross-section with one barn (b) being equal to $10^{-24}\ cm^{2}$. At the LHC cross-sections are frequently described in picobarns (pb), which are $10^{-36}\ cm^{2}$, or femtobarns (fb), which are $10^{-39}\ cm^{2}$. As described in greater detail in Section~\ref{SECTION-ATLAS-LHC}, if the cross-section and the amount of data collected are both known, the number of events expected can be calculated by $N_{events}=L\sigma$. In this equation $L$ is the luminosity, a measure of how much data has been collected, and $\sigma$ is the cross-section of the proces of interest. In the case of this analysis, our initial condition is a proton-proton collision at 7 TeV. From there, the cross-sections of interesting processes can be calculated. These theoretically predicted cross-sections can be compared to experimentally observed cross-sections as tests of the models.

Often when discussing interactions in the Standard Model a Feynman diagram is used to illustrate the process~\cite{Griffiths}. The Feynman diagrams not only have great utility for understanding the physics at work in a process, they also inform the resulting cross-section calculation, and consequently are common in both high energy theory and experiment as explanatory devices. In this analysis, the Feynman diagrams are drawn with space on the y-axis and time on the x-axis. For example, Fig.~\ref{FIGURE-THEORY-TTBAR} describes a process in which a quark-antiquark pair interact and form a gluon, which then splits into a \TTbar\ pair. The points at which particles connect are called vertices, and are identified by the particles involved.  For example, the rightmost vertex in this diagram is a $gt\bar{t}$ vertex.
\FIG{ttbarsimple}{An example of a basic Feynman diagram~\cite{JAXO}.}{FIGURE-THEORY-TTBAR}

In general, the Feynman diagrams shown in this analysis reflect the most basic interactions that result in the observed final state by showing only the tree-level diagrams. The tree-level diagrams are constructed such that there are a minimal number of vertices. These tree-level diagrams do not represent the only way that such a final state could occur. For every tree-level diagram, there are infinitely many higher-level diagrams with more vertices that contribute to the total cross-section. Physically these higher-level diagrams can have loops or additional radiation modifying the tree-level diagram. Mathematically, these diagrams represent expansion terms in a perturbative cross-section calculation. For example, in Fig.~\ref{FIGURE-THEORY-TTBAR} the interacting gluon could split into two gluons and reform in the middle of the interaction, as shown in Fig.~\ref{FIGURE-THEORY-TTLOOP}. This higher order diagram would be called a Next to Leading Order (NLO) diagram. There are also Next To Next To Leading Order (NNLO) diagrams and so on. The contributions by these higher order diagrams generally decrease with their complexity, but a critical part of correctly calculating the total cross-section of processes involve estimating the contribution of the higher order diagrams omitted in a given computation. These contributions are often given as a k-factor, a scaling factor which can be applied to the calculated cross-section to give the estimated cross-section for a higher order.

At the LHC we collide protons on protons, but those collisions have a high enough energy to break the proton, allowing the components particles to interact instead. Constructing Feynman diagrams involving all componenents of the proton would be a difficult task, so instead we take advantage of the fact that at high energies we can factorize the problem into two problems. The momentum contribution from each parton is measured to construct PDF(s), which give the the initial states for quark collisions. These initial states are used to solve the second problem of constructing Feynman diagrams for the processes of interest.

\FIG{ttbarsimpleloop}{An example of a Next to Leading Order diagram.}{FIGURE-THEORY-TTLOOP}

Another diagram correction that must be added for a realistic cross-section estimate is taking into account the effect of initial and final state radiation (ISR and FSR). These are diagrams in which the initial and/or final state particles radiate off an additional particle. These diagrams do give a new unique tree level-diagram from a theoretical standpoint, but from the experimental standpoint these diagrams are processes we will physically see in our detector. A single top event with final state radiation is still considered a single top event, thus these effects must be included in any cross-section calculation or simulation. An example of a \TTbar\ event with initial state radiation is given in Fig.~\ref{FIGURE-THEORY-ISR}.

\FIG{ttbarisr}{An example of a Feynman diagram with ISR.}{FIGURE-THEORY-ISR}


\subsection{Electroweak theory}

The Standard Model describes the unification of two of the four fundamental forces, the electromagnetic force and the weak force, into one force, the electroweak force. This section will first discuss the electromagnetic and weak forces separately, and then discuss their unification in the Standard Model.

The electromagnetic force is mediated by the photon. The photon interacts with electromagnetically charged particles, which are all known fundamental particles excluding the \Zboson\ boson, the gluon, the neutrinos, the Higgs boson, and the photon itself. 

The weak force describes interactions mediated by the $\Wboson^\pm$ and $\Zboson$ bosons. All quarks and leptons can participate in weak interactions. In addition, all force carriers except gluons can also participate. The weak force allows for several quantum number conservation laws to be ``broken'' in ways that the electromagnetic and strong force cannot. One symmetry that is broken by the weak force is chirality, a quantum number that represents the right- or left-handedness of a particle. If the spin of a massless particle is in the same direction as its momentum it right-handed, and if the spin is in the opposite direction as the momentum it is left-handed. Parity describes a possible symmetry in which the physics is identical if the coordinate system is inverted. The strong and electromagnetic interactions both interact with right- and left-handed particles in exactly the same way, but the weak force violates parity by only acting on left-handed particles (and their respective right-handed antiparticles)~\cite{PDG}. 

Charge-parity is another conservation law, requiring that the product of the charge of the initial state multiplied by the parity is conserved. However this conservation has also been discovered to be violated in some weak processes such as Kaon decay. Most relevant to this analysis, however, is the weak force's ability to change the generation of quarks. For example, the up and down quarks make up the first generation, while the top and bottom quarks make up the third generation. A non-weak interaction cannot change an up quark to a bottom quark. The weak force is capable of changing generations because the weak interaction eigenstates of the quarks are not the same as their flavor eigenstates. This allows the weakly interacting quarks to change not only their momentum and energy, but also the generation of particles. This quark flavor mixing is described by the Cabibbo-Kobayashi-Masakawa (CKM) matrix~\cite{Griffiths,PDG}

\begin{equation}\label{EQUATION-THEORY-CKM}
\begin{bmatrix}d'\\s'\\b'\\\end{bmatrix} =  V_{CKM}\begin{bmatrix} d\\s\\b\\\end{bmatrix} = \begin{bmatrix} V_{ud} & V_{us} & V_{ub}\\ V_{cd}& V_{cs}& V_{cb}\\V_{td}& V_{ts}& V_{tb}\\\end{bmatrix} \begin{bmatrix} d\\s\\b\\\end{bmatrix}
\end{equation}

\noindent
where d, s, and b are the down, strange, and bottom quarks. The matrix $V_{CKM}$ can also be parametrized with three mixing angles ($\theta_{12}$, $\theta_{23}$, $\theta_{13}$ and a CP-violating phase ($\delta$):

\begin{equation}\label{EQUATION-THEORY-CKMMATRIX2}
V_{CKM} =  \begin{bmatrix} c_{12}c_{23} & s_{12}c_{13} & s_{13}e^{-\imath\delta}\\ -s_{12}c_{23}-c_{12}s_{23}s_{13}e^{\imath\delta} & c_{12}c_{23} - s_{12}s_{23}s_{13}e^{\imath\delta} & s_{23}c_{13}\\s_{12}s_{23}-c_{12}c_{23}s_{13}e^{\imath\delta} & -c_{12}s_{23}-s_{12}c_{23}s_{13}e^{\imath\delta} & c_{23}c_{13}\\\end{bmatrix}
\end{equation}

\noindent
Here $c_{ij}$ and $s_{ij}$ represent $cos(\theta_{ij})$ and $sin(\theta_{ij})$, respectively. Each element represents the mixing between flavor eigenstates under the weak interaction. If there was no mixing between cross generational eigenstates the matrix would be the identity matrix. For example, $V_{tb}$ represents the relative strength of the $Wtb$ vertex (the coupling between W, t, and b), shown in Fig.~\ref{FIGURE-THEORY-TDECAY}.  If $V_{ub}$ was zero, there would be no $Wub$ vertex in the Standard Model and the u quark could not change flavor to or from the b-quark through the weak interaction. 

The Standard Model takes the mixing angles and $\delta$ as inputs that must be measured experimentally. From these measurements we know the diagonal elements are close to one, while the off-diagonal elements are close to zero. An interpretation of the matrix elements is that the interaction and flavor eigenstates for the quarks are almost identical, and consequently the weak force typically conserves quark generation. The measurement of each of these elements is an active field of research. The current best measured magnitudes for the CKM matrix elements are~\cite{PDG}:

\begin{equation}\label{EQUATION-THEORY-CKMMATRIXEXP}
V_{CKM} = \begin{bmatrix} 0.97425\pm 0.00022 & 0.2252\pm 0.0009 & (4.15\pm 0.49) \times 10^{-3}\\ 0.230\pm 0.011& 1.006 \pm 0.023 & (40.9 \pm 1.1)\times 10^{-3}\\(8.4 \pm 0.6) \times 10^{-3}& (42.9 \pm 2.6) \times 10^{-3}& 0.89 \pm 0.07\\\end{bmatrix}
\end{equation}

One of the goals of this analysis is to make a direct measurement of the $V_{tb}$ matrix element (The lower right hand element). In this analysis we look at a class of processes in which only one top quark is produced, referred to as single top processes. Single top processes uniquely allow for a simple direct measurement of $V_{tb}$ because they contain a $Wtb$ vertex. Consequently their cross-section (discussed in Section~\ref{SECTION-THEORY-FEYNMAN}) is proportional to the magnitude of the $V_{tb}$ matrix element squared, $\sigma_{sgtop}\propto \left|V_{tb}\right|^2$~\cite{sgtopvtb}. Without a direct measurement, an analysis must assume the unitarity of the CKM matrix and the existence of exactly three generations of quarks to make an indirect measurement of $V_{tb}$. 

While in general the experimental evidence is consistent with a unitary $3 \times 3$ CKM matrix and exactly three quark generations, measurements with a minimum number of assumptions are preferable. In addition, direct measurements of $V_{tb}$ can be sensitive to new physics that violate these assumptions. This is why it is critical to make direct measurements of $V_{tb}$ and why single top analyses are important.

The weak and electromagnetic forces unify at high energy (approximately the scale of the mass energy of the weak force carriers~\cite{PDG}). This unification is the manifestation of the gauge group $SU(2) \times U(1)$ with four massless gauge bosons. Three of these gauge bosons come from the generators of the $SU(2)$ symmetry, while the remaining one comes from generator of the $U(1)$ symmetry. The Standard Model also posits a Higgs potential of the form:

\begin{equation}
V(\phi) = \mu^2\phi^\dagger\phi + \frac{\lambda^2}{2}\left(\phi^\dagger\phi\right)^2
\end{equation}

\noindent
If $\mu^2$ is negative, then this potential has a symmetric minimum away from the central value. Once a point in the minimum is selected the symmetry is broken. In the Standard Model this leaves the massive \Zboson\ and \Wboson$\pm$ bosons that we observe. The $\Wboson\pm$ bosons come from the $SU(2)$ group, while the \Zboson\ boson and the photon originate from a mixing of the $SU(2)$ and $U(1)$ groups' bosons. This symmetry breaking also implies the existence of a scalar Higgs boson which prior to the LHC had not been observed. The search for the Higgs boson was one of the driving arguments to build the LHC and the ATLAS and CMS experiments. As of this writing, a Higgs-like particle has been observed with a mass of approximately 125 GeV~\cite{HiggsATLAS, HiggsCMS}.

\subsection{Quantum Chromodynamics}
\label{SECTION-THEORY-QCD}
QCD defines the interactions between particles with color charge (the origin of the ``chromo'' in chromodynamics) and is mediated by the gluons. The strong force has a much larger coupling strength than the other forces, and as a result the cross-section (discussed in Section~\ref{SECTION-THEORY-FEYNMAN}) of strong force interactions is generally larger than the cross-section of electroweak interactions. The strong force is represented by an $SU(3)$ group symmetry, and as a result there are three types of color charge, referred to conventionally as red, green, and blue. The selection of colors from the visible electromagnetic spectrum to represent the conserved quantities is to give some intuition to the concept of color charge, but there is no connection in the theory between the color red and the strong force color charge red. 

Like electric charge, these color charges can each have negative values, referred to as anti-red, anti-green, and anti-blue. Each quark has a color charge, and each anti-quark has an anti-color charge, while each of the gluons carries a superposition of color and anti-color states. One superposition with all three color anti-color combinations is a colorless state, which does not correspond to a gluon.  Consequently, there are a total of eight gluons we observe. 

%In addition, the symmetry can tell us the number of force carriers by examining the number of generators of the group. In general, the $SU(n)$ group has $n^2-1$ generators, so for $SU(3)$ there are eight force carriers, the gluons. 

Isolated color charge is disfavored by the strong force, and as a result, stable states must be color neutral, possessing an equal amount of red, green, and blue color charge or color and anti-color charges that sum to zero. 

This favoring of color neutral states is called color confinement and consequently they cannot exist in nature alone, instead grouping into bound states with other quarks. Mesons are two quark bound states with color-anticolor pairs, for example the $\pi^+$ particle is made up of an (up, anti-down) pair with a color state such as (blue, antiblue). Baryons are three particle bound states with a red, a blue, and a green component particle. States that are not color neutral, for example any bare quark or gluon, will quickly hadronize, creating quarks and antiquarks which combine to form color neutral baryons and mesons. If the quark has significant momentum, such as in a collider experiment, this hadronization manifests as a spray of hadrons, called a jet. These jets are how quarks are seen from the perspective of a detector, as described in more detail in Section~\ref{SECTION-DEFINEJETS}. 

Although the Standard Model includes a complete theory of QCD, the theory does not give a set of computations to calculate all quantities to arbitrary precision in closed form. As a result, many phenomena in QCD are modeled using both experimental and theoretical inputs. For example, the modeling of the hadronization of quarks and gluons is strongly dependent on experimental data. Another example is the use of experimental data for parton distribution functions (PDFs), the modeling of the interior momentum distribution of the components (also called partons) of particles such as the proton. In addition to the three valence quarks that make up the proton, there are many gluons and other quarks within that exist on short time scales. At high energies these other partons can have significant amounts of momentum, making them important to include in the PDFs. Because the calculations required for short range QCD are beyond present simulation capabilities, the composition of the proton cannot be computed from first principles and must be modeled using experimental data as an input. 


\section{Top quark physics}
\label{THEORY-TOPQUARK}

The top quark was first observed in 1995 at the Tevatron at Fermilab~\cite{Top-CDF,Top-D0}. The top quark's high mass makes it of great interest to high energy physics. Understanding the properties of the top quark and its associated production processes is critical to probing the Standard Model and searching for new physics. The mass of the top quark, along with the mass of the \Wboson\ boson, can constrain the mass of the Standard Model Higgs. This argument leads to the conclusion that the Higgs is relatively low mass (less than ~200 GeV), a prediction that turned out to be correct. From the perspective of a detector, top quark processes often appear similar to processes in many new physics models as well as rare Standard Model processes, such as the signal in the analysis described in Section~\ref{SECTION-BPRIME}.

Although it was predicted well before observation, its high mass of $172$ GeV made detecting it difficult. Due to the top quark's large width, it has the interesting quality of being the only quark with an observed decay lifetime (\textasciitilde $10^{-25}\ s$) much shorter than the strong force timescale (\textasciitilde $10^{-24}\ s$, the timescale for the quark to hadronize and turn into a jet). As a result, it decays instead of hadronizing and forming a colorless bound state. As a result, the detector does not see a jet; instead sees the top quark's decay products. As a result of the CKM matrix element $V_{tb}$ being close to 1, the top quark almost exclusively decays to a \Wboson\ boson and a bottom quark, as show in Fig.~\ref{FIGURE-THEORY-TDECAY}.

\FIG{tdecay}{The top quark typically decays into a \Wboson\ boson and a bottom quark.}{FIGURE-THEORY-TDECAY}

The \Wboson\ boson can decay to either a lepton and its corresponding anti-neutrino or hadronically into quarks which will produce jets. Approximately 32\% of the time it decays leptonically and the remaining 68\% of the time it decays to a pair of quarks. Leptonically includes tau leptons here, although when we talk about leptonic top decays from an experimental perspective we usually mean electron and muon decays, as those are directly detected by our detector.


The top quark was initially discovered by searching for \TTbar\ pair production, shown in Fig.~\ref{FIGURE-THEORY-TTBAR}, in which two top quarks are formed in the same QCD-mediated process. This production channel has a relatively high cross-section compared to processes in which only one top quark is formed. The relatively large cross-section for top pair production may be surprising, as the high mass of the top quark would lead one to expect that creating two simultaneously would be much less favorable than a single top quark because it requires much more energy. However, because \TTbar\ production can occur through the strong force while single top processes only happen through electroweak mechanisms (see Fig.~\ref{FIGURE-THEORY-WTCHAN}), the cross-section of \TTbar\ processes is much higher.

The top quark has two related properties that we will be measuring in this analysis: the top quark width and lifetime~\cite{D0TopWidth:2010}. In this analysis we indirectly measure the top quark width by taking advantage of the linear dependence of the signal cross-section on the width, shown in equation~\ref{EQUATION-THEORY-WIDTH}. 

\begin{equation}
\label{EQUATION-THEORY-WIDTH}
\Gamma_{t}^{obs} = \Gamma_{t}^{SM} \times \frac{\sigma^{obs}_{Wt}}{\sigma^{SM}_{Wt}}
\end{equation}

Here $\sigma^{obs}_{Wt}$ is the measured cross-section of the \Wtchan\ process, $\sigma^{SM}_{Wt}$ is the predicted Standard Model cross section of the \Wtchan\ process, and $\Gamma_{t}^{obs}$ and $\Gamma_{t}^{SM}$ are the measured and predicted top quark widths. 

The lifetime is a measure of the decay time of the top quark and can be calculated directly from the top quark width, as shown in equation~\ref{EQUATION-THEORY-LIFETIME}.

%If an experiment has infinite precision the width can be measured by looking at the mass distribution of a particle. In the case of the top quark, the uncertainty in the invariant mass distribution is much larger than the mass width, and a direct measurement of the width is impossible. Instead an indirect measure is employed by taking advantage of the linear dependence of the signal cross-section on the width, which means that the width can be measured by scaling the Standard Model width based on the ratio of the Standard Model cross-section to the measured cross-section, shown in equation~\ref{EQUATION-THEORY-WIDTH}. The lifetime is a measure of the decay time of the top quark and can be calculated directly from the top quark width, as shown in equation~\ref{EQUATION-THEORY-LIFETIME}.

\begin{equation}
\label{EQUATION-THEORY-LIFETIME}
\tau_{t} = \frac{\hbar}{\Gamma_{t}}
\end{equation}

\noindent
Single top processes, with their much lower cross-section than that for \TTbar\ production, are also important to particle physics, and were first observed in 2009 at the Tevatron~\cite{SGTOP-D0}. The existence and properties of single top processes reflect testable predictions of the Standard Model, and studying single top processes allows physicists to test these predictions. In addition, the single top processes uniquely allow for a direct measurement of $V_{tb}$, while previous measurements all required the assumption of three generations of quarks. 

Three main channels of single top processes are studied at the collider experiments: the t-channel, the s-channel, and associated production (also referred to as \Wtchan). The t-channel is the highest cross-section contributor, and has been observed independently of the other two channels at the LHC~\cite{TCHAN-ATLAS}. Its Feynman diagram is shown in Fig.~\ref{FIGURE-THEORY-STCHAN}. 

The s-channel cross-section is relatively small compared to the t-channel. At the LHC, it is even smaller than the \Wtchan, for reasons that will be discussed below. It has not been observed independently as of this writing, but it is an important channel with sensitivity to new physics. Its Feynman diagram is shown in Fig.~\ref{FIGURE-THEORY-STCHAN}.

The \Wtchan\ channel is the signal this analysis is searching for. The Feynman diagram for the \Wtchan\ process is given in Fig.~\ref{FIGURE-THEORY-WTCHAN}. 

The cross sections for the different single top production processes at a pp collider with $\sqrt{s}=7$ TeV are given in Table~\ref{TABLE-THEORY-SGTOP-XS}. Here $\sqrt{s}$ is the total center of mass energy of the proton-proton collision. The cross-sections are given in units of \pb. Examining the initial states of these three processes provides insight into the hierarchy of the cross-sections shown. The $t$-channel has the highest cross-section as it requires only an energetic gluon in addition to a quark. The \Wtchan\ process requires both an energetic gluon and an energetic b-quark. The $s$-channel is disfavored due to the energetic anti-quark required in addition to the quark. At the Tevatron the $s$-channel had a significantly higher cross-section than the \Wtchan\ because the Tevatron was a $p\bar{p}$ collider, making energetic anti-quarks much more common while the lower energies made energetic gluons less common. 

\begin{table}[!h!tbp] 
\begin{center}
\begin{tabular}{|l|r|}
\hline
$t$-channel  & 64.2 \pb\\
\hline
\Wtchan  & 15.6 \pb \\
\hline
$s$-channel & 4.6 \pb \\
\hline
\end{tabular}
\label{TABLE-THEORY-SGTOP-XS}
\caption{The cross-sections of the single top processes at the LHC at $\sqrt{s} = 7\ TeV$~\cite{SGTOP-XS1,SGTOP-XS2,SGTOP-XS3}.}
\end{center}
\end{table}

\DBLFIGL{tchannel}{schannel}{Feynman diagrams illustrating (a) the t-channel process and (b) the s-channel process.}{FIGURE-THEORY-STCHAN}



%\MEDIUMFIG{schannel}{The s-channel process}{FIGURE-THEORY-SCHAN}
 
\subsection{Wt-channel}
\label{THEORY-SIGNAL}
The signal in this analysis is the associated production of a \Wboson\ boson and a top quark, referred to as the \Wtchan. The process occurs primarily in two diagrams, shown in Fig.~\ref{FIGURE-THEORY-WTCHAN}. This process has not previously been observed independently of other single top measurements due to its relatively low cross-seciton at the Tevatron. The LHC's energy provides many more gluons with much more energy, significantly increasing the cross-section of the process. While the \Wtchan\ has a lower cross-section than the s-channel at the Tevatron, at the LHC the cross-section is significantly higher. Due to this small cross-section at the Tevatron, the LHC provides the first opportunity to observe the \Wtchan. 

\FIG{wtchannel}{The \Wtchan\ process.}{FIGURE-THEORY-WTCHAN}

Figure~\ref{FIGURE-THEORY-WTCHANFULL} shows \Wtchan\ production and decay. In this analysis we are looking in the dilepton subchannel, which means that both of the \Wboson\ bosons must decay leptonically to electrons or muons. This gives three lepton final states, two electron ($ee$), two muon ($\mu\mu$), and electron muon ($e\mu$). Despite the reduction of the size of the signal by an order of magnitude, the dilepton final state is much cleaner than final states that include hadronic \Wboson\ boson decays. Not only are leptons better measured in the detector, but the backgrounds to the dilepton final state are much better understood than the backgrounds to the single lepton final state. Note that the final state contains two oppositely signed leptons, two neutrinos, and a jet from the bottom quark. The neutrinos, while not directly detected, are observed as missing energy in the transverse direction denoted \MET, described in more detail in Section~\ref{SECTION-MET}. 

\FIG{wtchannelfull}{The decay chain of an example \Wtchan\ event.}{FIGURE-THEORY-WTCHANFULL}


\subsection{Backgrounds}
\label{THEORY-BACKGROUND}
The major backgrounds for this analysis are \TTbar, diboson, Drell-Yan, and \multijet. The background processes that contaminate this measurement each mimic the final state of the signal in some way. The \ttbar\ background is by far the largest background to our signal. Although the other backgrounds are much smaller, together they contribute about the same number of events as the \Wtchan\ signal itself.

The \TTbar\ background is shown in Fig.~\ref{FIGURE-THEORY-TTBAR2}. This is the top quark production channel through which the top quark was initially observed. The final state is similar to the \Wtchan, the only significant difference being an extra b-quark. However, this extra jet can be lost during the detection and reconstruction (discussed in Sections~\ref{SECTION-ATLAS-DET} and~\ref{SECTION-OBJDEF}), giving a reconstructed final state that matches the signal. In addition, the kinematics of these two processes are similar, making it difficult to design kinematic cuts that remove \TTbar\ without also removing the signal. In addition, the \TTbar\ cross-section is approximately an order of magnitude higher than that for the \Wtchan. 

\FIG{ttbar}{The \TTbar\ process. It has a final state with two b-quarks, two oppositely signed leptons, and two neutrinos.}{FIGURE-THEORY-TTBAR2}

The diboson backgrounds are shown in Fig.~\ref{FIGURE-THEORY-DIBOSON}. Although they are referred to as a single background, many processes contribute. There are two potential final states to consider. The first is a two lepton, two neutrino final state. For this to be mistaken as the \Wtchan\ process, an additional jet will need to be added to the event through ISR/FSR or \pileup\ (discussed in Section~\ref{SECTION-ATLAS-PILEUP}. The other final state contains two leptons and two jets. Here one of the jets must be lost during reconstruction and there must be significant fake \MET\ (MET not corresponding to a neutrino) added. \MET\ is how neutrinos can be indirectly observed in the detector, discussed in greater detail in Section~\ref{SECTION-MET}. The combined cross-section of these processes is marginally larger than the \Wtchan\ signal cross-section and even after the decrease in the events due to the difference in final state, the diboson background is the second largest background after \ttbar.

%The $WW$ process has a potential final state where both \Wboson\ bosons decay leptonically, the only difference from the signal being a missing b-quark, which can be faked by the misidentification of a jet produced by ISR/FSR radiation. The $WZ$ process has a final state where the \Wboson\ boson decays hadronically and the \Zboson\ boson decays leptonically, resulting in too many jets in the final state and no missing energy. The $ZZ$ process has a final state where one \Zboson\ decays hadronically and one Z decays leptonically, resulting in too many jets in the final state and no missing energy. These final states are good at mimicking the final state of the signal because they have two leptons and jets, and missing transverse energy can be faked if the energy from one of the jets is lost. Despite the low cross-section of the diboson processes, they remain a significant background in this analysis because of their similarity to the signal. 

\SEXFIGCUST{ww1}{ww2}{wz1}{wz2}{zz1}{zz2}{Feynman diagrams of diboson processes with dilepton final states. (a) and (b) are $WW$ processes. (c) and (d) are $WZ$ processes. (e) and (f) are $ZZ$ processes.}{FIGURE-THEORY-DIBOSON}

The Drell-Yan background, shown in Fig.~\ref{FIGURE-THEORY-DY}, makes up a significant fraction of the background contamination. It occurs when a \Zboson\ boson or $\gamma$ is created and then produces a lepton anti-lepton pair. For the kinematic region relevant to \Wtchan, this background is strongly dominated by the case where the mediating particle is a \Zboson\ boson, thus it is often referred to as the \Zjets\ background. The final state of this process does not strictly match the final state of the signal due to its lack of a jet and neutrinos. However, additional reconstructed jets can be added to an event in various ways, such as from ISR and FSR and \MET can be added through reconstruction errors.  Although most of the \Zjets\ events do not pass the jet requirement, because of its large cross-section relative to the \Wtchan\ signal cross-section it remains a significant background due to its large cross-section.

\FIG{dy}{The Drell-Yan background involves a photon or \Zboson\ boson.}{FIGURE-THEORY-DY}

The \multijet\ background is a difficult background to quantify, representing a wide range of processes. These processes are events where many jets are formed, but only one or zero leptons. A common example of this background is a \Wjets\ process containing many jets, but only one lepton, illustrated in Fig.~\ref{FIGURE-THEORY-WJETS}. The actual final state of these processes does not contain two real leptons, making them different from the signal final state. For a \multijet\ event to look similar to the signal in the detector at least one jet must be misreconstructed as a lepton. The ATLAS lepton reconstruction algorithms have a low rate of false positives, hence jets faking as leptons are uncommon ($<$ 1\% for high energy jets). Despite the rarity of faking a lepton, the \multijet\ events are so numerous that many still meet the selection criteria by chance. 

\FIG{wjets}{One contributing process to the multijet background is W+jets.}{FIGURE-THEORY-WJETS}
