\chapter{Introduction}
Science never rests. It constantly drives the boundaries of knowledge to new and unexpected realms. Through human history we have seen this knowledge progress from a practical, intuitive, and frequently incorrect understanding of the world to more rigorous models with greater predictive power than our ancestors could have ever dreamed. 
One of the themes seen throughout the history of science is the push to understand the basic building blocks of the universe. Ancient models posited four or five basic elements, made up of the most common materials found. In the 19th century, atomic theory was developed, which drove the smallest objects down to the atomic level, and then later even further when scientists discovered that atoms were made of protons, neutrons, and electrons. In the mid 20th century, scientists discovered that protons and neutrons were made of even smaller particles, which were named quarks~\cite{physicshistory}. Through the scientific process we probe the smallest scales, trying to understand the list of particles that we now consider fundamental.

Investigating these particles can be difficult, as the proton is tightly bound and high energies are required to break it apart. Even more energy is necessary to create the most massive particles we have discovered. To reach these massive energies an accelerator 24 kilometers in length, the Large Hadron Collider (LHC), has been constructed. At the LHC the proton is broken apart by accelerating two sets of protons to near the speed of light and colliding them. These collisions can create new particles, the products of which are detected at massive detectors built around the collision points. Through these collisions we study the properties of the known particles and, if we are lucky, discover new ones.

This dissertation will detail the search for a special kind of production of the most massive fundamental particle known, the top quark. This kind of production is known as the \Wtchan. In the following pages the workings of the LHC and the ATLAS detector will be discussed. From there I will explain the efforts required to go from a set of raw observations to a complete picture of the results of a collision. I will discuss how systematic uncertainties impact our measurement, and the steps we take to reduce them. Finally, the experimental and statistical methodology used to extract the measurements made will be detailed and the results will be shown.
