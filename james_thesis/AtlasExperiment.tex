\chapter{The LHC and the ATLAS Experiment}
\label{SECTION-ATLAS}
A vast experimental apparatus is required to investigate the physics of the single top \Wtchan\ process. An large and powerful accelerator must be designed to bring particles to near light speed and collide them. Also, an sensitive detector must be built around a collision point to study the collision products. An experiment of this scope rests on decades of planning, construction, and testing. This analysis uses proton-proton collisions from the Large Hadron Collider (LHC) measured by the ATLAS (A Toroidal LHC ApparatuS) detector.
\section{The Large Hadron Collider}
\label{SECTION-ATLAS-LHC}

The Large Hadron Collider (LHC) is a particle accelerator and collider 27 km in circumference situated on the French-Swiss border near Geneva, Switzerland~\cite{LHC}. It was designed to be the next generation high energy collider, surpassing the previous highest energy collider, the Tevatron~\cite{TEVATRON}. The Tevatron, which ran from 1983-2011, was the world's premier particle collider prior to the LHC. It made numerous discoveries, the most critical to this analysis being the first observation of the top quark~\cite{Top-CDF, Top-D0}, the first observation of single top quark production~\cite{SGTOP-D0, SGTOP-CDF}, and evidence for the Higgs~\cite{d0higgs}. The LHC is a circular accelerator which uses superconducting magnets and accelerating cavities to accelerate beams of particles to high energies and collide them together. Its primary function is to collide proton beams with proton beams. The total center of mass energy in each proton collision is a critically important quantity that determines the kind of physics one can study. The LHC was designed to run at 14 TeV, but due to technical problems with the superconducting magnets a collision center of mass energy of 7 TeV was used from 2009 through 2011. For 2012 the center of mass energy was increased to 8 TeV, and after a brief set of runs with lead ions, the LHC was shut down in 2013 until approximately 2015 to upgrade the collision energy to 14 TeV. For the purposes of this analysis, we use data collected between February 2011 and August 2011, and thus only use 7 TeV center of mass collisions.

The actual acceleration of protons to their final collision speed is performed in several steps. They are first accelerated to 50 MeV in the LINAC 2 linear accelerator, then the Proton Synchrotron Booster, a small circular accelerator further accelerates them to 1.4 GeV. The 1.4 GeV protons are delivered to the Proton Synchrotron which boosts them to 25 GeV. These protons are fed into the Super Proton Synchrotron and accelerated to 450 GeV. Finally, they are delivered to the LHC ring which accelerates them to their final collision energy. During this injection process starting in the Proton Synchrotron, the beam is divided into separated groups of protons called bunches. These bunches are capable of colliding at eight interaction points throughout the detector. Currently, there are four active interaction points spaced throughout the beamline with a 50-75 nanosecond separation, depending on the current running conditions.

%\FIG{lhcinjector}{The LHC injector chain~\cite{INJECTOR}.}{FIGURE-ATLAS-INJECT}

The LHC is host to seven major experiments:
\begin{enumerate}% {$\bullet$} {}
%\item ALICE (A Large Ion Collider Experiment) studies Pb-Pb collisions to investigate quark-gluon plasma~\cite{ALICE}.
%\item TOTEM (TOTal Elastic and diffractive cross-section Measurement) does specific measurements of total proton-proton cross-section and elastic scattering~\cite{TOTEM}.
%\item LHCb (Large Hadron Collider beauty) studies the physics of the bottom quark~\cite{LHCb}. 
%\item LHCf (Large Hadron Collider forward) investigates the physics of neutral pions produced in the forward region from proton-proton collisions~\cite{LHCf}.
%\item MoEDAL (Monopole and Exotics Detector At the LHC) will search for magnetic monopoles and other exotic physics. As of this writing, it is still being built~\cite{MoEDAL}.
%\item CMS (Compact Muon Solenoid)~\cite{CMS} is a large general purpose detector. It's intended to be able to do a variety of physics analyses while being sensitive to production of the Higgs boson, which was undiscovered at the time of design and construction.
%\item ATLAS is the detector used in this analysis. It is the largest experiment by volume and is designed to be a general purpose detector which can observe a wide range of physics phenomena. Like CMS, it has the dual goals of being able to observe as much physics as possible while still being sensitive to all of the unexcluded phase space of the Standard Model Higgs boson. 
\item ALICE (A Large Ion Collider Experiment) ~\cite{ALICE}
\item TOTEM (TOTal Elastic and diffractive cross-section Measurement) ~\cite{TOTEM}
\item LHCb (Large Hadron Collider beauty)~\cite{LHCb}
\item LHCf (Large Hadron Collider forward)~\cite{LHCf}
\item MoEDAL (Monopole and Exotics Detector At the LHC) ~\cite{MoEDAL}
\item CMS (Compact Muon Solenoid)~\cite{CMS} 
\item ATLAS (A Toroidal LHC ApparatuS)
\end{enumerate}


An important concept in high energy physics experiment is integrated luminosity, a measure of the interactions per unit cross-section. It is a measure of how much data has been collected. It can also be described as a rate, referred to as instantaneous luminosity, related to integrated luminosity by $L_{integrated}=\int L_{inst}(t)\ dt$. The relationship between luminosity, cross-section, and number of events is described by the following equation:

\begin{equation}
N_{events}=L\sigma,
\end{equation}

\noindent
where $N_{events}$ is the number of events of some process over some period of time, $\sigma$ representing the cross-section of the process, and $L$ representing the integrated luminosity over the period of time. The LHC is designed for a peak instantaneous luminosity of $10^{34}\ cm^{-2}s^{-1}$, or $10^{-5}\ fb^{-1}s^{-1}$, although it will almost certainly reach even higher luminosities as the operators become more experienced and as its hardware is upgraded. Fig.~\ref{FIGURE-ATLAS-LUMI} shows the measured delivered luminosity for the ATLAS experiment, and Fig.~\ref{FIGURE-ATLAS-INST} shows the peak instantaneous luminosity per run for the ATLAS experiment. Note the significant gains in rate that have been made in each year of running. The rapidly increasing luminosity from the LHC is a strong driver for the output of physics results from the experiments.

\VLARGEFIG{intlumivsyear}{The delivered luminosity to the ATLAS experiment in the years 2010, 2011, and 2012~\cite{LUMIPLOTS}. }{FIGURE-ATLAS-LUMI}
\FIGLAND{lumivstime}{The peak instantaneous luminosity per run delivered to the ATLAS experiment in the years 2010, 2011, and 2012~\cite{LUMIPLOTS}.}{FIGURE-ATLAS-INST}

While a higher rate of events is typically desired, especially by the larger experiments, there are difficulties when the rates get too high. At the high instantaneous luminosities at the LHC, multiple proton-proton interactions are likely to occur in each bunch crossing. This phenomenon is referred to as in-time \pileup, discussed in greater detail in Section~\ref{SECTION-ATLAS-PILEUP}.

\FIG{mu_2011_2012-nov}{The mean number of interactions per crossing taken in 2011 and between April 4th and November 26th in 2012~\cite{LUMIPLOTS}.}{FIGURE-ATLAS-CROSS}

\section{The ATLAS detector}
\label{SECTION-ATLAS-DET}
This analysis uses data collected by the ATLAS detector~\cite{ATLAS-TDR}. The ATLAS detector is large, the largest LHC experiment by volume at approximately 22,000 $m^3$ and has a mass of approximately 7,000 tons with over 100 million electronic readout channels. It is maintained and its data analyzed by a world-spanning collaboration of over 2900 scientists as of July 2012. It is able to detect a variety of particles, including photons, electrons, muons, and the products of quark hadronization. These particles are detected using many different technologies which will be discussed in the following Sections.

\subsection{Detector basics}
\label{SECTION-ATLAS-GEO}
There are a number of general concepts that must be discussed to understand the functioning of the detector. First consider the coordinate system describing the location of objects in the detector. The origin is defined as the interaction point. The proton beamline runs along the $z$-axis. The positive $z$ direction is counterclockwise around the LHC ring as viewed from above. The $x$-axis points towards the center of the ring, and the $y$-axis points up vertically. Typically, however, the coordinates are not discussed in Cartesian coordinates, instead using coordinates of $z$, $\eta$, and $\phi$, with $z$ remaining the same as in the Cartesian system. The angle $\phi$ is defined as the azimuthal angle from the $x$-axis in the $x$-$y$ plane, while $\eta$ is a more complex variable used for reasons described below. The vector $\vec{r}$ also sometimes represents the vector from the origin to the point.

The $\eta$ coordinate, also known as pseudorapidity, is derived from the more intuitive polar angle $\theta$, the angle between $\vec{r}$ and the $y$-axis. In high energy experiments, $\theta$ is no longer a useful variable because $\Delta\theta$ between two objects it is not relativistically invariant along the $z$-axis. Instead, angles are better measured using rapidity, defined as:

\begin{equation}
y = \frac{1}{2}ln\left(\frac{E+p_z}{E-p_z}\right)
\end{equation}

\noindent
In this equation we use natural units in which $c=1$. The rapidity transformation under a Lorentz boost $\beta = \frac{v}{c}$ along the $z$-axis is given below. It is shown that difference between rapidities is invariant under these transformations.

\begin{equation}
y \to y-tanh^{-1}(\beta),
\end{equation}
\begin{equation}
y_1 - y_2 = y_1^{\prime} - tanh^{-1}(\beta) - \left( y_2^{\prime} - tanh^{-1}(\beta) \right) = y_1^{\prime} - y_2^{\prime}.
\end{equation}

\noindent
Although the invariance of the rapidity is very useful, rapidity as a measurement of angle is problematic, as $E$ is dependent not only on the momentum of the particle, but also its mass. In other words, two particles with identical momentum traveling in identical directions but with different masses will have two different rapidities. There is also the practical concern that the mass of a given particle is not always known, thus rapidity cannot be calculated even if it were desirable. As a compromise, pseudorapidity is used instead, defined as:

\begin{equation}
 \eta = \frac{1}{2}ln\left(\frac{\left|\vec{p}\right|+p_z}{\left|\vec{p}\right|-p_z}\right)  = -ln\left(tan\left(\frac{\theta}{2}\right)\right)
\end{equation}

\noindent
This quantity has the benefit of $\Delta\eta$ being relativistically invariant for massless particles under boosts along the $z$-axis while being independent of mass. Note that in the case $m << E$, the equation for rapidity is equivalent to pseudorapidity. Since at ATLAS we often deal with particles with energies much higher than their mass, pseudorapidity proves to be a useful approximation for rapidity. The relationship between $\eta$ and $\theta$ is shown in Fig.~\ref{FIGURE-ATLAS-ETA}.

Often we consider the angular difference between two objects in the detector. Calculating this difference is straightforward if they lie on the $\eta-z$ or $\phi-z$ planes, but for the general case we need to define something more robust. This variable is called $\Delta R$, and is defined:

\begin{equation}
\Delta R = \sqrt{\left(\Delta\phi\right)^2+\left(\Delta\eta\right)^2}.
\end{equation}

\noindent
A concept often encountered in detector design is the radiation length. The radiation length is a material property that reflects the amount of energy lost by an EM particle passing through. When designing an experiment's EM calorimeter, it is important to maximize the number of radiation lengths in the calorimeter while minimizing the number of radiation lengths the particle will encounter before reaching the calorimeter. A similar concept exists for hadronic objects interacting with nuclei through the strong force called interaction length. The number of interaction lengths in the hadronic calorimeter must be maximized to capture all of the remaining energy of the hadronic shower.

\FIG{etatheta}{Relationship between $\eta$ and $\theta$.}{FIGURE-ATLAS-ETA}

A diagram of the ATLAS detector featuring the major subsystems is shown in Fig.~\ref{FIGURE-ATLAS-CUTAWAY}. Each of these subsystems is discussed briefly below.

\begin{list} {$\bullet$} {}
\item The magnet systems change the direction of charged particles, giving more information on their mass and momentum. There are two magnet systems, the solenoid magnet, used by the inner detector, and the toroid magnets, used by the muon detectors. These systems are discussed in more detail in Section~\ref{SECTION-ATLAS-MAGNETS}.
\item The tracking systems observe the path that particles take through the solenoid's magnetic fields to determine a particle's momentum and to aid in particle identification. The technical details on the tracking are discussed in Section~\ref{SECTION-ATLAS-TRACK}.
\item The calorimeters measure the energy of particles and help with particle identification. There are a large number of different technologies that are described in Section~\ref{SECTION-ATLAS-CAL}.
\item The muon systems are the largest system by volume. They detect and measure muons with the aid of the toroidal magnets. More detail is given in Section~\ref{SECTION-ATLAS-MUON}.
\end{list}


\FIGLAND{ATLAS_cutaway}{A diagram of the ATLAS detector and its subdetectors. Image of people added to the left side to illustrate scale.~\cite{ATLAS-EXP}}{FIGURE-ATLAS-CUTAWAY}

\subsection{Magnet systems}
\label{SECTION-ATLAS-MAGNETS}
The magnet systems in ATLAS curve the path of charged particles. By looking at the amount of deflection a particle experiences in a known magnetic field, the particle's momentum is better understood. These systems use superconducting magnets made of niobium-titanium, requiring them to be cooled to low temperature. Liquid helium at 4.5K is used for this cooling while the critical temperature of the superconductor is 1.9-2.7K above that. 

The solenoid magnet system generates a two Tesla magnetic field for use by the inner detector. Minimizing the number of radiation lengths present in the magnet system's structure is a critical constraint to maximize the sensitivity of the detectors. The solenoid is designed to present a maximum of 0.66 radiation lengths to an incoming particle. The magnetic field generated is axial along the z-axis, which means that it will cause a charged particle to bend in the x-y plane.

There are three sets of toroidal magnets at ATLAS, one in the barrel and one at each end-cap. These magnets bend the path of muons passing through the muon detectors. The barrel magnet provides a 0.5 T average magnetic field and 1.5-5.5 Tm of bending power, while the end cap magnets each provide a 1.0 T average magnetic field with 1-7.5 Tm of bending power. The barrel services the $\left|\eta\right| < 1.4$ region, while the end caps service the $1.6 < \left|\eta\right| < 2.7$ regions. The region $1.4 < \left|\eta\right| < 1.6$ is covered by a combination of the two. The magnetic field generated is inhomogeneous, but mostly perpendicular to the path of muons. Extensive testing was done to construct an detailed map of the magnetic fields created by the toroidal magnet systems. An example of the bending power of the magnetic field as a function of $\left|\eta\right|$ is shown in Fig.~\ref{FIGURE-ATLAS-MAGNET}. The bending power measures the amount of deflection on a charged particle as it passes through and it is an important quantity because it, along with the resolution of the detectors, determines what ranges of momenta can be measured and with what precision.

\FIG{magnetfield}{The predicted bending power through MDT layer as a function of $\left|\eta\right|$ for infinite momentum muons~\cite{ATLAS-EXP}.}{FIGURE-ATLAS-MAGNET}
\subsection{Inner detector tracking}
\label{SECTION-ATLAS-TRACK}

The inner detector tracking system gives high resolution information about the path particles take through the detector as they pass through the magnetic field of the inner solenoid~\cite{INNERDET}. Combined with information from other detectors, the inner detector is a powerful tool for correctly identifying particles, determining their momenta, and locating their origin. The inner detector has sensitivity in the range $\left|\eta\right| <2.5$. The ATLAS tracking system uses three different subdetectors to accomplish this task, as illustrated in Fig.~\ref{FIGURE-ATLAS-TRACK}. 


\FIG{tracking}{A diagram of the three subdetectors of the inner detector and their relative sizes~\cite{ATLAS-EXP}.}{FIGURE-ATLAS-TRACK}


The highest resolution tracking system is the pixel detector~\cite{PIXEL-DET}. It is made up of three barrel layers and six end-cap disk layers, three on each side of the detector. These layers contain approximately 80 million silicon sensors giving it a resolution of up to 10 $\mu$m in R-$\phi$ space and 115 $\mu$m along the $z$-axis. High resolution tracking so close to the interaction point allows for accurate measurement of the origin of each particle, which is useful in verifying that different particles originate from the same interaction and also provides discrimination power for particle identification. Due to its close proximity to the interaction point, the pixel detector is designed to be able to withstand the large amounts of radiation expected. 

The next system out from is the semiconductor tracker (SCT)~\cite{SCT}. These four cylindrical double-layers of sensors function similarly to the pixel detectors, but instead of being small pixels, they are long strips stretching in the $z$-direction. The pairs of sensors are angled slightly with respect to the $z$-axis to allow measurement of the $z$-coordinate. This angle makes the SCT more cost effective than simply extending the pixel detector while still fulfilling the physics requirements, as the high resolution of the pixel detector is not necessary farther from the beamline. The spatial resolution of the SCT is 17 $\mu$m in R-$\phi$ space, and 580 $\mu$m along the $z$-axis.

The final inner detector system is the transition radiation tracker (TRT), which takes advantage of transition radiation, the radiation emitted when a particle moves across the border between two materials with differing dielectric constants~\cite{TRT,TRT2}. It is formed from 73 (barrel) or 163 (endcap) layers of 4 mm diameter drift tubes containing a mixture of 70\% xenon, 27\% carbon dioxide, and 3\% $O_2$. When a particle passes through the surrounding layer made up of a  polypropylene-polyethylene fiber mat, it will produce transition radiation which ionizes the gas in the tube. The signal is picked up by a wire that runs through the middle of each straw, which is then interpreted as a hit. The energy released by transition radiation is dependent on the $\beta$ of the particle. Examining the energy profile as a particle passes through the TRT allows particles to be identified. In particular, the TRT is critical for discriminating electrons from charged pions, giving a rejection factor greater than 20 for pions at 90\% electron efficiency~\cite{ATLAS-TDR}. The TRT is the largest of the three tracking detectors and even though its absolute resolution of approximately 170 $\mu$m per straw is the lowest, the number of hits it receives makes it critical for particle identification and momentum measurements. 

\subsection{Calorimetry}
\label{SECTION-ATLAS-CAL}

The calorimetry systems measure the energy of certain particles in the detector and in identify particles. There are two layers of calorimetry, an EM (electromagnetic) calorimeter which is sensitive to low mass particles that interact electromagnetically, for example electrons and photons, and a hadronic calorimeter which is sensitive to hadrons. The calorimeters make up the second to last layer of the detector and should stop nearly all of the remaining outgoing particles with the exception of muons, neutrinos, and possibly exotic undiscovered particles. Figure~\ref{FIGURE-ATLAS-CAL} shows the layout of the layers of calorimeters.

\FIGLAND{calorimeter}{A diagram of the layers of the calorimeter~\cite{ATLAS-EXP}.}{FIGURE-ATLAS-CAL}

The EM calorimeter is made up of a barrel section and two end-cap sections (EMEC) covering the region $\left|\eta\right|< 2.5$. It contains sections of lead plates and electrodes with liquid argon as a sampling medium. High energy electrons shower Bremsstrahlung radiation while interacting with the lead plates~\cite{DETECTORS}. These high energy photons will then pair produce to form smaller energy electrons and positrons. The cycle repeats until the photons and leptons remaining are low enough energy to ionize the liquid argon. These ionized electrons are then detected by the electrodes. 

The EM calorimeter is designed to be thick enough to stop the propagation of all but the most energetic photons and electrons. Corrections are applied to account for energy lost in the previous layers of the detector to get an accurate estimate of the total energy. Note that since the mechanism for generating radiation is Bremsstrahlung, there is a mass dependence of $1/m^4$. This is why these calorimeters are so sensitive to electrons, but not sensitive to muons, which are approximately 200 times more massive, resulting in a $\left(1/200\right)^4=1/1,600,000,000$ reduction in sensitivity.

The hadronic tile calorimeters operate in the range $\left|\eta\right| < 1.7$, using steel as the absorber and scintillator tiles as the active medium~\cite{TILE}. The iron in the steel has an interaction length much larger than its radiation length. Scintillating tiles are used here and not in the interior layers because the scintillating tiles are not nearly as radiation hard as liquid argon systems but are much more affordable. Unlike the EM calorimeter which relies on electromagnetic interactions, hadronic calorimeters create cascades which rely primarily on the strong force. The basic concept of sampling is similar to the calorimeter, where the passive medium initiates cascades which are then measured in the active medium. In the case of the tile calorimeters, the showers originate through mostly through inelastic interactions with nuclei in the steel layers. The charged particles passing through the scintillating tiles excite the molecules to a higher energy state. Upon returning to their ground state, the molecules emit ultraviolet photons that are read out though fibers to photomultiplier tubes. Because the cascades created by the hadronic calorimeter are driven primarily by the strong force, muons will pass through this layer with minimal interaction. The hadronic end cap calorimeter (HEC) uses similar principles, but with copper as the absorber and liquid argon as the active medium.

The forward calorimeter (FCal) covers the extreme $\eta$ region of the detector, \mbox{$3.1 < \left|\eta\right| < 4.9$}. Due to its proximity to the beamline, it is sensitive to the \pileup\ effects described in Section~\ref{SECTION-ATLAS-PILEUP}. It is composed of three modules projecting away from the interaction point. The module closest to the interaction point is designed for EM interactions, using a copper absorber, while the other two use a tungsten absorber to create hadronic interactions. The interactions are sampled by thin layers of liquid argon. Copper was chosen to give high resolution for the EM interactions and its high conductivity allows for quick heat removal. Tungsten were chosen because it create showers with small lateral spread, giving better containment to the laterally thin FCAL.

\subsection{Muon spectrometer}
\label{SECTION-ATLAS-MUON}

The muon spectrometer is the outermost layer of the ATLAS detector. It tracks muons as they bend through the toroidal magnetic field in the region \mbox{$\left|\eta\right| < 2.7$}, allowing for their momenta to be measured. The amount of bending is determined by the magnets as discussed in Section~\ref{SECTION-ATLAS-MAGNETS}. The detection occurs in four subsystems: the monitored drift tubes (MDT) and the cathode strip chambers (CSC) make detailed measurements, while the resistive plate chambers (RPC) and the thin gap chambers (TGC) are primarily designed to allow quick trigger decisions to be made. Figure~\ref{FIGURE-ATLAS-MUON} illustrates the layout of the muon system. 

\VLARGEFIG{muonsystem}{A diagram of the muon detector systems~\cite{ATLAS-EXP}.}{FIGURE-ATLAS-MUON}

The MDTs are installed to cover \mbox{$\left|\eta\right| < 2.7$}. They are made up of many pressurized drift tubes approximately 3 cm in diameter running in the z direction. Muons ionize the gas as they pass through, releasing electrons, and these electrons are attracted to a central wire at high positive potential. As they approach the wire they pick up enough energy to ionize the surrounding gas. This ionization creates an avalanche of electrons hitting the wire and this signal is then propagated to the electronics. These chambers are located throughout the $\eta$ space of the detector, and the geometry varies throughout. The placement of the tubes and the deformation of their internal geometry are well known due to monitoring by built-in optical systems, allowing an optimal resolution of tracked muons of \mbox{50 $\mu$m}.

The Cathode Strip Chambers (CSCs) give a high resolution view of the region \mbox{$2 < \left|\eta\right| < 2.7$}. Similar to the MDT, the CSC is made up of chambers filled with pressurized gas. Muons pass through, ionizing the gas. In the CSC, instead of a single central wire, the chambers are strips filled with many wires. The wires induce a charge onto cathodes of the side of the strip. These cathodes are segmented, giving additional information about the angular coordinates of the muon. The CSCs are divided into smaller and larger wedge chambers which alternate around in the $\phi$ direction of each of the endcap regions. As a muon leaves the detector in the appropriate $\eta$ range, it will pass through four planes of CSCs, giving up to four measurements of its $\eta$ and $\phi$ coordinates. The CSC subsystem has a resolution of \mbox{40 $\mu$m} in the $R$ direction and 5 mm in the $\phi$ direction.

The Resistive Plate Chambers (RPCs) are used for triggering in the barrel region \mbox{$\left|\eta\right| < 1.05$}. The RPCs are made of parallel resistive plates \mbox{2 mm} apart. An electric field of \mbox{4.9 kV/mm} applied to these plates cause discharges along the ionized tracks as muons pass through. The discharge signal is read out to conducting strips attached to the plates. The plates are resistive so that the discharge is localized and doesn't immediately discharge the rest of the plate while the charge replenishes. The discharge is quick and consequently the RPCs are useful for triggering. There are three layers of RPCs in the barrel, each layer containing two detectors. Therefore, a muon going through the barrel region will be detected up to six times by the RPC, allowing a reasonable estimate of its path through the region. The distance between the RPCs determines the observable energies of the muons, as the muon energy determines the amount of bending applied to the muons between layers. The bending must be large enough to be observable given the resolution of the RPC. The design of the ATLAS RPCs allows muons in the range of \mbox{6-35 GeV} to be selected with a spatial resolution of \mbox{10 mm} in the z direction and \mbox{10 mm} in the $\phi$ direction.

The Thin Gap Chambers (TGCs) are used for triggering in the end-cap region \mbox{$1.05 < \left|\eta\right| <2.4$}. Additionally, they add information to measurements from the MDTs about the $\phi$ coordinate. The TCGs are made up of many wires enclosed in a gas volume between two plates separated by 2.8 mm that read out to conductive strips perpendicular to the wires. The information from the strips can also determine information about the $\phi$ coordinate. The TGCs are constructed in sets of doublets and triplets, the number of each depending on the location in the detector. The TGCs have a resolution of \mbox{2-6 mm} in the R direction and \mbox{3-7 mm} in the $\phi$ direction. Although their resolution is high compared to the MDTs and the CSCs, the TGCs have a fast response time and are critical for the triggering discussed in the next section.

\subsection{Triggering and data acquisition}
\label{SECTION-ATLAS-TRIG}

Given the bunch crossing rate of the LHC combined with the size and complexity of the detector systems, it is clear that ATLAS is collecting information at a rate that is unreasonable to store in real-time. A back of the envelope calculation reveals that with 25 ns bunch crossings and at least one event occurring per crossing, there are 40 million events every second. Given that the average event is 1.3 Mbytes in size~\cite{ATLAS-EXP}, 40 millions events per second is an impossible rate at which to collect and record data, thus a triggering system has been developed that makes decisions about which events to keep and which to discard. There are three levels of decision making that occur for each selected event: level 1, level 2, and the event filter.

The level 1 (L1) trigger system makes use of the calorimeter and muon systems to make rough and quick judgments about which to keep. It is the level that first evaluates each event, and so it must make a decision for every crossing. It accepts only one in one thousand events, reducing the incoming rate for the level 2 system to 40k events per second. Due to the short timescale to make decisions, the L1 system does minimal processing on the data. Typically it is limited to local phenomenon such as determining how many clumps of energy are in various detectors, but the calorimeter has additional hardware designed to also calculate the global missing transverse energy and the total transverse energy. The L1 detector calls these clumps of energy Regions of Interest (RoIs), and this list of regions is used as a seed for the level 2 systems.

The level 2 (L2) trigger system a factor of one thousand more time than the L1 trigger to make decisions, and as a result is able to use information from more of the detector subsystems, including the tracking, to construct a better picture of the events. It can evaluate criteria like the shape of showers, is able to do much better particle identification than the L1 trigger, and reconstructs the RoI energies with much better resolution. It has an input rate of 40 kHz from the L1 system, and outputs to the event filter at a rate below 3.5 kHz.

The event filter (EF or L3 trigger) uses advanced offline reconstruction techniques utilizing the full capabilities of the detector to make its decision. The events being processed by the EF are temporarily written to memory, so that even a failure in a computing node will not cause a loss of the event. It also classifies the events that pass into various streams. The most obvious stream is the set of events collected for physics analysis, but there are also streams for detector calibration and other such tasks. It outputs events to be saved at approximately 200 Hz.


\subsection{Pile-up}
\label{SECTION-ATLAS-PILEUP}

Often an interaction can interfere with the detector readout of another unrelated interaction. This phenomenon is referred to as \pileup. There are two types of \pileup, called in-time \pileup\ and out-of-time \pileup. In-time \pileup\ is caused by multiple interactions in the same bunch crossing creating many hits in detectors from events other than the one selected by the trigger, leading to objects and tracks being assigned to the wrong interactions. Sometimes this only impacts the lowest level triggers, but this \pileup\ can also have an effect on the offline reconstruction, too, especially at high instantaneous luminosities. Out-of-time \pileup\ occurs when interactions from an earlier or later bunch crossing bleed into the readings of the current bunch crossing. This \pileup is caused by detectors that have a response time longer than the bunch crossing time. Out-of-time \pileup\ has the most impact on the LAr calorimeters due to their long electronics response time (up to \~400 ns). The muon gas chambers also have a long electronics response time, but they aren't as sensitive to \pileup\ because the rate of detected particles is lower. Out-of-time \pileup\ can also occur in the in the inner detector when particles spiral in the detector for longer than the bunch crossing time. Any simulation of interactions in the detector must take into account both of these types of \pileup.

% There can also be interference from interactions immediately before and after the interaction of interest due both the finite response time of detector systems and, in the inner detector, particles which get spiral in the detector for longer than the bunch crossing time. This kind of interference is called out-of-time \pileup. Out-of-time \pileup\ has the most impact on the calorimeters, which have the longest electronics response time (up to ~400 ns). This can make it difficult for an experiment to separate out the effects of multiple interactions into different events. The procedure for dealing with pileup varies between experiments, and will be discussed in more detail in Sections~\ref{SECTION-SYSTEMATICS} and~\ref{SECTION-ATLAS-PILEUP}
