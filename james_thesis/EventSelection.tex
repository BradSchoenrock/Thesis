\chapter{Event Selection}
In this section the selection criteria applied to the data and simulated events and the reasoning behind them are described. These selection cuts are chosen because they keep clean signal events while rejecting background and poorly reconstructed signal events. These cuts come from the Top Working Group, although two of our cuts, the \Zboson\ veto cut and the \MET angular correlations cut are specific to this analysis.

\section{Selecting events from data}
The data used are 7 TeV proton-proton collision data from between February 2011 and August 2011. Unprescaled single electron and muon triggers are used to choose event candidates, and the event is required to be flagged as having taken place during a period of running where the LHC had stable beams and all detectors were running without issue. These quality criteria are applied using a list of sections of runs, called a Good Runs List (GRL). These data represent \LUMI\ of integrated luminosity.

\section{Selecting dilepton events}
To select dilepton events and reject our backgrounds a chain of cuts is applied to both the data and the simulated events. The cuts applied in this analysis are:

\begin{list} {$\bullet$} {}
\item Primary vertex cut.
\item Reject events with a ``Bad'' jet.
\item Reject events that may be contaminated by the LAr hole problem.
\item Reject events with an electron overlapping a muon.
\item Reject events with two muons satisfying $\Delta\phi > 3.10$.
\item Require two selected leptons.
\item Trigger selection of an electron ($ee$ subchannel only).
\item Trigger matched to reconstructed electron ($ee$ subchannel only).
\item $\MET\ > 50\ GeV$.
\item \Zboson\ veto cut, $81\ GeV < M_{ll} < 101\ GeV$ ($ee$ and $\mu\mu$ only).
\item $\Delta\phi(l_1,\MET)+\Delta\phi(l_2,\MET) > 2.5$.

\end{list}

I will now go into more detail on each of these selection cuts and discuss the rationale behind them.

A number of event quality cuts are applied to eliminate events that have been poorly reconstructed or otherwise do not represent a good collision event. These cuts are determined by the top working group and the implementation and rationale for each is well documented~\cite{TOPCOMMONOBJECTS}. To ensure that the event is from a collision event, a cut is applied requiring that the first primary vertex in the event have at least four tracks. 

Another cut is applied to remove events that contain ``Bad'' jets. These are jets that have been determined to not be associated with a real in-time energy deposit. Because the presence of a single high \pT\ ``Bad'' jet can pollute the event kinematics, any event with a bad jet with $p_T > 10\ GeV$ is removed from the selection. 

A third cut rejects events if they may have been impacted by LAr hardware issue during running discussed in section~\ref{SECTION-DEFINEELECTRONS}. 

The fourth cut rejects events if a selected muon and a selected electron share a track. This would be an indication that the electron is erroneously reconstructed from muon energy deposits in the calorimeter from the muon, and consequently these events are discarded. 

In the $\mu\mu$ channel, there is an additional veto to remove coincident cosmics events. Cosmic muons can appear as two muons in the detector, with one track from the muon going in and another back to back track of the muon leaving. As a result, this cut rejects events with pairs of tracks muon that match up closely. Specifically, muons are required to have been reconstructed with opposite charge, both having an impact parameter greater than $0.5\ mm$, and must have $\Delta\phi > 3.1$. 

After these cuts are applied, the remaining events have no obvious reconstruction errors and originate from collisions in our detector. These events are subjected to further cuts to enhance the signal to background ratio as much as possible. This analysis is divided into three channels with differing lepton combinations. Events are selected that have two electrons ($ee$ channel), an electron and a muon ($e\mu$ channel), or two muons ($\mu\mu$ channel). Each of these channels requires that the leptons selected meet the quality criteria defined in Sections~\ref{SECTION-DEFINEELECTRONS} and~\ref{SECTION-DEFINEMUONS}. 

In the $ee$ channel it is also ensured that the {\sc EF\_E20} electron trigger fired for this event. Furthermore, this triggering object must be consistent with at least one of our selected electrons by meeting the requirement $\Delta R(electron,\ trigger\ object) < 0.15$. Due to a bug in simulating the trigger conditions for the muons in the 2010 simulated events, the same procedure cannot be repeated for muons.

We consider three regions of our analysis defined by the number of jets: 1-jet, 2-jet, and 3+jet. As the largest background, $\ttbar$, contains two jets in the final state, 1-jet events are considered the primary signal region. Since the $\ttbar$ background yield dominates in the 2+jet bin, events with more than one jet are used to constrain the uncertainty in the $\ttbar$ normalization. One distinguishing characteristic of the $\Wt$ signal is the presence of two neutrinos, hence it is required that events have $\MET\ > 50\ GeV$. This cut eliminates much of the \multijet background. 

Even after all of the previous cuts, the $ee$ and $\mu\mu$ channels suffer from large contamination from $\Zee$ and $\Zmm$ events due to their relatively large cross-sections. To reduce the impact of these channels, an additional cut is made on events with a dilepton invariant mass near the \Zboson\ boson mass, $81\ GeV < M_{ll} < 101\ GeV$. This cut is independent of whether the channel is $ee$ or $\mu\mu$, because in this energy regime the energy resolutions of reconstructed electrons and muons are similar.

A powerful cut reduces the $\Ztt$ background significantly. This cut is performed by taking the sum of the $\Delta \phi$ of both leptons with the missing transverse energy vector. The cut value is optimized to maximize background rejection while minimizing signal loss. This triangle cut is defined as:

\begin{equation}
\Delta\phi(l_1,\MET)+\Delta\phi(l_2,\MET) > 2.5.
\end{equation}

\noindent
The resulting impact on the events is shown in Figs.~\ref{FIG-ZTAUTAU} and~\ref{FIG-ZTAUTAU2}. Although there is some discrimination power in the individual distributions, when they are summed together the reason for the triangle cut becomes obvious, as we are able to eliminate many background events without losing much signal. 

\TRPFIGLEG{paper_ll1+j_LP2fb_v4_DeltaPhiLeadingLeptonMissingEt_flat}{paper_ll1+j_LP2fb_v4_DeltaPhiSubLeadingLeptonMissingEt_flat}{legend}{The impact of the triangle cut on signal and background: (a) the angle between leading lepton and $\MET$ (b) the angle between the second lepton and $\MET$. The simulated events are represented by the solid regions, while the data are represented with a black dot.}{FIG-ZTAUTAU}
\FIG{plhc_presel_triangularDeltaPhiLepMET}{The effect of the triangle \Ztt\ veto cut in two dimensions. }{FIG-ZTAUTAU2}

An event selection is applied that divides the events into three exclusive bins: dielectron, dimuon, and electron-muon. These bins are examined separately in the control regions to make sure that the backgrounds are well modeled. Plots of selected variables in these bins are shown in Appendix~\ref{APPENDIX-CONTROLREGIONS}. Examining them in the bins independently gives us a useful tool for diagnosing the cause of disagreements between the data and the simulated events. Since the kinematics of these three subchannels are similar, for the final analysis they are merged together. 

\section {Event yields}
Table~\ref{TABLE-EVENT-YIELD-PRESEL} shows the resulting yields after selection along with the simulated statistical and data-driven uncertainties. 
We expect 3003.5 events in our signal region and observe 3059. 
The data are in reasonable agreement with our background and signal estimates given the data statistical uncertainty and the simulated event yield uncertainty. 
In addition, agreement is also observed individually in the $ee$, $e\mu$, and $\mu\mu$ channels. Distributions of relevant kinematic variables are shown in Fig.~\ref{FIGURE-PRESEL-CBD1} 
for the combined channel. Similar Figs. 
are available for the three individual $ee$, $e\mu$ and $\mu\mu$ channels in 
Appendix~\ref{APPENDIX-CONTROLREGIONS}.
 
\begin{table}[htdp]
\begin{center}
   \begin{tabular}{l|cccr}
    \hline
    Process      & $ee$               & $\mu\mu$        & $e\mu$   & All combined \\
    \hline \hline 
    $Wt$  &       38.6 $\pm$ 0.8 &       65.3 $\pm$ 1.0 &      119.7 $\pm$ 1.3 &      223.6 $\pm$ 1.8\\
\hline
   \TTB\  &      438.1 $\pm$ 4.5 &      738.5 $\pm$ 5.8 &     1336.0 $\pm$ 7.8 &     2509.6 $\pm$ 10.7\\
    $WW$  &       16.7 $\pm$ 2.4 &       29.0 $\pm$ 2.9 &       55.3 $\pm$ 4.1 &      101.0 $\pm$ 5.6\\
    $WZ$  &        4.9 $\pm$ 0.7 &       13.8 $\pm$ 1.2 &        8.1 $\pm$ 0.9 &       26.8 $\pm$ 1.7\\
    $ZZ$  &        0.9 $\pm$ 0.1 &        4.5 $\pm$ 0.3 &        0.4 $\pm$ 0.1 &        5.8 $\pm$ 0.3\\
    \Zee\ (DD)&   35.7 $\pm$ 2.5 &        ---           &        ---           &       35.7 $\pm$ 2.5\\
    \Zmm\ (DD)&    ---           &      69.5  $\pm$ 3.1 &        ---           &       69.5 $\pm$ 3.1\\   
   \Ztt\  (DD)&    1.1 $\pm$ 0.6 &        5.7 $\pm$ 3.4 &        2.6 $\pm$ 1.6 &       9.4 $\pm$ 3.8\\
 Fake dilepton (DD)    &    9.0 $\pm$ 9.0 &        ---           &        6.9 $\pm$ 6.9 &       15.9 $\pm$ 15.9 \\ 
 \hline
 Total expected& 542.0 $\pm$ 10.7 &      926.3 $\pm$ 8.1 &       1529.0 $\pm$ 11.4 &   2997.3 $\pm$ 17.6 \\
 \hline
 Data Observed  &          573 $\pm$ 24  &                905 $\pm$ 30  &               1581 $\pm$ 40  &     3059 $\pm$ 55\\
    \hline\hline
   \end{tabular}
 \caption{The observed and predicted event yields in the selected dilepton sample with at least one jet and for an integrated luminosity of \LUMI. Uncertainties represent the effect of MC statistics for the MC-based estimates and the total uncertainty for the data-driven estimates.}
\label{TABLE-EVENT-YIELD-PRESEL}
\end{center}
\end{table}

\SEXFIGLEG{paper_ll1+j_LP2fb_v4NoShift_NJets_flat}{paper_ll1+j_LP2fb_v4_Jet1Pt_flat}{paper_ll1+j_LP2fb_v4_HT_AllJets_flat}{paper_ll1+j_LP2fb_v4_MET_flat}{paper_ll1+j_LP2fb_v4_LeadingLeptonPt_flat}{legend}{Histograms of the selected sample with combined $ee$, $e\mu$ and $\mu\mu$ channels. The simulated events are represented by the solid regions, while the data are represented with a black dot. (a) Jet multiplicity, (b) Leading jet \pT, (c)$H_T(jet)$, (d) \MET, (e) Leading lepton \pT.}{FIGURE-PRESEL-CBD1}

