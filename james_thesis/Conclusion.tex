\chapter{Conclusion}
We have analyzed \LUMI\ of data collected with the ATLAS detector. In our search for the \Wtchan\ we have seen a statistically significant excess of 3.3$\sigma$. This is sufficient to claim evidence, and although this does not meet the $>5\sigma$ criteria to claim observation, it is a significant step to verifying the Standard Model prediction. The estimated cross-section is also extracted from the data, giving a result of $\sigma(pp\rightarrow Wt + X) = 16.8 ^{+2.9}_{-2.9} \mathrm{(stat)} ^{+4.9}_{-4.9} \mathrm{(syst)}~pb$. This analysis also allowed us to make measurements of other Standard Model parameters. The CKM matrix element $V_{tb}$ is measured to be $|V_{tb}| = 1.03^{+0.16}_{-0.19}$. The width of the top quark is measured at $\Gamma_{t}^{obs.} = 1.4\pm 0.5~\rm GeV$ (Note the increase in the percent uncertainty due to the $|V_{tb}|^2$ dependence), giving a lifetime of $\tau_{t}=(4.7^{+1.2}_{-1.2})\times 10^{-25}~s$. These measurements are all consistent with theoretical Standard Model predictions and other experimental measurements. This analysis is published in Physics Letters B~\cite{WTEVIDENCE}.

In this analysis I implemented the BDT used, which includes the variable selection and testing, the training procedure, and the parameter optimization. I implemented the ATLAS and top group recommendations for the object definitions, event selection, and studied most of the systematics (the jet energy scale, jet reconstruction, jet ID, lepton ID, lepton resolution, \MET, and pile-up uncertainties). The data-driven \Ztt\ normalization is estimated by me. I prepared the plots of the BDT and plots of the variables used. During the preparation of the paper and the associated note, I gave many single top working group talks and the approval talk to the top working group. I also collaborated with Huaqiao Zhang to perform many cross-checks while going through review.

With time the systematic uncertainties will be better understood and in the future this analysis will be repeated with more data. However, there is ample room for improvement in the analysis procedure itself. Note that the BDT optimization is done using only the nominal \MC. A look at the uncertainty composition of the final cross-section measurement will reveal that this analysis is quite systematically limited. A BDT optimization using information from the systematically shifted datasets could bring significant improvement to the result as a whole. This is not a trivial undertaking, as the existing toolsets are not equipped to do this kind of optimization out of the box, however implementing a systematics-sensitive optimization has the potential to greatly increase the significance.

This evidence for the existence of the \Wtchan\ was also confirmed independently by the CMS collaboration~\cite{CMSEVIDENCE}. Both the CMS and ATLAS collaborations will continue to update these analyses with better analysis techniques, a better understanding of the systematic uncertainties, and more data. The discovery of the \Wtchan\ is not the end, of course. Precision measurements of $V_{tb}$ and the top quark properties and searches for new physics in the \Wtchan\ signal region are all exciting new analyses waiting to be explored.

The LHC era is already showing its promise, giving exciting results like the recent Higgs discovery~\cite{HiggsATLAS,HiggsCMS} and confirming the predictions of the Standard Model. Even with the Higgs boson discovered, there remains much discovery ahead. The LHC will be running for years, pushing our understanding forward. With each collision we strive for a better understanding of our universe, and with time and hard work, these efforts will be rewarded.
