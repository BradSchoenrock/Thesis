\chapter{Signal and Background Estimation}
\label{SECTION-BACKGROUND-ESTIMATE}

The signal and backgrounds processes are modeled using a variety of techniques. Primarily they are based on \MC\ models using a pseudo-random number generator (PRNG) to simulate many events. These simulations contain many steps, chaining together several pieces of software to arrive at a complete simulated event. Different software is used to simulate different processes, as some software is known to simulate certain classes of processes better than others. In addition, the same process is simulated using several different software combinations to investigate the dependence of the software on the result. These estimates are done by the analyzers themselves. In particular, I performed the $\Ztt$ estimate.

\begin{enumerate}
\item The events are generated at the parton level and simulated through the initial interaction, which takes into account the parton distribution function of the proton and the underlying Standard Model physics. This analysis uses the physics generators \AcerMC\ 3.5~\cite{AcerMC}, \ALPGEN\ 2.13~\cite{ALPGEN}, \POWHEG\ 1.0 patch 4~\cite{POWHEG1,POWHEG2}, and \MCNLO\ 3.41~\cite{MCNLO1,MCNLO2}. The processes created by each generator are detailed in Tables~\ref{TABLE-BACKGROUND-MCSAMPLES1} and~\ref{TABLE-BACKGROUND-MCSAMPLES2}.

\item Bare quarks and gluons are showered into jets using hadronization and parton showering software. The two hadronization simulation software packages are \Pythia\ 6.423~\cite{PYTHIA} and \HERWIG\ 6.510~\cite{HERWIG}.

\item The detector is simulated using the \GEANT\ 3.5~\cite{GEANT4} software package. This simulates the geometry of the ATLAS detector in detail, such as the energy resolution of detector elements and \pileup\ effects.

\item For the remainder of the chain the same reconstruction steps are applied to the simulated events\ as to the data.

\end{enumerate}

\section{\MC\ modeling}

\label{SECTION-BACKGROUND-MC}

Good simulation of high energy physics events is difficult. It is for this reason that many of the systematic uncertainties shown in Section~\ref{SECTION-SYSTEMATICS} are related to the simulation steps discussed above. Additionally, many cross checks are done to ensure that the simulation is is an accurate model of the physics and detector. Simulated samples are shared across the collaboration, therefore many of the cross checks are done at the collaboration or physics group level. However we independently compare our data with the simulation to verify that the modeling is good. The simulated samples are discussed in detail below and summarized in Tables~\ref{TABLE-BACKGROUND-MCSAMPLES1} and~\ref{TABLE-BACKGROUND-MCSAMPLES2}.

The \Wt\ signal is calculated to have a cross-section approximately 20\% the magnitude of the total single top cross-section at 7 TeV~\cite{SGTOP-XS1,SGTOP-XS2,SGTOP-XS3}, a theoretical cross-section of $\sigma_{\Wt} = 15.74\ pb$~\cite{SGTOP-XS2}. It has been simulated using a variety of generator and hadronization model combinations. The nominal sample uses \AcerMC\ 3.5 as the generator and \Pythia\ 6.423 as the hadronization model. The top quark decays almost exclusively to a \Wboson\ boson and a $b$-quark, while the resulting two \Wboson\ bosons follow the decay branching ratios of the \Wboson\ boson. For the purposes of this analysis, we examine final states in which both of the \Wboson\ bosons decay leptonically into either an electron/neutrino pair or a muon/neutrino pair. This occurs for approximately 5\% of the \Wt\ events~\cite{PDG}. The tau lepton decays of the \Wboson\ boson are also simulated in the simulated events\ and some events may make it past the selection, but they are a small fraction of the total yield due to the approximately 35\% branching ratio of $\tau$ to electrons and muons.

The \TTbar\ background makes up the largest background in this analysis. The total cross-section at 7 TeV is $\sigma_{\TTbar}\ = 161^{+11}_{-16}\ pb$~\cite{TTBARXS}, approximately ten times larger than the \Wt\ signal. Like the \Wtchan, the top quarks in the \TTbar\ process almost exclusively decay into a \Wboson\ boson and b-quark pair, and in this analysis we are interested in the case where both of the \Wboson\ bosons decay leptonically. The major difference is the second b-quark in the final state, but a second b-quark can go undetected if it has low energy or is reconstructed incorrectly. For example, particles with significant momentum may diverge from the cone of the jet and be left out of the reconstruction, giving the reconstructed jet energy lower than the selection threshold. It is for this reason that the \TTbar\ background is by far the most significant background for a \Wtchan\ analysis. The nominal sample uses the \MCNLO\ generator with the \HERWIG\ hadronization model.

Additional simulated events\ have been generated to analyze the contribution from several different systematic uncertainty. For more information on the systematics, refer to Section~\ref{SECTION-SYSTEMATICS}. For comparison in generator and hadronization studies, two \Wt\ samples have been created, one using \MCNLO\ as the generator and \HERWIG\ for the hadronization, and a second using \AcerMC\ as the generator and \Pythia\ for the hadronization. Additionally, two \TTbar\ samples have been created, one using \POWHEG\ as the generator and \HERWIG\ for hadronization, and another using \POWHEG\ as the generator and \Pythia\ for hadronization. For both the \TTbar\ and \Wt\ processes, six different samples have been created exploring a range of ISR/FSR parameter phase space. This scheme allows us to probe the ISR and FSR contributions independently and in combination with each other.

The \Zjets\ background is significant. While its tree level final state is not similar to the \Wt\ signal (it has no real neutrinos to provide \MET) its cross-section is over sixty times higher. Our selection leaves the events where the \Zboson\ boson decays to two leptons. The \Zjets\ background is divided into several different samples, depending on the number of jets in the final state. These samples are used to determine the shape of the \Zjets\ distributions, and the overall normalization is provided by a data-driven method described in Sections~\ref{SECTION-DY-ESTIMATE} and~\ref{SECTION-ZTAUTAU-ESTIMATE} to minimize impact of systematic uncertainties. They are generated with \ALPGEN\ and hadronized with \HERWIG. Their respective cross-sections are given in Table~\ref{TABLE-BACKGROUND-MCSAMPLES2}.

The \Wjets\ background is similar to the \Zjets\ background in that its final state does not resemble the final state of the \Wt\ signal, but its cross-section is higher still, approximately 10 times as large as the \Zjets\ background. This simulated sample is not used directly as an estimate, but is instead used to provide a shape to the data-driven estimate of the \multijet\ background. This background's normalization must be estimated from data because doing a simulation is much more difficult than using data-driven methods. Due to its large cross-section and low acceptance, it would require generating many orders of magnitude more events than the other backgrounds. In addition, generating these events accurately would be difficult, as the low acceptance means that the software would have to accurately simulate even rare events. The \Wjets\ sample is generated using \ALPGEN\ and hadronized with \HERWIG. The samples are generated based on how many additional partons are involved in the interaction and additional samples are constructed specifically for the heavier quark flavors~\cite{LEPJETINTNOTE}. The samples and their respective cross-sections are given in Table~\ref{TABLE-BACKGROUND-MCSAMPLES2}.

The diboson backgrounds $WW$, $WZ$, and $ZZ$ are simulated with at least one of the bosons decaying leptonically. These backgrounds were generated using \ALPGEN\ and hadronized with \HERWIG. The NLO k-factors were calculated with MCFM for $WW$ and $ZZ$ and extrapolated from calculations for $\sqrt{s} = 14\ TeV$~\cite{Campbell1999} for $WZ$.

The simulated events are weighted to a total integrated luminosity of \LUMI. It simulates the effect of \pileup\ by reweighting individual events to compensate for the variation in the mean number of interactions per collision observed in the data. The accuracy of this simulation is evaluated by producing the histograms showing the number of primary vertices detected as in Fig.~\ref{FIGURE-BACKGROUND-NPVX} and verifying that the simulation agrees with the data within the expected uncertainty.

\TRPFIGLEG{paper_ll1+j_LP2fb_v4_Pvtx_n_flat}{paper_ll1j_LP2fb_v4_Pvtx_n_flat}{legend}{Histograms of the number of primary vertices in data and simulated events for (a) the selected sample and (b) the signal enhanced region. The simulated events are represented by the solid regions, while the data are represented with a black dot.}{FIGURE-BACKGROUND-NPVX}


\begin{table}[htdp]
\begin{center}
\begin{tabular}{lrrrr}
\hline
Description         & $\sigma$ [pb]  & $L_{int}$ [$fb^{-1}$]&  $N_{MC}$& Generator+Hadronization \\[1mm]
\hline \hline
\Wt\   all decays            &  15.74 & 9.5      & 150k   & AcerMC+HERWIG  \\[1mm]
\Wt\   all decays            &  15.74 & 19      & 300k   & AcerMC+Pythia  \\[1mm]
\Wt\   all decays            &  15.74 & 19      & 300k   & MC@NLO+HERWIG  \\[1mm]
$Wt$ ISR up    & 15.74 & 32k & 19   & ACERMC+Pythia     \\[1mm]
$Wt$ ISR down  & 15.74 & 32k & 19   & ACERMC+Pythia     \\[1mm]
$Wt$ FSR up    & 15.74 & 32k & 19    & ACERMC+Pythia     \\[1mm]
$Wt$ FSR down  & 15.74 & 32k & 19    & ACERMC+Pythia     \\[1mm]
$Wt$ ISR/FSR up    & 15.74 & 32k & 19    & ACERMC+Pythia     \\[1mm]
$Wt$ ISR/FSR down  &  15.74 & 32k & 19    & ACERMC+Pythia     \\[1mm]
\hline
$t\bar{t}$ not fully hadronic  & 89.7   & 2.2          & 200k   & MC@NLO+HERWIG   \\[1mm]
$t\bar{t}$ not fully hadronic  & 89.4   & 2.2          & 200k   & POWHEG+HERWIG  \\[1mm]
$t\bar{t}$ not fully hadronic  & 89.4   & 2.2          & 200k    & POWHEG+Pythia \\[1mm]
$t\bar{t}$ not fully hadronic ISR up    & 89.1 & 2.2 & 200k     & AcerMC+Pythia  \\[1mm]
$t\bar{t}$ not fully hadronic ISR down  & 89.1 & 2.2 & 200k     & AcerMC+Pythia  \\[1mm]
$t\bar{t}$ not fully hadronic FSR up    & 89.1 & 2.2 & 200k      & AcerMC+Pythia  \\[1mm]
$t\bar{t}$ not fully hadronic FSR down  & 89.1 &  2.2& 200k      & AcerMC+Pythia  \\[1mm]
$t\bar{t}$ not fully hadronic ISR/FSR up    & 89.1 & 2.2 & 200k       & AcerMC+Pythia \\[1mm]
$t\bar{t}$ not fully hadronic ISR/FSR down  & 89.1 & 2.2 & 200k       & AcerMC+Pythia \\[1mm]
\hline
single top t-channel (e)      &  7.09 & 28   & 200k  & AcerMC+Pythia   \\[1mm]
single top t-channel ($\mu$)  &  7.09 & 28   & 200k  & AcerMC+Pythia   \\[1mm]
single top t-channel ($\tau$) &  7.09 & 28   & 200k  & AcerMC+Pythia   \\[1mm]
\hline
single top s-channel (e)      &  0.47 &  21  & 10k    & MC@NLO+HERWIG  \\[1mm]
single top s-channel ($\mu$)  &  0.47 &  21  & 10k   & MC@NLO+HERWIG  \\[1mm]
single top s-channel ($\tau$) &  0.47 &  21  & 10k   & MC@NLO+HERWIG  \\[1mm]
%\hline
\hline\hline
\end{tabular}
\caption{The simulated samples and their respective cross-sections.}
\label{TABLE-BACKGROUND-MCSAMPLES1}
\end{center}
\end{table}

\begin{table}[phtdp]
\begin{center}
\begin{tabular}{lrrrr}
\hline
 Description         & $\sigma$ [pb]  & $L_{int}$ [$fb^{-1}$] &  $N_{MC}$& Generator+Hadronization \\[1mm]
\hline \hline
$Z\to \ell\ell$ + 0 parton   & 827.4  &  8.0          &  6,600k & ALPGEN+HERWIG \\[1mm]     
$Z\to \ell\ell$ + 1 partons  & 166.6  &  8.0          &  1,340k & ALPGEN+HERWIG \\[1mm]     
$Z\to \ell\ell$ + 2 partons  & 50.4   &  5.7          &    285k & ALPGEN+HERWIG \\[1mm]     
$Z\to \ell\ell$ + 3 partons  & 14.0   &  7.9          &    110k & ALPGEN+HERWIG \\[1mm]  
$Z\to \ell\ell$ + 4 partons  & 3.4    &  8.8          &     30k & ALPGEN+HERWIG \\[1mm]  
$Z\to \ell\ell$ + 5 partons  & 1.0    &  9          &     9k & ALPGEN+HERWIG \\[1mm]  
\hline
$W\to \ell\nu$ + 0 parton   & 8,296   &  2.0         &  3,500k & ALPGEN+HERWIG \\[1mm]     
$W\to \ell\nu$ + 1 partons  & 1,551   &  1.5         &  2,500k & ALPGEN+HERWIG \\[1mm]     
$W\to \ell\nu$ + 2 partons  &   452   &  6.1         &  3,770k & ALPGEN+HERWIG \\[1mm]     
$W\to \ell\nu$ + 3 partons  &   121   &  8.3         &  1,000k & ALPGEN+HERWIG \\[1mm]  
$W\to \ell\nu$ + 4 partons  &  30.3   &  8.3         &    250k & ALPGEN+HERWIG \\[1mm]  
$W\to \ell\nu$ + 5 partons  &   8.3   &  8.4         &     70k & ALPGEN+HERWIG \\[1mm]  
\hline
$W\to \ell\nu+b\bar{b}$ + 0 parton   & 54.7  &  8.7  &    475k & ALPGEN+HERWIG \\[1mm]     
$W\to \ell\nu+b\bar{b}$ + 1 partons  & 40.4  &  5.1  &    205k & ALPGEN+HERWIG \\[1mm]     
$W\to \ell\nu+b\bar{b}$ + 2 partons  & 20.0  &  8.8  &    175k & ALPGEN+HERWIG \\[1mm]     
$W\to \ell\nu+b\bar{b}$ + 3 partons  & 7.6   &  9.2  &     70k & ALPGEN+HERWIG \\[1mm] \hline
\hline
$W\to \ell\nu+c$ + 0 parton   & 517.6 &  1.7   &  860k & ALPGEN+HERWIG \\[1mm]     
$W\to \ell\nu+c$ + 1 partons  & 192.1 &  1.7   &  318k & ALPGEN+HERWIG \\[1mm]     
$W\to \ell\nu+c$ + 2 partons  & 51.0  &  1.7 &     85k& ALPGEN+HERWIG \\[1mm]     
$W\to \ell\nu+c$ + 3 partons  & 11.9  &  1.7    &  20k & ALPGEN+HERWIG \\[1mm] 
$W\to \ell\nu+c$ + 4 partons  & 2.8   & 1.8   &    5k & ALPGEN+HERWIG \\[1mm] 
\hline
$WW$                         & 17.0    &  15         &  250k & ALPGEN+HERWIG        \\[1mm]
$WZ$                         & 5.5     &  45         &  250k & ALPGEN+HERWIG        \\[1mm]
$ZZ$                         & 1.3     &  192        &  250k & ALPGEN+HERWIG        \\[1mm]
\hline\hline
\end{tabular}
\caption{The simulated samples and their respective cross-sections.}
\label{TABLE-BACKGROUND-MCSAMPLES2}
\end{center}
\end{table}


\section{Fake dilepton data-driven estimate}
\label{SECTION-QCD-ESTIMATE}

The contributions from \Wjets\ and multijet effects are difficult to model correctly in simulation. Instead, a data-driven method is employed to more accurately model this background. These backgrounds are significantly reduced in magnitude by the requirement of two leptons, as the tree level diagrams for these processes have one or fewer lepton. In order for these backgrounds to pass the event selection criteria, one of the quark or gluon jets in the events must be reconstructed as a lepton by mistake. This misreconstructed jet is referred to as a ``fake'', and this data-driven method relies on estimates of the prevalence of these fakes using a sideband of the data. A sideband region is a set of data with selection criteria orthogonal to the selection. 

The matrix method is used in this estimate. It defines a selection of the data using a loose electron and muon requirement and divides the events into one of four categories ($N_{TT}$, $N_{TL}$, $N_{LT}$, $N_{LL}$) depending on which of the leptons fit the loose definition or the tight definition. The loose selection is made as to select events with an increased contribution from mis-reconstructed leptons. From these regions we can use equations~\ref{EQN-MATRIXMETHOD} and~\ref{EQN-MATRIXMETHODMATRIX} below to estimate the real prevalence of real and fake leptons in the analysis selected sample. In this equation $r$ represents the real-to-tight efficiency, the probability that a real lepton that passes the loose cut will be identified as a tight lepton. Also, $f$ represents the fake-to-tight efficiency, the probability that a jet that passes the loose lepton cut will be identified as a tight lepton. 

\begin{equation}\label{EQN-MATRIXMETHOD}
\begin{bmatrix}N_{TT}\\N_{TL}\\N_{LT}\\N_{LL}\end{bmatrix} = E\begin{bmatrix}N_{RR}\\N_{RF}\\N_{FR}\\N_{FF}\end{bmatrix}
\end{equation}
\begin{equation}\label{EQN-MATRIXMETHODMATRIX}
E=\begin{bmatrix}rr& rf& fr& ff\\ r(1-r)& r(1-f)& f(1-r)& f(1-f)\\ (1-r)r& (1-r)f& (1-f)r& (1-f)f\\ (1-r)(1-r)& (1-r)(1-f)& (1-f)(1-r)& (1-f)(1-f)\end{bmatrix}
\end{equation}

\noindent
Both leptons are selected using looser requirements which are a subset of the tight selection requirements. This will allow us to look at leptons which pass this loose requirement and see how they compare to leptons that pass the more stringent tight analysis selection. Electrons are selected by replacing the ``isEM tight'' and track match requirement from the analysis with an ``isEM medium'', track match and b-layer hit requirement. Additionally the isolation requirement is removed. The muon selection is modified by removing the ID hit, the $Etcone$, and the $Ptcone$ isolation cut requirements. Two methods are used to estimate $r$ and $f$ separately.

The real-to-tight efficiencies for the real leptons uses a enhanced sample of \Zjets\ events. Events which have one tight and one opposite signed loose lepton with an invariant mass within 5 \GeV\ of $M_{\ell\ell} = 91\ GeV$ are selected. This selection is dominated by \Zjets\ events decaying to two leptons, as the events selected have leptons with an invariant mass close to the \Zboson\ boson mass of 91 GeV. As a result, it provides a high probability that the loose lepton is a real lepton. This loose lepton can be divided into categories of leptons that pass the event selection and leptons that don't, giving the efficiency of a real lepton passing the tight lepton selection. 

The fake-to-tight efficiencies are estimated by selecting events with a single loose lepton also with $\MET\ < 10\ GeV$. Although this selection is primarily made up of QCD events, it still has significant contamination of real leptons by \Wjets\ and \Zjets\ events. An iterative procedure has been developed to remove these events~\cite{LEPJETINTNOTE}. The initial step assumes no contamination, given an estimate of the fake-to-tight efficiency. This estimate is used to extract a scale factor between the total number of events and the numbers of \Wjets\ and \Zjets\ events that pass selection without a \MET\ cut.

\begin{equation}
k^{n}_{W/Z+jets} = \frac{N^{tight}-N^{n,tight}_{fake}}{N^{tight}_{W+jets,MC}+N^{tight}_{Z+jets,MC}}
\end{equation}

\noindent
The estimate of the fake-to-tight efficiency is then repeated, this time subtracting off the scale factor adjusted \Wjets\ and \Zjets\ contributions and the MC estimated \TTbar\ contribution. This procedure is iterated until the efficiency converges to a stable value. The results of these calculations are

\begin{equation}
  \begin{split}
    r_e &= 85.40 \pm 0.10\% \\
    f_e &= 4.86  \pm 0.01\% \\
    r_\mu &= 98.27 \pm 0.03\% \\
    f_\mu &= 21.07 \pm 0.05\%.
  \end{split}
\end{equation}

\noindent
Once these values are known, the matrix given in~\ref{EQN-MATRIXMETHOD} is inverted to give the estimated composition of the analysis selection, the tight-tight contribution. The dataset used to estimate the yield contains a luminosity of 0.7 $fb^{-1}$, and the resulting estimate is rescaled by \LUMI/0.7 $fb^{-1}$.

This estimate is affected by several systematic and statistical uncertainties. The statistical uncertainties come from the event counts in the regions used to estimate the efficiencies and the data statistics in the regions used to estimate the final yield. Systematic uncertainty contributitions arise from four sources:
\begin{list} {$\bullet$} {}
\item Choice of parametrization of the real and fake efficiencies.
\item Additional unmodeled contamination from backgrounds ignored in the signal enhanced regions.
\item Differing composition of the control regions from the signal region.
\item Differing data-taking conditions between the first 0.7 $fb^{-1}$ and the full \LUMI.
\end{list}

Another single top ATLAS analysis has done a thorough estimate of these uncertainties~\cite{LEPJETINTNOTE}. Instead of repeating these studies in detail for such a small background, we use a conservative estimate based on the findings of the other single-top analysis. We use a normalization uncertainty of 100\% to account for these systematic uncertainties. Although this means that the lower end is not modeled properly, the yield of this background is small ($<1\%$) and the effort required to better understand the systematic does not provide significant gain, as it has little impact on the measured cross-section.
 Overall it is found that due to the strict muon selection, the only significant background comes from $ee$ and \emu\ channels. The shape for this background is estimated using \Wjets\ simulated events as described in Section~\ref{SECTION-BACKGROUND-MC}. The estimated \multijet\ background and its associated uncertainties are given in Table~\ref{TABLE-MM-RESULTS}.

\begin{table}[!h]
\begin{center}
   \begin{tabular}{l | ll}
    \hline\hline
    Channel & 1-jet         & 2-jet and higher \\
    \hline\hline
    $ee$       & $6.6\pm 6.6$  & $2.4\pm 2.4$  \\
    $\mu\mu $  & negl.        & negl. \\
    $e\mu$     & $4.5\pm 4.5$  & $3.6\pm 3.6$ \\
    \hline
   \end{tabular}
 \caption{Fake dilepton background estimated for a luminosity of \LUMI. Both statistical and systematic uncertainties are included.}
\label{TABLE-MM-RESULTS}
\end{center}
\end{table}


\section{Drell-Yan data-driven estimate}
\label{SECTION-DY-ESTIMATE}

There is a significant background from Drell-Yan events in which a \Zboson\ boson or a virtual photon decay into a pair of leptons. A diagram of these processes is shown in Fig.~\ref{FIGURE-THEORY-DY}. Here a data-driven procedure called the ABCDEF method estimates the magnitude of the background for the dielectron and dimuon decays. This method uses independent uncorrelated regions in phase space to divide the data into signal and background enriched regions (shown in Fig.~\ref{FIGURE-BACKGROUND-ABCDPOP}) and then estimates the ratio of the background population across one of the cuts using the two background enriched regions. This ratio is used to extrapolate the contamination of the third region into the signal region, as shown in equations~\ref{EQN-BACKGROUND-ABCD1} and~\ref{EQN-BACKGROUND-ABCD2}. 

\begin{equation}
\label{EQN-BACKGROUND-ABCD1}
N_{A}^{predicted} = N_{D}^{data} \times (N_{B}^{data}/N_{E}^{data})
\end{equation} 
\begin{equation}
\label{EQN-BACKGROUND-ABCD2}
N_{C}^{predicted} = N_{F}^{data} \times (N_{B}^{data}/N_{E}^{data})
\end{equation} 

Two variables must be selected which are uncorrelated and have good separation between signal and background. The regions of phase space thus created must have enough events so that the statistical uncertainty on the estimate will be small. The variables chosen for this estimate are the dilepton invariant mass $M_{\ell\ell}$ and the missing transverse energy \MET. Typically this method uses only four regions, but because the dilepton invariant mass is used as one of the variables, the cut applied, $81\ GeV < M_{\ell\ell} <\ 101\ GeV$, gives a total of six regions. These regions and their relative populations are shown in Fig.~\ref{FIGURE-BACKGROUND-ABCDPOP}.
\VLARGEFIG{ll1+j_LP2fb_v4_ABCDEF}{A scatter plot illustrating the division of phase space into six regions and their relative population sizes. A larger dot indicates a higher density of events. }{FIGURE-BACKGROUND-ABCDPOP}

A simplistic model will allow the Drell-Yan contribution to the signal regions A and C to be estimated by using the populations of the other regions according to the following equations~\ref{EQN-BACKGROUND-ABCD1} and~\ref{EQN-BACKGROUND-ABCD2}. This simple model neglects several potential sources of error and a more robust model must be used. This more sophisticated method must take into account possible contamination from non-Drell-Yan backgrounds in the background control regions B, D, E, and F. In fact, the simulated sample estimates predict a significant contamination in region B, hence this must certainly be modeled. Also, although we have selected two variables minimally correlated with each other, even a weak correlation can cause uncertainty in the estimate. To model the effect of these two systematics, two additional scale factors are added, one as an overall scale factor, and one as a $k$-factor modifying the non-Drell-Yan simulated background estimate, which is then subtracted from the total event count in that region,

\begin{equation}
N_{A}^{predicted} = N_{f}^{A} \times \frac{N_{B}^{data}-k_{A}\times N_{B}^{MCBG}}{N_{E}^{data}-k_{A}\times N_{E}^{MCBG}} \times (N_{D}^{data}-k_{A}\times N_{D}^{MCBG})
\end{equation} 

\begin{equation}
N_{C}^{predicted} = N_{f}^{C} \times \frac{N_{B}^{data}-k_{C}\times N_{B}^{MCBG}}{N_{E}^{data}-k_{C}\times N_{E}^{MCBG}} \times (N_{F}^{data}-k_{C}\times N_{F}^{MCBG}).
\end{equation} 

\noindent
These parameters are found by constructing a likelihood function and fitting. To make the fit more robust, the likelihood functions for several possible \MET\ cuts ranging from \mbox{10 to 50 GeV} in \mbox{5 GeV} increments are combined. The event counts are modeled as Poisson distributions and the following likelihood function is maximized:

\begin{equation}
\mathcal{L}(N_f,k) = \prod_{\MET\ cut \in 10}^{50\ GeV} \textrm{Pois} \left (N^{obs}|N^{exp}_{MC}+N^{est}_{DY}  \right) (\MET\ ~cut ).
\end{equation}

\noindent
This fit is computed independently for regions $A$ and $C$, since the contaminating backgrounds in these regions may have a strong dependence on the two selection cuts. An additional variable modeling linear dependence on the \MET\ was considered, but an analysis showed that having no such dependence was more consistent with the data, and hence the final estimate is done assuming no dependence on \MET. The overall scale factors derived from this fit are $N_{f}^A = 1.0 \pm 0.1$ for region $A$ and $N_f^C=1.2\pm 0.1$ for region $C$. For the final computation, these two fits were combined into an average value of $N_{f} = 1.1 \pm 0.1$. The background contamination scale factor $k$ was fitted to region C and determined to be $k = 1.3 \pm 0.2$ for $ee$ and $1.4\pm0.2$ for $\mu\mu$. Region $D$ was excluded due the contaminating presence of the multijet background in the low $M_{\ell\ell}$ and low \MET\ region.

The systematic uncertainty is estimated by independently varying the fitted $N_f$ and $k$ parameters by 1$\sigma$ and calculating the change in the background estimate. These are considered to be independent and are added in quadrature to give an overall uncertainty for the estimate. This procedure is repeated for each of the 1-jet, 2-jet, and 3-jet inclusive bins and the results are displayed in Table~\ref{TABLE-ABCD-RESULTS}.

\begin{table}[!h]
\begin{center}
   \begin{tabular}{l | lll}
    \hline
    Channel    & 1-jet       & 2-jet & 3-jet and higher \\
    \hline
    $ee$       & $20.1\pm 2.0$ & $10.7\pm 2.0$ & $4.9\pm 2.0$\\
    $\mu\mu $  & $29.1\pm 3.3$ & $28.4\pm 3.1$ & $12.0\pm 3.1$\\
    \hline
   \end{tabular}
 \caption{Drell-Yan background estimates for selected events in the 1-jet, 2-jet and 3-jet and higher bins, obtained using the ABCDEF method with \LUMI\ of data. The combined statistical and systematic uncertainty is shown.}
\label{TABLE-ABCD-RESULTS}
\end{center}
\end{table}

It can be seen that the overall yield is largest in the 1-jet bin, where it makes up approximately 10\% of the overall background. As the jet multiplicity rises, the relative contribution from Drell-Yan decreases. The shape for these backgrounds is modeled using the simulation samples described in Section~\ref{SECTION-BACKGROUND-MC}.

\section{$Z\to\tau\tau$ data-driven estimate}
\label{SECTION-ZTAUTAU-ESTIMATE}

A data-driven estimate was also performed for the \Ztt\ background. This background is much less significant than the other backgrounds, especially given the powerful discrimination against it during selection. As a result, after selection \Ztt\ makes up approximately 1\% of the total background. This estimate uses a method similar to the Drell-Yan estimate, using a background enriched region $B$ to estimate the contamination in the signal region $A$. Again the Drell-Yan rejection window is chosen as the discriminating variable. The other contaminating backgrounds are subtracted from the yields using their simulation estimated yields, and then the \Ztt\ contribution to region A is estimated using the following formula:

\begin{equation}
DY_{A}^{EST} = \frac{DY_{A}^{MC}}{DY_{B}^{MC}} \times (Data_{B} - MC_{B}^{Backgrounds}).
\end{equation}

\noindent
The uncertainty is taken to be the difference between the data-driven estimate and the simulation estimate, giving an overall uncertainty of 60\%. The estimate is done separately for the $ee$, \emu, and $\mu\mu$ channels for the 1-jet, 2-jet, and 3-jet inclusive bins. The shape of the distributions is provided by the simulated events discussed in Section~\ref{SECTION-BACKGROUND-MC}.

\begin{table}[!h]
\begin{center}
   \begin{tabular}{l | lll}
    \hline
    Channel &  1-jet & 2-jet & 3-jet and higher \\
    \hline
    $ee$       & $1.1\pm 0.6$  & $1.1\pm 0.6$ & $0.0\pm 0.6$\\
    $\mu\mu $  & $5.7\pm 3.4$  & $1.7\pm 1.0$ & $0.7\pm 0.4$\\
    $e\mu$     & $2.6\pm 1.6$  & $1.2\pm 0.7$ & $0.8\pm 0.5$\\
    \hline
   \end{tabular}
 \caption{\Ztt\ background estimates for selected events in the 1-jet, 2-jet and 3-jet and higher bins. The errors include statistical and systematic uncertainties.}
\label{TABLE-ZTAUTAU}
\end{center}
\end{table}
