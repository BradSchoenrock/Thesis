\chapter{Event selection}
\label{SECTION-SELECTION}
Before the final analysis described in Chapter~\ref{SECTION-ANALYSIS} can be performed, an event selection specific to the signal kinematics is applied. This event selection is designed to remove background events while having minimal impact on the signal and defines the control regions used to perform the data-driven background estimates described in Section~\ref{SECTION-BG-DD}. Events are also separated into different channels by the event selection and these channels are individually optimized.

\section{Composite objects}
\label{SECTION-SELECTION-COMPOBJ}
While Chapter~\ref{SECTION-OBJ} details how basic analysis objects are reconstructed from the raw detector response, Figure~\ref{FIGURE-SELECTION-WPRIME} shows several intermediate particles that can also be reconstructed. These intermediate states of the W boson, top quark, and ultimately the \Wprime\ boson are what define this channel as unique from any other process with the same final state, such as $Wbb$. These intermediate particles also have unique kinematics that distinguish \Wprimechan\ from other processes.

\VLARGEFIG{Wprime}{Illustration of the \Wprimechan\ process.}{FIGURE-SELECTION-WPRIME}

\subsection{W boson and neutrino reconstruction}
\label{SECTION-SELECTION-COMPOBJ-W}
The W boson in Figure~\ref{FIGURE-SELECTION-WPRIME} is the only intermediate particle composed entirely of final state objects and its reconstruction is as simple as adding the 4-momenta of the lepton and neutrino together. The complication with this is that the 4-momentum of the neutrino is not known. Section~\ref{SECTION-OBJ-MET} describes how the neutrino's $p_T$ can be determined from the MET by assuming that the momentum is balanced in the transverse plane. This same technique cannot be used to determine $p_z$ for the neutrino because there is no reason the interacting partons should have the same momentum along the beamline as each other. Instead the W boson and neutrino are defined simultaneously by requiring that the lepton (a single lepton selection is applied in Section~\ref{SECTION-SELECTION-CUTS}) and neutrino combine to form an on-shell W boson with a mass of 80.4 GeV. Both the lepton and neutrino are assumed to be massless and the neutrino's $p_T$ is assumed to be equivalent to the MET. This gives rise to a quadratic equation for the neutrino's $p_z$, with solutions given by Equation~\ref{EQ-SELECTION-NEUTRINO}. 

\begin{equation}
\label{EQ-SELECTION-NEUTRINO}
p_{z,\nu} = \frac{\mu p_{z,l}}{p_{T,l}^2} \pm \sqrt{\frac{\mu^2p_{z,l}^2}{p_{T,l}^4} - \frac{E_l^2p_{T,\nu}^2-\mu^2}{p_{T,l}^2}}
\end{equation}
\begin{equation}
\label{EQ-SELECTION-NEUTRINOMU}
\mu = \frac{M_W^2}{2} + cos(\Delta\phi_{l,\nu})p_{T,\nu}p_{T,l}
\end{equation}

\noindent
In Equation~\ref{EQ-SELECTION-MTW}, $p_{T,l}$ and $p_{T,\nu}$ are the transverse momenta of the lepton and neutrino respectively,and $p_{z,l}$ and $p_{z,\nu}$ are the z-momenta of the lepton and neutrino. $\Delta\phi_{l,\nu}$ is the difference in $\phi$ between the lepton and neutrino. There are three possible categories of solution to Equation~\ref{EQ-SELECTION-NEUTRINO} based on the sign of the discriminant. If the discriminant is positive then there are two real solutions to Equation~\ref{EQ-SELECTION-NEUTRINO} and the solution with the lowest $|p_z|$ is chosen to define the neutrino, creating a less energetic final state. If the discriminant is 0 then there is only one $p_z$ solution then the neutrino is uniquely defined. If the discriminant is negative then the solutions for $p_z$ are imaginary, in this case the $p_T$ of the neutrino is rescaled so that the discriminant becomes 0, then the neutrino $p_z$ is uniquely defined and the neutrino $p_T$ is taken to be the rescaled value.

\subsection{Top quark reconstruction}
\label{SECTION-SELECTION-COMPOBJ-T}
While it is possible to reconstruct the top quark in Figure~\ref{FIGURE-SELECTION-WPRIME}, there is an ambiguity about which jet originated from the top quark decay. The indeterminacy is resolved differently depending on which channel the event belongs to. If the event contains only 1 b-tagged jet then the invariant mass of each jet and the reconstructed W boson is calculated and the combination with a mass closest to the top quark mass of 172.5 GeV forms the reconstructed top quark. For events that contain 2 b-tagged jets the mass of the W boson and each b-tagged jet is calculated, with the pair producing a mass closest to 172.5 GeV forming the top quark. The cut flow for each channel is described in greater detail in Section~\ref{SECTION-SELECTION-CUTS}.

\subsection{\Wprime\ reconstruction}
\label{SECTION-SELECTION-COMPOBJ-WPRIME}
Similar to how the top quark is reconstructed, the \Wprime\ boson is reconstructed differently depending on which analysis channel the event falls into. For events that contain 2 b-tagged jets the \Wprime\ boson is reconstructed by combining the reconstructed top quark with the b-tagged jet that was not used to reconstruct the top quark. For events with 1 b-tagged jet the \Wprime\ boson is reconstructed by combining the reconstructed top quark with the highest $p_T$ jet not used to reconstruct the top quark, requiring that the b-tagged jet is included in the \Wprime\ reconstruction. This means that for events where the b-tagged jet was included in the top quark reconstruction that the jet combined with the top quark is not b-tagged. For events where the top quark reconstruction does not include the b-tagged jet, the jet combined with the top quark to form the \Wprime\ boson must be b-tagged.

\section{Data triggers}
\label{SECTION-SELECTION-TRIG}
In order for an event to be recorded by the ATLAS detector and included in an analysis it must pass the trigger selection described in Section~\ref{SECTION-ATLAS-TDAQ}. To search for \Wprimechan\ the ATLAS single lepton triggers are used. The single electron triggers require that electrons either have an $E_T$ $>$ 24 GeV and pass medium isolation requirements for the hadronic leakage, shower width in $\eta$, and track isolation as described in Section~\ref{SECTION-OBJ-EL} or have an $E_T$ $>$ 60 GeV without any isolation requirement. The single muon triggers require that muons either have a $p_T$ $>$ 24 GeV and pass medium isolation requirements for the ID track isolation described in Section~\ref{SECTION-OBJ-MU} or have a $p_T$ $>$ 36~GeV without any isolation requirement. The complete set of requirements for electrons and muons detailed in Chapter~\ref{SECTION-OBJ} is applied offline, after the data has been recorded. Events must also have been taken during an LHC stable beam period and during a time when all of the ATLAS subsystems were properly operating. The combination of these requirements corresponds to an integrated luminosity of \LUMI.

\section{Cut flow}
\label{SECTION-SELECTION-CUTS}
Before performing the multivariate analysis described in Chapter~\ref{SECTION-ANALYSIS}, it is useful to apply a set of event selection cuts. These cuts are designed to remove background events with large kinematical differences from the signal samples so that the multivariate analysis can be more focused on discriminating between the hard to classify events. The event selection also defines the separate analysis channels which will undergo individually optimized multivariate analyses. The event selection cuts are as follows:

\begin{list}{$\bullet$}{}
\item Exactly 1 lepton.
\item Lepton $p_T$ $>$ 35 GeV.
\item MET $>$ 35 GeV
\item W boson transverse mass ($m_T(W)$) + MET $>$ 60 GeV, where $m_T(W)$ is defined in Equation~\ref{EQ-SELECTION-MTW}.
\item Exactly 2 or 3 jets.
\item Exactly 1 or 2 b-tagged jets.
\item \Wprime\ boson mass ($m(\Wprime)$) $>$ 330 GeV.
\end{list}

\begin{equation}
\label{EQ-SELECTION-MTW}
m_T(W) = \sqrt{2p_{T,l}p_{T,\nu}(1-cos(\Delta\phi_{l,\nu}))}
\end{equation}

\noindent
The number of jets and the number of b-tagged jets defines a unique analysis channel which is referred to by the number of jets and b-tagged jets in events in that particular channel, for example the 2jets 1tag channel contains events with exactly 2 jets and exactly 1 b-tagged jet. This produces four separate analysis channels, 2jets 1tag, 2jets 2tag, 3jets 1tag, and 3jets 2tag. 

The cuts on the lepton number, lepton $p_T$, and MET are chosen to match the decay channel seen in Figure~\ref{FIGURE-SELECTION-WPRIME} where we expect a single high $p_T$ lepton and large MET from the W boson decay. The cut on $m_T(W)$ + MET is called the triangular cut and is commonly used in single top analyses to discriminate against the multijets background. The cut on $m(\Wprime)$ is chosen to define a control region used to perform a data driven normalization of the W+jets background as described in Section~\ref{SECTION-BG-DD-WJETS}. The cut value of 330 GeV was chosen to maximize the size of the control region while keeping the signal contamination to less than 5\% for all of the signal samples. 

Both of the 1tag channels have significantly larger backgrounds than the 2tag channels so two additional cuts are applied to the 1tag channels only:

\begin{list}{$\bullet$}{}
\item $E_T$ of the leading jet ($E_T(jet1)$) $>$ 140 GeV.
\item Transverse energy of the reconstructed top quark ($E_T(Top)$) $>$ 175 GeV.
\end{list}

\noindent
These cuts are chosen by ranking a list of event kinematics variables by their discrimination power after performing the initial event selection cuts. The discrimination power of each variable is determined by mapping the signal efficiency ($\epsilon_S$) versus the background efficiency ($\epsilon_B$) for successively raised cuts on the variable. The area between the curve this process maps out and the line of $\epsilon_S\ =\ \epsilon_B$ is defined to be the discrimination power of the variable. For the two most discriminating variables, $p_T(jet1)$ and $E_T(Top)$, the cut is chosen to be at least 95\% efficient for all signal samples. The final event yields are shown in Table~\ref{TABLE-SELECTION-YIELDS}.


\begin{table}
\begin{center}
\begin{tabular}{|c|cccc|}
\hline
Sample & 2jets 1tag & 3jets 1tag & 2jets 2tag & 3jets 2tag \\
\hline
\WprimeR 500 & 12601.14 & 5599.62 & 8874.10 & 5120.80 \\ 
\WprimeR 750 & 4018.08 & 2723.38 & 2468.55 & 2172.28 \\ 
\WprimeR 1000 & 1117.67 &  937.33 &  606.33 &  657.60 \\ 
\WprimeR 1250 &  337.63 &  311.93 &  155.44 &  189.40 \\ 
\WprimeR 1500 &  101.52 &  107.72 &   41.97 &   57.29 \\ 
\WprimeR 1750 &   32.09 &   36.60 &   12.11 &   18.28 \\ 
\WprimeR 2000 &   11.07 &   13.18 &    4.12 &    6.20 \\ 
\WprimeR 2250 &    4.09 &    5.01 &    1.48 &    2.06 \\ 
\WprimeR 2500 &    1.71 &    1.94 &    0.63 &    0.84 \\ 
\WprimeR 2750 &    0.80 &    0.86 &    0.34 &    0.40 \\ 
\WprimeR 3000 &    0.42 &    0.42 &    0.19 &    0.22 \\ 
\hline
\WprimeL 500 & 6680.34 & 3078.17 & 5235.86 & 3129.40 \\ 
\WprimeL 750 & 2307.41 & 1551.99 & 1556.07 & 1310.33 \\ 
\WprimeL 1000 &  681.85 &  577.36 &  411.47 &  438.10 \\ 
\WprimeL 1250 &  229.29 &  213.66 &  112.84 &  138.45 \\ 
\WprimeL 1500 &   75.38 &   77.88 &   32.26 &   44.19 \\ 
\WprimeL 1750 &   25.97 &   30.55 &   10.65 &   14.45 \\ 
\WprimeL 2000 &    9.75 &   11.16 &    3.43 &    4.99 \\ 
\WprimeL 2250 &    3.81 &    4.55 &    1.37 &    1.98 \\ 
\WprimeL 2500 &    1.65 &    1.89 &    0.60 &    0.78 \\ 
\WprimeL 2750 &    0.73 &    0.82 &    0.30 &    0.40 \\ 
\WprimeL 3000 &    0.40 &    0.41 &    0.18 &    0.21 \\ 
\hline
single top s-channel  &  138.10 &   73.92 &   98.82 &   58.71 \\  
single top t-channel  & 1957.69 & 1080.64 &  242.73 &  373.00 \\  
single top Wt-channel &  624.74 &  979.43 &   80.10 &  270.67 \\  
$t\bar{t}$ & 4586.93 & 9410.07 & 1480.34 & 5108.30 \\  
W+lf & 2950.59 & 1255.28 &   45.78 &   45.27 \\  
W+c  & 4877.90 & 1989.33 &   78.18 &   71.61 \\  
W+cc & 3471.35 & 2470.32 &   81.00 &  127.04 \\  
W+bb & 3395.41 & 2086.27 &  455.80 &  675.72 \\  
Z+jets &  361.64 &  379.02 &    2.32 &    9.45 \\  
diboson &  214.10 &  119.48 &   16.34 &   15.36 \\  
multijets & 1132.34 &  540.02 &   59.54 &   58.68 \\  
\hline
total background & 23710.80 & 20383.78 & 2640.95 & 6813.83 \\ 
\hline
data & 21106.00 & 18317.00 & 2632.00 & 6666.00 \\ 
\hline
\end{tabular}
\caption{Event yields for signal samples, background samples, and data by analysis channel.}
\label{TABLE-SELECTION-YIELDS}
\end{center}
\end{table}
