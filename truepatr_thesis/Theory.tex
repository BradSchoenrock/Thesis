\chapter{Theory}
\label{SECTION-THEORY}
High energy physics attempts to deal with the fundamental particles and forces of the universe, and the Standard Model (SM) of high energy physics is the theoretical framework used in this analysis. The Standard Model describes the universe as being composed of 17 fundamental particles and their interactions through three of the four fundamental forces. This analysis is a search for a \Wprime\ particle not included in the Standard Model which would be indicative of other physical theories, collectively called Beyond the Standard Model (BSM) theories. There are a wide variety of BSM theories therefore this analysis is performed in a model independent manner using an effective Lagrangian. The motivation behind searching for a \Wprime\ is explored by examining some representative BSM theories and their consequences. The focus of this chapter is not to derive the Standard Model or any BSM theories from first principles, but rather to provide a practical framework and context in which to understand this analysis and the implications of the results.

\section{The Standard Model}
\label{SECTION-THEORY-SM}
The Standard Model has provided accurate predictions of experimental observables for over 40 years. It was developed after decades of experimentation had catalogued a myriad of particle states. The properties of these states were observed to follow patterns and symmetries, and eventually these symmetries were developed into the Standard Model. The symmetries of the Standard Model are described in group theory terms as $SU(3) \otimes SU(2) \otimes U(1)$ with each symmetry giving rise to its own conservation law. The Standard Model is a quantum field theory arising from a unification of quantum mechanics and special relativity, and is mathematically described by a Lagrangian~\cite{RYDER}.

\subsection{The forces}
\label{SECTION-THEORY-SM-FORCES}
The Standard Model includes three of the four fundamental forces of nature, the electromagnetic, weak, and strong forces. The electromagnetic and weak forces can be unified into a single electroweak force similar to the unification of the electric and magnetic forces into the electromagnetic force. The electroweak force is described by the $SU(2) \otimes U(1)$ symmetry of the Standard Model and is mediated by the massless photon as well as the massive W and Z bosons. One of the greatest theoretical achievements of the Standard Model was the prediction of the existence and masses of the W and Z bosons well before their experimental discovery. The strong force is described by the $SU(3)$ symmetry of the Standard Model and is mediated by massless gluons. The strong force differs from the electroweak force in that the strong force grows with increased distance between objects while the electroweak force diminishes, which has unique consequences. There have been many attempts to unify the strong force with the electroweak force and even to include a quantum field theory of gravity, such as supersymmetry or string theory. However, there is no clear experimental evidence to support these theories and they are not considered part of the Standard Model~\cite{GARCIA}.

The electroweak force is a unification of the electric, magnetic, and weak forces. The electric and magnetic forces were unified by Maxwell in 1879. The resulting electromagnetic force decreases as the distance between objects increases, and the force is carried to infinite distance by its massless mediator, the photon. The weak force is responsible for a wide range of observed phenomena, including beta decay and the violation of parity and charge-momentum conservation. These processes can be described with phenomenological theories at low and intermediate energies, however at higher energies above a few GeV the weak theories are unstable on their own. The weak force is similar to the electromagnetic in that the force decreases as the distance between objects increases, however the weak force is mediated by the massive W and Z bosons and so has a limited range of typically 2.5 am. It is only after unification that electroweak theory provides consistent predictions for the energy ranges probed by modern accelerators of up to several TeV~\cite{GARCIA}.

The strong force is responsible for holding baryons, mesons, and nuclii together and is described in the Standard Model by quantum chromodynamics (QCD). QCD describes the strong force using a type of charge called ``color'' which comes in three colors and their anticolors. Quarks each cary either a color or anticolor charge and gluons, the mediating particles of the strong force, carry both a color and an anticolor charge. The color charge carried by the gluons is a key difference between QCD and quantum electrodynamics (QED) in which the mediating particle is charge neutral. This means that gluons are self-interacting and do not form a free gluon field and also that QCD is not locally gauge invariant and so it is a non-Abelian gauge theory. This leads to antishielding of bare color charges by the vaccuum and the force between colored objects becoming larger as the distance between them increases. This corresponds with the fact that only colorless objects are observed in nature and with the formation of particle shower ``jets'', as discussed further in Section~\ref{SECTION-THEORY-SM-PARTICLES-JETS}, from what would otherwise be bare color charged objects~\cite{GARCIA}.

\subsection{The particles}
\label{SECTION-THEORY-SM-PARTICLES}
The Standard Model contains 17 fundamental particles and their anti-particles which compose all objects. These particles can be classified into leptons, quarks, and bosons as shown in Table~\ref{TABLE-THEORY-SM-PARTICLES}. For each particle in Table~\ref{TABLE-THEORY-SM-PARTICLES} there is a corresponding anti-particle with opposite electric charge. In general a particle name or symbol refers to both the particle and its anti-particle except where they are explicitly distinguished between, thus ``electron'' refers to both electrons and positrons in general. The structure visible in Table~\ref{TABLE-THEORY-SM-PARTICLES} is not accidental and is vital to our understanding of the particles.

The six lepton flavors can be classified into 3 generations, each containing a charged lepton and a neutrino. The charged leptons all are massive and carry an electrical charge of -1, while the neutrinos are electrically neutral and their masses have not been directly observed. The current best limits on the mass of each neutrino flavor are given in Table~\ref{TABLE-THEORY-SM-PARTICLES} because the observation of neutrino flavor oscillations~\cite{Neutrino_Oscillations} implies that neutrinos are not massless but no mass measurements have been made yet. The leptons do not interact through the strong force, so the charged leptons only interact through the weak and electromagnetic forces and the neutrinos can only interact weakly. Because neutrinos can only interact through the weak force their interaction with matter, such as a detector, is rare and specialized experiments are necessary to study them. In contrast the first generation charged lepton, the electron, is easily detected through electromagnetic interactions and is readily available in nature. This difference in detectability has lead to the term ``lepton'' generally indicating the charged leptons with the neutrinos being indicated separately.

Similar to the leptons, the quarks can also be described by 3 generations, each containing 2 flavors. Each generation contains one quark with an electric charge of $\nicefrac{2}{3}$ and one quark with an electric charge of $-\nicefrac{1}{3}$, called up-type and down-type respectively based on the first generation quarks with those charges. Quarks interact through all 3 of the forces in the Standard Model and thus are readily detectable using a variety of methods. Since quarks have a color charge, bare quarks will typically form jets as described in Section~\ref{SECTION-THEORY-SM-PARTICLES-JETS} and cannot be directly measured.

The final group of particles in Table~\ref{TABLE-THEORY-SM-PARTICLES} is the bosons. The bosons all have integer spins, with the photon, gluon, W and Z bosons all being spin 1 and the Higgs boson being spin 0. The photon, W and Z bosons are the mediating particles of electroweak theory with the W and Z bosons gaining their masses through the Higgs mechanism. The Higgs mechanism adds a quartic complex scalar field potential to the theory which is locally gauge invariant. Through an appropriate choice of parameters the field is made to have a non-zero expectation value and induce spontaneous symmetry breaking in the $SU(2) \otimes U(1)$ electroweak group. After further reparameterization and an appropriate choice of gauge, what is left is the massive W and Z bosons, the massless photon, and a new massive Higgs boson, which was just recently discovered at the LHC~\cite{ATLAS_Higgs,CMS_Higgs}. The final boson is the gluon which mediates the strong force. The gluon has a color and anticolor charge which makes it self-interacting, and a bare gluon will form a jet as described in Section~\ref{SECTION-THEORY-SM-PARTICLES-JETS}.

\begin{table}[!h!tbp]
\begin{center}
\begin{tabular}{|l|c|c|c|c|}
\hline
Particle & Symbol & Mass & Charge [e] & Spin \\
\hline
\multicolumn{5}{|>{\columncolor{green!20}}c|}{\textbf{Leptons}}\\
\hline
\lc Electron         & e            & 511 KeV    & -1 & $\nicefrac{1}{2}$\\
\lc Electron Neutrino& $e_\nu$      & $< $2.05 eV & 0  & $\nicefrac{1}{2}$\\
\lc Muon             & $\mu$        & 106 MeV  & -1 & $\nicefrac{1}{2}$\\
\lc Muon Neutrino    & $\mu_\nu$    &$< $0.17 MeV& 0  & $\nicefrac{1}{2}$\\
\lc Tau              & $\tau$       & 1.78 GeV   & -1 & $\nicefrac{1}{2}$\\
\lc Tau Neutrino     & $\tau_\nu$   &$< $18.2 MeV& 0    & $\nicefrac{1}{2}$\\
\hline
\qc \multicolumn{5}{|>{\columncolor{orange!20}}c|}{\textbf{Quarks}}\\
\hline
\qc Up               & u  & 2.3 MeV    & $\nicefrac{2}{3}$  & $\nicefrac{1}{2}$\\
\qc Down             & d  & 4.8 MeV    & $-\nicefrac{1}{3}$ & $\nicefrac{1}{2}$\\
\qc Charm            & c  & 1.28 GeV   & $\nicefrac{2}{3}$  & $\nicefrac{1}{2}$\\
\qc Strange          & s  & 95 MeV    & $-\nicefrac{1}{3}$ & $\nicefrac{1}{2}$\\
\qc Top              & t  & 173 GeV    & $\nicefrac{2}{3}$  & $\nicefrac{1}{2}$\\
\qc Bottom           & b  & 4.18 GeV    & $-\nicefrac{1}{3}$ & $\nicefrac{1}{2}$\\
\hline
\bc \multicolumn{5}{|>{\columncolor{purple!20}}c|}{\textbf{Bosons}}\\
\hline 
\bc Photon                  & $\gamma$  & 0        & 0                  & 1\\
\bc \Wboson$^\pm$ Boson      & $W^\pm$    & 80.4 GeV & $\pm$ 1            & 1\\
\bc \Zboson\ Boson          & $Z$       & 91.2 GeV & 0                  & 1\\
\bc Gluon                   & $g$       & 0        & 0                  & 1\\
\bc Higgs                   & $H$       & 126 GeV  & 0                  & 0\\
\hline
\end{tabular}
\caption{The fundamental particles of the Standard Model and their properties~\cite{PDG}.}  
\label{TABLE-THEORY-SM-PARTICLES}
\end{center}
\end{table}

\subsubsection{Jets}
\label{SECTION-THEORY-SM-PARTICLES-JETS}
Jets are phenomenological objects that are formed when individual colored particles, single quarks and gluons, are produced at sufficiently high energies. As the colored particle moves away from the initial colored object it is connected to, the energy of the strong interaction between them increases until it becomes energetically favorable to produce a quark-antiquark pair from the vaccuum for the particle and the initial colored object to be bound to. This production of new hadrons is called hadronization and it absorbs a small ammount of the initial colored particle's energy. This hadron will decay and the process will repeat itself until there is insufficient energy remaining for further hadronization, creating a narrow shower of hadrons that in total have the same energy and momentum as the original colored particle. This particle shower is called a jet and it is used as the detectable stand-in for the original particle. This is a general picture of what happens to bare quarks and gluons and there are subtle differences depending on the flavor of the initial particle, in particular if the initial particle is a top or bottom quark. Because of their large mass, top quarks almost exclusively decay into a W boson and bottom quark before hadronization can occur creating a very different signal from other quarks. Bottom quarks also have unique phenomenology in that the hadron produced with the initial bottom quark has an unusually long lifetime and will travel a detectable distance before the b quark decays and further hadronization takes place, creating a secondary vertex~\cite{GARCIA}. 

\subsection{The Lagrangian}
\label{SECTION-THEORY-SM-LAGRANGIAN}
The mathematics of the Standard Model is typically formulated in terms of a Lagrangian $L$ and the Lagrangian density $\mathcal{L}$ such that $L=\int\mathcal{L}d^3x$. The Standard Model includes many different phenomena so it is useful to group the terms of the Lagrangian density by the physical motivation as seen in Equations~\ref{EQ-THEORY-SM}~\cite{ITZYKSON}.

\begin{equation}
\label{EQ-THEORY-SM}
\mathcal{L} = \mathcal{L}_{EW} + \mathcal{L}_{QCD} + \mathcal{L}_{Yuk}
\end{equation}

The Lagrangian density that describes the electroweak force is given in Equation~\ref{EQ-THEORY-SM-EW}. 

\begin{equation}
\label{EQ-THEORY-SM-EW}
\mathcal{L}_{EW} = -\frac{1}{4}F_{\mu\nu a}F^{\mu\nu}_{a} + D_{\mu}\phi D^{\mu}\phi - \mu^{2}\phi^{2} - \lambda(\phi^{2})^{2}
\end{equation}

\noindent
The first term describes the electroweak interactions with the index $a$ running over the photon, W and Z bosons. $F_{\mu\nu}$ is the electroweak field tensor and is defined as:

\begin{equation}
\label{EQ-THEORY-SM-FMUNU}
F_{\mu\nu} = \partial_{\mu}A_{\nu} - \partial_{\nu}A_{\mu} - [A_{\mu},A_{\nu}]
\end{equation}

\noindent
Since electroweak theory is abelian, $[A_{\mu},A_{\nu}]\ =\ 0$ and $F_{\mu\nu}$ is simplified. The last three terms of Equation~\ref{EQ-THEORY-SM-EW} describe the Higgs field. The first of these is the kinetic term where $D_{\mu}$ is the covariant derivative defined as:

\begin{equation}
\label{EQ-THEORY-SM-DMU}
D_{\mu} = \partial_{\mu} - gA_{\mu}
\end{equation}

\noindent
The final two terms of the electroweak Lagrangian density describe the Higgs potential.

The QCD Lagrangian density is given in Equation~\ref{EQ-THEORY-SM-QCD}.

\begin{equation}
\label{EQ-THEORY-SM-QCD}
\mathcal{L}_{QCD} = \sum\limits_j\bar{\psi}(i\gamma^{\mu}D_{\mu} - m_j)\psi - \frac{1}{4}G_{\mu\nu a}G_{a}^{\mu\nu}
\end{equation}

\noindent
The first term describes the quarks with index j running over all six of the quarks with masses $m_{j}$. The covariant derivative $D_{\mu}$ is similar to Equation~\ref{EQ-THEORY-SM-DMU} but now contains eight gauge fields corresponding to the eight gluons denoted by the index $a$:

\begin{equation}
\label{EQ-THEORY-SM-DMUA}
D_{\mu} = \partial_{\mu} - g_{a}A_{\mu a}
\end{equation}

\noindent
The second term of the QCD Lagrangian density describes the gluons. $G_{\mu\nu a}$ ia analogous to $F_{\mu\nu}$ in the electroweak Lagrangian density, but for each of the eight gauge fields.

 \begin{equation}
\label{EQ-THEORY-SM-GMUNUA}
G_{\mu\nu a} = \partial_{\mu}A_{\nu a} - \partial_{\nu}A_{\mu a} - [A_{\mu a},A_{\nu a}]
\end{equation}

\noindent
Since QCD is a non-abelian theory, $[A_{\mu a},A_{\nu a}]\ \neq\ 0$ and $G_{\mu\nu a}$ does not simplify in the same way as $F_{\mu\nu}$.

The Yukawa Lagrangian density describes the fermions and is given in Equation~\ref{EQ-THEORY-SM-YUK}.

\begin{equation}
\label{EQ-THEORY-SM-YUK}
\mathcal{L}_{Yuk} = \sum\limits_a\bar{\psi}(i\gamma^{\mu}D_{\mu} - G_a\phi)\psi
\end{equation}

\noindent
The index a runs over the fermions, with the covariant derivative defined in Equation~\ref{EQ-THEORY-SM-DMU}. The mass of each fermion is determined by $G_a\phi$ where $G_a$ is the fermion's coupling to the Higgs field $\phi$, and in this way the Higgs field gives the fermions their masses. 

\section{Beyond the Standard Model theories}
\label{SECTION-THEORY-BSM}
While the Standard Model has described the observations of particle physics experiments for over 40 years, there are known problems with the theory. The Standard Model does not include gravity, which has been experimentally verified many times. There is no mechanism to produce the amount of matter-antimatter asymmetry that is observed in the universe. The Standard Model does not include a suitable dark matter particle to match astronomical observations. While mathematically possible, the observed masses of the W and Z bosons require very precise cancellations of parameters which seem unnatural. While this list is by no means exhaustive, there have been decades of work to solve these problems with BSM theories.

\subsection{Extensions of $SU(2) \otimes U(1)$}
\label{SECTION-THEORY-BSM-EXTENSIONS}
Theories that include a \Wprime\ boson often extend the $SU(2) \otimes U(1)$ electroweak symmetry which describes the W boson. The simplest such extension is $SU(2) \otimes SU(2) \otimes U(1)$, where the new $SU(2)$ can be a right handed extension of the left handed $SU(2)$ group which describes the Standard Model weak interactions or some other $SU(2)$ symmetry. The $SU(2) \otimes U(1)$ symmetry can also be extended by embedding the $SU(2)$ into a group of higher degree, resulting in symmetries such as $SU(3) \otimes U(1)$ or $SU(4) \otimes U(1)$. Each of these extensions contains a myriad of specific theories with different coupling structures and a variety of experimental predictions~\cite{PDG}. Since there is currently no strong experimental evidence to distinguish between these theories, this analysis does not assume any specific theory but instead uses an effective Lagrangian approach.

\subsection{Effective Lagrangian approach}
\label{SECTION-THEORY-BSM-EFFECTIVE}
In all of these theories the \Wprime\ boson is described by a Lagrangian density term of the form given in Equation~\ref{EQ-THEORY-BSM-EFFECTIVE}.

\begin{equation}
\label{EQ-THEORY-BSM-EFFECTIVE}
\mathcal{L}_{\Wprime} = \frac{1}{2\sqrt{2}}V^\prime_{ij}W^\prime_{\mu}\bar{f}^i\gamma^{\mu}(g^\prime_R(1+\gamma_5)+g^\prime_L(1-\gamma_5))f^j
\end{equation}

\noindent
This Lagrangian density includes arbitrary right handed and left handed coupling strengths $g^\prime_R$ and $g^\prime_L$ respectively. These coupling strengths are a common metric across all models regardless of how they are determined within each theory, and thus they are a model independant parameter which can be experimentally measured or constrained. For this analysis we use benchmark \WprimeR\ and \WprimeL\ models where $g^\prime_R$ and $g^\prime_L$ are equal to $g_L$ for the Standard Model W.
