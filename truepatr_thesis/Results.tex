\chapter{Statistical analysis}
\label{SECTION-RESULTS}
With the separation of background and signal events optimized by the multivariate analysis (MVA) described in Chapter~\ref{SECTION-ANALYSIS}, the statistical significance of the analysis must be determined. Possible sources of systematic error and their effects are described in Section~\ref{SECTION-RESULTS-SYSTEMATICS}. With no statistically significant excess of events found by the analysis, 95$\%$ confidence level (C.L.) limits are placed on the cross-section times branching ratio of \Wprime\ bosons with masses in the range 0.5 TeV - 3.0 TeV. The methodology used to derive these results is described in Section~\ref{SECTION-RESULTS-LIMITS}, and the results are given in Section~\ref{SECTION-RESULTS-RESULTS}.

\section{Systematic uncertainties}
\label{SECTION-RESULTS-SYSTEMATICS}
ATLAS is a complicated experiment as described in Chapter~\ref{SECTION-ATLAS}, and the calibration and interpretation of the detector response from the many systems introduce possible biases to the data. Additionally, the simulated Monte Carlo signal and background events may have biases in their generation and simulation which must be accounted for. Tables~\ref{TABLE-RESULTS-1TAGSYSTEMATICS} and \ref{TABLE-RESULTS-2TAGSYSTEMATICS} summarize the effects of these uncertainties on the background and signal event yields.

\begin{table}
\begin{center}
\begin{tabular}{|c|cc|cc|}
\hline
 & \multicolumn{2}{|c|}{2jets 1tag} & \multicolumn{2}{|c|}{3jets 1tag} \\
Systematic Uncertainty & Signal & Background & Signal & Background \\
\hline
JES & & & & \\
JER & & & & \\
Jet Reconstruction & & & &\\
JVF & & & & \\
b-tagging & & & & \\
LES/LER & & & & \\
Lepton Reconstruction & & & & \\
MET & & & & \\
PDF & & & & \\
ISR/FSR & & & & \\
MC Generator & & & & \\
Theory Normalization & & & & \\
W+jets Normalization & & & & \\
Multijet Normalization & & & & \\
Luminosity & & & & \\
\hline
\end{tabular}
\caption{Summary table of the systematic shifts in signal and background event yields of the 1tag analysis channels.}
\label{TABLE-RESULTS-1TAGSYSTEMATICS}
\end{center}
\end{table}

\begin{table}
\begin{center}
\begin{tabular}{|c|cc|cc|}
\hline
 & \multicolumn{2}{|c|}{2jets 2tag} & \multicolumn{2}{|c|}{3jets 2tag} \\
Systematic Uncertainty & Signal & Background & Signal & Background \\
\hline
JES & & & & \\
JER & & & & \\
Jet Reconstruction & & & &\\
JVF & & & & \\
b-tagging & & & & \\
LES/LER & & & & \\
Lepton Reconstruction & & & & \\
MET & & & & \\
PDF & & & & \\
ISR/FSR & & & & \\
MC Generator & & & & \\
Theory Normalization & & & & \\
W+jets Normalization & & & & \\
Multijet Normalization & & & & \\
Luminosity & & & & \\
\hline
\end{tabular}
\caption{Summary table of the systematic shifts in signal and background event yields of the 2tag analysis channels.}
\label{TABLE-RESULTS-2TAGSYSTEMATICS}
\end{center}
\end{table}

\subsection{Jet energy scale (JES)}
\label{SECTION-RESULTS-SYSTEMATICS-JES}
The jet energy scale (JES) uncertainty represents the potential bias in the measurement of the jet energies and includes both theoretical and experimental components. The theoretical components of the uncertainty come from the use of Monte Carlo simulations to derive the nominal JES correction factors for jets, while the experimental components of the uncertainty come from an imperfect knowledge of the conditions a collision takes place in, such as the pileup. To estimate these uncertainties additional Monte Carlo samples are produced with varied simulation conditions and compared to data. In order to derive the JES calibration the energy of a jet is balanced in $p_T$ against a reference object. This reference object can be a photon, a leptonically decaying Z boson, a jet in a different pseudorapidity region of the detector, or several jets with lower $p_T$. Each of these types of reference objects contributes to the overall calibration differently depending on the pseudorapidity and $p_T$ of the jet~\cite{JES}. The effects of the JES uncertainty on the \Wprime\ search are evaluated by generating Monte Carlo samples with $+1\sigma$ and $-1\sigma$ shifts in JES and applying the event selection and multivariate analysis to them.

\subsection{Jet energy resolution (JER)}
\label{SECTION-RESULTS-SYSTEMATICS-JER}
The jet energy resolution (JER) uncertainty is the precision with which the energy of jets can be measured by the ATLAS detector. The JER uncertainty is estimated by measuring the $p_T$ balance in dijet events~\cite{JER}. In order to assess the effect of this resolution on the \Wprime\ search Monte Carlo samples are generated with all of the jet energies modified by a random offset chosen using a Gaussian distribution with a standard deviation equal to the measured JER uncertainty and the event selection and multivariate analysis are applied to the modified sample.

\subsection{Jet reconstruction efficiency}
\label{SECTION-RESULTS-SYSTEMATICS-JETRECO}
The effects of the efficiency of the jet reconstruction algorithm on the \Wprime\ search are estimated by randomly removing jets from the simulated events according to the jet reconstruction efficiency. This efficiency is determined by matching jets reconstructed from tracking information with jets reconstructed from the calorimeter. The modified sample has the event selection and multivariate analysis applied to it and half of the difference in the BDT weight distribution between the modified sample and nominal sample is taken as a symmetric uncertainty around the nominal values~\cite{JETRECO}.

\subsection{Jet vertex fraction (JVF)}
\label{SECTION-RESULTS-SYSTEMATICS-JVF}
The jet vertex fraction (JVF) is the fraction of each jet's total $p_T$ with tracks pointing to the event's primary vertex. This is a useful variable for removing pilup jets and a cut is applied to it for each jet as described in Section~\ref{SECTION-OBJ-JET}. The uncertainty in the efficiency of this cut is determined by taking the difference in cut values necessary to attain the same jet efficiency in data and Monte Carlo Z+jets samples~\cite{JVF}. New Monte Carlo samples are produced with the JVF cut varied up and down from the nominal value by this uncertainty and these modified samples have the event selection and multivariate analysis applied to them to determine the effect on the BDT weight distribution.

\subsection{b-tagging performance}
\label{SECTION-RESULTS-SYSTEMATICS-TAGGING}
The MV1 b-tagging algorithm is described in Section~\ref{SECTION-OBJ-JET-BTAG} and is central to the analysis for background rejection and the definition of the analysis channels. The MV1 algorithm provides an output for each jet which a cut is applied to in order to achieve 70$\%$ b-tagging efficiency. The b-tagging efficiency, c-tagging efficiency, and mis-tagging rate are all measured in data by applying the MV1 algorithm to analysis channels with high purities of b-jets, c-jets, and light flavor jets respectively. The b-tagging efficiency, c-tagging efficiency, and mis-tagging rate thus all have uncertainties derived from the systematic and statistical uncertainties in their measurement~\cite{BTAGGING,CTAGGING,MISTAGGING}. The effects of these uncertainties is evaluated by reweighting the Monte Carlo events by $\pm 1 \sigma$ shifts in each.

\subsection{Lepton energy scale and resolution}
\label{SECTION-RESULTS-SYSTEMATICS-LEP}
The lepton energy scale and resolution uncertainties are determined by comparing $Z \to ll$ data and Monte Carlo events. The lepton energy scale is binned in $E_T$ and pseudorapidity and the difference in the dilepton invariant mass peak in each bin determines the lepton energy scale uncertainty for that bin. The lepton energy resolution is not binned and the difference in the width of the simulated and measured widths of the dilepton invariant mass peak determines the uncertainty in the lepton energy resolution~\cite{ELECTRONRECO,MUONRECO}. The effects of the lepton energy scale uncertainty are evaluated by producing Monte Carlo samples with $+1\sigma$ and $-1\sigma$ shifts in the lepton energy scale and applying the event selection and multivariate analysis to them. The effects of the lepton energy resolution are determined by modifying the nominal lepton energies by a random offset chosen using a Gaussian distribution with a standard deviation equal to the lepton energy resolution uncertainty  and applying the event selection and multivariate analysis to the modified events.

\subsection{Lepton trigger and reconstruction}
\label{SECTION-RESULTS-SYSTEMATICS-LEPRECO}
The simulated Monte Carlo events are weighted by lepton trigger and reconstruction efficiencies measured in the $Z \to ll$ and $W \to l\nu$ channels. These efficiencies are functions of the lepton kinematics and are determined by relaxing the respective requirements and measuring how often the original trigger or reconstruction requirements are met. The uncertainty in these efficiencies is determined by taking half the difference in the measured data and Monte Carlo efficiencies as a symmetric uncertainty about the nominal value~\cite{ELECTRONRECO,MUONRECO}.

\subsection{Missing transverse energy (MET)}
\label{SECTION-RESULTS-SYSTEMATICS-MET}
As described in Section~\ref{SECTION-OBJ-MET}, the MET calculation contains terms for the leptons, jets, soft jets, and cell-out energy. The effects of the uncertainty in the lepton or jet terms are included in the lepton and jet energy scale and resolution uncertainties. The uncertainty in the soft jet and cell-out terms is estimated using a $Z \to \mu\mu$ sample with no jets having $p_T$ $>$ 20 GeV. The projection of the MET onto the reconstructed Z boson's transverse direction is calculated for a data sample and Monte Carlo simulated sample. The average deviation from unity, defined as the ratio of this variable between the data and Monte Carlo samples, is taken as the uncertainty in the soft jets and cell-out energy~\cite{MET}. New Monte Carlo samples are generated with each of these varied up and down by this uncertainty and the event selection and multivariate analysis are applied to these samples to evaluate the effect on the final BDT weight distribution.

\subsection{Parton distribution function (PDF)}
\label{SECTION-RESULTS-SYSTEMATICS-PDF}
Estimates of the uncertainties in the parton distribution functions are provided by their authors as a set of uncertainty eigenvectors. The Monte Carlo samples are reweighted according to each of the 68$\%$ C.L. uncertainty eigenvectors for CT10, MWST2008NLO68CL~\cite{MSTW2008}, and NNPDF23~\cite{NNPDF23} and half of the largest variation from the nominal sample is taken as a symmetric uncertainty on each Monte Carlo sample.

\subsection{Initial state radiation and final state radiation (ISR/FSR)}
\label{SECTION-RESULTS-SYSTEMATICS-ISRFSR}
Initial state radiation and final state radiation (ISR/FSR) is the radiation of a particle immediately before or after the hard interaction of a process. The uncertainty in modeling these effects is estimated by varying the ISR/FSR settings of PYTHIA within the range consistent with previous measurements of $t\bar{t}$~\cite{ISRFSR}. Six additional $t\bar{t}$ samples are produced with these varied ISR/FSR settings and have the event selection and multivariate analysis applied to them. The maximum deviation of the resulting BDT weight distributions from the nominal distribution is taken as a symmetric systematic uncertainty on the sample.

\subsection{Monte Carlo event generator and parton showering}
\label{SECTION-RESULTS-SYSTEMATICS-GENERATOR}
The different Monte Carlo event generation and parton showering programs make a variety of different approximations and produce events with different kinematic distributions. To reduce the analysis' sensitivity to the effects of a specific event generator or parton shower program a systematic uncertainty is included for all top quark processes which is estimated by taking the difference from the nominal Monte Carlo sample and a Monte Carlo sample produced with another event generator or parton shower program. For the $t\bar{t}$ sample the nominal POWHEG+PYTHIA sample is compared to samples produced with POWHEG+HERWIG, MC@NLO+HERWIG, and ALPGEN+HERWIG and the largest difference is taken as a symmetric uncertainty. The nominal single top quark t-channel sample produced with ACERMC+PYTHIA is compared to a sample produced with MC@NLO+HERWIG and the difference between the samples is taken as a symmetric uncertainty. For the single top quark s-channel the nominal POWHEG+PYTHIA sample is compared to a sample produced with MC@NLO+HERWIG and the difference is taken as a symmetric uncertainty. For the single top quark Wt channel, the nominal POWHEG+PYTHIA sample is compared to a sample produced with MC@NLO+HERWIG and samples produced with POWHEG+PYTHIA with the two different NLO calculation schemes (diagram removal and diagram subtraction) are compared to each other, and the larger difference is taken as a symmetric uncertainty on the nominal sample. 

\subsection{Theoretical cross-sections}
\label{SECTION-RESULTS-SYSTEMATICS-XS}
All of the Monte Carlo samples, with the exception of the W+jets and multijets samples, are normalized to a calculated theoretical cross-section. These cross-sections have associated uncertainties in their calculation and each channel is assigned an independent flat uncertainty to account for this. The $t\bar{t}$ sample has an cross-section uncertainty of $-5.9$/$+5.1\%$. For the single top quark processes the uncertainty in the s-channel cross-section is $\pm 3.9\%$, the uncertainty in the t-channel cross-section is  $-2.1$/$+3.9\%$, and the uncertainty in the Wt channel cross-section is $\pm 6.8\%$. The Z+jets and diboson samples both have a cross-section uncertainty of $42\%$.

\subsection{W+jets normalization}
\label{SECTION-RESULTS-SYSTEMATICS-WJETS}
The W+jets Monte Carlo samples are normalized using a data driven technique described in Section~\ref{SECTION-BG-DD-WJETS}. An uncertainty in this normalization is determined by applying each of the applicable uncertainties described in this Section to the control region used to derive the W+jets normalization. The variation in the normalization caused by each of these individual uncertainties is then added in quadrature to calculate the total W+jets normalization uncertainty.

\subsection{Multijet normalization}
\label{SECTION-RESULTS-SYSTEMATICS-MULTIJET}
The multijet background is normalized using the matrix method as described in Section~\ref{SECTION-BG-DD-QCD}. Since this is a small background and a detailed estimation of the uncertainty in its normalization would be very difficult, a $50\%$ uncertainty on the sample's normalization is applied.

\subsection{Luminosity}
\label{SECTION-RESULTS-SYSTEMATICS-LUMI}
There is a $\pm 3.6\%$ uncertainty in the luminosity of the data set used for this analysis, which is preliminary result derived from updated van der Meer scans performed in April 2012~\cite{LUMI}. Since all of the Monte Carlo samples are normalized to this luminosity, this uncertainty is assessed as a flat $\pm 3.6\%$ uncertainty in all of the Monte Carlo samples.


\section{Limit setting procedure}
\label{SECTION-RESULTS-LIMITS}
The statistical analysis for the \Wprimechan\ search uses the $CL_S$ method~\cite{CLS,CLS2} implemented in the MClimit software package. The $CL_S$ method is a binned likelihood method, meaning the likelihood of each bin of the BDT weight distribution occuring is computed and then all of the bins are combined into a global likelihood. This method was chosen because it is more powerful than a single bin counting experiment and is less sensitive to fluctuations in the background, as described below.

The $CL_S$ method is derived from basic Poisson and Gaussian statistics. The likelihood of observing N events if there are $\mu$ expected events is given by Poisson statistics as:

\begin{equation}
\label{EQ-RESULTS-POISSON}
\mathcal{L}(N|\mu) = \frac{\mu^{N}e^{-\mu}}{N!}
\end{equation}

The inclusion of uncertainties into this likelihood is done by modifying the expected number of events through nuisance parameters ($\theta$), each representing a single uncertainty. The expected number of events of each samples ($N_i$) is modified by the probability density function of each of the uncertainties that affect that sample (G), and then all of the samples are summed over to produce the final expected number of events as shown in Equation~\ref{EQ-RESULTS-NIUSSANCE}

\begin{equation}
\label{EQ-RESULTS-NIUSSANCE}
\mu = \sum\limits_iN_i\prod\limits_jG(\theta_{ij},\delta_{ij})
\end{equation}

\noindent
The probability density function for each of the uncertainties can in principle be any non-negative unitary distribution, but for this analysis they are all assumed to be Gaussian distributions with a mean of 1 and a standard deviation of $\delta$. One of the consequences of Equation~\ref{EQ-RESULTS-NIUSSANCE} is that $\mu$ is now a multi-dimensional probability distribution and not a single value.

So far this procedure has described a single counting experiment. Each bin of the BDT weight distribution can be treated as an independent counting experiment and the likelihood of observing an event distribution ($N_{obs}$) with $N_k$ events in the $k^{th}$ bin is given by the product of the independent bins' likelihoods as shown in Equation~\ref{EQ-RESULTS-MULTIBIN}.

\begin{equation}
\label{EQ-RESULTS-MULTIBIN}
\mathcal{L}(N_{obs}|\theta_{ij},\delta_{ij}) = \prod\limits_k\frac{\mu_k^{N_k}e^{-\mu_k}}{N_k!}
\end{equation}

In order to determine if a signal process at a given mass point is excluded, 10,000 psuedo-experiments are generated for each of two hypotheses: one hyopthesis ($H_1$) includes the signal and all of the background processes while the other hypothesis ($H_0$) includes only the background processess. These pseudo-experiments are generated by randomly choosing a value for each nuisance parameter based on that parameter's probability distribution function. This shifted distribution is used to set the expected number of events of a Poisson distribution in each bin. These Poisson distributions are then randomly sampled to produce the final psuedo experiment distribution. For each psuedo experiment the log-likelihood ratio (LLR) is computed according to Equation~\ref{EQ-RESULTS-LLR}.

\begin{equation}
\label{EQ-RESULTS-LLR}
LLR = -2ln\left(\frac{\mathcal{L}(N|H_1)}{\mathcal{L}(N|H_0)}\right)
\end{equation}

\noindent
The LLR of the observed data set is also calculated and used to determine the strength of the limit on the signal process. $CL_{S+B}$ is defined as the fraction of pseudo-experiments generated with the signal-plus-background hypothesis ($H_1$) with an LLR greater than the LLR of the data set. Similarly, $CL_B$ is defined as the fraction of pseudo-experiments generated with the background-only hypothesis ($H_0$) with an LLR greater than that of the data set. $CL_S$ is the likelihood of the signal being included in the data and is defined as the ratio of $CL_{S+B}$ to $CL_B$ as shown in Equation~\ref{EQ-RESULTS-CLS}.  An advantage of using $CL_S$ rather than $CL_{S+B}$ is that any mis-modeling or fluctuations in the background estimates will affect $CL_{S+B}$ and $CL_B$ similarly and will partially cancel when determining $CL_S$.

\begin{equation}
\label{EQ-RESULTS-CLS}
CL_S = \frac{CL_{S+B}}{CL_B}
\end{equation}

A signal is excluded at the 95$\%$ confidence level if it has a $CL_S$ $<$ 0.05. After the signal has been excluded or not at its nominal cross-section the process is repeated with the normalization of the signal sample decreased (if the current cross-section was excluded) or increased (if the current cross-section was not excluded) by a factor $k$. This is iteratively repeated until $CL_S$ equals 0.05. When this occurs the value of k can be used to determine the 95$\%$ limit on the cross-section as shown in Equation~\ref{EQ-RESULTS-95XS}.

\begin{equation}
\label{EQ-RESULTS-95XS}
\sigma_{95\%\ limit} = k\sigma_{nominal}
\end{equation}

This can be taken one step further and 95$\%$ confidence level limits can be placed on the coupling strengths $g^\prime_R$ and $g^\prime_L$. The effective Lagrangian density used to generate the signal samples was given in Equation~\ref{EQ-THEORY-BSM-EFFECTIVE} and is repeated here in Equation~\ref{EQ-RESULTS-BSM-EFFECTIVE}.

\begin{equation}
\label{EQ-RESULTS-BSM-EFFECTIVE}
\mathcal{L}_{\Wprime} = \frac{1}{2\sqrt{2}}V^\prime_{ij}W^\prime_{\mu}\bar{f}^i\gamma^{\mu}(g^\prime_R(1+\gamma_5)+g^\prime_L(1-\gamma_5))f^j
\end{equation}

\noindent
The cross-sections of these processes are thus proportional to $g^{\prime 2}_R$ or $g^{\prime 2}_L$. The signal processes were all generated with $g^\prime_R$ or $g^\prime_L$ equal to the Standard Model W coupling stength $g$, so the ratio of the observed limit on the cross-section to the nominal cross-section for the \WprimeR\ or \WprimeL\ signals can be used to compute an equivalent limit on $\frac{g^{\prime}_R}{g}$ or $\frac{g^{\prime}_L}{g}$ respectively using Equation~\ref{EQ-RESULTS-GPRIME}.

\begin{equation}
\label{EQ-RESULTS-GPRIME}
\sqrt{k} = \sqrt{\frac{\sigma_{95\%\ limit}}{\sigma_{nominal}}} = \frac{g^\prime}{g}
\end{equation}


\section{Results}
\label{SECTION-RESULTS-RESULTS}
The limit setting procedure described in Section~\ref{SECTION-RESULTS-LIMITS} is performed for each of the 22 \WprimeR\ and \WprimeL\ mass points (11 mass points each) in all four of the analysis channels (2jet 1tag, 2jet 2tag, 3jet 1tag, 3jet 2tag). The limits on cross-section times branching ratio for the \WprimeR\ mass points in each of the analysis channels are plotted in Figure~\ref{FIG-RESULTS-WPRIMERIGHTLIMITS}, and the limits on cross-section times branching ratio for the \WprimeL\ mass points in each of the analysis channels are plotted in Figure~\ref{FIG-RESULTS-WPRIMELEFTLIMITS}. The red line in these plots shows the theoretical cross-section times branching ratio as a function of \Wprime\ mass which is calculated with $g^\prime$ = $g$. The solid black line shows the observed 95$\%$ confidence level limit on the cross-section times branching ratio as a function of \Wprime\ mass. The region above this line has been excluded at the 95$\%$ confidence level and regions where the theoretical cross-section times branching ratio are above the observed limit have been excluded for $g^\prime$ = $g$. The dashed line is the expected limit, which is derived similarly to the observed limit but is the median result of 10,000 background only pseudo-data sets used in place of the data. The green and yellow bands are the 68$\%$ (1$\sigma$) and 95$\%$ (2$\sigma$) confidence interval of the expected limit results. The mass value where the theoretical cross-section times branching ratio crosses the observed or expected limit is known as the mass limit and is a commonly used figure of merit for exclusions such as this. The mass limits for each of the analysis channels is given in Table~\ref{TABLE-RESULTS-WPRIMERLIMITS} for a \WprimeR\ boson and in Table~\ref{TABLE-RESULTS-WPRIMELLIMITS} for a \WprimeL\ boson.

%\QUADFIGLEG{Muon}{Muon}{Muon}{Muon}{95$\%$ confidence level limits on the cross-section times branching ratio of the \WprimeR\ boson in the (a) 2jet 1tag channel (b) 3jet 1tag channel (c) 3jet 1tag channel (d) 3jet 2tag channel}{FIG-RESULTS-WPRIMERIGHTLIMITS}

%\QUADFIGLEG{Muon}{Muon}{Muon}{Muon}{95$\%$ confidence level limits on the cross-section times branching ratio of the \WprimeL\ boson in the (a) 2jet 1tag channel (b) 3jet 1tag channel (c) 3jet 1tag channel (d) 3jet 2tag channel}{FIG-RESULTS-WPRIMELEFTLIMITS}

\begin{table}
\begin{center}
\begin{tabular}{|l|l|l|}
\hline
Analysis Channel & Observed Mass Limit & Expected Mass Limit\\
\hline
2jet 1tag & & \\
3jet 1tag & & \\
2jet 2tag & & \\
3jet 2tag & & \\
\hline
\end{tabular}
\caption{Expected and observed mass limits for a \WprimeR\ boson in the individual analysis channels.}
\label{TABLE-RESULTS-WPRIMERLIMITS}
\end{center}
\end{table}

\begin{table}
\begin{center}
\begin{tabular}{|l|l|l|}
\hline
Analysis Channel & Observed Mass Limit & Expected Mass Limit\\
\hline
2jet 1tag & & \\
3jet 1tag & & \\
2jet 2tag & & \\
3jet 2tag & & \\
\hline
\end{tabular}
\caption{Expected and observed mass limits for a \WprimeL\ boson in the individual analysis channels.}
\label{TABLE-RESULTS-WPRIMELLIMITS}
\end{center}
\end{table}

The limits on the cross-section times branching ratio can also be used to calculate limits on $\frac{g^\prime}{g}$ as described by Equation~\ref{EQ-RESULTS-GPRIME}. By using the observed or expected limits on the cross-section times branching ratio, observed or expected limits on $\frac{g^\prime_R}{g}$ and $\frac{g^\prime_L}{g}$ can be calculated for each of the tested mass points for the \WprimeR\ and \WprimeL\ bosons respectively. These limits are plotted as a function of the \Wprime\ mass and an exclusion region is formed, as seen in Figures~\ref{FIG-RESULTS-WPRIMERGLIMITS} and \ref{FIG-RESULTS-WPRIMELGLIMITS}.

%\QUADFIGLEG{Muon}{Muon}{Muon}{Muon}{95$\%$ confidence level limits on $\frac{g^\prime_R}{g}$ in the (a) 2jet 1tag channel (b) 3jet 1tag channel (c) 3jet 1tag channel (d) 3jet 2tag channel}{FIG-RESULTS-WPRIMERGLIMITS}

%\QUADFIGLEG{Muon}{Muon}{Muon}{Muon}{95$\%$ confidence level limits on $\frac{g^\prime_L}{g}$ in the (a) 2jet 1tag channel (b) 3jet 1tag channel (c) 3jet 1tag channel (d) 3jet 2tag channel}{FIG-RESULTS-WPRIMELGLIMITS}

The individual analysis channels can also be combined by treating each channel as more bins in the BDT weight distribution. The resulting combination is analyzed in an identical manner to the individual analysis channels and similar limits can be made. Because of their kinematic similarities, the 2jet 1tag and 3jet 1tag channels are combined to form the 1tag combined channel, and the 2jet 2tag and 3jet 2tag channels are combined to form the 2tag combined channel. A final combination of all of the analysis channels forms the full combination channel. The limits on the cross-section times branching ratio for a \WprimeR\ and \WprimeL\ boson are shown in Figure~\ref{FIG-RESULTS-WPRIMERIGHTCOMBLIMITS} and \ref{FIG-RESULTS-WPRIMELEFCOMBTLIMITS} respectively. The derived mass limits from the combined channels for a \WprimeR\ or \WprimeL boson are given in Table~\ref{TABLE-RESULTS-WPRIMERCOMBLIMITS} and \ref{TABLE-RESULTS-WPRIMELCOMBLIMITS}. The limits on $\frac{g^\prime_R}{g}$ and $\frac{g^\prime_L}{g}$ are shown in Figure~\ref{FIG-RESULTS-WPRIMERGCOMBLIMITS} and \ref{FIG-RESULTS-WPRIMELGCOMBLIMITS}.

%\TRPFIGLEG{Muon}{Muon}{Muon}{95$\%$ confidence level limits on the cross-section times branching ratio of the \WprimeR\ boson in the (a) 1tag combined channel (b) 2tag combined channel (c) full combination channel}{FIG-RESULTS-WPRIMERIGHTCOMBLIMITS}

%\TRPFIGLEG{Muon}{Muon}{Muon}{95$\%$ confidence level limits on the cross-section times branching ratio of the \WprimeL\ boson in the (a) 1tag combined channel (b) 2tag combined channel (c) full combination channel}{FIG-RESULTS-WPRIMELEFCOMBTLIMITS}

\begin{table}
\begin{center}
\begin{tabular}{|l|l|l|}
\hline
Analysis Channel & Observed Mass Limit & Expected Mass Limit\\
\hline
1tag combined & & \\
2tag combined & & \\
full combination & & \\
\hline
\end{tabular}
\caption{Expected and observed mass limits for a \WprimeR\ boson in the combined analysis channels.}
\label{TABLE-RESULTS-WPRIMERCOMBLIMITS}
\end{center}
\end{table}

\begin{table}
\begin{center}
\begin{tabular}{|l|l|l|}
\hline
Analysis Channel & Observed Mass Limit & Expected Mass Limit\\
\hline
1tag combined & & \\
2tag combined & & \\
full combination & & \\
\hline
\end{tabular}
\caption{Expected and observed mass limits for a \WprimeL\ boson in the combined analysis channels.}
\label{TABLE-RESULTS-WPRIMELCOMBLIMITS}
\end{center}
\end{table}

%\TRPFIGLEG{Muon}{Muon}{Muon}{95$\%$ confidence level limits on $\frac{g^\prime_R}{g}$ in the (a) 1tag combined channel (b) 2tag combined channel (c) full combination channel}{FIG-RESULTS-WPRIMERGCOMBLIMITS}

%\TRPFIGLEG{Muon}{Muon}{Muon}{95$\%$ confidence level limits on $\frac{g^\prime_L}{g}$ in the (a) 1tag combined channel (b) 2tag combined channel (c) full combination channel}{FIG-RESULTS-WPRIMELGCOMBLIMITS}



