\chapter{Introduction}
\label{SECTION-INTRO}
Humanity has always sought to understand. People have progressed from an intuitive understanding of the universe based on practical experience to a more scientific understanding based on logic and the rigorous study of phenomena. The understanding brought about by science has allowed humanity to achieve feats which would otherwise be impossible, and we continue to seek more understanding. At the forefront of this quest, particle physicists work to understand the most basic components of the universe and the most fundamental of interactions between them. The field has progressed from classical theories of the four natural elements, through the chemical elements and nuclear structure, to modern relativistic quantum field theories that describe the most fundamental objects yet imagined. Continuing research in the field takes many forms: precision measurements of trapped and isolated particles over months allow theoretical models to be tested, detection of particles accelerated by supernovae allows access to an energy regime physicists are unable to recreate on Earth, and experiments at particle colliders allow for the investigation of the rarest phenomena.

This dissertation describes the search for \Wprimechan\ using the ATLAS detector at CERN. This is a search for a new particle using the largest detector at the largest, highest energy particle collider ever built and is part of an international research program in collaboration with a multinational research group from around the world. The dissertation is divided into chapters intended to first establish the environment the analysis takes place in, to build up the analysis from basic components, and finally to present the results and put them into context with the rest of the field. Towards this end the chapters are as follows:

\begin{list}{$\bullet$}{}
\item \textbf{Chapter 2} discusses the theoretical background for the analysis.
\item \textbf{Chapter 3} describes the physical apparatus of the experiment.
\item \textbf{Chapter 4} defines the analysis level objects reconstructed from the detector response.
\item \textbf{Chapter 5} details the simulation methods used to model the backgrounds and potential signal processes.
\item \textbf{Chapter 6} defines the criteria for an event to be included in the analysis.
\item \textbf{Chapter 7} describes the multivartiate analysis technique and its specific implementation.
\item \textbf{Chapter 8} details the results of the analysis and their statistical significance.
\item \textbf{Chapter 9} discusses the results of the analysis and their context in current research.
\end{list}
