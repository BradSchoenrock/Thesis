%% Before beginning to type your dissertation, read the formatting guide, 
%% which can be found at http://grad.msu.edu/etd/docs/formattingguide.pdf
%% Also get the latest version of  msuphddissertation.cls and the template file
%% at http://www.math.msu.edu/~weil/MSU_Ph.D._Dissertation.zip
%% Send questions to weil@math.msu.edu

\documentclass{msuphddissertation}
\graphicspath{{figures/}}
%\usepackage{todo}
%\usepackage[hide]{todo}
%\usepackage{epstopdf}
\usepackage{lineno}
\usepackage{subfigure}
\usepackage{modatlasstyle}
\usepackage{amsmath,amssymb,amsthm,paralist}
\usepackage{extra_definitions}
\usepackage{multirow}
\usepackage{mathtools}
\usepackage{authblk}
\usepackage{mathrsfs}
\usepackage{graphicx}
\usepackage{pdflscape}
\usepackage{array}
\usepackage{units}
\usepackage{bigstrut}
\usepackage{url}
%\usepackage{ulem}
\usepackage[table]{xcolor}
\renewcommand\bibname{BIBLIOGRAPHY}
%\usepackage{hyperref}
%% Include other packages you wish to use except setspace. 
%% That package is loaded automatically.
%% IMPORTANT: Load only those packages you know you will use.
%% Some packages can cause conflicts resulting in improper formatting.
\author{Patrick True} %% Put your name in full as it is officially recognized by Michigan State University here.
\title{Search for W' production in the single-top channel with the ATLAS detector} %% Put the title of your dissertation here.
\munit{}
%\munit{High Energy Physics} %% Put the name of the field of your degree (NOT department or division, or college) here.
%% For example Dynamical Systems, Psychology, String Theory, etc.

%% Put additional preamble items here.
%%%%%% LANDSCAPE %%%%%%
%% Put a page you want in landscape inside the environment msulandscape, 
%% which is defined in msuphddissertation.cls. No extra package is needed.

%%%%%%%%%%%%%%%%%%%%%%%%%%%%
%%%%%%%%  NOTE   %%%%%%%%%%%%%%
%% PREPARING A DISSERTATION WITH THIS CLASS FILE DOES NOT %%%
%% GUARANTEE THAT THE GRADUATE SCHOOL WILL APPROVE IT %%%
%%%%%%%%%%%%%%%%%%%%%%%%%%%%%%%

%%%%%%%%%%%%%%%%%%%%%%%%%%%%%%%%%%
%%%%%%%%%%%% WARNING %%%%%%%%%%%%%%%
%% The Graduate School requires that all text, including superscripts %%
%% and subscripts at all levels, as well as that in imported %%
%% graphics files be in 12 point. For that reason it's recommended %%
%% that no text be part of any imported files. %%

%% Once your document has been filed with the Graduate School,
%% if you wish to produce a version of it whose subscripts and superscripts
%% are in traditional smaller proportion, remove the "%" sign 
%% in front of following command. 
%\DeclareMathSizes{12}{12}{10}{8}
%% If your document has footnotes, remove the "%" sign 
%% in front of following command. 
%\renewcommand{\footnotesize}{\small}
%% To single space your document, remove the 
%% two commands \begin{doublespace}
%% and \end{doublespace below.

\begin{document}

\maketitlepage %%This command will produce the title page of your thesis.
\begin{abstract}
This thesis presents the search for W'$\rightarrow$tb using the LHC pp collision data collected with the ATLAS detector at a center-of-mass energy of 8 TeV. The primary backgrounds to this search are ttbar, W+jets, and multijets processes. To reduce the contributions of these backgrounds we require a leptonic final state and use Boosted Decision Trees to discriminate between signal and background-like events. This measurement gives limits on the W'$\rightarrow$tb cross-section times branching ratio and on the ratio of coupling constants g'/g as functions of the W' mass.


%formatted for stupid proquest
%This thesis discusses a search for the Standard Model single top Wt-channel process. An analysis has been performed searching for the Wt-channel process using 4.7 fb<super>-1<\super> of integrated luminosity collected with the ATLAS detector at the Large Hadron Collider. A boosted decision tree is trained using machine learning techniques to increase the separation between signal and background. A profile likelihood fit is used to measure the cross-section of the Wt-channel process at 16.8 &plusmn;2.9 (stat) &plusmn; 4.9(syst) pb, consistent with the Standard Model prediction. This fit is also used to generate pseudoexperiments to calculate the significance, finding an observed (expected) 3.3 &sigma; (3.4 &sigma;) excess over background.

%% Type your abstract here. An abstract is REQUIRED and limited to two pages.
%% The abstract must not include any figures.
\end{abstract}

%% If you wish to have a copyright page, remove the "%" in front of  \begin{copyrt}
%% and remove the "%" in front of \end{copyrt}.
%% The mandatory form of the Copyright will be generated automatically. 
%% A copyright statement is optional.

%\begin{copyrt}
%\end{copyrt}

%% If you wish to have a dedication, remove the "%" in front of
%\begin{dedication}
%% and remove the "%" in front of 
%\end{dedication}
%% A dedication must be single-spaced and 
%% centered on the page.  Both will be done automatically. 

%\begin{dedication} 
%% Type your dedication here. A dedication is optional.
%\end{dedication}
%% If you wish to have an acknowledgment, remove the "%" in front of  \begin{acknowledgment}
%% and remove the "%" in front of  \end{acknowledgment}  
\begin{acknowledgment}
Thanks Amber.
%% Type your acknowledgment here. An acknowledgment is optional.
\end{acknowledgment}
\newpage
%% If you wish to have a preface, remove the "%" in front of  \begin{preface}
%% and remove the "%" in front of  \end{preface}  
%\begin{preface}
%% Type your acknowledgment here. An acknowledgment is optional.
%\end{preface}

\TOC

%% If your document contains tables, remove the "%" in front of 
%%  the following line.
\listoftables

%% If your document contains figures, remove the "%" in front of
%% the following line.
\LOF
%% If any of your figures contain color, you must
%% include the following disclaimer in the caption of your first figure.
%% "For interpretation of the references to color in this and all other figures, 
%% the reader is referred to the electronic version of this dissertation."

%%%% LIST OF SYMBOLS AND ABBREVIATIONS %%%%
%% Such a list is possible using the environment
%% abbreviationskey
%% at the place in the document where you wish the list to appear.
% The list will be included in the TOC as KEY TO SYMBOLS AND ABBREVIATIONS
%%%%%%%%%%

\newpage
\pagenumbering{arabic}
\begin{doublespace}
%\linenumbers

\chapter{Introduction}
\label{SECTION-INTRO}

If you have knowledge, let others light their candles in it. -Margaret Fuller

\vspace{5mm} %5mm vertical space

High energy physics is concerned with obtaining the most fundamental understanding of the universe. In practice this means categorizing all fundamental particles and their interactions in order to understand what the world is made of. Assorted scientific fields question what the world is made of in various detail. Chemistry asks which atoms and molecules comprise the things around us, nuclear physics investigates what makes up the nuclei of atoms and how nuclei are formed, and high energy physics studies what we currently think are the most fundamental particles in existence. In order to understand high energy physics we need a framework to describe the elementary particles and their interactions. This framework is referred to as the Standard Model (SM). 


\section{The Standard Model}
\label{SECTION-THEORY-SM}

The SM of high energy physics has been among the most successful theories of the past century. It has been tested again and again and has encountered few unexplained anomalies. It started as an effort to combine the fundamental forces we know into one overarching theory. Electricity and magnetism had been combined into electromagnetism long ago and in the last century the SM was developed. Electromagnetism was combined with weak interactions, followed by the inclusion of the Higgs mechanism and strong interactions to form the SM we know today~\cite{Griffiths,QFT-PS,QFT-IZ}. 

\LARGEFIG{StdMdl-wheel}{The SM of high energy physics~\cite{SMPic}.}{FIGURE-STDMDL}
%\LARGEFIG{StdMdl}{The SM of high energy physics~\cite{Figure-StdMdl}.}{FIGURE-STDMDL}

The SM particles are classified based on their properties and interactions and are shown in Figure~\ref{FIGURE-STDMDL}. One way we can classify particles is by their spin. A particle with half integer spin is called a fermion (colored red or green in Figure~\ref{FIGURE-STDMDL}) while a particle with integer spin is a boson (colored blue or black in Figure~\ref{FIGURE-STDMDL}). All discovered fundamental particles are either spin 0, spin $\frac{1}{2}$ or spin 1. We further break down the fermions into two categories, the first set are the leptons which have an electric charge $\pm 1$ (electron, muon, and tau) to interact with the electroweak force, and three neutral neutrinos which only interact via the weak force. The other type of fermion is the quark. Quarks interact via the weak, electromagnetic, and strong forces carrying half integer spins, $\pm \frac{1}{3}$ or $\pm \frac{2}{3}$ electrical charges, and color charges. The strong force, at low energies, imparts color confinement onto individual quarks which binds them together in mesons (quark antiquark pairs) or baryons (three quark systems such as the proton or neutron). If quarks are high enough energy they undergo a process known as hadronization where new quark-antiquark pairs are created from that energy until all that remain are many mesons and baryons. Quarks also interact electromagnetically and weakly like their charged leptonic counterparts. The vector bosons (spin 1) moderate the forces involved in the Standard Model. The gluon interacts via the strong force, the photon and the $W^\pm$ interact electromagnetically, and the $W^\pm$ and $Z$~bosons interact weakly. The final particle we have is the recently discovered Higgs boson~\cite{Aad:2012tfa} which took nearly 50~years to discover. A history of particle discovery can be seen in Figure~\ref{FIGURE-TIMELINE}. 

\VLARGEFIG{Timeline}{History of high energy physics illustrating the time it took from theorizing the existence of the particles until discovery~\cite{Timeline}.}{FIGURE-TIMELINE}

A deeper understanding of the SM can be obtained through the Lagrange density~\cite{QFT-PS}, 
 
%\mathscr{L}
\begin{equation}
\begin{split}%\begin{multline}
\mathscr{L} = & -\frac{1}{2}tr[G_{\mu \nu}G^{\mu \nu}] -\frac{1}{2}tr[W_{\mu \nu}W^{\mu \nu}] -\frac{1}{4} B_{\mu \nu} B^{\mu \nu} \\ 
& + i \bar{\psi} [ \cancel{D} -m] \psi + \bar{\psi}_{iL}y_{ij}\psi_{jR}\phi  +  h.c. +|D_\mu \phi|^2 - V(\phi)
\end{split}%\end{multline}
\label{EQUATION-STDLAG}
\end{equation}

\noindent where $\psi$ is the Dirac field with a sum over the matter particles with $L$ denoting left-handed particles and $R$ denoting right-handed particles, $\phi$ is the Higgs field, $y_{ij}$ are the Yukawa couplings, $\cancel{D}$ is the covariant derivative defined through Dirac slash notation as 

\begin{equation}
\cancel{D}=\gamma_\mu D^\mu
\end{equation}

\begin{equation}
D^\mu=\partial^\mu - i g_S T^a G^{a\mu} - i Y g_Y B^\mu - \frac{i g_L}{2}\sigma ^a W^{a \mu} 
\end{equation}


\noindent where Y is the hypercharge of a particle. Hypercharge for left-handed particles are $-\frac{1}{2}$ for leptons and $\frac{1}{6}$ for quarks while for right-handed particles hypercharge is the electric charge of the particle. The tensor $T^a$ is defined as half of $\lambda_a$ (which are the 8 Gell Mann matrices) for quarks, and is zero for leptons. The $\sigma_a$ matrices are the 3 Pauli matrices. The covariant derivative also applies to the Higgs boson, with $T^a=0$ (no coupling to strong force) and $Y=-\frac{1}{2}$. The gauge field strength tensor is denoted by $B_{\mu \nu}$ and is defined by 
 
\begin{equation}
B_{\mu \nu} = \frac{\partial B_\nu}{\partial \mu} - \frac{\partial B_\mu}{\partial \nu}
\end{equation}


\noindent where $B_\nu$ is the hypercharge gauge potential. The QCD field tensor $G_{\mu\nu}$ defines the gluon fields and are defined as 


\begin{equation}
G_{\mu\nu}=\frac{\lambda_a}{2} G_{\mu\nu}^a=\frac{i}{g_s}[D_\mu ,D_\nu]
\end{equation}
\begin{equation}
G_{\mu\nu}^a=\partial_\mu G_\nu^ a - \partial_\nu G_\mu^a + g_s f^{abc} G_{\mu b} G_{\nu c}
\end{equation}

\noindent where $G_\nu$ is the strong gauge potential. The weak tensor $W_{\mu\nu}$ is defined as 

\begin{equation}
W_{\mu\nu}=\frac{\sigma_a}{2} W_{\mu\nu}^a=\frac{i}{g}[D_\mu ,D_\nu]
\end{equation}
\begin{equation}
W_{\mu\nu}^a=\partial_\mu W_\nu^ a - \partial_\nu W_\mu^a + g f^{abc} W_{\mu b} W_{\nu c}
\end{equation}

\noindent where $W_\nu$ is the weak gauge potential. Note that the $W^{a \mu}$ term in equation 1.3 only couples to left-handed particles. 

The SM Lagrangian in equation \ref{EQUATION-STDLAG} contains a lot of information on the SM in one concise equation. The first three terms in the Lagrange density formula contains the strong and electroweak forces, the fourth term describes how the particles interact with these fields, the fifth term and its Hermitian conjugate ($h.c.$) describes how the fermions get their masses (note the $\phi$ dependence means that the Higgs contributes but does not determine the value of their masses), the next to last term describes how the Higgs gives mass to the bosons, and the final term is the Higgs potential~\cite{Griffiths,QFT-PS,QFT-IZ}. 

This formulation represents a group with a $SU(3) \times SU(2) \times U(1)$ symmetry. The $SU(3)$ represents the strong force, with the threefold symmetry in color charge. The eight generators of this $SU(3)$ symmetry correspond to the various color combinations of the gluon which can be mathematically represented by the Gell-Mann matrices. The $SU(2) \times U(1)$ represents the electroweak force which unified electricity, magnetism, and the weak forces whose generators can be represented by the Pauli matrices. The Higgs mechanism breaks this symmetry and this phenomenon is known as electroweak symmetry breaking. By breaking this symmetry the massless electroweak bosons ($W_1$, $W_2$, $W_3$), and the hypercharge boson ($B$) are recombined as the massive $W^+$, $W^-$ (which are linear combinations of $W_1$ and $W_2$), the massive $Z^0$ (which is a linear combination of $W_3$ and $B$) and the massless photon (which is a combination of $W_3$ and $B$ as well). The Higgs doublet has 4 degrees of freedom, three of which are consumed by the longitudinal components of the massive $W^+$, $W^-$, and $Z^0$. The remaining degree of freedom is a neutral scalar particle, the Higgs boson~\cite{GROUPTHEORY,GROUPTHEORY2}. 

\section{Feynman Diagrams}
\label{SECTION-FEYNMANN-DIAGRAMS}

Thanks to Richard Feynman we can obtain an intuitive understanding of particles and their interactions through Feynman diagrams~\cite{QFT-PS}. We can view these pictures as having direct correlation with the processes involved and even set up the relevant equations to compute the scattering amplitude of a particular process. In these diagrams we compact the spacial dimensions into one vertical axis while time is represented on the horizontal axis. %A classic example is Bhabha scattering in Figure~\ref{FIGURE-baba}.

%\MEDIUMFIG{BaBa}{Bhabha scattering~\cite{JAXO}.}{FIGURE-baba}

%In Figure~\ref{FIGURE-baba} the electron and positron meet, annihilate into a photon or $Z$~boson, and then decay into an electron positron pair. Notice that the electron lines have a directionality to them. The arrow pointing forward in time is the particle (electron) while the arrow pointing backward in time is interpreted as the antiparticle (the positron). Diagrams with loops and/or diagrams which have gluons radiating from the incoming quarks (known as initial-state radiation or ISR) or have gluons radiating from the outgoing particles (known as final-state radiation FSR) that still have the same underlying event structure are known as higher order diagrams, which can be seen in Figure~\ref{FIGURE-babaloop} or~\ref{FIGURE-OneLoopbaba}. 

%This Feynman diagram represents the annihilation term in the Bhabha scattering process but, when doing calculations, there is another term that corresponds to one electron emitting a photon and the other electron absorbing it (Moller scattering). Together these diagrams will give a LO calculation of the cross-section for Bhabha scattering. There are also diagrams with loops and/or diagrams which have initial-state radiation(ISR) or final-state radiation(FSR) that still have the same underlying event structure as seen in Figure~\ref{FIGURE-babaloop} or~\ref{FIGURE-OneLoopbaba}. 

%\SMALLFIG{babaloop}{An example of an NLO diagram for Bhabha scattering~\cite{Actis:2009uq}.}{FIGURE-babaloop}

%\FIG{OneLoopbaba}{Examples of NNLO diagrams for Bhabha scattering with loops and real emissions~\cite{Actis:2009uq}.}{FIGURE-OneLoopbaba}

The \xs~for a particular scattering process is defined as the ratio of number of particles scattered per unit time ($dN(t)$) to number of particles passing through a defined area per unit time ($n$), see equation~\ref{EQUATION-xsecs}. 

\begin{equation}
d\sigma =dN(t)/n 
\label{EQUATION-xsecs}
\end{equation}

\begin{equation}
N_{events}=\sigma \int L(t) dt
\label{EQUATION-NEV}
\end{equation}

\begin{equation}
 L(t)= \frac{n_1 n_2}{4\pi \sigma_x \sigma_y} 
\label{EQUATION-LUMIN}
\end{equation}

Informally this is how probable that process is to occur in each interaction. The standard unit for \xs~($\sigma$) is the Barn ($10^{-24}\ $cm$^{2}$) but we commonly use picobarn or femtobarn to describe \xs s. Consequently we define the beam intensity, or luminosity ($L$), in inverse picobarns or inverse femtobarns. That way we can easily calculate the expected number of events from equation \ref{EQUATION-NEV}. Luminosity can be calculated from the beam parameters of the accelerator by equation~\ref{EQUATION-LUMIN} where $n_1$ and $n_2$ are the number of particles in each beam, and $\sigma_x$ and $\sigma_y$ are the Gaussian RMS beam sizes in their respective directions~\cite{QFT-PS}.


%\subsection{The Higgs Boson}
%\label{SECTION-HIGGS}

%The most recently discovered particle was the Higgs Boson. on July 2, 2012 D0 and CDF at the Tevatron announced they had evidence of a particle resembling the SM Higgs boson with a mass of approximately 125 GeV~\cite{d0higgs}~\cite{tevatron-bbbar}. Two days later the \atlas and CMS experiments jointly announced the discovery of a particle resembling the SM Higgs boson with a mass of approximately 125 GeV that has since been verified to be the SM Higgs boson. The excess in \atlas data (solid line) at 125 GeV can be seen as distinctly above the one standard deviation band (green) and the two standard deviation band (yellow) away from the expected background only hypothesis (dashed line inside the bands) in Figure~\ref{FIGURE-HIGGS-DISC}~\cite{Higgs\atlas}~\cite{HiggsCMS}. 

%\VLARGEFIG{HiggsDisc}{A graph showing an excess in \atlas data above the expected limit without the Higgs boson at 125 GeV~\cite{HiggsATLAS}.}{FIGURE-HIGGS-DISC}

%\VLARGEFIG{Timeline}{History of high energy physics illustrating the time it took from theorizing the existence of the particles until discovery~\cite{Timeline}.}{FIGURE-TIMELINE}

%This concluded a nearly five decade search that stands as the longest search for a fundamental particle that has been discovered. The history of fundamental particle discovery can be seen in Figure~\ref{FIGURE-TIMELINE}. The Theory of electroweak symmetry breaking and the subsequent search for the Higgs started with the work by Nambu and Goldstone in 1960 which predicted the massless Nambu-Goldstone boson as a consequence of electroweak symmetry breaking~\cite{PhysRev.117.648}. Anderson pointed out in 1963 that in non-relativistic theories these massless Nambu-Goldstone bosons could be reparametrize to give rise to massive bosons~\cite{PhysRev.130.439}. All that was left was to show that this could be done in relativistic theories as well, and that was accomplished independently by Higgs~\cite{Higgs:1964ia}~\cite{Higgs:1964pj}~\cite{Higgs:1966ev}, Englert and Brout~\cite{Englert:1964et}, and Guralnik Hagen and Kibble~\cite{Guralnik:1964eu} in 1964. In 2013 Peter Higgs and Francois Englert shared the Nobel prize ``for the theoretical discovery of a mechanism that contributes to our understanding of the origin of mass of subatomic particles, and which recently was confirmed through the discovery of the predicted fundamental particle, by the \atlas and CMS experiments at CERN's Large Hadron Collider''. 


\section{Top Quark Physics}
\label{SECTION-TOP-PHYSICS}

The \at~is of specific interest to high energy physics and in particular this thesis. It has a mass that makes it the heaviest fundamental particle that we know today, 173.2~GeV, which is about the mass of a gold atom~\cite{topmass}.

 Due to the \at's large natural width, which is defined as the probability per unit time that a particle decays, it is the only quark with an observed decay lifetime ($10^{-25}$~s) shorter than the timescale for strong interactions ($10^{-24}$~s)~\cite{TOPWidth:1993,CDF-topwidth,D0-topwidth,D0TopWidth:2010}. 

\noindent Because of this, and that the \ckm~matrix element \vtb~(\vtb~corresponds to the strength of the \at~flavor changing to bottom quark through a weak decay) is approximately equal to 1, the \at~almost always decays into a \aw~and a \ab~before it hadronizes into a jet~\cite{sgtopvtb,QFT-PS}. 

The \at~was originally discovered through pair production at the Tevatron in 1995~\cite{Top-CDF,Top-D0}. Later the production of a single \at~was discovered at the Tevatron~\cite{SGTOP-D0,SGTOP-CDF} and its width measured~\cite{CDF-topwidth,D0-topwidth,D0TopWidth:2010}. These production channels have also been investigated at the LHC~\cite{Aad:2015yem,Aad:2014fwa,Aad:2012ux,Chatrchyan:2011vp,Schilling:2012dx}.

There are three channels of single \at~physics that have been studied at the LHC. They are \tchan, \schan, and associated production (also referred to as \Wt). The largest contribution to single top at the LHC is \tchan, followed by \Wt, with \schan~being the smallest of the three. Being the largest, \tchan~was observed first and has been observed independent of the other single \athyph~production modes~\cite{TCHAN-ATLAS}. \Wt~has also been observed in \atlas~\cite{Aad:2015eto}and CMS~\cite{Chatrchyan:2014tua}. Cross-sections for the different single \athyph~processes at a proton-proton collider with $\sqrt{s} =$ 8~TeV are given in Table~\ref{TABLE-THEORY-SGTOP-XS}. The center-of-mass energy is denoted as $\sqrt{s}$ for the proton-proton collision. The LHC's high beam energies make gluons in the proton more prevalent then when compared to energetic quarks so a look into the initial states of these processes shown in Figures~\ref{FIGURE-tchan},~\ref{FIGURE-Wtchan}, and~\ref{FIGURE-schan} reveal the hierarchical nature of their \xs s. 

\SMALLFIG{tchan}{Representative Feynman diagram for the \tchan~single \at~process~\cite{SGTOP-DIAG}.}{FIGURE-tchan}
\MEDIUMFIG{Wtchan}{Representative Feynman diagrams for the \Wt~single \at~process~\cite{SGTOP-DIAG}.}{FIGURE-Wtchan}
\SMALLFIG{schan}{Representative Feynman diagram for the \schan~single \at~process~\cite{SGTOP-DIAG}.}{FIGURE-schan}

The \tchan~process has an initial state of an energetic gluon as well as a light quark, \Wt~has an initial state of an energetic gluon as well as an energetic \ab~(which will be harder to get from a proton when compared to a light quark which is naturally in a proton), and \schan~has an energetic antiquark in its initial state making it difficult to produce at the LHC. While \schan~has a comparatively small \xs~at the LHC it was not so disfavored at the Tevatron because the Tevatron was a proton anti-proton collider, making  energetic anti-quarks more prevalent. 

\begin{table}[!h!tbp] 
\begin{center}
\begin{tabular}{|l|r|}
\hline
\tchan &   216.99 +9.04 -7.71 pb\\
\hline
\Wt  &  84.4 +5.00 -6.80 pb \\
\hline
\schan &  10.32  +0.40 -0.36  pb \\
\hline
\end{tabular}
\label{TABLE-THEORY-SGTOP-XS}
\caption{The \xs~for different modes of single \athyph~production at the LHC at $\sqrt{s} = 8$ TeV~\cite{SGTOP-XS}~\cite{Kant:2014oha}.}
\end{center}
\end{table}

\section{tZ Associated Production}
\label{SECTION-TZ}

The production of a \at~in association with a \az~has not been considered at the LHC until now. The Feynman diagram for \tz~can be seen in Figure~\ref{FIGURE-tZ}. The related \ttz~\xs~has been measured, and although the uncertainty is quite high, the \at~+ \az~processes are a potentially fruitful one to investigate~\cite{ATLAS-CONF-2016-003}. The rate of the \tz~process suggests that it should be visible in the 8 TeV data set as seen in Figure~\ref{FIGURE-TZRATES} which shows NLO \xs s for the processes shown at various energies. The \tz~signature investigated includes three charged leptons, missing transverse energy, and two jets, one of which may be identified as a \ab~\cite{Campbell:2013yla}. 

Several histograms can be seen in Figures~\ref{TruthJetFig} and~\ref{TruthObjFig} which show simulations of particles before any detector interaction or decays. Some notable features of \tz~are the disparity between the $\eta$ ($\eta$ is defined in Section~\ref{SECTION-ATLAS-DET}) of the light jet vs. \ab, the higher transverse momentum (\PT) of the light jet compared to the \ab, the similarity in \PT~of leptons from the \az~and \aw~, and the \PT~of the neutrino which will manifest as \met. The variable \met~is discussed in more detail in Section~\ref{SECTION-OBJ-MET}.

\LARGEFIG{tZ}{Representative Feynman diagram for the \tz~associated production decaying to three leptons via a \az~and a \aw~\cite{JAXO}.}{FIGURE-tZ}

\VLARGEFIG{TZRATES}{Top-Quark pair and single \at~\xs s with and without accompanying \az~\cite{Campbell:2013yla}.}{FIGURE-TZRATES}

\QUAFIG{LightJetPt}{LightJetEta}{bQuarkPt}{bQuarkEta}{Information drawn from simulation of $tZ$. Light quark \PT~and $\eta$ as well as \abhyph~\PT~and $\eta$. This simulation is described in detail in Section~\ref{SECTION-MC-SIG} with added simulation steps taken for a more complete analysis. }{TruthJetFig}

\SIXFIG{ZLepPt}{WLepPt}{ZbosonPt}{WbosonPt}{TopQuarkPt}{NuPt}{Information drawn from simulation of $tZ$. The \PT~of other objects in \tz~including the lepton from the decay of the \az~and the lepton from the decay of the \aw. This simulation is described in detail in Section~\ref{SECTION-MC-SIG} with added simulation steps taken for a more complete analysis.}{TruthObjFig}


Standard model \tz~is important to measure because it is able to probe the coupling of the \at~with a \az~\cite{Campbell:2013yla}. Standard model \tz~is also a background to several SM processes and Beyond the SM processes. Anomalous \tz~couplings are one model that are of interest~\cite{Dror:2015nkp}. Monotop-quark production is one of these involving a \at~and large missing transverse energy coming from theorized dark mater candidates. Single \at~production in association with a Higgs boson is important to look for to probe the coupling of a Higgs boson to the \at. One can also consider \tz~as a background to Flavor Changing Neutral Current (FCNC) decays from \TTB~where one of the \at s decays to a \az~and a light quark which would enhance the cross section for this analysis. 

For this analysis a cut and count method is used. By examining the kinematic properties of the particles, as we have begun to do in Figures~\ref{TruthJetFig} and~\ref{TruthObjFig}, regions of phase space can be created to isolate backgrounds to ensure proper data modeling through simulation as well as isolating the \tz~signal to improve sensitivity for a statistical analysis. 


%\section{\tz~Beyond The Standard Model}
%\label{SECTION-THEORY-BSM}



%\section{Beyond The Standard Model}
%\label{SECTION-THEORY-BSM}

%The Standard Model, through all its successes,does not explain all phenomena observed. Perhaps the oldest phenomena not explained by the SM is gravity. The best theory of gravity to date is general relativity, which is incompatible with the Standard Model. More recently dark matter and dark energy have been found to comprise approximately 95\% of all matter in the universe. We currently do not know what dark matter is, but there is a popular opinion that it takes the form of a gravity only interacting particle. This is the Weakly Interacting Massive Particle (WIMP) theory. The neutrino sector also holds mysteries. According to the SM neutrinos are massless, but the observation of neutrino oscillation confirms that neutrinos do have mass. In the SM neutrino masses can be added, but they must be extremely small and its not clear if their masses come from the Higgs mechanism as the other particles do.~\cite{2009APS..HAW.KD010C}

%Another confusing observation is that our universe has a lot of matter, but not so much antimatter. The SM predicts equal parts matter and antimatter upon creation, and has no sufficient explanation concerning the lack of antimatter observed. The final phenomena i will discuss here is the top-quark anti-top-quark forward backward asymmetry (denoted $A_{FB}$). D0 and CDF at the Tevatron noticed that $A_{FB}$ was dramatically higher than the SM predicted value.~\cite{Aaltonen:2008hc}~\cite{Berger:2011ua}~\cite{Baumgart:2013yra}. The LHC is a proton-proton collider as opposed to the Tevatron's proton-antiproton collider, meaning this could not be studied further at the LHC. 

%There are several theories to account for the shortcomings of the SM as well as to extend it to cover other hypothetical particles. Some of these include SUSY, technicolor, Kaluza-Klien theory, string theory, M-theory, extra dimensions, and modified gravity. Each of these attempt to move one step closer to a Theory Of Everything(TOE) that will describe every aspect of the composition of the universe.  
%(more here)

\chapter{Theory}
\label{SECTION-THEORY}


\section{The Standard Model}
\label{SECTION-THEORY-SM}


\subsection{Problems with the Standard Model}
\label{SECTION-THEORY-SM-PROBLEMS}


\section{Beyond The Standard Model}
\label{SECTION-THEORY-BSM}

\chapter{CERN, the LHC, and \atlas~}
\label{SECTION-EXPERIMENT}

When one tugs at a single thing in nature, he finds it attached to the rest of the world -John Muir. 

\vspace{5mm} %5mm vertical space

In 1954 the Conseil Europ�en pour la Recherche Nucl�aire (CERN) formed a nuclear physics laboratory just outside of Geneva, Switzerland. CERN has since delivered on their promise to give us dozens of experiments that study everything from meteorology to biology. Some of the labs accomplishments include: the discovery of the \aw~\cite{Arnison:1983rp} and \az~\cite{Arnison:1983mk}; the determination of the number of light neutrino families~\cite{neutrinofamilies}; the creation of the world wide web~\cite{www}; the creation, isolation, and stabilization of anti-hydrogen for up to 15~minutes~\cite{Variola:818451}; and the discovery of the Higgs boson~\cite{HiggsATLAS,HiggsCMS}. 

Over the past few decades CERN has focused on accelerator physics, housing the Large Electron-Positron Collider (LEP)~\cite{Myers:226776} which ran from 1989 until 2000. LEP was then replaced with the Large Hadron Collider (LHC)~\cite{O'Luanaigh:1998498} starting operations in 2009 after a faulty start in 2008 due to a failure in an electrical connection leading to a rupture of the liquid helium enclosure of one of the superconducting magnets. The LHC and LEP are often thought of hand in hand because they both used the same 27~km tunnel.


\section{The Accelerator Chain}
\label{SECTION-ACCELERATORCHAIN}

The LHC is capable of colliding protons as well as heavy ions, although we focus on the proton accelerator chain shown in Figure~\ref{FIGURE-LHCchain}. The protons used in the LHC start from a hydrogen bottle where a magnetic field strips the electrons from $H_2$ and the resulting protons are sent through linear accelerator 3 (Linac3). Linac3 uses radio-frequency cavities that charge cylindrical conductors which are alternately positively and negatively charged. The conductors directly behind the protons are positively charged while the conductors in front of the protons are negatively charged, with both working to accelerate the protons. Once the protons are through Linac3 they will be bunched with 100~ms bunch spacing and will be up to 50~MeV in energy~\cite{Linac2}. From here they are sent through the 157~m circumference Proton Synchrotron Booster which accelerate the protons to an energy of 1.4~GeV in only 530~ms~\cite{PSB}. From there the protons go to the 628~m circumference Proton Synchrotron (PS) for tighter bunching of 25~ns, and are accelerated to 25~GeV~\cite{PS}. The final step before the LHC is the Super Proton Synchrotron (SPS) which is 7~km in circumference. The SPS can accelerate protons to 450~GeV in 4.3~seconds~\cite{SPS}. The SPS is notable for the 1984 Nobel prize winning discovery of the \aw~and \az~\cite{nobelWZ}.

\VLARGEFIG{LHCchain}{Diagram of the accelerator complex for protons to get to the LHC~\cite{LHCchain}.}{FIGURE-LHCchain}

Finally the protons make it to the LHC to be ramped up to the desired energy for collision. A segment of the LHC can be seen in Figure~\ref{FIGURE-LHC-Tunnel} which shows the housing for the magnets with the beam pipe located inside.

\VLARGEFIG{LHC-Tunnel}{A Segment of the LHC beampipe~\cite{LHC-Tunnel}.}{FIGURE-LHC-Tunnel}


\section{The Large Hadron Collider}
\label{SECTION-LHC}

It takes a few minutes to fill each LHC ring (one in each direction) forming the beams with thousands of bunches of protons which get accelerated together. After a 20~minute wait time after injection to stabilize and tighten the beams they are accelerated over half an hour to get up to full energy. In total it takes between 5 and 20~seconds to get the protons from Linac3 to the LHC, then a little less than an hour to get them up to energy. Once set up they can be stored for collisions for around 10~hours. The lifetime of the usable beam is limited mostly by protons in the beam exchanging momentum between the transverse and longitudinal directions. This is known as the Touschek effect~\cite{PhysRevLett.10.407}. Particles are lost from the beam if their longitudinal momentum deviation is great enough for them to escape the RF bucket (the longitudinal space that defines bunches) or the momentum aperture (the transverse space that defines how large a bunch can be in the transverse plane). After approximately 10~hours of beam collisions the beam is exhausted and is dumped and the injection process is repeated~\cite{LHC-TDR}. 

Given that the necessary conditions for the discovery of new physics were so extreme, the LHC was designed with unprecedented capabilities. While most people think of the LHC as the highest energy collider in the world, which it is, there are more considerations when building an accelerator. In order to discover rare processes we consider instantaneous luminosity in order to collect as many interesting events as can be produced as quickly as possible. Peak \atlas~online luminosity is around $5*10^{33}$~cm$^{-2}~s^{-1}$ (as seen in Figure~\ref{FIGURE-LUMI1}) which is around 20 times the peak Tevatron luminosity~\cite{TevatronLumi,LUMIPLOTS}. A greater instantaneous luminosity leads to a greater integrated luminosity, which is a measure of how much data has been collected over time, as seen in Figure~\ref{FIGURE-LUMI2} for previous the 7~TeV and 8~TeV run (Run 1) and Figure~\ref{FIGURE-LUMI3} for the 13~TeV run (run 2). As run 2 went on instantaneous luminosity was increased to maximize data collection showing the dramatic increase in data collection in August and September~\cite{ATLAS-LUMI}. The generic term luminosity will usually refer to integrated luminosity in this thesis unless otherwise stated. 

\VLARGEFIG{peakLumiByFill}{Peak instantaneous luminosity over time~\cite{ATLAS-LUMI}.}{FIGURE-LUMI1}

~\DBLFIG{LUMI-INTEG}{Total LHC delivered integrated luminosity over time for run 1~\cite{ATLAS-LUMI}}{FIGURE-LUMI2}{intlumivstime2015DQ}{Total LHC delivered integrated luminosity over time for run 2~\cite{ATLAS-LUMI}}{FIGURE-LUMI3}

%\VLARGEFIG{intlumivstime2015DQ}{Total LHC delivered integrated luminosity over time for run 2~\cite{ATLAS-LUMI}}{FIGURE-LUMI3}

In order to keep the beams on track and together, the LHC has 1232 dipole magnets to steer the beam and 392 quadrupole magnets for focusing and a total of around 9600 superconducting magnets. The beams are segmented into 2808 buckets which can be filled with bunches of protons or not. The LHC was designed to deliver bunches that are spaced so that the resulting collisions are 25~ns apart (corresponding to approximately 10 meters between bunches). In 2015 the LHC operated at 50 ns bunch spacing (leaving every other bucket empty) to help with pile up. Pile up is when two separate proton proton collisions are read in at the same time and can come in two forms. The first is out-of-time pile up and refers to two different bunch crossings interacting with the detector more quickly than the detectors response time. Running with 50ns bunch spacing helps with out-of-time pile up while running with 25ns bunch spacing helps with the other type of pile up, in-time pile up. In-time pile up is when two parton collisions happen within the same bunch crossing and both interact with the detector at the same time. Our data collection techniques are designed around some degree of pileup. Raising the pileup allows us to collect more data, potentially at the cost of data quality if it is not carefully monitored. With this in mind pile up was increased from the 7~TeV run with an average of 9.1 interactions per bunch crossing to the 8~TeV run with 20.7 interactions per bunch crossing as seen in Figure~\ref{FIGURE-NUM-INTERACTIONS}. In run 2 a more conservative 13.7 interactions per bunch crossing was used as seen in Figure~\ref{FIGURE-NUM-INTERACTIONS13}. Pile up is an important consideration in triggering and is discussed in this capacity in chapter~\ref{SECTION-TRIGGERS}~\cite{LHC-TDR}. 

~\DBLFIG{NUM-INTERACTIONS}{Number of interactions per bunch crossing for 7 and 8~TeV~\cite{ATLAS-LUMI}.}{FIGURE-NUM-INTERACTIONS}{mu_2015}{Number of interactions per bunch crossing for 13~TeV~\ref{FIGURE-NUM-INTERACTIONS}.}{FIGURE-NUM-INTERACTIONS13}

%~\VLARGEFIG{mu_2015}{Number of interactions per crossing is a measure of in-time pile up. This figure is for run 2 which is in comparison to Figure~\ref{FIGURE-NUM-INTERACTIONS} which is for run 1~\cite{ATLAS-LUMI}.}{FIGURE-NUM-INTERACTIONS13}

 With an accelerator of this magnitude and a diversity of possible research topics, investigation through multiple experiments is merited. CMS~\cite{CMS} and \atlas~\cite{ATLAS-TDR} are the largest general purpose detectors designed to search for the broadest range of possible new physics models and precision measurements. MoEDAL~\cite{MoEDAL} searches for magnetic monopoles. TOTEM~\cite{Berardi:2004ku} and LHCf~\cite{Adriani:2006jd} are looking for forward particles and are positioned near CMS and \atlas, respectively. ALICE~\cite{ALICE} was specially designed to study heavy ion collisions at the LHC to search for a state of matter known as quark-gluon plasma. LHCb~\cite{LHCb} is an asymmetric detector studying the effects of matter antimatter asymmetry in proton-proton collisions. 

With the LHC at such high energies and luminosities, the detectors had to be designed to be fast, radiation hard, and finely segmented all while maintaining a sensible budget. 


\section{\atlas~}
\label{SECTION-ATLAS-DET}


A Large Toroidal LHC AparatuS (also known as \atlas~or the \atlas~detector) is among the largest and most complex particle detectors in the world and can be seen in Figure~\ref{FIGURE-ATLAS-OPEN} with its namesake toroidal magnets in full view before much of the detector was added. A schematic view can be seen in Figure~\ref{FIGURE-ATLAS} It utilizes a multilayer design which has become ubiquitous in high energy physics. With this multilayer design comes a coordinate system that is vital to the design and use of the detector. There is a Cartesian coordinate system superimposed in \atlas~with the $\hat{y}$ coordinate running vertically to the surface, the $\hat{x}$ coordinate running toward the center of the LHC ring, and the $\hat{z}$ coordinate running the length of \atlas~pointing in the counter clockwise direction around the LHC ring when viewed from above. There is also a spherical coordinate system defined with $\phi$ running around the detector sweeping from the $\hat{x}$ axis toward the $\hat{y}$ axis while $\theta$ runs away from the $\hat{z}$ axis.While useful for construction and planning purposes, these variables are not as useful for analysis. A Lorentz invariant variable is desirable so particle properties in the detector can be measured in any reference frame. One is rapidity which is defined through energy ($E$) and momentum ($\vec{p}$) as

\begin{equation}
y = \frac{1}{2}ln\left(\frac{E+p_z}{E-p_z}\right)
\end{equation}

\noindent which has the unfortunate property of being dependent on the particle's mass. Another is the widely used pseudorapidity, defined as 

\begin{equation}
 \eta = \frac{1}{2}ln\left(\frac{\left|\vec{p}\right|+p_z}{\left|\vec{p}\right|-p_z}\right)
\end{equation}

\noindent which can be rewritten in terms of detector geometry variables as 

\begin{equation}
 \eta = -ln\left(tan\left(\frac{\theta}{2}\right)\right)
\end{equation}

\noindent where $\eta=\infty$ corresponds to the beamline. $\eta$ is Lorentz invariant as long as $m << E$ which is true in the low mass regime. In this regime pseudorapidity approximates rapidity. Pseudorapidity, therefore, has both the properties of describing detector geometry and describing particles in the detector that have boosts along the $\hat{z}$ axis~\cite{ATLAS-EXP}. 

\VLARGEFIG{ATLAS-OPEN}{\atlas~with its namesake toroidal magnets prominently visible~\cite{ATLAS-OPEN}.}{FIGURE-ATLAS-OPEN}

\atlas~can be segmented into several parts: the inner detector, the calorimeters, the muon spectrometer, and the magnets. Overall these systems are designed to work together to give measurements of particle energies as well as particle identification as diagrammed in Figure~\ref{FIGURE-IDWEDGE}. An electron can be identified by tracks in the inner detector and a shower in the electromagnetic calorimeter, and it is distinguished from the photon which has no tracks in the inner detector. Jets get stopped in the hadronic instead of the electromagnetic calorimeter and muons will go all the way through the detector leaving hits in all detector elements. Many particles like the \az~and \at~decay before reaching the detector. These objects must be reconstructed from their decay products. Neutrinos can be difficult to reconstruct because they go through the entire detector without interacting at all. The object reconstruction is described in chapter~\ref{SECTION-OBJ} but for now we can take a deeper look into the subsystems of \atlas.

\VLARGEFIG{Atlas}{Cutaway diagram of \atlas~\cite{Figure-Atlas}.}{FIGURE-ATLAS}

\VLARGEFIG{IDWEDGE}{A figure diagramming how particle identification can be achieved using multiple layers of the detector~\cite{Pequenao:1505342}.}{FIGURE-IDWEDGE}

\subsection{Magnet System}
\label{SECTION-ATLAS-MAGNETS}

\atlas~has a magnet system designed to assist in particle identification by curving the path of charged particles through the detector systems. There are three parts to the magnet systems; the solenoidal magnet around the inner detector, the barrel toroids, and the endcap toroids. A schematic diagram of the layouts of these magnets can be seen in Figure~\ref{FIGURE-MAGNETS}. The magnetic field can be seen in Figure~\ref{FIGURE-BFIELD} which shows the inhomogeneous nature of the toroidal fields of the main toroids (positioned at approximately $4.5 < R < 10$) and the endcap toroids (positioned at approximately $R < 4.5$ and $8 < z < 12$) and the comparative constant nature of the solenoidal field (positioned at approximately $R < 1.5 $and$ z < 3$).

\MEDIUMFIG{Magnets}{Illustration of the \atlas~magnet system, showing the barrel solenoid, barrel toroid, and endcap toroid coils~\cite{ATLAS-EXP}.}{FIGURE-MAGNETS}

\VLARGEFIG{ATLAS-BField}{A mapping of the magnetic fields in \atlas~\cite{ASalzburger}.}{FIGURE-BFIELD}

The solenoidal magnet provides a nearly uniform 2~T magnetic field for the inner detector. The Solenoid is designed to be as thin as possible to minimize the interaction of the particles from physics events to aid in calorimetry. Any interaction in the solenoid will begin the showering process which means energy from the interacting particle will be lost and will have to be accounted for in calorimetry. 

The eight barrel toroids are visible in Figure~\ref{FIGURE-ATLAS-OPEN} and run $\eta < 1.6$ providing a peak magnetic field of 3.9~T around the muon spectrometers and are highly irregular as seen in Figure~\ref{FIGURE-BFIELD}. Because of this irregularity the magnetic field must be mapped carefully for accurate muon tracking. 

The endcap toroids complete the \atlas~magnet systems providing a peak magnetic field of 4.1T for the forward detectors at $1.4 < \eta < 2.7$. The endcap magnets are offset from the barrel toroids by $\frac{1}{16}$ of a turn so that they bisect the angle (in $\phi$) between the barrel toroids seen in Figure~\ref{FIGURE-BFIELD}.

These magnets are crucial for particle identification and momentum measurements of charged particles and are strategically placed around the detector subsystems described hereafter~\cite{MAGNET}.


\subsection{Inner Detector}
\label{SECTION-ATLAS-ID}

The inner detector provides tracking information for tracked particles close to the beamline. Track reconstruction consists of finding sets of measurements coming from one charged particle and building the associated trajectory through the detector.
 In order to achieve this the inner detector was designed to be as hermetic as possible with high granularity as close to the beamline as possible~\cite{Capeans:1291633,IOPATLASBLAYERPROJ}.

The inner detector has three parts and can be seen in its entirety in Figure~\ref{FIGURE-ATLAS-ID}. Those parts are the pixel detector, the SemiConducting Tracker (SCT), and the Transition Radiation Tracker (TRT)~\cite{INNERDET}. In May of 2014 another layer was inserted inside the ID known as the Insertable B Layer (IBL) for improved tracking closer to the interaction point. 

\VLARGEFIG{InnerDetector}{Cutaway diagram of the \atlas~inner detector~\cite{Figure-InnerDetector}.}{FIGURE-ATLAS-ID}

The pixel detector and the SCT work on ionization of silicon which is separated into positive and negative charges which can be separated by an electric field into read out electronics. The read out can be either binary or non-binary. A binary read out registers a hit over some threshold or registers no hit. The non-binary readout registers the charge collected over time over some threshold and reports the amount of charge collected to assist in track reconstruction. Non-binary readouts give better tracks, but are more expensive and the read out electronics take up more space in the valuable real estate near the beamline. The pixel detector and SCT cover $\eta < 2.5$~\cite{PIXEL-DET}.

The SCT strip detectors use a stereo-angle technique to get position measurements where concentric layers are constructed so a small angle of $40$~mRad. Without an angle between layers of the strip detector only $\phi$ could be read out, but with the angle $\eta$ can be read out as well. By using strips of silicon a lot of money and space can be saved in comparison to pixel detectors, mostly in read out electronics, and there won't be as much supporting material in the way for the calorimeters~\cite{SCT_Barrel,SCT_Endcap}.

The TRT works on the principle of transition radiation. When a high energy particle goes between media with differing dielectric constants, the result will be the emission of radiation or as Jackson puts it ``the fields must reorganize themselves as the particle approaches and passes through the interface. In this process, some pieces of the fields are shaken off as transition radiation''~\cite{jackson91}. The TRT uses this by filling tubes with a gas of $Xe$, $CO_2$, and $O_2$ which is ionized by charged particles passing through. All charged particles will interact with the TRT giving tracking information, but the TRT has two separate thresholds for readout. The first threshold tracks charged particles while the second higher threshold determines if transition radiation is being detected. This occurs when a particle which is traveling faster than the speed of light in the medium it is entering, creating a sort of shock wave of radiation known as transition radiation. Because the electron participates in transition radiation more strongly than the pion we can obtain good pion rejection while maintaining electron reconstruction efficiency~\cite{TRT,TRT_Barrel,TRT_Endcap}.

The largest source of track reconstruction inefficiency is hadronic interaction. When a hadron interacts with the nucleus of the detector material it is usually destroyed, and creates a hadronic and electromagnetic shower. The primary track stops when the original particle undergoes this process and a series of other tracks begin, but unfortunately the track can not be reconstructed. Another problem is electron bremsstrahlung. When a charged particle passes near the nucleus of the detector material it will radiate, loosing energy. This affects electrons more than other particles because the energy loss as it traverses the detector is proportional to energy over mass squared so the light electron will undergo bremsstrahlung more strongly than its heavier counterparts. Another consideration in tracking is multiple scattering which can cause a random change in direction not caused by curvature in the magnetic field. The method of detection can actually be a problem as well because when the particle ionizes the atoms of the detector it looses energy~\cite{ASalzburger}. Even with these effects the track reconstruction efficiency is quite good varying from 90\% in the central regions to 80\% in the forward regions shown in Figure~\ref{FIGURE-ATLAS-IDEFF}~\cite{ATL-PHYS-PUB-2015-051}. 


\LARGEFIG{IDeff}{Track reconstruction efficiencies for the ID in ATLAS~\cite{ATL-PHYS-PUB-2015-051}.}{FIGURE-ATLAS-IDEFF}


\subsection{Calorimeters}
\label{SECTION-ATLAS-CALO}

There are two calorimeter systems for detecting electromagnetically interacting particles and strongly interacting particles referred to as the electromagnetic calorimeter, and hadronic calorimeter respectively, as seen in Figure~\ref{FIGURE-ATLAS-CALO}. One high energy particle from the hard interaction of an event will shower into many particles creating a wave of energy deposition in the calorimeters. This process is particularly useful for neutral particles that can not be tracked in the inner detector, but is also useful for a more complete picture of a particular object. 

\VLARGEFIG{Calorimeters}{Cutaway diagram of the \atlas~calorimeter systems~\cite{Figure-Calo}.}{FIGURE-ATLAS-CALO}

Both the electromagnetic and hadronic calorimeters are sensitive to several different types of interactions. The first is radiative interactions where the incoming particle will scatter off a constituent atom creating a Rutherford scattering. They can also Compton scatter off atomic electrons, ionize the atoms of the detector, or have other similar low momentum transfers. After this, particle energies fall to when they are absorbed by atomic interactions and the number of particles in the shower begins to fall. It is often noted that muons and protons are minimally ionizing in the electromagnetic calorimeters. This is because the radiative interactions which begin the particle cascade fall by m$^2$, so their large mass relative to the electron gets them through the electromagnetic calorimeter~\cite{Wigmans}. 

The Liquid Argon (LAr) calorimeters use sampling calorimetry. Because the primary interactions are radiative in nature, it is desirable to have high atomic number in the calorimeter material. The principal of sampling calorimetry is to have two materials in the calorimeter; one to facilitate the radiative interaction and begin the showering process, and another to detect the lower energy interactions that provide the signal that is read out. The LAr calorimeter is accordion shaped lead coated with stainless steel for radiative interactions with liquid argon to collect the resulting shower. The accordion shape is useful to increase the path length of particles in the material, thereby increasing the probability of an interaction and lowering the total amount of material needed. Wires in the liquid argon are held at high voltage to attract the ionized particles and read out the resulting current~\cite{Wigmans}. 

Tile calorimeters (TileCal) are placed outside the LAr calorimeters. They work on the same principles as the LAr calorimeters. The sampler for the TileCal is sheet steel without the accordion shape, and the readout material is a collection of plastic scintillators that emit light when hit with the resulting shower. The plastic scintillators are coupled to optical wavelength shifting fibers to redirect light to photomultiplier tubes. The light emitted from the scintillating plastic is typically in the UV range, and is shifted into the blue or green visible wavelengths to help limit attenuation while propagating~\cite{TILE}~\cite{Proudfoot:2006tr}. 

The geometry of the calorimeter systems are complex, but a simplified version is given here and can be seen in Figure~\ref{FIGURE-ATLAS-CALO}. The barrel region of the detector (with $\eta < 1.475$) has both LAr calorimetry and TileCal~\cite{EMCAL_Barrel}. The endcap calorimeters cover $1.375 < \eta < 3.2$ ~\cite{EMCAL_Endcap}. There is also a LAr forward calorimeter (FCal) that covers the extremely forward region $3.1 < \eta < 4.9$ with a copper absorber for the electromagnetic portion and a tungsten absorber for the hadronic part. There is also an inner presampler to catch how intensively radiative interactions from interactions with the inner detector took place~\cite{EMCAL_Presampler}.

This gives an overall energy resolution of the electromagnetic calorimeter of less than 1\%~\cite{Aharrouche:2006nf}. The energy resolution of the hadronic calorimeter is significantly worse, with calibrations from dijet events showing variations of 2-4\%~\cite{ATLAS-CONF-2015-017} but in practice this evaluation is a large source of systematic uncertainties to analysis. Energy resolution effects are described in more detail in Section~\ref{SECTION-systematics}. 


\subsection{Muon Systems}
\label{SECTION-ATLAS-MUON}

Muons provide an interesting challenge because they do not interact strongly with the electromagnetic or hadronic calorimeters and pass through the detector. The muon systems have four layers as shown in Figure~\ref{FIGURE-ATLAS-MUON}; they are the Monitored Drift Tubes (MDT), the Cathode Strip Chambers (CSC), the Resistive Plate Chambers (RPC) and the Thin Gap Chambers (TGC). 

\VLARGEFIG{Muon}{Cutaway diagram of the \atlas~ muon spectrometer and toroid magnet systems~\cite{Figure-Muon}.}{FIGURE-ATLAS-MUON}

The MDT provide most of the precision muon tracking in \atlas. The MDT have a barrel and endcap section. The barrel section covers $\eta < 1.0$ while the endcaps cover $1.0 < \eta < 2.7$. The MDT works on ionizing gases and is composed of straw tubes filled with gas which is composed of argon, nitrogen, and methane. The muon traverses the straw tube and ionizes the gas and the resulting charged particles are collected on a wire in the center of the tube which is held at high voltage. The amount of time it takes from the first current to reach the center wire until the last current reaches the center wire tells us how far away the muon came to the center of the tube. With this information we can make measurements on the path of the muon. 

The CSC are multiwire proportional chambers used in high radiation zones around \atlas~at $2.0 < \eta < 2.7$. They work on the same principles as the MDT, but with a different gas mixture. 

The RPC are designed to supplement the MDT in the barrel region $\eta < 1.05$ and has a fairly simple design. Each RPC chamber is two resistive plates held at 8900~V across a 2~mm gap. An incoming charged particle ionizes the gas and cause a localized discharge of the capacitor. The location of this discharge can then be read out. This method does not give very good spacial resolution (approximately 1~cm) but is quite fast with a timing uncertainty of 1.5~ns. Due to this excellent timing resolution, the RPC are utilized primarily by the Level 1 trigger described in Section~\ref{SECTION-TRIGGERS-L1}. 

The TGC (as seen in Figure~\ref{FIGURE-ATLAS-TGC-wheel}) are also designed to supplement the MDT but in the forward region of the detector $1.05 < \eta < 2.4$. It uses the same technology as the RPC and consequently is used for triggering as well. The TGC have a different gas mixture in order to decrease spacial resolution for bunch identification (down to 9~mm) but suffers in timing response (7~ns) which is still fast enough to be used by the level~1 trigger system~\cite{Green:1221848,ATLAS-TDR,DETECTORS}.
 
\VLARGEFIG{ATLAS-TGC-wheel}{The TGC wheel~\cite{ATLAS-TGC-wheel}.}{FIGURE-ATLAS-TGC-wheel}

Overall this system gives excellent momentum resolutions of less than 0.1\% for momenta used in this analysis~\cite{ATLAS-TDR}. This is possible thanks to the fine segmentation and position resolution of the layers of the muon spectrometers containing over 1 million readout channels~\cite{Airapetian:391176}. 

\chapter{Object Definitions}
\label{SECTION-OBJ}


\section{Electron definition}
\label{SECTION-OBJ-EL}


\section{Muon definition}
\label{SECTION-OBJ-MU}


\section{Jet definition}
\label{SECTION-OBJ-JET}

\begin{equation}
\label{EQ-OBJ-ANTIKT}
d_{ij} = min(p^{-2}_{T,i},p^{-2}_{T,j})\frac{\Delta\eta^{2}_{ij}+\Delta\phi^{2}_{ij}}{R^{2}}
\end{equation}

\begin{equation}
\label{EQ-OBJ-ANTIKTBEAM}
d_{i} = p^{-2}_{T}
\end{equation}


\subsection{Jet b-tagging}
\label{SECTION-OBJ-JET-BTAG}


\section{Missing transverse energy definition}
\label{SECTION-OBJ-MET}

\chapter{Background Simulation}
\label{SECTION-BG}

In order to devise and optimize the analysis strategy both signal and background events are modeled. Most of these events are simulated using Monte Carlo (MC) techniques where each event is generated, showered and hadronized, run through a detector simulation, and reconstructed using a variety of software packages. The exception to this are the W+jets and multijet backgrounds which are modeled using either partially or wholely data driven techniques as described in Sections~\ref{SECTION-BG-DD-WJETS} and~\ref{SECTION-BG-DD-QCD} respectively.

\section{Monte Carlo simulation}
\label{SECTION-BG-MC}
The MC simulation of events is broken down into four stages. Event generation simulates the initial physics event and its decay. Showering and hadronization simulate the formation of jets from any bare quarks or gluons in the generated events. Detector simulation models the interaction of the physics event with the ATLAS detector using  a \GEANT~\cite{GEANT4} simulation of the ATLAS detector, resulting in a detector response for the event. The final step is event reconstruction where the same algorithms used to analyze data events are applied to the simulated detector responses to build analysis objects.

There are a plethora of software packages available to perform MC simulation of events, and these packages make a variety of different assumptions and simplifications of the physics they are simulating. This leads to the situation that different packages are able to more accurately simulate different physics processes and careful consideration and investigation is necessary to ensure the simulations used in the analysis are as accurate as possible. Since \Wprimechan\ is a single top process it was extremely useful to consult the extensive work already done comparing the different MC generator and showering programs for each process by the ATLAS single top group. 

For all processes except \Wprime\ the current group recomendation has been used. For the \Wprime\ signal processes the \MADGRAPH~\cite{MADGRAPH} generator has been used due to its ease of implementation and handling of spin correlations of decays. The \Wprime\ events were showered with \Pythia~\cite{PYTHIA} similar to most of the background signals. Table~\ref{TABLE-BG-MC} shows which programs were chosen to simulate each sample's generation and showering~\cite{MADGRAPH}~\cite{PYTHIA}~\cite{POWHEG}~\cite{HERWIG}~\cite{ALPGEN}~\cite{ACERMC}. With the exception of the data driven methods described in Section~\ref{SECTION-BG-DD}, the background and signal samples are normalized using their theoretical cross-sections ($\sigma$), the total luminosity (\Lumi), and a k-factor (k) which estimates the higher order corrections to the cross-section. Equation~\ref{EQ-BG-NORM} gives the normalized number of events expected for each sample (N). The cross-section and k-factor values for the signal and background samples are  given in Table~\ref{TABLE-BG-SIGNAL} and Table~\ref{TABLE-BG-MC} respectively.

\begin{equation}
\label{EQ-BG-NORM}
N = k\sigma\Lumi
\end{equation}


\begin{table}
\begin{center}
\begin{tabular}{|l|cc|cc|}
\hline
\Wprime\ Mass [GeV] & \WprimeL\ $\sigma$ [pb] & \WprimeL\ k & \WprimeR\ $\sigma$ [pb] & \WprimeR\ k \\[1mm]
\hline
500 & 12.333 & 1.3684 & 17.510 & 1.2906 \\[1mm]
750 & 2.7223 & 1.3144 & 3.7174 & 1.2779 \\[1mm]
1000 & 0.81915 & 1.2564 & 1.0652 & 1.2796 \\[1mm]
1250 & 0.28025 & 1.2405 & 0.37278 & 1.2260 \\[1mm]
1500 & 0.10618 & 1.2202 & 0.13932 & 1.2183 \\[1mm]
1750 & 0.043693 & 1.1893 & 0.055667 & 1.2062 \\[1mm]
2000 & 0.018551 & 1.1774 & 0.023718 & 1.1740 \\[1mm]
2250 & 0.0082073 & 1.1638 & 0.010283 & 1.1669 \\[1mm]
2500 & 0.0038171 & 1.1512 & 0.0046794 & 1.1485 \\[1mm]
2750 & 0.0018512 & 1.1529 & 0.0021970 & 1.1522 \\[1mm]
3000 & 0.00095811 & 1.1687 & 0.0011035 & 1.1592 \\[1mm]
\hline
\end{tabular}
\caption{Cross-sections and k-factors for generated \Wprime\ samples.}
\label{TABLE-BG-SIGNAL}
\end{center}
\end{table}


\begin{table}[htdp]
\begin{center}
\begin{tabular}{|l|cc|rr|}
\hline
Process & $\sigma$ [pb] & k & Generator & Showering \\[1mm]
\hline 
single top s-channel & 1.6424 & 1.1067 & \POWHEG\ & \Pythia\ \\[1mm]
single top t-channel & 25.750 & 1.1042 & \AcerMC\ & \Pythia\ \\[1mm]
single top Wt-channel & 20.461 & 1.0933 & \POWHEG\ & \Pythia\ \\[1mm]
$t\bar{t}$ & 114.51 & 1.1992 & \POWHEG\ & \Pythia\ \\[1mm]
W+lf & 31994 & 1.133 & \ALPGEN\ & \Pythia\ \\[1mm]
W+c & 1126.0 & 1.52 & \ALPGEN\ & \Pythia\ \\[1mm]
W+cc & 403.44 & 1.133 & \ALPGEN\ & \Pythia\ \\[1mm]
W+bb & 133.99 & 1.133 & \ALPGEN\ & \Pythia\ \\[1mm]
Z+jets & 2804.4 & 1.229 & \ALPGEN\ & \HERWIG\ \\[1mm] %Alpgen+Pythia in v11
diboson & 17.075 & 1.7223 & \HERWIG\ & \HERWIG\ \\[1mm]
\hline
\end{tabular}
\caption{Simulated background samples with associated cross-sections, k-factors, generating programs and showering programs.}
\label{TABLE-BG-MC}
\end{center}
\end{table}


\section{Data driven estimates} 
\label{SECTION-BG-DD}
While the above method works well to simulate many background processes, it is sometimes useful to use control regions of data to estimate some backgrounds. For W+jets it is necessary to correct the overall normalization as well as the relative abundance of the simulated samples based on the flavor associated jet. Multijets has a very high rate of occurence and a very low acceptance making it very difficult to predict, so this analysis uses the matrix method to estimate this background from data.

\subsection{W+jets normalization} 
\label{SECTION-BG-DD-WJETS}
The W+jets samples in this analysis are globally normalized using the charge asymmetry method in the region $m(W')\ <\ 330\ GeV$. This region has a signal contamination $<$ 5\% for all signal mass points considered in the analysis. This method normalizes the W+jets sample in each analysis channel using the theoretical asymmetry ratio $r_{MC} = \frac{W^{+}}{W^{-}}$ to account for the observed asymmetry in data. The ratio between the observed asymmetery and the expected asymmetry is applied as a normalization factor to the entire channel, as shown in Equation~\ref{EQ-BG-DD-WJETS}.

\begin{equation}
\label{EQ-BG-DD-WJETS}
N_{W^{+}} + N_{W^{-}} = \frac{r_{MC} + 1}{r_{MC} - 1}(D^{+} - D^{-})
\end{equation}

\noindent
$N_{W^{+}}\ +\ N_{W^{-}}$ is the normalized W+jets yield and $D^{+}$ and $D^{-}$ are the number of data events with positive and negative leptons respectively. The fraction of W+jets composed by W+lf, W+c, W+cc, and W+bb is determined by simultaneously varying the fraction of the total W+jets sample each sub-channel composes and fitting the MET distribution. For this fit the W+cc and W+bb samples are merged into a single W+hf sample and so they recieve the same normalization factor. The normalization factors for each sample are given in Table~\ref{TABLE-BG-DD-WJETS}.

\begin{table}[htdp]
\begin{center}
\begin{tabular}{|l|cccc|}
\hline
Process & 2jets 1tag & 2jets 2tag & 3jets 1tag & 3jets 2tag \\[1mm]
\hline 
W+lf & 0.941462 & 1.31867 & 0.883688 & 1.96718 \\[1mm]
W+c & 0.801521 & 1.12266 & 0.752335 & 1.67477 \\[1mm]
W+cc & 1.39795 & 1.95806 & 1.31217 & 2.92102 \\[1mm]
W+bb & 1.39795 & 1.95806 & 1.31217 & 2.92102 \\[1mm]
\hline
\end{tabular}
\caption{W+jets normalization factors.}
\label{TABLE-BG-DD-WJETS}
\end{center}
\end{table}

\subsection{Multijets estimate} 
\label{SECTION-BG-DD-QCD}
The contribution of the multijet process to this analysis is estimated using the matrix method. The matrix method uses data events which have passed the event selection in Chapter~\ref{SECTION-SELECTION} except with a loose lepton which has relaxed requirements compared to the tight lepton required for the signal region. Both the loose and tight lepton are defined in Chapter~\ref{SECTION-OBJ}. For both electrons and muons

\begin{equation}
\label{EQN-BG-DD-QCD1}
N^{loose} = N^{loose}_{real} + N^{loose}_{fake}
\end{equation}
\begin{equation}
\label{EQN-BG-DD-QCD2}
N^{tight} = \epsilon_{real}N^{loose}_{real} + \epsilon_{fake}N^{loose}_{fake}
\end{equation}

\noindent
where N is the number of data events containing a lepton of the indicated type. $\epsilon_{real} = \frac{N^{tight}_{real}}{N^{loose}_{real}}$ and $\epsilon_{fake} = \frac{N^{tight}_{fake}}{N^{loose}_{fake}}$ are the conversion efficiencies for loose leptons to tight leptons. $\epsilon_{real}$ is estimated using the tag and probe method on $Z \rightarrow ll$ events, while $\epsilon_{fake}$ is estimated using a multijets enhanced data sample where the lepton isolation criteria have been removed. With the total number of events with loose and tight leptons known from the dataset, Equation~\ref{EQN-BG-DD-QCD1} and Equation~\ref{EQN-BG-DD-QCD2} can be inverted and combined with the fakes conversion efficiency to solve for $N^{tight}_{fake}$ which is the multijets estimate for the analysis.

\chapter{Event Selection}
In this section the selection criteria applied to the data and simulated events and the reasoning behind them are described. These selection cuts are chosen because they keep clean signal events while rejecting background and poorly reconstructed signal events. These cuts come from the Top Working Group, although two of our cuts, the \Zboson\ veto cut and the \MET angular correlations cut are specific to this analysis.

\section{Selecting events from data}
The data used are 7 TeV proton-proton collision data from between February 2011 and August 2011. Unprescaled single electron and muon triggers are used to choose event candidates, and the event is required to be flagged as having taken place during a period of running where the LHC had stable beams and all detectors were running without issue. These quality criteria are applied using a list of sections of runs, called a Good Runs List (GRL). These data represent \LUMI\ of integrated luminosity.

\section{Selecting dilepton events}
To select dilepton events and reject our backgrounds a chain of cuts is applied to both the data and the simulated events. The cuts applied in this analysis are:

\begin{list} {$\bullet$} {}
\item Primary vertex cut.
\item Reject events with a ``Bad'' jet.
\item Reject events that may be contaminated by the LAr hole problem.
\item Reject events with an electron overlapping a muon.
\item Reject events with two muons satisfying $\Delta\phi > 3.10$.
\item Require two selected leptons.
\item Trigger selection of an electron ($ee$ subchannel only).
\item Trigger matched to reconstructed electron ($ee$ subchannel only).
\item $\MET\ > 50\ GeV$.
\item \Zboson\ veto cut, $81\ GeV < M_{ll} < 101\ GeV$ ($ee$ and $\mu\mu$ only).
\item $\Delta\phi(l_1,\MET)+\Delta\phi(l_2,\MET) > 2.5$.

\end{list}

I will now go into more detail on each of these selection cuts and discuss the rationale behind them.

A number of event quality cuts are applied to eliminate events that have been poorly reconstructed or otherwise do not represent a good collision event. These cuts are determined by the top working group and the implementation and rationale for each is well documented~\cite{TOPCOMMONOBJECTS}. To ensure that the event is from a collision event, a cut is applied requiring that the first primary vertex in the event have at least four tracks. 

Another cut is applied to remove events that contain ``Bad'' jets. These are jets that have been determined to not be associated with a real in-time energy deposit. Because the presence of a single high \pT\ ``Bad'' jet can pollute the event kinematics, any event with a bad jet with $p_T > 10\ GeV$ is removed from the selection. 

A third cut rejects events if they may have been impacted by LAr hardware issue during running discussed in section~\ref{SECTION-DEFINEELECTRONS}. 

The fourth cut rejects events if a selected muon and a selected electron share a track. This would be an indication that the electron is erroneously reconstructed from muon energy deposits in the calorimeter from the muon, and consequently these events are discarded. 

In the $\mu\mu$ channel, there is an additional veto to remove coincident cosmics events. Cosmic muons can appear as two muons in the detector, with one track from the muon going in and another back to back track of the muon leaving. As a result, this cut rejects events with pairs of tracks muon that match up closely. Specifically, muons are required to have been reconstructed with opposite charge, both having an impact parameter greater than $0.5\ mm$, and must have $\Delta\phi > 3.1$. 

After these cuts are applied, the remaining events have no obvious reconstruction errors and originate from collisions in our detector. These events are subjected to further cuts to enhance the signal to background ratio as much as possible. This analysis is divided into three channels with differing lepton combinations. Events are selected that have two electrons ($ee$ channel), an electron and a muon ($e\mu$ channel), or two muons ($\mu\mu$ channel). Each of these channels requires that the leptons selected meet the quality criteria defined in Sections~\ref{SECTION-DEFINEELECTRONS} and~\ref{SECTION-DEFINEMUONS}. 

In the $ee$ channel it is also ensured that the {\sc EF\_E20} electron trigger fired for this event. Furthermore, this triggering object must be consistent with at least one of our selected electrons by meeting the requirement $\Delta R(electron,\ trigger\ object) < 0.15$. Due to a bug in simulating the trigger conditions for the muons in the 2010 simulated events, the same procedure cannot be repeated for muons.

We consider three regions of our analysis defined by the number of jets: 1-jet, 2-jet, and 3+jet. As the largest background, $\ttbar$, contains two jets in the final state, 1-jet events are considered the primary signal region. Since the $\ttbar$ background yield dominates in the 2+jet bin, events with more than one jet are used to constrain the uncertainty in the $\ttbar$ normalization. One distinguishing characteristic of the $\Wt$ signal is the presence of two neutrinos, hence it is required that events have $\MET\ > 50\ GeV$. This cut eliminates much of the \multijet background. 

Even after all of the previous cuts, the $ee$ and $\mu\mu$ channels suffer from large contamination from $\Zee$ and $\Zmm$ events due to their relatively large cross-sections. To reduce the impact of these channels, an additional cut is made on events with a dilepton invariant mass near the \Zboson\ boson mass, $81\ GeV < M_{ll} < 101\ GeV$. This cut is independent of whether the channel is $ee$ or $\mu\mu$, because in this energy regime the energy resolutions of reconstructed electrons and muons are similar.

A powerful cut reduces the $\Ztt$ background significantly. This cut is performed by taking the sum of the $\Delta \phi$ of both leptons with the missing transverse energy vector. The cut value is optimized to maximize background rejection while minimizing signal loss. This triangle cut is defined as:

\begin{equation}
\Delta\phi(l_1,\MET)+\Delta\phi(l_2,\MET) > 2.5.
\end{equation}

\noindent
The resulting impact on the events is shown in Figs.~\ref{FIG-ZTAUTAU} and~\ref{FIG-ZTAUTAU2}. Although there is some discrimination power in the individual distributions, when they are summed together the reason for the triangle cut becomes obvious, as we are able to eliminate many background events without losing much signal. 

\TRPFIGLEG{paper_ll1+j_LP2fb_v4_DeltaPhiLeadingLeptonMissingEt_flat}{paper_ll1+j_LP2fb_v4_DeltaPhiSubLeadingLeptonMissingEt_flat}{legend}{The impact of the triangle cut on signal and background: (a) the angle between leading lepton and $\MET$ (b) the angle between the second lepton and $\MET$. The simulated events are represented by the solid regions, while the data are represented with a black dot.}{FIG-ZTAUTAU}
\FIG{plhc_presel_triangularDeltaPhiLepMET}{The effect of the triangle \Ztt\ veto cut in two dimensions. }{FIG-ZTAUTAU2}

An event selection is applied that divides the events into three exclusive bins: dielectron, dimuon, and electron-muon. These bins are examined separately in the control regions to make sure that the backgrounds are well modeled. Plots of selected variables in these bins are shown in Appendix~\ref{APPENDIX-CONTROLREGIONS}. Examining them in the bins independently gives us a useful tool for diagnosing the cause of disagreements between the data and the simulated events. Since the kinematics of these three subchannels are similar, for the final analysis they are merged together. 

\section {Event yields}
Table~\ref{TABLE-EVENT-YIELD-PRESEL} shows the resulting yields after selection along with the simulated statistical and data-driven uncertainties. 
We expect 3003.5 events in our signal region and observe 3059. 
The data are in reasonable agreement with our background and signal estimates given the data statistical uncertainty and the simulated event yield uncertainty. 
In addition, agreement is also observed individually in the $ee$, $e\mu$, and $\mu\mu$ channels. Distributions of relevant kinematic variables are shown in Fig.~\ref{FIGURE-PRESEL-CBD1} 
for the combined channel. Similar Figs. 
are available for the three individual $ee$, $e\mu$ and $\mu\mu$ channels in 
Appendix~\ref{APPENDIX-CONTROLREGIONS}.
 
\begin{table}[htdp]
\begin{center}
   \begin{tabular}{l|cccr}
    \hline
    Process      & $ee$               & $\mu\mu$        & $e\mu$   & All combined \\
    \hline \hline 
    $Wt$  &       38.6 $\pm$ 0.8 &       65.3 $\pm$ 1.0 &      119.7 $\pm$ 1.3 &      223.6 $\pm$ 1.8\\
\hline
   \TTB\  &      438.1 $\pm$ 4.5 &      738.5 $\pm$ 5.8 &     1336.0 $\pm$ 7.8 &     2509.6 $\pm$ 10.7\\
    $WW$  &       16.7 $\pm$ 2.4 &       29.0 $\pm$ 2.9 &       55.3 $\pm$ 4.1 &      101.0 $\pm$ 5.6\\
    $WZ$  &        4.9 $\pm$ 0.7 &       13.8 $\pm$ 1.2 &        8.1 $\pm$ 0.9 &       26.8 $\pm$ 1.7\\
    $ZZ$  &        0.9 $\pm$ 0.1 &        4.5 $\pm$ 0.3 &        0.4 $\pm$ 0.1 &        5.8 $\pm$ 0.3\\
    \Zee\ (DD)&   35.7 $\pm$ 2.5 &        ---           &        ---           &       35.7 $\pm$ 2.5\\
    \Zmm\ (DD)&    ---           &      69.5  $\pm$ 3.1 &        ---           &       69.5 $\pm$ 3.1\\   
   \Ztt\  (DD)&    1.1 $\pm$ 0.6 &        5.7 $\pm$ 3.4 &        2.6 $\pm$ 1.6 &       9.4 $\pm$ 3.8\\
 Fake dilepton (DD)    &    9.0 $\pm$ 9.0 &        ---           &        6.9 $\pm$ 6.9 &       15.9 $\pm$ 15.9 \\ 
 \hline
 Total expected& 542.0 $\pm$ 10.7 &      926.3 $\pm$ 8.1 &       1529.0 $\pm$ 11.4 &   2997.3 $\pm$ 17.6 \\
 \hline
 Data Observed  &          573 $\pm$ 24  &                905 $\pm$ 30  &               1581 $\pm$ 40  &     3059 $\pm$ 55\\
    \hline\hline
   \end{tabular}
 \caption{The observed and predicted event yields in the selected dilepton sample with at least one jet and for an integrated luminosity of \LUMI. Uncertainties represent the effect of MC statistics for the MC-based estimates and the total uncertainty for the data-driven estimates.}
\label{TABLE-EVENT-YIELD-PRESEL}
\end{center}
\end{table}

\SEXFIGLEG{paper_ll1+j_LP2fb_v4NoShift_NJets_flat}{paper_ll1+j_LP2fb_v4_Jet1Pt_flat}{paper_ll1+j_LP2fb_v4_HT_AllJets_flat}{paper_ll1+j_LP2fb_v4_MET_flat}{paper_ll1+j_LP2fb_v4_LeadingLeptonPt_flat}{legend}{Histograms of the selected sample with combined $ee$, $e\mu$ and $\mu\mu$ channels. The simulated events are represented by the solid regions, while the data are represented with a black dot. (a) Jet multiplicity, (b) Leading jet \pT, (c)$H_T(jet)$, (d) \MET, (e) Leading lepton \pT.}{FIGURE-PRESEL-CBD1}


\chapter{Analysis}
\label{SECTION-ANALYSIS} 

Let's think the unthinkable, let's do the undoable. Let us prepare to grapple with the ineffable itself, and see if we may not eff it after all. -Douglas Adams

\vspace{5mm} %5mm vertical space

After the \az, the \at, the \aw~from the \at~decay, and the neutrino from the \aw~decay are reconstructed as described in Chapter~\ref{SECTION-OBJ}, they are used to help separate \tz~from the various backgrounds. The energies and momenta of each of these objects in our detector, as well as the multiplicity of the objects, are used to achieve this seperation. The decisions made in the preselection and cut flow are informed by the kinematic properties of the \tz~process. The \tz~Feynman diagram is shown in Figure~\ref{FIGURE-tZ}. 



\section{Preselection}
\label{SECTION-PRESELECTION}

One goal in setting up an analysis is to understand the background model in relation to the observed data. To accomplish this, defining characteristics of the signal region are determined in order to limit the number of \MC~samples needed. Because the signal has three leptons and a \az, cuts on the number of leptons and \azhyph~mass (for instance) are applied to limit any contribution from certain low lepton multiplicity non-\azhyph~sources such as \aw +jets and multijets. 

The following cuts are optimized by maximizing $S/\sqrt{B}$ where $S$ is the total expected signal contribution and $B$ is the total expected background contribution. This is done to improve agreement between data and the background model while maintaining as much signal statistics as possible. 

\begin{itemize}
\item Exactly 3 leptons with \PT~\textgreater~10~GeV. Exactly 3 leptons is a defining feature of this analysis. Two of the leptons come from the \az~decay, while the third comes from the \at~decay. Because these leptons are required to be electrons or muons their distributions are mirrors of each other by definition which can be seen in Figure~\ref{FIGURE-NMUON}.
\item At least one OSSF pair. Because we are concerned with processes that contain a real \az, this requirement ensures that we can always attempt to reconstruct a valid \az~candidate even if it is an event that is mis-identified as containing a \az. 
\item Leading lepton \PT~\textgreater~40~GeV. The leading lepton's threshold is higher than the second and third due to being more likely that it is the candidate that is required to pass the single lepton trigger or a candidate in the case of a multi-lepton trigger. Figure~\ref{TRIPFIG1} shows that this cut removes some background, but little signal is lost.
\item Second lepton \PT~\textgreater~20~GeV. This lepton is not required to have passed a single lepton trigger, but may have been required to pass the di-electron trigger threshold. Figure~\ref{TRIPFIG1} shows that this cut removes some background, but little signal is lost. Tightening this cut is also investigated because \zjets~and \TTB~peak at a lower momentum than the signal, but in the interest of maintaining statistics a lower \PT~threshold is chosen. 
\item Third lepton \PT~\textgreater~10~GeV. The \PT~of this lepton is significantly lower than the rest. 10 GeV is chosen due to the thresholds that define how electrons are reconstructed. Figure~\ref{TRIPFIG1} shows that the signal peaks at higher \PT~than \zjets~and \TTB~, and tightening this cut is investigated, but in the interest of maintaining statistics, a lower \PT~threshold is chosen.
\item 2, 3, or 4 jets with \PT~\textgreater~25~GeV. Figure~\ref{TRIPFIG2} shows that the one jet region contains virtually no signal, so removing it eliminates background at no cost, while events with $>=$ 5 jets have little signal and are not as well modeled.
\item Leading jet \PT~\textgreater~40~GeV. Figure~\ref{TRIPFIG1} shows that below 40 GeV, there is less than 1 expected signal events, so very little is removed.
\item Exactly 1 \bjet. Figure~\ref{TRIPFIG2} shows that there is little signal outside the one \bjet~region. The signal and top backgrounds have one \bjet~from the top decay while non-top backgrounds are unlikely to have one.
\item 80 GeV \textless~\azhyph~mass \textless~100~GeV. The \az~is a defining feature of this analysis. The signal and all backgrounds except \TTB~and single \at~production have a real \az. Figure~\ref{TRIPFIG3} shows how much ttbar is off the $Z$ peak and a cut here will give substantial gains in removing \TTB~background without removing a significant amount of signal events. 
\item \met~\textgreater~20~GeV. This cut defines processes that have a real source of \met~such as the neutrino from \athyph~decays. Figure~\ref{TRIPFIG2} shows that the low \met~region is populated heavily by \zjets~with little signal.
\item If \wtm~\textless~40~GeV then \met~\textgreater~40~GeV is required. If viewed in the two dimentional plane, cases where jets are mis-reconstructed as leptons are expected to have low \wtm~and low \met. Distributions for \wtm~and \met~can be seen independently in Figure~\ref{TRIPFIG2}. The 2D plane of \wtm~vs. \met~is shown for both signal and data in Figure~\ref{TRIPFIG3}. Here we can see that the signal peaks above 20~GeV in \met~and above 40~GeV in \wtm~while data preferentially resides in the region where both \wtm~and \met~are below 40~GeV. This cut is primarily targeted at \zjets~events (as well as potential backgrounds with mis-identified \aw s) which heavily populate the low \wtm~region. This cut is referred to as the notch cut because of its unique shape. 
\end{itemize}


\QUADFIGLEG{LeadingLeptonPt}{SecondLeptonPt}{ThirdLeptonPt}{LeadingJetPt}{Distributions of Lepton \PT~for (a) leading, (b) second, and (c) third leptons as well as (d) leading jet \PT~with preselection applied except the cuts on minimum \PT~thresholds shown which are 40~GeV for the leading lepton, 20~GeV for the second lepton, 10~GeV for the third lepton, or 40 GeV for the leading jet. There are minimum \PT~reconstruction thresholds for these objects which are 25~GeV for the leading lepton and leading jet, and 10~GeV for the second and third leptons.}{TRIPFIG1}

\QUADFIGLEG{njet}{nbjet}{Wtransversemass}{met}{Distributions of (a) number of jets, (b) number of \bjet s, (c) \wtm, (d) \met. At least one jet is required at this level in all cases, but the cut on the variable shown is omitted in order to assess the full distribution. The distribution of the number of jets does not include the cut on the number of jets, the distribution of the number of \bjet s does not include the cut on the number of \bjet s, and the \wtm~and \MET~distributions do not contain the \MET~or the notch cuts.}{TRIPFIG2}

\TRPFIGLEG{METvsWtmSignal}{METvsWtmData}{Zmass}{Distributions of (a) two-dimentional map of \wtm~vs \met~for the signal, (b) two dimentional map of \wtm~vs \met~for the data, and (c) invariant mass of the \az. For both (a) and (b) the \met~cut and the notch cut are not applied and for (c) the \azhyph~mass window cut is not applied in order to show the full distribution.}{TRIPFIG3}


\clearpage

\section{Control Regions}
\label{SECTION-CONTROL-REGIONS}

Three control regions are considered for the three primary backgrounds to ensure that the background model describes the data well. The control regions are for \TTB, Diboson, and \zjets~and their yields are summarized in Table~\ref{tab:CRyields} where it can be seen that \tz~contamination is small and that the control regions are fairly pure in their respective backgrounds. The control region for \TTB~is defined by the preselection cuts with the exception of the \azhyph~mass window which is inverted. This has the effect of cutting out large contributions which contain a real \az, leaving primarily \TTB. In every distribution shown in Figures~\ref{FIGURE-CRttbar} and~\ref{FIGURE-CRttbar2}, there is good agreement between data and simulated events with a quite pure sample of \TTB. In order to isolate Diboson and \zjets , we begin with the preselection again, but instead of requiring exactly 1 \bjet, we require exactly 0 \bjet s in order to eliminate \athyph~contributions. This defines an intermediate control region with Diboson and \zjets~mixed as shown in Figures~\ref{FIGURE-CRint} and~\ref{FIGURE-CRint2}. This is expected because both Diboson and \zjets~have a real \az~and do not have a \ab~that would come from a \athyph~decay. To isolate Diboson more precisely, a cut is placed on \wtm~to constrain it to higher than 80~GeV as shown in Figures~\ref{FIGURE-CRDiboson} and~\ref{FIGURE-CRDiboson2}. This provides a region with high Diboson purity to evaluate the quality of its modeling. In order to isolate \zjets~from the intermediate control region a cut on \met~is made to constrain it to lower than 60~GeV as shown in Figures~\ref{FIGURE-CRZjets} and~\ref{FIGURE-CRZjets2}. This region has lower purity in \zjets~when compared to the \TTB~control region and the Diboson control region, and shows areas of mis-modeling in low to mid \wtm~(less than 70~GeV). Low lepton \PT~also seems to be poorly modeled (20-40~GeV for each of the three leptons). This is likely because the third lepton must be a mis-reconstructed one. Despite also having a mis-reconstructed lepton, due to only having two real leptons, \TTB~does not show similar mis-modeling for two primary reasons. The first reason is that \TTB~MC statistics is much better than \zjets. The second reason is that \TTB~has more hard objects (extra jets) stemming from the primary interactions, while \zjets~has extra hard objects come from initial-state or final-state radiation. This mis-modeling is mitigated by cuts on \PT, \wtm, and \met~as well as cuts made to the signal region. Even with these measures taken the mis-modeling reflects itself as large uncertainties on \zjets~which is shown in Chapter~\ref{SECTION-RESULTS}. Collectively these control regions give insight to the contribution of the largest backgrounds to this analysis. 



\QUADFIGLEG{LeadingLeptonPtCRttbar}{SecondLeptonPtCRttbar}{ThirdLeptonPtCRttbar}{LeadingJetPtCRttbar}{Distributions of transverse momenta for (a) the leading lepton, (b) the second lepton, (c) the third lepton, and (d) the leading jet in the control region for \TTB.}{FIGURE-CRttbar}
\QUADFIGLEG{njetCRttbar}{nbjetCRttbar}{WtransversemassCRttbar}{metCRttbar}{Distributions of (a) jet multiplicity, (b) \bjet~multiplicity, (c) \wtm, and (d) \met~in the control region for \TTB.}{FIGURE-CRttbar2}



\QUADFIGLEG{LeadingLeptonPtCRint}{SecondLeptonPtCRint}{ThirdLeptonPtCRint}{LeadingJetPtCRint}{Distributions of transverse momenta for (a) the leading lepton, (b) the second lepton, (c) the third lepton, and (d) the leading jet in the intermediate control region for Diboson and \zjets.}{FIGURE-CRint}
\QUADFIGLEG{njetCRint}{nbjetCRint}{WtransversemassCRint}{metCRint}{Distributions of (a) jet multiplicity, (b) \bjet~multiplicity, (c) \wtm, and (d) \met~in the intermediate control region for Diboson and \zjets.}{FIGURE-CRint2}



\QUADFIGLEG{LeadingLeptonPtCRDiboson}{SecondLeptonPtCRDiboson}{ThirdLeptonPtCRDiboson}{LeadingJetPtCRDiboson}{Distributions of transverse momenta for (a) the leading lepton, (b) the second lepton, (c) the third lepton, and (d) the leading jet in the control region for Diboson.}{FIGURE-CRDiboson}
\QUADFIGLEG{njetCRDiboson}{nbjetCRDiboson}{WtransversemassCRDiboson}{metCRDiboson}{Distributions of (a) jet multiplicity, (b) \bjet~multiplicity, (c) \wtm, and (d) \met~in the control region for Diboson.}{FIGURE-CRDiboson2}



\QUADFIGLEG{LeadingLeptonPtCRZjets}{SecondLeptonPtCRZjets}{ThirdLeptonPtCRZjets}{LeadingJetPtCRZjets}{Distributions of transverse momenta for (a) the leading lepton, (b) the second lepton, (c) the third lepton, and (d) the leading jet in the control region for \zjets.}{FIGURE-CRZjets}
\QUADFIGLEG{njetCRZjets}{nbjetCRZjets}{WtransversemassCRZjets}{metCRZjets}{Distributions of (a) jet multiplicity, (b) \bjet~multiplicity, (c) \wtm, and (d) \met~in the control region for \zjets.}{FIGURE-CRZjets2}



\begin{table} [ht!]
\setlength{\tabcolsep}{2pt}
\footnotesize
\centering
\begin{tabular}{| l | c | c | c | c | c | c |}
\hline
\hline
Event Yields & Preselection & \TTB~CR & intermediate CR & Diboson CR & \zjets~CR & final selection\\

\hline
\hline

$t\bar{t}$ & 45 & 196 & 16 & 4.0 & 5.7 & 10 $\pm$ 45\%\\
single \at & 1.4 & 7.7 & 0.85 & 0.26 & 0.30 & 0.34 $\pm$ 66\%\\
$ttV$ & 4.4 & 2.7 & 1.0 & 0.38 & 0.30 & 0.61 $\pm$ 66\%\\
$Z$ + jets & 32 & 10 & 110 & 4.0 & 78 & 1.7 $\pm$ 413\%\\
Diboson & 18 & 5.0 & 100 & 31 & 48 & 3.3 $\pm$ 32\%\\

\hline

$tZ$ & 5.7 & 0.63 & 1.6 & 0.4 & 0.68 & 2.9 $\pm$ 11\%\\

\hline

Total Expected & 108 & 223 & 232 & 41 & 134 & 19 $\pm$ 71\%\\
Data Observed & 108 & 237 & 214 & 52 & 131 & 22\\

\hline

 S/B & 0.06 & 0.00 & 0.01 & 0.01 & 0.01 & 0.18 \\ 
 S/$\surd$B & 0.57 & 0.04 & 0.10 & 0.07 & 0.06 & 0.71 \\ 

\hline
\hline

\end{tabular}
\caption{Event yields for various stages of analysis to compare with control region (CR) yields. The final selection is described in Section ~\ref{SECTION-SELECTION-CUTS} and uncertainties provided on the final selection are described in Chapter~\ref{SECTION-RESULTS} taken in quadrature for each sample. They are provided here for reference.}
\label{tab:CRyields}
\end{table}



\clearpage

\section{Cut Flow}
\label{SECTION-SELECTION-CUTS}

Once the preselection region is defined, our goal is to improve the sensitivity of the analysis. We do this by searching for kinematic variables where the shape of the signal distribution significantly differs from the shape of one or all of the background distributions and evaluating its effect on the value $S/\sqrt{B}$. The variable $S/\sqrt{B}$ is used to optimize because it ensures both strong signal to background ratios while also ensuring that we limit the contribution of statistical errors. Many distributions are considered for their background rejection, and/or physical motivations but distributions of special interest are the angular variables and \athyph~mass shown in Figure~\ref{FIGURE-TRIPFIG-FINAL2} because they display the properties of the \at. The polarization of the \at~is most notable in Figure~\ref{FIGURE-TRIPFIG-FINAL2} where the optimal basis shows both \TTB~and \tz~have a distribution favoring values closer to 1, while Diboson is comparatively flat. In principle these variables could be used to distinguish backgrounds without a \at~from the signal which does. In practice the discrimination power of these variables is not as strong as that of others. The variables with the best discriminating power are shown in Table~\ref{tab:eventyieldFullSelec} and are, 

\begin{itemize}

\item \wtm~\textgreater~50~GeV. This selects for events with higher energy \aw s. 

\item Leading non-\bjet~|\eta|~\textgreater~1.5. This selects for events with a forward jet as is the case with single top \tchan~and \tz.

\item $\Delta R$ between the \bjet~and Leading non-\bjet~\textgreater~2.5. $\Delta R$ is calculated as the $\Delta \eta$ and the $\Delta \phi$ added in quadrature. These two objects are expected to not be near each other in the signal selecting for events where the jets do not both come from the same source.

\end{itemize}

 The distributions of these variables are shown in Figure~\ref{cutfig} and are re-optimized sequentially to show that any correlations are minor, and to ensure optimal sensitivity. Table~\ref{tab:eventyieldFullSelec} also shows what background each cut is preferentially removing. The \wtm~cut targets \zjets, while also eliminating \TTB~and some Diboson. The cut on the leading non-\bjet~\eta~is less obviously targeted at a specific background, but is removing approximately half of all backgrounds while removing comparatively little signal. This is due to the forward jet, a characteristic kinematic property of single \athyph~production. The cut on the $\Delta R$ between the \bjet~and leading non-\bjet~performs well because the \bjet~and the leading non-\bjet~are coming from opposite legs of the hard interaction. This creates a distribution where the \at~and its decay products (in this case the \bjet) come out preferentially far apart in $\Delta R$ in the signal compared to the backgrounds. 








\begin{table} [ht!]
\setlength{\tabcolsep}{2pt}
\footnotesize
\centering
\begin{tabular}{| l | r | r | r | r |}
\hline
\hline
Process & Preselection & \wtm~& Leading-non \bjet~\eta & full selection \\ 
\hline
$t\bar{t}$ & 45 & 28 & 15 & 10 $\pm$ 45\% \\ 
single \at & 1.4 & 1.0 & 0.49 & 0.34 $\pm$ 66\% \\ 
$ttV$ & 4.4 & 3.1 & 1.0 & 0.61 $\pm$ 66\%\\ 
$Z$ + jets & 32 & 5.3 & 2.3 & 1.7 $\pm$ 413\%\\ 
Diboson & 18 & 13 & 5.2 & 3.3 $\pm$ 32\%\\ 
\hline
$tZ$ & 5.7 & 4.3 & 3.2 & 2.9 $\pm$ 11\%\\ 
\hline
Total Expected & 108 & 55 & 27 & 19 $\pm$ 71\% \\ 
Data Observed & 108 & 62 & 29 & 22 \\ 
\hline
 S/B & 0.06 & 0.08  & 0.13 & 0.18 \\ 
 S/$\surd$B & 0.57 & 0.60 & 0.66 & 0.71 \\ 

\hline
\hline
\end{tabular}
\caption{Event yields after selection cuts are applied. Uncertainties provided on the final selection are the uncertainties described in Chapter~\ref{SECTION-RESULTS} taken in quadrature for each sample.}
\label{tab:eventyieldFullSelec}
\end{table}


\TRPFIGLEG{W_transverse_mass}{LeadingNonb-jetEta}{b-jet+LeadingNonb-jetdR}{Distributions of (a) \wtm which is required to be~\textgreater~50~GeV, (b) the \eta~of the leading non b-tagged jet which is required to be~\textgreater~1.5, and (c) the $\Delta R$ between the \bjet~and leading non-\bjet which is required to be~\textgreater~2.5. Each has the entire selection applied except the variable plotted to view the full distribution.}{cutfig}


Once we have applied the full cut flow, we are left with the remaining distributions to analyze. These represent the kinematic properties of events selected by this analysis which are shown in Figures~\ref{FIGURE-SigRegion} and~\ref{FIGURE-SigRegion2}. The application of the full selection takes us from an $S/B$ of 0.06 to 0.18. These efforts are to improve the sensitivity of our analysis as shown in the next chapter. There is reasonable agreement throughout the signal region, and in \PT~distributions it can be seen that the signal peaks higher when the cuts are placed. These cuts were chosen because they optimized $S/\sqrt{B}$ which in this case prioritized preserving statistics over improving signal purity. With more data collected these cuts could be tightened to further improve $S/\sqrt{B}$. 



\QUADFIGLEG{FinalLeadingLeptonPt}{FinalSecondLeptonPt}{FinalThirdLeptonPt}{FinalLeadingJetPt}{Distributions of transverse momenta for (a) the leading lepton, (b) the second lepton, (c) the third lepton, and (d) the leading jet in the signal region.}{FIGURE-SigRegion}
\QUADFIGLEG{Finalnjet}{Finalnbjet}{WtransM}{Finalmet}{Distributions of (a) jet multiplicity, (b) \bjet~multiplicity, (c) \wtm, and (d) \met~in the signal region.}{FIGURE-SigRegion2}


\DBLFIGLEG{nelec}{nmuon}{(a) Number of electrons and (b) number of muons.}{FIGURE-NMUON}


\QUADFIGLEG{TopPolOpt}{TopPolHel}{Whelicity}{HISTO-TOPMASS}{Distributions of the \athyph~polarization in the (a) Optimal basis and (b) the helicity basis, (c) the \awhyph~helicity, and (d) the mass of the \at.}{FIGURE-TRIPFIG-FINAL2}



\chapter{Results}
\label{SECTION-RESULTS}

Bayesian address the question everyone is interested in by using assumptions no one believes. Frequentest use impeccable logic to deal with an issue of no interest to anyone. - L.Lyons

\vspace{5mm} %5mm vertical space

A single bin profile likelihood calculation is performed to extract limits on the \tz~\xs~at the 95\% confidence limit using roostats~\cite{ROOSTAT}. Profile likelihood calculations can produce confidence intervals on non-normal distributions more accurately than maximum or partial likelihood functions~\cite{st0132} and for this reason they have become a popular statistical method for high energy physics. However, before we can do this evaluation, a series of systematic uncertainties must be adressed and evaluated. The expected sensitivity to larger data-sets from the LHC is also evaluated. 

\section{Systematic Uncertainties}
\label{SECTION-systematics}

Systematic uncertainties on the object reconstruction, event reconstruction, normalization, and theoretical modeling affect the acceptance and expected event yield for each source. Tables~\ref{tab:systematicsMODEL} and~\ref{tab:jetsystematics} contain evaluated uncertainties. Some uncertainties are symmetric in nature, while others have distinct up and down variations. Nearly all uncertainties have been symmetrized either because of practical reasons (their effect is small so we can simplify them or they happen to come out symmetric) or for theoretical reasons (there is a physical motivation for them to be symmetric). For these symmetric systematic uncertainties both up and down variations are considered and the greater of the two is used~\cite{TOPCOMMONSYSTEMATICS}. 

\begin{itemize}

\item Luminosity - The uncertainty on the integrated luminosity is $\pm$2.1\%. It is obtained from Van Der Meer scans are performed in which the beam positions in the $x-y$ plane are varied~\cite{Balagura:2011yw,Aad:2013ucp}.

\item Pile Up - Pile up is discussed briefly in Chapter~\ref{SECTION-TRIGGERS}. Here we need to evaluate how well we estimate the degree to which pile up interferes with our ability to distinguish events from each other. This is one of the few uncertainties that was not symmetrized having different uncertainties for the up and down variations~\cite{JETUNCERTAINTIES}. 

\item Lepton efficiency scale factors - Leptons from our simulated Monte Carlo samples are needed to replicate our data in identification criteria (Electron ID, Muon ID Systematic, and Muon ID Statistics), isolation criteria, and trigger simulation (Electron Trigger, Muon Trigger). A prescription for how to assess this uncertainty is provided by the EGamma (which evaluates electrons and photons) and Muon groups which are derived from $Z->\ell\ell$ samples.~\cite{ELECTRON-RECO,muonSFWiki}

\item Electron calibration - Electron momentum scale (Electron Scale) and resolution (Electron Resolution) are handled separately from lepton efficiency scale factors. Scale corrections are derived for data and smearing corrections for Monte Carlo. These corrections assess the systematic uncertainties associated with the processing of photons, and in this case, electrons~\cite{Fayard:2060328}. %(EG_SCALE_ALL) (EG_RESOLUTION_ALL)

\item Muon calibration - Muon momentum scale and resolution are handled separately from lepton efficiency scale factors. Muon track identification (Muon track ID), transverse momentum scale (Muon Scale), and resolution (Muon Resolution) are corrected as well as~\cite{muonSFWiki}. %MUONS SCALE MUONS MS MUONS ID

\item \met~calibration - Lepton and jet energy and momentum scale and resolution uncertainties propagate into calculations of \met~(giving MET Scale and MET Resolution). How we include soft tracks into this calculation corresponds to a source of uncertainty~\cite{METWiki}.

%There are three sources of uncertainties that manifest from comparing the soft track components of \met~to the hard track components. The vector sum of hard objects creates an axis known as ptHard. The three sources are the scale of the soft tracks shifted along ptHard, the \PT~smearing of soft tracks along ptHard, and the \PT~smearing of soft tracks perpendicular to ptHard. The MissingETUtility tool is used to evaluate these effects.~\cite{METWiki}

\item Jet energy scale (JES) - JES  and  its  uncertainty  are  derived  combining  information from test-beam data, collision data, and simulation. The JES uncertainty is split into several orthogonal components using $in situ$ techniques resulting in independent effective uncertainties. This is determined with 8 TeV data and extrapolated to 13 TeV running conditions~\cite{ATL-PHYS-PUB-2015-015}.
%NPScinario1 Grouped NP stuff

\item Jet energy resolution (JER) - The precision with which a jet's energy is measured has an uncertainty associated with it. A mis-modeling of this energy resolution can lead to varying acceptances in final state kinematics~\cite{ATL-PHYS-PUB-2015-015,JERWiki}.

\item \bjet~tagging - \btag ing scale factors are used on a per-event basis to correct \btag ing efficiency. This is determined with 8 TeV data and extrapolated to 13 TeV running conditions using three independent eigenvectors for the efficiency of \bjet s, \cjet s, and light jets as well as two parameters to account for the extrapolation from 8 to 13~TeV~\cite{Aad:2015ydr}.

%A 6 element eigenvector is needed for the efficiency of \bjet s, a 4 element eigenvector is needed for the efficiency of $c$-jets, and a 12 element eigenvector is needed for the efficiency of light jets. 

\item Initial-state radiation and final-state radiation (ISR/FSR) - ISR/FSR is evaluated on the \TTB~sample by varying the renormalization and factorization scales up and down by a factor of two from the nominal value of 1. This process is done to \TTB~because it is the dominant background to single top analyses and is small when compared to other uncertainties that affect the other backgrounds. 

%(RADHI and RADLO) 

\item NLO subtraction - The uncertainties of how the NLO subtraction method is applied is evaluated on the \TTB~sample. Powheg and aMC@NLO are two tools that are used to calculate higher order corrections. A comparison of the two tools applied to \TTB~is used to estimate this uncertainty.
% aMcAtNloHerwigpp

\item Parton showering (PS) and Hadronization - The uncertainty on parton showering and hadronization is evaluated by comparing the cluster model in Herwig and the Lund string model in Pythia applied to \TTB. A comparison of the two techniques implemented in these tools is used to estimate the uncertainty on this process. 
%PowhegHerwigppEvtGen

\item Parton Distribution Function (PDF) - The uncertainties that come from the choice of PDF is evaluated on the \TTB~sample by comparing PDF4LHC15 and CT10. 
%PowhegPythia8

\item Normalization - Normalization uncertainties for Diboson and \zjets~are estimated from control regions. For \TTB~\cite{ttbarxsecUNCERT}, single top~\cite{sgtopxsecUNCERT}, and \ttz~\cite{ttVxsecUNCERT}, theory uncertainties on scale variations, PDF, and top-quark mass are used.

\item MC Statistics - The uncertainty due to limited statistics in our simulated samples is assessed by taking the sum of the square of the weights of each event in each sample. When selecting for a narrow piece of phase space in order to look for small signals as is done in this analysis, it becomes increasingly difficult to both separate signal from background and maintain meaningful statistics both for MC and data.


\end{itemize}




\begin{table} [ht!]
\setlength{\tabcolsep}{2pt}
\footnotesize
\centering
\begin{tabular}{| l | c | c | c | c | c | c | c |}
\hline
\hline
Systematics & $t\bar{t}$ & Other Top & $Z$ + jets & Diboson & $tZ$ & background total \\
\hline
\hline

Pile Up UP & -11\% & 64\% & 94\% & 10\% & -2.4\% & 5.8\% \\
Pile Up DOWN & 2.1\% & -13\% & 6.8\% & 1.8\% & 0.69\% & 1.3\% \\

\hline
Normalization & $\pm $ 5.5\% & $\pm $ 10\% & $\pm $ 20\% & $\pm $ 20\% & $\pm $ - & $\pm $ 10\% \\
\hline

MC Statistics & $\pm $ 8.3\% & $\pm $ 9.2\% & $\pm $ 47\% & $\pm $ 3.0\% & $\pm $ 1.6\% & $\pm $ 9.9\% \\

\hline

PDF & $\pm $ 4.3\% & - & - & - & - & $\pm $ 2.3\% \\
PS and Hadronization & $\pm $ 0.86\% & - & - & - & - & $\pm $ 0.46\% \\
NLO subtraction & $\pm $ 20\% & - & - & - & - & $\pm $ 11\% \\
ISR/FSR RadLo & 27\% & - & - & - & - & 14\% \\
ISR/FSR RadHi & -24\% & - & - & - & - & -11\% \\

\hline
\hline

\end{tabular}
\caption{Systematic uncertainties related to background normalization and theory modeling. Other Top is the combination of \TTB V and single top.}
\label{tab:systematicsMODEL}
\end{table}



\clearpage



\begin{table} [ht!]
\setlength{\tabcolsep}{2pt}
\footnotesize
\centering
\begin{tabular}{| l | c | c | c | c | c | c | c |}
\hline
\hline
Systematics & $t\bar{t}$ & Other Top & $Z$ + jets & Diboson & $tZ$ & background total \\
\hline
\hline


Muon ID Systematic & $\pm $ 0.57\% & $\pm $ 0.90\% & $\pm $ 1.1\% & $\pm $ 0.90\% & $\pm $ 1.0\% & $\pm $ 0.77\% \\
\hline

Muon ID Statistics & $\pm $ 0.48\% & $\pm $ 0.90\% & $\pm $ 0.57\% & $\pm $ 0.60\% & $\pm $ 0.69\% & $\pm $ 0.570\% \\
\hline

Electron ID & $\pm $ 2.6\% & $\pm $ 1.8\% & $\pm $ 1.1\% & $\pm $ 1.5\% & $\pm $ 1.7\% & $\pm $ 2.1\% \\
\hline

Electron Trigger & $\pm $ 2.1\% & $\pm $ 5.4\% & $\pm $ 6.8\% & $\pm $ 1.8\% & $\pm $ 0.69\% & $\pm $ 1.3\% \\
\hline

Electron Reconstruction & $\pm $ 1.2\% & $\pm $ 0.90\% & $\pm $ 1.1\% & $\pm $ 0.90\% & $\pm $ 0.69\% & $\pm $ 1.0\% \\
\hline

Electron Scale & $\pm $ 1.7\% & $\pm $ 3.8\% & $\pm $ 0\% & $\pm $ 1.2\% & $\pm $ 0\% & $\pm $ 0.99\% \\
\hline

Electron Resolution & $\pm $ 0.86\% & $\pm $ 2.9\% & $\pm $ 0\% & $\pm $ 0.30\% & $\pm $ 0.34\% & $\pm $ 0.72\% \\
\hline

Muon Scale & $\pm $ 0.48\% & $\pm $ 2.1\% & $\pm $ 0.57\% & $\pm $ 0.60\% & $\pm $ 0.69\% & $\pm $ 0.57\% \\
\hline
Muon Resolution & $\pm $ 0.48\% & $\pm $ 7.0\% & $\pm $ 2.8\% & $\pm $ 0.90\% & $\pm $ 0.69\% & $\pm $ 0.41\% \\
\hline
Muon track ID & $\pm $ 1.3\% & $\pm $ 1.9\% & $\pm $ 0.57\% & $\pm $ 0.60\% & $\pm $ 0.69\% & $\pm $ 0.82\% \\
\hline

MET Scale & $\pm $ 0.67\% & $\pm $ 0\% & $\pm $ 0\% & $\pm $ 0.60\% & $\pm $ 0.34\% & $\pm $ 0.46\% \\
\hline
MET Resolution & $\pm $ 1.3\% & $\pm $ 0\% & $\pm $ 0\% & $\pm $ 0\% & $\pm $ 0.34\% & $\pm $ 0.77\% \\
\hline

JER & $\pm $ 4.7\% & $\pm $ 4.5\% & $\pm $ 130\% & $\pm $ 8.4\% & $\pm $ 1.40\% & $\pm $ 10\% \\
\hline

bTagSF \bjet s& $\pm $ 5.7\% & $\pm $ 1.2\% & $\pm $ 1.1\% & $\pm $ 0.30\% & $\pm $ 1.2\% & $\pm $ 3.2\% \\

\hline

bTagSF \cjet s & $\pm $ 0.48\% & $\pm $ 1.3\% & $\pm $ 6.0\% & $\pm $ 10\% & $\pm $ 0\% & $\pm $ 2.0\% \\

\hline

bTagSF light jets & $\pm $ 1.7\% & $\pm $ 2.1\% & $\pm $ 12\% & $\pm $ 13\% & $\pm $ 1.0\% & $\pm $ 2.4\% \\

\hline

JES 1 up & 1.2\% & -3.0\% & 150\% & 9.6\% & 0\% & 15\% \\
JES 1 down & -4.7\% & -3.0\% & 130\% & 8.4\% & 1.3\% & 10\% \\

JES 2 up & 0.76\% & -2.1\% & 290\% & 4.5\% & 0.34\% & 27\% \\
JES 2 down & -2.1\% & -1.2\% & 0\% & -3.4\% & 0\% & -1.7\% \\

JES 3 up & 4.1\% & -0.90\% & 190\% & 10\% & 0.69\% & 20.90\% \\
JES 3 down & -4.6\% & 2.1\% & 0\% & -10.0\% & -0.69\% & -4.1\% \\

\hline
\hline
\end{tabular}
\caption{Systematic uncertainties related to object identification, resolution, and scale. Other Top is the combination of \TTB V and single top.}
\label{tab:jetsystematics}
\end{table}



\clearpage



\section{Statistical Analysis}
\label{SECTION-stats}

Maximum likelihood ratio tests are among the most used methods in statistics because of their strength in hypothesis testing and generality. A popular variant of this method is the profile likelihood ratio test which considers nuisance parameters which are not of primary interest ($\theta$) to be functions of the  parameter which is of interest ($\beta$). The parameter of interest, $\beta$, in this case is defined as the ratio of the measured cross section to the standard model cross section. The nuisance parameters, $\theta$, are measures of systematic uncertainties which are modeled by Gaussian statistics. By profiling we simplify the problem of finding $\beta$ and $\theta$ which optimizes the likelihood function in Equation~\ref{EQUATION-MaxLike} to constrain $\theta = f(\beta )$ so that we can optimize Equation~\ref{EQUATION-ProfLike} which is often a preferable procedure when only one nuisance parameter is important.

\begin{equation}
\mathcal{L}(\beta ,\theta |data) 
\label{EQUATION-MaxLike}
\end{equation}

\begin{equation}
\mathcal{L}(\beta ,f(\beta )|data)
\label{EQUATION-ProfLike}
\end{equation}

Because we have only one parameter which needs to be optimized for a profile likelihood fit is performed. This procedure is further simplified by only considering a distribution of a single bin. This simplification makes the profiling of the nuisance parameters easy, as each is simply a Gaussian, not dependent on the parameter of interest at all. These nuisance parameters are treated as correlated between sources of signal and background in the optimization procedure. The signal \xs~is then extracted from the likelihood function. The extracted \xs~measurement is $\sigma_{tZ}$~=~448~$\pm$~672~(stat)~$\pm$~448~(syst)~fb. This is 1.9 times the expected Standard Model \xs~of 236~fb which is due to the data excess over the expected background shown in Table~\ref{tab:eventyieldFullSelec}. Because of the large uncertainties this is still in agreement with the standard model expectation. This corresponds to an upper bound at the 95\% confidence limit on the \tz~\xs~of $\sigma_{tZ}$ = 1345~fb. A lower bound can not be set due to large systematic uncertainties. The most notable systematic uncertainties in this analysis are the experimental uncertainties JES and JER, MC statistics for samples that have a mis-reconstructed lepton, and normalization uncertainties.  

\section{Outlook}
\label{SECTION-outlook}

In order to estimate the potential sensitivity to \tz~with increased data collection, a series of simplified statistical analyses is performed. Systematic uncertainties are removed in order to see the effects of increased statistics in an idealized way. The expected yields obtained by this analysis are scaled up by a factor of 10 to estimate the expected precision of the cross-section measurement with the full 2016 data set, which corresponds to an integrated luminosity of approximately 30~\fb. The expected yields are then also scaled up by a factor of 100 in order to estimate the expected precision of the cross-section measurement with the full Run~2 and Run~3 data set, which corresponds to an integrated luminosity of approximately 300~\fb. When this is performed with the event yields of this analysis we can get an expected uncertainty on the \xs~of 150\%. With the full 2016 data set the expected uncertainty drops to 50\%. With the full set of run 2 data the expected uncertainty falls to 20\%. This analysis is currently statistics limited, but with the full run 2 data set we will become systematics limited. The most immediate gains can be made from increasing MC statistics, with longer-term gains to be made from better understanding JES and JER. To improve the sensitivity of this analysis, more complex multivariate analysis methods could be employed, the profile likelihood could be performed on a strong discriminating distribution, and/or control regions could be fit and included in the statistical analysis. Beyond that we will need to wait for the LHC to deliver more data in order to put further constraints on the \tz~\xs. 



\chapter{Conclusion}
We have analyzed \LUMI\ of data collected with the ATLAS detector. In our search for the \Wtchan\ we have seen a statistically significant excess of 3.3$\sigma$. This is sufficient to claim evidence, and although this does not meet the $>5\sigma$ criteria to claim observation, it is a significant step to verifying the Standard Model prediction. The estimated cross-section is also extracted from the data, giving a result of $\sigma(pp\rightarrow Wt + X) = 16.8 ^{+2.9}_{-2.9} \mathrm{(stat)} ^{+4.9}_{-4.9} \mathrm{(syst)}~pb$. This analysis also allowed us to make measurements of other Standard Model parameters. The CKM matrix element $V_{tb}$ is measured to be $|V_{tb}| = 1.03^{+0.16}_{-0.19}$. The width of the top quark is measured at $\Gamma_{t}^{obs.} = 1.4\pm 0.5~\rm GeV$ (Note the increase in the percent uncertainty due to the $|V_{tb}|^2$ dependence), giving a lifetime of $\tau_{t}=(4.7^{+1.2}_{-1.2})\times 10^{-25}~s$. These measurements are all consistent with theoretical Standard Model predictions and other experimental measurements. This analysis is published in Physics Letters B~\cite{WTEVIDENCE}.

In this analysis I implemented the BDT used, which includes the variable selection and testing, the training procedure, and the parameter optimization. I implemented the ATLAS and top group recommendations for the object definitions, event selection, and studied most of the systematics (the jet energy scale, jet reconstruction, jet ID, lepton ID, lepton resolution, \MET, and pile-up uncertainties). The data-driven \Ztt\ normalization is estimated by me. I prepared the plots of the BDT and plots of the variables used. During the preparation of the paper and the associated note, I gave many single top working group talks and the approval talk to the top working group. I also collaborated with Huaqiao Zhang to perform many cross-checks while going through review.

With time the systematic uncertainties will be better understood and in the future this analysis will be repeated with more data. However, there is ample room for improvement in the analysis procedure itself. Note that the BDT optimization is done using only the nominal \MC. A look at the uncertainty composition of the final cross-section measurement will reveal that this analysis is quite systematically limited. A BDT optimization using information from the systematically shifted datasets could bring significant improvement to the result as a whole. This is not a trivial undertaking, as the existing toolsets are not equipped to do this kind of optimization out of the box, however implementing a systematics-sensitive optimization has the potential to greatly increase the significance.

This evidence for the existence of the \Wtchan\ was also confirmed independently by the CMS collaboration~\cite{CMSEVIDENCE}. Both the CMS and ATLAS collaborations will continue to update these analyses with better analysis techniques, a better understanding of the systematic uncertainties, and more data. The discovery of the \Wtchan\ is not the end, of course. Precision measurements of $V_{tb}$ and the top quark properties and searches for new physics in the \Wtchan\ signal region are all exciting new analyses waiting to be explored.

The LHC era is already showing its promise, giving exciting results like the recent Higgs discovery~\cite{HiggsATLAS,HiggsCMS} and confirming the predictions of the Standard Model. Even with the Higgs boson discovered, there remains much discovery ahead. The LHC will be running for years, pushing our understanding forward. With each collision we strive for a better understanding of our universe, and with time and hard work, these efforts will be rewarded.

%\newpage
%\vspace*{\fill}
%\begin{center}
%\Huge \textbf{APPENDICIES}
%\end{center}
%\vfill
%\newpage
%\appendix
%\chapter{Data/MC Agreement in Control Regions}
\label{APPENDIX-CONTROLREGIONS}
This appendix shows the BDT variables in the background-enhanced 2-jet and 3-jet regions. The 2-jet and 3-jet regions clearly show how dominant of a background \ttbar\ is for this analysis. Due to the strong \ttbar\ contribution we are able to use these regions to constrain the \ttbar\ normalization, which would otherwise be a dominating uncertainty. Selected variables are also shown in the three dilepton channels: $ee$, $e\mu$, and $\mu\mu$. The dilepton subchannels show that the good data-simulation agreement does not break down when these subchannels are examined independently. 

\section{2-jet events}
\label{APPENDIX-CONTROLREGIONS-2J}
\SEXFIGLEG{paper_ll2j_LP2fb_v4_pT_sys_flat}{paper_ll2j_LP2fb_v4_pT_sys_sig_flat}{paper_ll2j_LP2fb_v4_AllJetsLepton_Centrality_flat}{paper_ll2j_LP2fb_v4_ThrustEta_flat}{paper_ll2j_LP2fb_v4_SystemLep1Lep2_eta_flat}{legend}{The top five variables in the BDT ranked by separation power, comparing the signal and background estimate to the data in the 2-jet bin.}{FIGURE-CONTROL-2JVARIABLES1}

\SEXFIGLEG{paper_ll2j_LP2fb_v4_eta_sys_lepsJet1_flat}{paper_ll2j_LP2fb_v4_LeadingLeptonEta_flat}{paper_ll2j_LP2fb_v4_SystemLep1Lep2_E_flat}{paper_ll2j_LP2fb_v4_HT_AllJets_flat}{paper_ll2j_LP2fb_v4_pT_sys_lepsJet1_flat}{legend}{The 6th-10th top variables in the BDT ranked by separation power, comparing the signal and background estimate to the data in the 2-jet bin.}{FIGURE-CONTROL-2JVARIABLES2}

\SEXFIGLEG{paper_ll2j_LP2fb_v4_Thrust_flat}{paper_ll2j_LP2fb_v4_InvariantMass_Lep2Jet1_flat}{paper_ll2j_LP2fb_v4_SystemLep1Jet1_eta_flat}{paper_ll2j_LP2fb_v4_SubLeadingLeptonEta_flat}{paper_ll2j_LP2fb_v4_Jet1Eta_flat}{legend}{The 11th-15th top variables in the BDT ranked by separation power, comparing the signal and background estimate to the data in the 2-jet bin.}{FIGURE-CONTROL-2JVARIABLES3}

\SEXFIGLEG{paper_ll2j_LP2fb_v4_DeltaMinPhiLeptonLeadingJet_flat}{paper_ll2j_LP2fb_v4_InvariantMass_Lep1Jet1_flat}{paper_ll2j_LP2fb_v4_DeltaPhi_SLep1Jet1_Lep2_flat}{paper_ll2j_LP2fb_v4_MET_flat}{paper_ll2j_LP2fb_v4_DeltaEtaLeadingLeptonLeadingJet_flat}{legend}{The 16th-20th top variables in the BDT ranked by separation power, comparing the signal and background estimate to the data in the 2-jet bin.}{FIGURE-CONTROL-2JVARIABLES4}

\TRPFIGLEG{paper_ll2j_LP2fb_v4_DeltaRSubLeadingLeptonLeadingJet_flat}{paper_ll2j_LP2fb_v4_InvariantMass_MaxLepJet1_flat}{legend}{The 21st and 22nd top variables in the BDT ranked by separation power, comparing the signal and background estimate to the data in the 2-jet bin.}{FIGURE-CONTROL-2JVARIABLES5}

\newpage

\section{3-jet inclusive events}
\label{APPENDIX-CONTROLREGIONS-3J}
\SEXFIGLEG{paper_ll3jinc_LP2fb_v4_pT_sys_flat}{paper_ll3jinc_LP2fb_v4_pT_sys_sig_flat}{paper_ll3jinc_LP2fb_v4_AllJetsLepton_Centrality_flat}{paper_ll3jinc_LP2fb_v4_ThrustEta_flat}{paper_ll3jinc_LP2fb_v4_SystemLep1Lep2_eta_flat}{legend}{The top five variables in the BDT ranked by separation power, comparing the signal and background estimate to the data in the 3-jet inclusive bin.}{FIGURE-CONTROL-3JVARIABLES1}

\SEXFIGLEG{paper_ll3jinc_LP2fb_v4_eta_sys_lepsJet1_flat}{paper_ll3jinc_LP2fb_v4_LeadingLeptonEta_flat}{paper_ll3jinc_LP2fb_v4_SystemLep1Lep2_E_flat}{paper_ll3jinc_LP2fb_v4_HT_AllJets_flat}{paper_ll3jinc_LP2fb_v4_pT_sys_lepsJet1_flat}{legend}{The 6th-10th top variables in the BDT ranked by separation power, comparing the signal and background estimate to the data in the 3-jet inclusive bin.}{FIGURE-CONTROL-3JVARIABLES2}

\SEXFIGLEG{paper_ll3jinc_LP2fb_v4_Thrust_flat}{paper_ll3jinc_LP2fb_v4_InvariantMass_Lep2Jet1_flat}{paper_ll3jinc_LP2fb_v4_SystemLep1Jet1_eta_flat}{paper_ll3jinc_LP2fb_v4_SubLeadingLeptonEta_flat}{paper_ll3jinc_LP2fb_v4_Jet1Eta_flat}{legend}{The 11th-15th top variables in the BDT ranked by separation power, comparing the signal and background estimate to the data in the 3-jet inclusive bin.}{FIGURE-CONTROL-3JVARIABLES3}

\SEXFIGLEG{paper_ll3jinc_LP2fb_v4_DeltaMinPhiLeptonLeadingJet_flat}{paper_ll3jinc_LP2fb_v4_InvariantMass_Lep1Jet1_flat}{paper_ll3jinc_LP2fb_v4_DeltaPhi_SLep1Jet1_Lep2_flat}{paper_ll3jinc_LP2fb_v4_MET_flat}{paper_ll3jinc_LP2fb_v4_DeltaEtaLeadingLeptonLeadingJet_flat}{legend}{The 16th-20th top variables in the BDT ranked by separation power, comparing the signal and background estimate to the data in the 3-jet inclusive bin.}{FIGURE-CONTROL-3JVARIABLES4}

\TRPFIGLEG{paper_ll3jinc_LP2fb_v4_DeltaRSubLeadingLeptonLeadingJet_flat}{paper_ll3jinc_LP2fb_v4_InvariantMass_MaxLepJet1_flat}{legend}{The 21st and 22nd top variables in the BDT ranked by separation power, comparing the signal and background estimate to the data in the 3-jet inclusive bin.}{FIGURE-CONTROL-3JVARIABLES5}
\newpage
\section {Dilepton subchannels}
This section contains selected variables of the different dilepton final states.  This illustrates that our backgrounds are well modeled for each of the final states individually.

\SEXFIGLEG{paper_ee1+j_LP2fb_v4_NJets_flat}{paper_ee1+j_LP2fb_v4_Jet1Pt_flat}{paper_ee1+j_LP2fb_v4_HT_AllJets_flat}{paper_ee1+j_LP2fb_v4_MET_flat}{paper_ee1+j_LP2fb_v4_LeadingLeptonPt_flat}{legend}{Distributions of variables comparing the signal and background estimate to the data  in the $ee$ channel. (a) Jet multiplicity, (b) Leading jet \pT, (c)$H_T(jet)$, (d) \MET, (e) Leading lepton \pT}{FIGURE-PRESEL-EE}
\SEXFIGLEG{paper_em1+j_LP2fb_v4_NJets_flat}{paper_em1+j_LP2fb_v4_Jet1Pt_flat}{paper_em1+j_LP2fb_v4_HT_AllJets_flat}{paper_em1+j_LP2fb_v4_MET_flat}{paper_em1+j_LP2fb_v4_LeadingLeptonPt_flat}{legend}{Distributions of variables comparing the signal and background estimate to the data  in the $e\mu$ channel. (a) Jet multiplicity, (b) Leading jet \pT, (c)$H_T(jet)$, (d) \MET, (e) Leading lepton \pT}{FIGURE-PRESEL-EM}
\SEXFIGLEG{paper_mm1+j_LP2fb_v4_NJets_flat}{paper_mm1+j_LP2fb_v4_Jet1Pt_flat}{paper_mm1+j_LP2fb_v4_HT_AllJets_flat}{paper_mm1+j_LP2fb_v4_MET_flat}{paper_mm1+j_LP2fb_v4_LeadingLeptonPt_flat}{legend}{Distributions of variables comparing the signal and background estimate to the data  in the $\mu\mu$ channel. (a) Jet multiplicity, (b) Leading jet \pT, (c)$H_T(jet)$, (d) \MET, (e) Leading lepton \pT}{FIGURE-PRESEL-MM}

%\input{BPrimeSearch}
%\newpage
%\appendix
%\part*{Appendices}
%\addcontentsline{toc}{part}{Appendices}
%\chapter{Data/MC Agreement in Control Regions}
\label{APPENDIX-CONTROLREGIONS}
This appendix shows the BDT variables in the background-enhanced 2-jet and 3-jet regions. The 2-jet and 3-jet regions clearly show how dominant of a background \ttbar\ is for this analysis. Due to the strong \ttbar\ contribution we are able to use these regions to constrain the \ttbar\ normalization, which would otherwise be a dominating uncertainty. Selected variables are also shown in the three dilepton channels: $ee$, $e\mu$, and $\mu\mu$. The dilepton subchannels show that the good data-simulation agreement does not break down when these subchannels are examined independently. 

\section{2-jet events}
\label{APPENDIX-CONTROLREGIONS-2J}
\SEXFIGLEG{paper_ll2j_LP2fb_v4_pT_sys_flat}{paper_ll2j_LP2fb_v4_pT_sys_sig_flat}{paper_ll2j_LP2fb_v4_AllJetsLepton_Centrality_flat}{paper_ll2j_LP2fb_v4_ThrustEta_flat}{paper_ll2j_LP2fb_v4_SystemLep1Lep2_eta_flat}{legend}{The top five variables in the BDT ranked by separation power, comparing the signal and background estimate to the data in the 2-jet bin.}{FIGURE-CONTROL-2JVARIABLES1}

\SEXFIGLEG{paper_ll2j_LP2fb_v4_eta_sys_lepsJet1_flat}{paper_ll2j_LP2fb_v4_LeadingLeptonEta_flat}{paper_ll2j_LP2fb_v4_SystemLep1Lep2_E_flat}{paper_ll2j_LP2fb_v4_HT_AllJets_flat}{paper_ll2j_LP2fb_v4_pT_sys_lepsJet1_flat}{legend}{The 6th-10th top variables in the BDT ranked by separation power, comparing the signal and background estimate to the data in the 2-jet bin.}{FIGURE-CONTROL-2JVARIABLES2}

\SEXFIGLEG{paper_ll2j_LP2fb_v4_Thrust_flat}{paper_ll2j_LP2fb_v4_InvariantMass_Lep2Jet1_flat}{paper_ll2j_LP2fb_v4_SystemLep1Jet1_eta_flat}{paper_ll2j_LP2fb_v4_SubLeadingLeptonEta_flat}{paper_ll2j_LP2fb_v4_Jet1Eta_flat}{legend}{The 11th-15th top variables in the BDT ranked by separation power, comparing the signal and background estimate to the data in the 2-jet bin.}{FIGURE-CONTROL-2JVARIABLES3}

\SEXFIGLEG{paper_ll2j_LP2fb_v4_DeltaMinPhiLeptonLeadingJet_flat}{paper_ll2j_LP2fb_v4_InvariantMass_Lep1Jet1_flat}{paper_ll2j_LP2fb_v4_DeltaPhi_SLep1Jet1_Lep2_flat}{paper_ll2j_LP2fb_v4_MET_flat}{paper_ll2j_LP2fb_v4_DeltaEtaLeadingLeptonLeadingJet_flat}{legend}{The 16th-20th top variables in the BDT ranked by separation power, comparing the signal and background estimate to the data in the 2-jet bin.}{FIGURE-CONTROL-2JVARIABLES4}

\TRPFIGLEG{paper_ll2j_LP2fb_v4_DeltaRSubLeadingLeptonLeadingJet_flat}{paper_ll2j_LP2fb_v4_InvariantMass_MaxLepJet1_flat}{legend}{The 21st and 22nd top variables in the BDT ranked by separation power, comparing the signal and background estimate to the data in the 2-jet bin.}{FIGURE-CONTROL-2JVARIABLES5}

\newpage

\section{3-jet inclusive events}
\label{APPENDIX-CONTROLREGIONS-3J}
\SEXFIGLEG{paper_ll3jinc_LP2fb_v4_pT_sys_flat}{paper_ll3jinc_LP2fb_v4_pT_sys_sig_flat}{paper_ll3jinc_LP2fb_v4_AllJetsLepton_Centrality_flat}{paper_ll3jinc_LP2fb_v4_ThrustEta_flat}{paper_ll3jinc_LP2fb_v4_SystemLep1Lep2_eta_flat}{legend}{The top five variables in the BDT ranked by separation power, comparing the signal and background estimate to the data in the 3-jet inclusive bin.}{FIGURE-CONTROL-3JVARIABLES1}

\SEXFIGLEG{paper_ll3jinc_LP2fb_v4_eta_sys_lepsJet1_flat}{paper_ll3jinc_LP2fb_v4_LeadingLeptonEta_flat}{paper_ll3jinc_LP2fb_v4_SystemLep1Lep2_E_flat}{paper_ll3jinc_LP2fb_v4_HT_AllJets_flat}{paper_ll3jinc_LP2fb_v4_pT_sys_lepsJet1_flat}{legend}{The 6th-10th top variables in the BDT ranked by separation power, comparing the signal and background estimate to the data in the 3-jet inclusive bin.}{FIGURE-CONTROL-3JVARIABLES2}

\SEXFIGLEG{paper_ll3jinc_LP2fb_v4_Thrust_flat}{paper_ll3jinc_LP2fb_v4_InvariantMass_Lep2Jet1_flat}{paper_ll3jinc_LP2fb_v4_SystemLep1Jet1_eta_flat}{paper_ll3jinc_LP2fb_v4_SubLeadingLeptonEta_flat}{paper_ll3jinc_LP2fb_v4_Jet1Eta_flat}{legend}{The 11th-15th top variables in the BDT ranked by separation power, comparing the signal and background estimate to the data in the 3-jet inclusive bin.}{FIGURE-CONTROL-3JVARIABLES3}

\SEXFIGLEG{paper_ll3jinc_LP2fb_v4_DeltaMinPhiLeptonLeadingJet_flat}{paper_ll3jinc_LP2fb_v4_InvariantMass_Lep1Jet1_flat}{paper_ll3jinc_LP2fb_v4_DeltaPhi_SLep1Jet1_Lep2_flat}{paper_ll3jinc_LP2fb_v4_MET_flat}{paper_ll3jinc_LP2fb_v4_DeltaEtaLeadingLeptonLeadingJet_flat}{legend}{The 16th-20th top variables in the BDT ranked by separation power, comparing the signal and background estimate to the data in the 3-jet inclusive bin.}{FIGURE-CONTROL-3JVARIABLES4}

\TRPFIGLEG{paper_ll3jinc_LP2fb_v4_DeltaRSubLeadingLeptonLeadingJet_flat}{paper_ll3jinc_LP2fb_v4_InvariantMass_MaxLepJet1_flat}{legend}{The 21st and 22nd top variables in the BDT ranked by separation power, comparing the signal and background estimate to the data in the 3-jet inclusive bin.}{FIGURE-CONTROL-3JVARIABLES5}
\newpage
\section {Dilepton subchannels}
This section contains selected variables of the different dilepton final states.  This illustrates that our backgrounds are well modeled for each of the final states individually.

\SEXFIGLEG{paper_ee1+j_LP2fb_v4_NJets_flat}{paper_ee1+j_LP2fb_v4_Jet1Pt_flat}{paper_ee1+j_LP2fb_v4_HT_AllJets_flat}{paper_ee1+j_LP2fb_v4_MET_flat}{paper_ee1+j_LP2fb_v4_LeadingLeptonPt_flat}{legend}{Distributions of variables comparing the signal and background estimate to the data  in the $ee$ channel. (a) Jet multiplicity, (b) Leading jet \pT, (c)$H_T(jet)$, (d) \MET, (e) Leading lepton \pT}{FIGURE-PRESEL-EE}
\SEXFIGLEG{paper_em1+j_LP2fb_v4_NJets_flat}{paper_em1+j_LP2fb_v4_Jet1Pt_flat}{paper_em1+j_LP2fb_v4_HT_AllJets_flat}{paper_em1+j_LP2fb_v4_MET_flat}{paper_em1+j_LP2fb_v4_LeadingLeptonPt_flat}{legend}{Distributions of variables comparing the signal and background estimate to the data  in the $e\mu$ channel. (a) Jet multiplicity, (b) Leading jet \pT, (c)$H_T(jet)$, (d) \MET, (e) Leading lepton \pT}{FIGURE-PRESEL-EM}
\SEXFIGLEG{paper_mm1+j_LP2fb_v4_NJets_flat}{paper_mm1+j_LP2fb_v4_Jet1Pt_flat}{paper_mm1+j_LP2fb_v4_HT_AllJets_flat}{paper_mm1+j_LP2fb_v4_MET_flat}{paper_mm1+j_LP2fb_v4_LeadingLeptonPt_flat}{legend}{Distributions of variables comparing the signal and background estimate to the data  in the $\mu\mu$ channel. (a) Jet multiplicity, (b) Leading jet \pT, (c)$H_T(jet)$, (d) \MET, (e) Leading lepton \pT}{FIGURE-PRESEL-MM}

%\input{BPrimeSearch}
%\clearpage
%\newpage
%% Put the body of your dissertation here. 
%% DO NOT include  the bibliography
%% If you wish to include one or more appendices, remove the "%" from the 
%% following eight (8) lines.
%\newpage
%\vspace*{\fill}
%\begin{center}
%\Huge \textbf{APPENDICIES}
%\end{center}
%\vfill
%\newpage
%\appendix
%%To start your first appendix, which will be labeled as Appendix A  
%% just type \chapter{<appendix 1 name>}
%%%%%%% A NOTE ABOUT APPENDICES %%%%%%%%%
%% Some appendices may be single spaced such as survey examples or letters.
%% Contact the Graduate School for details.
%% To single space an appendix first remove the % from 
%% the following two lines.
% \end{doublespace}
% \chapter{<appendix  name>}
%% After entering the appendix remove the % from 
%% the following line
% \begin{doublespace}
%% Any text entered now will be double spaced.
\end{doublespace}

%%Bibliography 

\bibliographystyle{atlasnote}
%\bibliographystyle{unsrt}
\bibliography{bibmain}

%% A bibliography is required. It may be made using BibTeX.
%% If it's made from scratch,
%% remove the "%" in front of \begin{thebibliography}{???}
%% replacing the ??? with the appropriate entry and 
%% remove the "%" in front of \end{thebibliography}
% \begin{thebibliography}{???}
%%  Enter the bibliography here.
% \end{thebibliography}
\end{document}
