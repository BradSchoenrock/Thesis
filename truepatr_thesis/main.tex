%% Before beginning to type your dissertation, read the formatting guide, 
%% which can be found at http://grad.msu.edu/etd/docs/formattingguide.pdf
%% Also get the latest version of  msuphddissertation.cls and the template file
%% at http://www.math.msu.edu/~weil/MSU_Ph.D._Dissertation.zip
%% Send questions to weil@math.msu.edu

\documentclass{msuphddissertation}
\graphicspath{{figures/}}
%\usepackage{todo}
%\usepackage[hide]{todo}
%\usepackage{epstopdf}
\usepackage{lineno}
\usepackage{subfigure}
\usepackage{modatlasstyle}
\usepackage{amsmath,amssymb,amsthm,paralist}
\usepackage{extra_definitions}
\usepackage{multirow}
\usepackage{mathtools}
\usepackage{authblk}
\usepackage{mathrsfs}
\usepackage{graphicx}
\usepackage{pdflscape}
\usepackage{array}
\usepackage{units}
\usepackage{bigstrut}
\usepackage{url}
%\usepackage{ulem}
\usepackage[table]{xcolor}
\renewcommand\bibname{BIBLIOGRAPHY}
%\usepackage{hyperref}
%% Include other packages you wish to use except setspace. 
%% That package is loaded automatically.
%% IMPORTANT: Load only those packages you know you will use.
%% Some packages can cause conflicts resulting in improper formatting.
\author{Patrick True} %% Put your name in full as it is officially recognized by Michigan State University here.
\title{Search for W' production in the single-top channel with the ATLAS detector} %% Put the title of your dissertation here.
\munit{}
%\munit{High Energy Physics} %% Put the name of the field of your degree (NOT department or division, or college) here.
%% For example Dynamical Systems, Psychology, String Theory, etc.

%% Put additional preamble items here.
%%%%%% LANDSCAPE %%%%%%
%% Put a page you want in landscape inside the environment msulandscape, 
%% which is defined in msuphddissertation.cls. No extra package is needed.

%%%%%%%%%%%%%%%%%%%%%%%%%%%%
%%%%%%%%  NOTE   %%%%%%%%%%%%%%
%% PREPARING A DISSERTATION WITH THIS CLASS FILE DOES NOT %%%
%% GUARANTEE THAT THE GRADUATE SCHOOL WILL APPROVE IT %%%
%%%%%%%%%%%%%%%%%%%%%%%%%%%%%%%

%%%%%%%%%%%%%%%%%%%%%%%%%%%%%%%%%%
%%%%%%%%%%%% WARNING %%%%%%%%%%%%%%%
%% The Graduate School requires that all text, including superscripts %%
%% and subscripts at all levels, as well as that in imported %%
%% graphics files be in 12 point. For that reason it's recommended %%
%% that no text be part of any imported files. %%

%% Once your document has been filed with the Graduate School,
%% if you wish to produce a version of it whose subscripts and superscripts
%% are in traditional smaller proportion, remove the "%" sign 
%% in front of following command. 
%\DeclareMathSizes{12}{12}{10}{8}
%% If your document has footnotes, remove the "%" sign 
%% in front of following command. 
%\renewcommand{\footnotesize}{\small}
%% To single space your document, remove the 
%% two commands \begin{doublespace}
%% and \end{doublespace below.

\begin{document}

\maketitlepage %%This command will produce the title page of your thesis.
\begin{abstract}
This thesis presents the search for W'$\rightarrow$tb using the LHC pp collision data collected with the ATLAS detector at a center-of-mass energy of 8 TeV. The primary backgrounds to this search are ttbar, W+jets, and multijets processes. To reduce the contributions of these backgrounds we require a leptonic final state and use Boosted Decision Trees to discriminate between signal and background-like events. This measurement gives limits on the W'$\rightarrow$tb cross-section times branching ratio and on the ratio of coupling constants g'/g as functions of the W' mass.


%formatted for stupid proquest
%This thesis discusses a search for the Standard Model single top Wt-channel process. An analysis has been performed searching for the Wt-channel process using 4.7 fb<super>-1<\super> of integrated luminosity collected with the ATLAS detector at the Large Hadron Collider. A boosted decision tree is trained using machine learning techniques to increase the separation between signal and background. A profile likelihood fit is used to measure the cross-section of the Wt-channel process at 16.8 &plusmn;2.9 (stat) &plusmn; 4.9(syst) pb, consistent with the Standard Model prediction. This fit is also used to generate pseudoexperiments to calculate the significance, finding an observed (expected) 3.3 &sigma; (3.4 &sigma;) excess over background.

%% Type your abstract here. An abstract is REQUIRED and limited to two pages.
%% The abstract must not include any figures.
\end{abstract}

%% If you wish to have a copyright page, remove the "%" in front of  \begin{copyrt}
%% and remove the "%" in front of \end{copyrt}.
%% The mandatory form of the Copyright will be generated automatically. 
%% A copyright statement is optional.

%\begin{copyrt}
%\end{copyrt}

%% If you wish to have a dedication, remove the "%" in front of
%\begin{dedication}
%% and remove the "%" in front of 
%\end{dedication}
%% A dedication must be single-spaced and 
%% centered on the page.  Both will be done automatically. 

%\begin{dedication} 
%% Type your dedication here. A dedication is optional.
%\end{dedication}
%% If you wish to have an acknowledgment, remove the "%" in front of  \begin{acknowledgment}
%% and remove the "%" in front of  \end{acknowledgment}  
\begin{acknowledgment}
Thanks Amber.
%% Type your acknowledgment here. An acknowledgment is optional.
\end{acknowledgment}
\newpage
%% If you wish to have a preface, remove the "%" in front of  \begin{preface}
%% and remove the "%" in front of  \end{preface}  
%\begin{preface}
%% Type your acknowledgment here. An acknowledgment is optional.
%\end{preface}

\TOC

%% If your document contains tables, remove the "%" in front of 
%%  the following line.
\listoftables

%% If your document contains figures, remove the "%" in front of
%% the following line.
\LOF
%% If any of your figures contain color, you must
%% include the following disclaimer in the caption of your first figure.
%% "For interpretation of the references to color in this and all other figures, 
%% the reader is referred to the electronic version of this dissertation."

%%%% LIST OF SYMBOLS AND ABBREVIATIONS %%%%
%% Such a list is possible using the environment
%% abbreviationskey
%% at the place in the document where you wish the list to appear.
% The list will be included in the TOC as KEY TO SYMBOLS AND ABBREVIATIONS
%%%%%%%%%%

\newpage
\pagenumbering{arabic}
\begin{doublespace}
%\linenumbers

\chapter{Introduction}
Science never rests. It constantly drives the boundaries of knowledge to new and unexpected realms. Through human history we have seen this knowledge progress from a practical, intuitive, and frequently incorrect understanding of the world to more rigorous models with greater predictive power than our ancestors could have ever dreamed. 
One of the themes seen throughout the history of science is the push to understand the basic building blocks of the universe. Ancient models posited four or five basic elements, made up of the most common materials found. In the 19th century, atomic theory was developed, which drove the smallest objects down to the atomic level, and then later even further when scientists discovered that atoms were made of protons, neutrons, and electrons. In the mid 20th century, scientists discovered that protons and neutrons were made of even smaller particles, which were named quarks~\cite{physicshistory}. Through the scientific process we probe the smallest scales, trying to understand the list of particles that we now consider fundamental.

Investigating these particles can be difficult, as the proton is tightly bound and high energies are required to break it apart. Even more energy is necessary to create the most massive particles we have discovered. To reach these massive energies an accelerator 24 kilometers in length, the Large Hadron Collider (LHC), has been constructed. At the LHC the proton is broken apart by accelerating two sets of protons to near the speed of light and colliding them. These collisions can create new particles, the products of which are detected at massive detectors built around the collision points. Through these collisions we study the properties of the known particles and, if we are lucky, discover new ones.

This dissertation will detail the search for a special kind of production of the most massive fundamental particle known, the top quark. This kind of production is known as the \Wtchan. In the following pages the workings of the LHC and the ATLAS detector will be discussed. From there I will explain the efforts required to go from a set of raw observations to a complete picture of the results of a collision. I will discuss how systematic uncertainties impact our measurement, and the steps we take to reduce them. Finally, the experimental and statistical methodology used to extract the measurements made will be detailed and the results will be shown.

\chapter{Theory}
\label{SECTION-THEORY}
High energy physics attempts to deal with the fundamental particles and forces of the universe, and the Standard Model (SM) of high energy physics is the theoretical framework used in this analysis. The Standard Model describes the universe as being composed of 17 fundamental particles and their interactions through three of the four fundamental forces. This analysis is a search for a \Wprime\ particle not included in the Standard Model which would be indicative of other physical theories, collectively called Beyond the Standard Model (BSM) theories. There are a wide variety of BSM theories therefore this analysis is performed in a model independent manner using an effective Lagrangian. The motivation behind searching for a \Wprime\ is explored by examining some representative BSM theories and their consequences. The focus of this chapter is not to derive the Standard Model or any BSM theories from first principles, but rather to provide a practical framework and context in which to understand this analysis and the implications of the results.

\section{The Standard Model}
\label{SECTION-THEORY-SM}
The Standard Model has provided accurate predictions of experimental observables for over 40 years. It was developed after decades of experimentation had catalogued a myriad of particle states. The properties of these states were observed to follow patterns and symmetries, and eventually these symmetries were developed into the Standard Model. The symmetries of the Standard Model are described in group theory terms as $SU(3) \otimes SU(2) \otimes U(1)$ with each symmetry giving rise to its own conservation law. The Standard Model is a quantum field theory arising from a unification of quantum mechanics and special relativity, and is mathematically described by a Lagrangian~\cite{RYDER}.

\subsection{The forces}
\label{SECTION-THEORY-SM-FORCES}
The Standard Model includes three of the four fundamental forces of nature, the electromagnetic, weak, and strong forces. The electromagnetic and weak forces can be unified into a single electroweak force similar to the unification of the electric and magnetic forces into the electromagnetic force. The electroweak force is described by the $SU(2) \otimes U(1)$ symmetry of the Standard Model and is mediated by the massless photon as well as the massive W and Z bosons. One of the greatest theoretical achievements of the Standard Model was the prediction of the existence and masses of the W and Z bosons well before their experimental discovery. The strong force is described by the $SU(3)$ symmetry of the Standard Model and is mediated by massless gluons. The strong force differs from the electroweak force in that the strong force grows with increased distance between objects while the electroweak force diminishes, which has unique consequences. There have been many attempts to unify the strong force with the electroweak force and even to include a quantum field theory of gravity, such as supersymmetry or string theory. However, there is no clear experimental evidence to support these theories and they are not considered part of the Standard Model~\cite{GARCIA}.

The electroweak force is a unification of the electric, magnetic, and weak forces. The electric and magnetic forces were unified by Maxwell in 1879. The resulting electromagnetic force decreases as the distance between objects increases, and the force is carried to infinite distance by its massless mediator, the photon. The weak force is responsible for a wide range of observed phenomena, including beta decay and the violation of parity and charge-momentum conservation. These processes can be described with phenomenological theories at low and intermediate energies, however at higher energies above a few GeV the weak theories are unstable on their own. The weak force is similar to the electromagnetic in that the force decreases as the distance between objects increases, however the weak force is mediated by the massive W and Z bosons and so has a limited range of typically 2.5 am. It is only after unification that electroweak theory provides consistent predictions for the energy ranges probed by modern accelerators of up to several TeV~\cite{GARCIA}.

The strong force is responsible for holding baryons, mesons, and nuclii together and is described in the Standard Model by quantum chromodynamics (QCD). QCD describes the strong force using a type of charge called ``color'' which comes in three colors and their anticolors. Quarks each cary either a color or anticolor charge and gluons, the mediating particles of the strong force, carry both a color and an anticolor charge. The color charge carried by the gluons is a key difference between QCD and quantum electrodynamics (QED) in which the mediating particle is charge neutral. This means that gluons are self-interacting and do not form a free gluon field and also that QCD is not locally gauge invariant and so it is a non-Abelian gauge theory. This leads to antishielding of bare color charges by the vaccuum and the force between colored objects becoming larger as the distance between them increases. This corresponds with the fact that only colorless objects are observed in nature and with the formation of particle shower ``jets'', as discussed further in Section~\ref{SECTION-THEORY-SM-PARTICLES-JETS}, from what would otherwise be bare color charged objects~\cite{GARCIA}.

\subsection{The particles}
\label{SECTION-THEORY-SM-PARTICLES}
The Standard Model contains 17 fundamental particles and their anti-particles which compose all objects. These particles can be classified into leptons, quarks, and bosons as shown in Table~\ref{TABLE-THEORY-SM-PARTICLES}. For each particle in Table~\ref{TABLE-THEORY-SM-PARTICLES} there is a corresponding anti-particle with opposite electric charge. In general a particle name or symbol refers to both the particle and its anti-particle except where they are explicitly distinguished between, thus ``electron'' refers to both electrons and positrons in general. The structure visible in Table~\ref{TABLE-THEORY-SM-PARTICLES} is not accidental and is vital to our understanding of the particles.

The six lepton flavors can be classified into 3 generations, each containing a charged lepton and a neutrino. The charged leptons all are massive and carry an electrical charge of -1, while the neutrinos are electrically neutral and their masses have not been directly observed. The current best limits on the mass of each neutrino flavor are given in Table~\ref{TABLE-THEORY-SM-PARTICLES} because the observation of neutrino flavor oscillations~\cite{Neutrino_Oscillations} implies that neutrinos are not massless but no mass measurements have been made yet. The leptons do not interact through the strong force, so the charged leptons only interact through the weak and electromagnetic forces and the neutrinos can only interact weakly. Because neutrinos can only interact through the weak force their interaction with matter, such as a detector, is rare and specialized experiments are necessary to study them. In contrast the first generation charged lepton, the electron, is easily detected through electromagnetic interactions and is readily available in nature. This difference in detectability has lead to the term ``lepton'' generally indicating the charged leptons with the neutrinos being indicated separately.

Similar to the leptons, the quarks can also be described by 3 generations, each containing 2 flavors. Each generation contains one quark with an electric charge of $\nicefrac{2}{3}$ and one quark with an electric charge of $-\nicefrac{1}{3}$, called up-type and down-type respectively based on the first generation quarks with those charges. Quarks interact through all 3 of the forces in the Standard Model and thus are readily detectable using a variety of methods. Since quarks have a color charge, bare quarks will typically form jets as described in Section~\ref{SECTION-THEORY-SM-PARTICLES-JETS} and cannot be directly measured.

The final group of particles in Table~\ref{TABLE-THEORY-SM-PARTICLES} is the bosons. The bosons all have integer spins, with the photon, gluon, W and Z bosons all being spin 1 and the Higgs boson being spin 0. The photon, W and Z bosons are the mediating particles of electroweak theory with the W and Z bosons gaining their masses through the Higgs mechanism. The Higgs mechanism adds a quartic complex scalar field potential to the theory which is locally gauge invariant. Through an appropriate choice of parameters the field is made to have a non-zero expectation value and induce spontaneous symmetry breaking in the $SU(2) \otimes U(1)$ electroweak group. After further reparameterization and an appropriate choice of gauge, what is left is the massive W and Z bosons, the massless photon, and a new massive Higgs boson, which was just recently discovered at the LHC~\cite{ATLAS_Higgs,CMS_Higgs}. The final boson is the gluon which mediates the strong force. The gluon has a color and anticolor charge which makes it self-interacting, and a bare gluon will form a jet as described in Section~\ref{SECTION-THEORY-SM-PARTICLES-JETS}.

\begin{table}[!h!tbp]
\begin{center}
\begin{tabular}{|l|c|c|c|c|}
\hline
Particle & Symbol & Mass & Charge [e] & Spin \\
\hline
\multicolumn{5}{|>{\columncolor{green!20}}c|}{\textbf{Leptons}}\\
\hline
\lc Electron         & e            & 511 KeV    & -1 & $\nicefrac{1}{2}$\\
\lc Electron Neutrino& $e_\nu$      & $< $2.05 eV & 0  & $\nicefrac{1}{2}$\\
\lc Muon             & $\mu$        & 106 MeV  & -1 & $\nicefrac{1}{2}$\\
\lc Muon Neutrino    & $\mu_\nu$    &$< $0.17 MeV& 0  & $\nicefrac{1}{2}$\\
\lc Tau              & $\tau$       & 1.78 GeV   & -1 & $\nicefrac{1}{2}$\\
\lc Tau Neutrino     & $\tau_\nu$   &$< $18.2 MeV& 0    & $\nicefrac{1}{2}$\\
\hline
\qc \multicolumn{5}{|>{\columncolor{orange!20}}c|}{\textbf{Quarks}}\\
\hline
\qc Up               & u  & 2.3 MeV    & $\nicefrac{2}{3}$  & $\nicefrac{1}{2}$\\
\qc Down             & d  & 4.8 MeV    & $-\nicefrac{1}{3}$ & $\nicefrac{1}{2}$\\
\qc Charm            & c  & 1.28 GeV   & $\nicefrac{2}{3}$  & $\nicefrac{1}{2}$\\
\qc Strange          & s  & 95 MeV    & $-\nicefrac{1}{3}$ & $\nicefrac{1}{2}$\\
\qc Top              & t  & 173 GeV    & $\nicefrac{2}{3}$  & $\nicefrac{1}{2}$\\
\qc Bottom           & b  & 4.18 GeV    & $-\nicefrac{1}{3}$ & $\nicefrac{1}{2}$\\
\hline
\bc \multicolumn{5}{|>{\columncolor{purple!20}}c|}{\textbf{Bosons}}\\
\hline 
\bc Photon                  & $\gamma$  & 0        & 0                  & 1\\
\bc \Wboson$^\pm$ Boson      & $W^\pm$    & 80.4 GeV & $\pm$ 1            & 1\\
\bc \Zboson\ Boson          & $Z$       & 91.2 GeV & 0                  & 1\\
\bc Gluon                   & $g$       & 0        & 0                  & 1\\
\bc Higgs                   & $H$       & 126 GeV  & 0                  & 0\\
\hline
\end{tabular}
\caption{The fundamental particles of the Standard Model and their properties~\cite{PDG}.}  
\label{TABLE-THEORY-SM-PARTICLES}
\end{center}
\end{table}

\subsubsection{Jets}
\label{SECTION-THEORY-SM-PARTICLES-JETS}
Jets are phenomenological objects that are formed when individual colored particles, single quarks and gluons, are produced at sufficiently high energies. As the colored particle moves away from the initial colored object it is connected to, the energy of the strong interaction between them increases until it becomes energetically favorable to produce a quark-antiquark pair from the vaccuum for the particle and the initial colored object to be bound to. This production of new hadrons is called hadronization and it absorbs a small ammount of the initial colored particle's energy. This hadron will decay and the process will repeat itself until there is insufficient energy remaining for further hadronization, creating a narrow shower of hadrons that in total have the same energy and momentum as the original colored particle. This particle shower is called a jet and it is used as the detectable stand-in for the original particle. This is a general picture of what happens to bare quarks and gluons and there are subtle differences depending on the flavor of the initial particle, in particular if the initial particle is a top or bottom quark. Because of their large mass, top quarks almost exclusively decay into a W boson and bottom quark before hadronization can occur creating a very different signal from other quarks. Bottom quarks also have unique phenomenology in that the hadron produced with the initial bottom quark has an unusually long lifetime and will travel a detectable distance before the b quark decays and further hadronization takes place, creating a secondary vertex~\cite{GARCIA}. 

\subsection{The Lagrangian}
\label{SECTION-THEORY-SM-LAGRANGIAN}
The mathematics of the Standard Model is typically formulated in terms of a Lagrangian $L$ and the Lagrangian density $\mathcal{L}$ such that $L=\int\mathcal{L}d^3x$. The Standard Model includes many different phenomena so it is useful to group the terms of the Lagrangian density by the physical motivation as seen in Equations~\ref{EQ-THEORY-SM}~\cite{ITZYKSON}.

\begin{equation}
\label{EQ-THEORY-SM}
\mathcal{L} = \mathcal{L}_{EW} + \mathcal{L}_{QCD} + \mathcal{L}_{Yuk}
\end{equation}

The Lagrangian density that describes the electroweak force is given in Equation~\ref{EQ-THEORY-SM-EW}. 

\begin{equation}
\label{EQ-THEORY-SM-EW}
\mathcal{L}_{EW} = -\frac{1}{4}F_{\mu\nu a}F^{\mu\nu}_{a} + D_{\mu}\phi D^{\mu}\phi - \mu^{2}\phi^{2} - \lambda(\phi^{2})^{2}
\end{equation}

\noindent
The first term describes the electroweak interactions with the index $a$ running over the photon, W and Z bosons. $F_{\mu\nu}$ is the electroweak field tensor and is defined as:

\begin{equation}
\label{EQ-THEORY-SM-FMUNU}
F_{\mu\nu} = \partial_{\mu}A_{\nu} - \partial_{\nu}A_{\mu} - [A_{\mu},A_{\nu}]
\end{equation}

\noindent
Since electroweak theory is abelian, $[A_{\mu},A_{\nu}]\ =\ 0$ and $F_{\mu\nu}$ is simplified. The last three terms of Equation~\ref{EQ-THEORY-SM-EW} describe the Higgs field. The first of these is the kinetic term where $D_{\mu}$ is the covariant derivative defined as:

\begin{equation}
\label{EQ-THEORY-SM-DMU}
D_{\mu} = \partial_{\mu} - gA_{\mu}
\end{equation}

\noindent
The final two terms of the electroweak Lagrangian density describe the Higgs potential.

The QCD Lagrangian density is given in Equation~\ref{EQ-THEORY-SM-QCD}.

\begin{equation}
\label{EQ-THEORY-SM-QCD}
\mathcal{L}_{QCD} = \sum\limits_j\bar{\psi}(i\gamma^{\mu}D_{\mu} - m_j)\psi - \frac{1}{4}G_{\mu\nu a}G_{a}^{\mu\nu}
\end{equation}

\noindent
The first term describes the quarks with index j running over all six of the quarks with masses $m_{j}$. The covariant derivative $D_{\mu}$ is similar to Equation~\ref{EQ-THEORY-SM-DMU} but now contains eight gauge fields corresponding to the eight gluons denoted by the index $a$:

\begin{equation}
\label{EQ-THEORY-SM-DMUA}
D_{\mu} = \partial_{\mu} - g_{a}A_{\mu a}
\end{equation}

\noindent
The second term of the QCD Lagrangian density describes the gluons. $G_{\mu\nu a}$ ia analogous to $F_{\mu\nu}$ in the electroweak Lagrangian density, but for each of the eight gauge fields.

 \begin{equation}
\label{EQ-THEORY-SM-GMUNUA}
G_{\mu\nu a} = \partial_{\mu}A_{\nu a} - \partial_{\nu}A_{\mu a} - [A_{\mu a},A_{\nu a}]
\end{equation}

\noindent
Since QCD is a non-abelian theory, $[A_{\mu a},A_{\nu a}]\ \neq\ 0$ and $G_{\mu\nu a}$ does not simplify in the same way as $F_{\mu\nu}$.

The Yukawa Lagrangian density describes the fermions and is given in Equation~\ref{EQ-THEORY-SM-YUK}.

\begin{equation}
\label{EQ-THEORY-SM-YUK}
\mathcal{L}_{Yuk} = \sum\limits_a\bar{\psi}(i\gamma^{\mu}D_{\mu} - G_a\phi)\psi
\end{equation}

\noindent
The index a runs over the fermions, with the covariant derivative defined in Equation~\ref{EQ-THEORY-SM-DMU}. The mass of each fermion is determined by $G_a\phi$ where $G_a$ is the fermion's coupling to the Higgs field $\phi$, and in this way the Higgs field gives the fermions their masses. 

\section{Beyond the Standard Model theories}
\label{SECTION-THEORY-BSM}
While the Standard Model has described the observations of particle physics experiments for over 40 years, there are known problems with the theory. The Standard Model does not include gravity, which has been experimentally verified many times. There is no mechanism to produce the amount of matter-antimatter asymmetry that is observed in the universe. The Standard Model does not include a suitable dark matter particle to match astronomical observations. While mathematically possible, the observed masses of the W and Z bosons require very precise cancellations of parameters which seem unnatural. While this list is by no means exhaustive, there have been decades of work to solve these problems with BSM theories.

\subsection{Extensions of $SU(2) \otimes U(1)$}
\label{SECTION-THEORY-BSM-EXTENSIONS}
Theories that include a \Wprime\ boson often extend the $SU(2) \otimes U(1)$ electroweak symmetry which describes the W boson. The simplest such extension is $SU(2) \otimes SU(2) \otimes U(1)$, where the new $SU(2)$ can be a right handed extension of the left handed $SU(2)$ group which describes the Standard Model weak interactions or some other $SU(2)$ symmetry. The $SU(2) \otimes U(1)$ symmetry can also be extended by embedding the $SU(2)$ into a group of higher degree, resulting in symmetries such as $SU(3) \otimes U(1)$ or $SU(4) \otimes U(1)$. Each of these extensions contains a myriad of specific theories with different coupling structures and a variety of experimental predictions~\cite{PDG}. Since there is currently no strong experimental evidence to distinguish between these theories, this analysis does not assume any specific theory but instead uses an effective Lagrangian approach.

\subsection{Effective Lagrangian approach}
\label{SECTION-THEORY-BSM-EFFECTIVE}
In all of these theories the \Wprime\ boson is described by a Lagrangian density term of the form given in Equation~\ref{EQ-THEORY-BSM-EFFECTIVE}.

\begin{equation}
\label{EQ-THEORY-BSM-EFFECTIVE}
\mathcal{L}_{\Wprime} = \frac{1}{2\sqrt{2}}V^\prime_{ij}W^\prime_{\mu}\bar{f}^i\gamma^{\mu}(g^\prime_R(1+\gamma_5)+g^\prime_L(1-\gamma_5))f^j
\end{equation}

\noindent
This Lagrangian density includes arbitrary right handed and left handed coupling strengths $g^\prime_R$ and $g^\prime_L$ respectively. These coupling strengths are a common metric across all models regardless of how they are determined within each theory, and thus they are a model independant parameter which can be experimentally measured or constrained. For this analysis we use benchmark \WprimeR\ and \WprimeL\ models where $g^\prime_R$ and $g^\prime_L$ are equal to $g_L$ for the Standard Model W.

\chapter{ATLAS and the LHC}
\label{SECTION-ATLAS}
The search for \Wprimechan\ requires a very large and extensive experimental setup. In order to set limits competitive to those currently in the literature, particles need to be collided with atleast several TeV of energy, and in order to correctly identify \Wprime\ events from the background the products of these collisions need to be carefully measured. The ATLAS (A Toroidal Lhc ApparatuS) experiment meets these criteria, it is the largest collider detector ever built and is capable of very precise measurements of the products of particle collisions. The collisions it measures are produced by the Large Hadron Collider (LHC) which is designed to collide particles with up to 14 TeV center of mass energy.

\section{The Large Hadron Collider}
\label{SECTION-ATLAS-LHC}
\subsection{The Accelerator Chain}
The LHC is only the final accelerator in a chain designed to take ions from rest, accelerate and collide them at up to 14 TeV center of mass energy. This acceleration occurs in stages, with protons being accelerated through a separate chain than other ions such as lead. Since my analysis uses only proton-proton collisions I will detail only their acceleration here. The proton source is a bottle of hydrogen gas, which is stripped of its electrons and accelerated to a 50 MeV proton beam by the Linac2 linear accelerator~\cite{Linac2}. This 50 MeV proton beam is then passed to the Proton Synchroton Booster (PSB) which accelerates the beam to 1.4 GeV in four superimposed synchrotron rings before injecting the bunches into the Proton Synchrotron (PS). By adjusting the timings of the four superimposed rings of the PSB and varying which rings the PS is filled from a plethora of bunch patterns can be selected based on the current operating goals~\cite{PSB}. The PS accelerates the proton beam from 1.4 GeV to 25 GeV in a 628 meter circumference synchrotron. The PS also does the final bunch splitting, creating the bunch pattern which will be kept through the remainder of the beam acceleration and collision~\cite{PS}. After being accelerated and bunched by the PS, the beams enter the Super Proton Synchrotron (SPS) for final acceleration and tuning before injection into the LHC. The SPS is a synchrotron nearly 7 km in circumference which accelerates the proton beam up to 450 GeV before injecting it into the LHC~\cite{SPS}.

The final stage of the accelerator chain is the LHC itself. The LHC is a 27 km circumference synchrotron with 2 superimposed rings which resides in what was previously the Large Electron-Positron collider (LEP) tunnel at CERN. It consists of 1104 superconducting dipole magnets designed to reach a peak field of 8.33 T to bend the proton beams around the ring, and 384 quadrupole magnets per ring to control the focusing of the beams. Each ring has a further 536 quadrupole, 1608 sextupole, 784 octupole, and 616 decapole magnets to control the beams and correct instabilities in the beams due to couplings during acceleration and collision. Nominally the LHC is designed to collide proton bunches at ATLAS every 25 nanoseconds with a center of mass energy of 14 TeV, however it is still early in the LHC program and these were not the conditions that the 2012 dataset was taken under. Due to difficulties with the magnet fault protection system the collisions took place with 8 TeV center of mass energy, and because the accelerator and beams are being actively studied a variety of beam configurations were used with bunches separated by 25-75 nanoseconds~\cite{LHC}.   

\section{The ATLAS detector}
\label{SECTION-ATLAS-DET}
The ATLAS detector is one of two large general purpose experiments which studies collisions produced by the LHC. It is designed to able to perform a wide variety of searches and measurements by collecting as much information as possible about the products of each collision. ATLAS uses a multilayered design that has become standard for large collider experiments and can be seen in Figure~\ref{FIGURE-ATLAS}. The innermost portion is called the inner detector which provides fine granularity tracking of charged particles. Moving radially outwards from the interaction point the next detectors are the calorimeters which measure the energy of the particles, and the outermost portion of the detector consists of the muon system which detects and tracks muons traveling through ATLAS. Each of these portions of ATLAS are made up of sub-detector systems designed to work together with the other systems and provide more information than any single technology detector~\cite{TDR1}.

\VLARGEFIG{Atlas}{Cutaway diagram of the ATLAS detector~\cite{Figure-Atlas}.}{FIGURE-ATLAS}

\subsection{Detector geometry}
\label{SECTION-ATLAS-GEO}
Before detailing each detector system that makes up ATLAS it is useful to discuss the coordinate system used in the experiment. The center of the detector is taken to be the origin and the z-axis extends along the beam line with positive being counterclockwise around the LHC when viewed from above. The x-axis points towards the center of the LHC and the y-axis points vertically upwards. While this forms a complete basis to describe the detector and it is sometimes used, it is not the most common coordinate system. Ignoring gravitational effects all directions transverse to the beams are equivalent and can be described by an angle $\phi$ taken to be 0 along the x-axis and increasing right-handedly with respect to the z-axis. The angle from the beam line is a common parameter for describing decays. However because objects are produced with Lorentz boosts in the z direction ranging from 0 to nearly 1 it is more useful to use a relativistic invariant to describe this angle. The equation for the Lorentz invariant rapidity ($y$) is:

\begin{equation}
y = \frac{1}{2}ln\left(\frac{E+p_z}{E-p_z}\right)
\end{equation}

While useful, the rapidity of a particle is dependent on the particle's mass and a different rapidity coordinate system to describe the detector would be necessary for each mass. Almost all particles produced by the LHC have $m << E$ so we can calculate rapidity with $m=0$ and it is approximately the rapidity for all particles produced by the LHC. This is called the pseudorapidity ($\eta$).

\begin{equation}
 \eta = \frac{1}{2}ln\left(\frac{\left|\vec{p}\right|+p_z}{\left|\vec{p}\right|-p_z}\right)
\end{equation}

\noindent
Which can be rewritten using the angle from the z-axis ($\theta$) as:

\begin{equation}
 \eta = -ln\left(tan\left(\frac{\theta}{2}\right)\right)
\end{equation}

Thus pseudorapidity is a purely geometric quantity, with $\eta = 0$ corresponding to the transverse plane and $|\eta| = \infty$ corresponding to the beamline. In detector parlance regions with small $|\eta|$ are called ``central'' and regions with larger $|\eta|$ are called ``forward.''


\subsection{Magnet system}
\label{SECTION-ATLAS-MAGNETS}
The ATLAS detector has three large superconducting magnet systems, the superconducting solenoid, the barrel toroid, and the endcap toroids as shown in Figure~\ref{FIGURE-MAGNETS}. The purpose of these magnets is to bend the path of charged particles as they propagate through the detector. With careful tracking of a charged particle's path through the magnetic field, it is possible to determine the particle's momentum~\cite{TDR1}.

The superconducting solenoid is a cylinder 5.3 m long and 2.63 m in diameter. It has 1173 turns of superconducting wire in a single layer along its length with an operating current of 7.6 kA. The inner volume contained by the solenoid has a central magnetic field of 2 T with a peak field of 2.6 T at the superconducting wire. The solenoid is designed to be as thin as possible in order to minimize the interaction between itself and particles from physics events. The particles pass through the 19 cm thick (at most 0.66 radiation lengths) solenoid before they enter the calorimeters~\cite{MAGNET}. 

The barrel toroid consists of 8 coils each of which is a 25.3 m long and 5.35 m wide ``racetrack'' design. These magnets run the length of ATLAS with their long dimension running parallel to z and their short dimension running radially. They are spaced evenly around the detector with their outer edge at a radius of 10.05 m. Each coil contains 120 turns of superconducting wire with an operating current of 20.5 kA which produces a peak field of 3.9 T~\cite{MAGNET}.

The two endcap toroids complete the ATLAS magnet system. Each endcap contains 8 coils of a racetrack design similar to the barrel toroid, however these coils are 4.5 m in the radial direction and 5 m in the z direction. The endcap coils are offset from the barrel toroid coils by $22.5^\circ$ in $\phi$ so that they bisect the angle between adjacent barrel toroid coils. They are aligned in z to share a common outer edge with the barrel toroid, and are placed radially from 0.825 m to 5.35 m. With 116 turns per coil of superconducting wire carrying 20 kA, the endcap toroids produce a peak field of 4.1 T~\cite{MAGNET}.

\VLARGEFIG{Magnets}{Illustration of the ATLAS magnet system, showing the barrel solenoid, barrel toroid, and endcap toroid coils~\cite{ATLAS-EXP}.}{FIGURE-MAGNETS}


\subsection{Inner detector}
\label{SECTION-ATLAS-ID}
The ATLAS inner detector is designed to provide excellent tracking information for charged particles with $|\eta| < 2.5$ produced by the LHC and is comprised of three concentric subsystems as shown in Figure~\ref{FIGURE-ATLAS-ID}. The pixel detector is nearest the beamline and provides the most precise position information with 97 million channels across three layers in the barrel region and with 43 million channels across five disk layers at both ends. Moving radially outwards from the beamline the next subsystem is the semiconductor tracker (SCT) which uses eight layers of thin slicon microstrip sensors in the barrel and 44 sensor layers in each endcap, with alternating layers at a 40 mrad angle to each other to allow full determination of position. The final subsystem of the inner detector is the transition radiation tracker (TRT) which is a straw tube system consisting of a barrel section containing axial straws and 18 radial straw wheel segments in each endcap, designed so that most particles will transverse 36 detecting straws~\cite{TDR1}.

\VLARGEFIG{InnerDetector}{Cutaway diagram of the ATLAS inner detector~\cite{Figure-InnerDetector}.}{FIGURE-ATLAS-ID}

\subsubsection{Pixel detector}
\label{SECTION-ATLAS-ID-PIXEL}
The pixel detector has the highest granularity and offers the best positioning and tracking information of charged particles in ATLAS. The system contains three barrel layers with three transverse disk layers at each end. The barrel layers are all 801 mm long, with the innermost layer located at a mean radius of 50.5 mm, the middle layer at 88.5 mm, and the outermost layer at 122.5 mm. The disk layers are all identical annuli with an inner radius of 89 mm and outer radius of 150 mm. These disks are placed at a mean $|z|$ of 495 mm, 580 mm, and 650 mm. This gives the pixel detector a total detecting area of 1.7 $m^{2}$ and coverage of $|\eta| \le 2.5$~\cite{TDR1}~\cite{PIXEL}. 

The active medium in the detector is silicon sensor cells 50 $\mu$m $\times$ 400 $\mu$m in size. In the barrel layers the long dimension is in the z direction and in the disk layers the long dimension is radial. These sensor cells are bump bonded to readout chips with each chip reading an 18 $\times$ 160 cell array. The signal from each cell is amplified and compared to a programmable threshold on each chip. If the signal exceeds the threshold the location is stored in a buffer on the chip to be read out via optical link in the case of a level 1 trigger acceptance, as detailed in Section~\ref{SECTION-ATLAS-TDAQ}~\cite{TDR1,PIXEL}.

\subsubsection{Semiconductor tracker}
\label{SECTION-ATLAS-ID-SCT}
While the pixel detector provides the highest resolution for tracking particles, the technology is not cost-effective to use to cover the larger areas corresponding to larger radii. The next subdetector is the semiconductor tracker (SCT) with four cylinders at radii of 299 mm, 371 mm, 443 mm, and 514 mm and nine disks at both endcaps with mean $|z|$ of 853.8 mm - 2720.2 mm. Each cylinder is 1492 mm long and contains two layers of silicon microstrip sensors at a 40 mrad angle to each other. The microstrip sensors are each 63.6 mm wide and 64 mm long rectangles divided into 768 microstrips each 16 $\mu$m wide~\cite{SCT_Barrel}. 
The endcap disks have a more complicated geometry with each disk containing 1-3 ``rings'' of sensor modules depending on position. Each endcap module has two layers of microstrip silicon sensors at a 40 mrad angle to eachother, similar to the barrel modules, however unlike the barrel modules the endcap microstrip sensors are tapered to form trapezoidal segments rather than rectangular. This tapering also causes variable microstrip widths of 16 $\mu$m - 20 $\mu$m. Each module for both the endcap and barrel regions has four silicon sensors (two per layer) attached to central logic circuits which amplify the signals from each microstrip and compare them to a programable threshold. Similar to the pixel detector, the channels with signals exceeding the threshold are stored in a buffer to be read out if the event is accepted by the level 1 trigger system~\cite{TDR1,SCT_Barrel,SCT_Endcap}.

%While the pixel detector provides the highest resolution for tracking particles, the technology is not feasible to use to cover the larger areas corresponding to larger radii, thus the next subdetector is the SCT with 4 cylinders at radii of 299 mm, 371 mm, 443 mm, and 514 mm and 9 disks at both endcaps with mean $|z|$ of 853.8 mm, 934 mm, 1091.5 mm, 1299.9 mm, 1399.7 mm, 1771.4 mm, 2115.2 mm, 2505 mm, and 2720.2 mm.

\subsubsection{Transition radiation tracker}
\label{SECTION-ATLAS-ID-TRT}
The final subdetector of the inner detector is the TRT. While both the pixel detector and SCT use variations of silicon detector technology, the TRT uses a modification of drift tube technology to detect particles. The TRT is divided into one barrel and two endcap sections. The barrel section consists of 52544 straw-tubes arranged in 73 layers parallel to the beam axis. Each straw-tube is a drift tube 1441 mm long and 4 mm in diameter and contains a 70\% Xe, 27\% $CO_{2}$ and 3\% $O_{2}$ gas mixture. Each straw-tube also contains a central 31 $\mu$m diameter gold-plated W-Re wire which is held at a potential of -1.53 kV relative to the straw-tube wall~\cite{TRT_Barrel}. 

The endcaps are each made up of 122,880 straw-tubes arranged radially in 160 layers. These straw-tubes are identical to those used in the barrel except that they are each 370 mm long. These endcap straw-tubes are bundled into modules called wheels of 8 layers each, and the wheels are distributed with $848 mm \le |z| \le 2710 mm$ to give nearly uniform coverage in $\eta$~\cite{TRT_Endcap}. Overall the barrel covers $|\eta| < 1.0$ and the endcaps cover $1.0 < |\eta| < 2.0$ with most particles traversing a total of 30 straw-tubes. 

As a charged particle traverses each straw it causes primary ionization within the gas, which undergoes avalanche multiplication as it accumulates toward the wire giving an amplification factor of $2.5 \times 10^{4}$ with the operating gas mixture and voltage. The unique feature of the ATLAS TRT is that surrounding each straw-tube is a layer of transition radiation (TR) material. The TR material is made up of many layers of polypropylene and polyethylene, and is designed to maximize the production of transition radiation produced by charged particles traversing the boundary between the two materials with different dielectric constants. The transition radiation produced in the TR material is generally a soft x-ray photon which is absorbed by the xenon in the straw-tubes, ionizing the xenon and producing an energy cascade much larger than a typical ionizing particle does when traversing a straw tube. This is particularly useful because electrons and charged pions are difficult to discriminate between, however the energy of transition radiation is proportional to $\gamma=E/m$ which allows for an additional rejection factor of 50-100 depending on the electron quality definition as described in Section~\ref{SECTION-OBJ-EL}~\cite{TDR1,TRT}.

\subsection{Calorimeters}
\label{SECTION-ATLAS-CALO}
Having measured the positions of particles as precisely as possible in the inner detector, the next detector systems particles will encounter are designed to measure their energy. The electromagnetic (EM) calorimeter is nearest the beamline covering $|\eta| < 3.2$ and uses liquid argon (LAr) technology with lead absorber plates in a distinctive accordion pattern. The hadronic calorimeter resides around the EM calorimeter, using scintillating tiles with iron absorbers in the barrel region of $|\eta| < 1.7$ and using LAr technology with copper and tungsten absorbers in the $1.5 < |\eta| < 3.2$ and $3.1 < |\eta| < 4.9$ regions respectively. The layout of these systems can be seen in Figure~\ref{FIGURE-ATLAS-CALO}. It is important that the calorimeter system provides the best containment of particles possible while maintaining good energy resolution so that the total energy of events can be determined~\cite{TDR1}.
  
\VLARGEFIG{Calorimeters}{Cutaway diagram of the ATLAS calorimeter systems~\cite{Figure-Calo}.}{FIGURE-ATLAS-CALO}

\subsubsection{Electromagnetic Calorimeter}
\label{SECTION-ATLAS-CALO-EM}
The ATLAS EM calorimeter is divided into a barrel section, a presampler, and two endcap sections. The barrel calorimeter is made up of two half barrels which surround the superconducting solenoid and covers the range $|\eta| < 1.475$ with one half barrel covering $\eta > 0$ and the other half barrel covering $\eta < 0$.  Each half barrel is a cylinder 3.2 m long and has a 2.8 m inner radius and 4 m outer radius. There are 1024 accordion-shaped absorber plates arrayed radially in each half barrel with the oscillations increasing in amplitude as radius increases which provides uniform density in $\phi$. The absorbers are 1.53 mm thick lead for $|\eta| < 0.8$ and 1.13 mm thick lead for $0.8 < |\eta| < 1.475$ with 0.2 mm thick stainless steel sheets glued to each side to provide structural support. Centered between consecutive absorbers is a readout electrode held at 2 kV relative to the absorber with the 2 mm gap between the electrode and absorber filled with liquid argon. Electrically charged incident particles will shower via Bremsstrahlung in the absorber and this shower will exit the thin absorber layer and enter the liquid argon. The shower ionizes the liquid argon and this ionization is collected at the electrode where it is amplified and read out at both the inner and outer edges of the calorimeter~\cite{EMCAL_Barrel}. 

The presampler is a 22 mm thick detector covering the interior of the barrel calorimeter. It is similar to the barrel detector in that uses liquid argon with 1.9-2.0 mm gaps between electrodes, however unlike the barrel calorimeter the presampler has no absorbers. The purpose of the presampler is to measure showers produced by interactions with the material between the interaction point and the EM calorimeter and improve the energy resolution of the EM calorimeter~\cite{EMCAL_Presampler}. 

The endcap sections are each a wheel 630 mm thick with a 330 mm inner radius and 2098 mm outer radius which covers $1.375 < |\eta| < 3.2$. Each wheel is further divided into an inner wheel and an outer wheel by a 3 mm gap located at $|\eta| = 2.5$. The endcaps have a design similar to the barrel calorimeter, with accordion-shaped absorber plates placed radially and readout electrodes interleaved. Each outer wheel contains 768 absorbers of 1.7 mm thick lead, while the inner wheels each have 256 absorbers of 2.2 mm thick lead. The endcap sections also have an 11 mm thick presampler of similar design to that used in in the barrel section~\cite{TDR1,EMCAL_Endcap}.

\subsubsection{Hadronic Calorimeter}
\label{SECTION-ATLAS-CALO-HAD}
The hadronic calorimeter makes up the remainder of the ATLAS calorimeter system and is comprised of four sub-systems; the tile barrel calorimeter and tile extended barrel calorimeter are both based on using iron absorber plates with plastic scintillator tiles interspersed, while the hadronic endcap calorimeter and the forward calorimeter are both based on LAr technology similar to the EM calorimeter. The tile barrel calorimeter has an inner radius of 1144 mm and an outer radius of 2115mm and a length of 5640 mm. The tile barrel calorimeter consists of 64 modules, each of which is a radial slice of the detector. Each module consists of 64 steel plates that are each 5 mm thick and run the radial length of the module. Between consecutive full length plates there are 11 alternating layers of scintillating plastic tiles and steel spacing tiles which is 4 mm thick. These layers progressively increase in length from the inner radius to the outer radius in order to provide high precision measurements while maintaining the necessary depth of interaction lengths to contain very energetic jets. A 1.5 mm gap along both edges of each alternating layer contains a wavelength shifting fiber which carries the scintillation light to photomultiplier tubes located along the outer radius of the modules where the signals are amplified, digitized and processed by readout electronics. The extended barrel calorimeter consists of two sections, one at each end of the tile barrel calorimeter. Each of these sections is 2900 mm long but otherwise follows the same general design as the tile barrel calorimeter with minor modifications to 12 of the 64 modules in each extended barrel calorimeter to accommodate necessary structural supports for the LAr cryostat. A gap region exists between the tile barrel calorimeter and the extended barrel calorimeter on each side. This gap is necessary to provide services to the LAr calorimeters and the inner detector, and while approximately 750 mm wide it is adjusted as needed to accomodate these necessary services. The gap region contains the intermediate tile calorimeter which consists of an irregular arrangement of absorber and scintilator tiles used to estimate the energy lost in the dead material of the gap region. In total the tile barrel calorimeter covers the $|\eta| < 1.0$ region while the extended barrel calorimeter covers $0.8 < |\eta| < 1.7$ and the intermediate tile calorimeter covers $0.8 < |\eta| < 1.0$~\cite{HCAL_Barrel}.

The hadronic endcap calorimeter consists of two wheels located outside of the electromegnetic endcap calorimeters at both ends of the detector, for a total of four wheels. Each of these wheels is further made up of 32 identical wedge-shaped modules. The front wheel on each side starts at a $|z|$ of 4,277 mm and is 816.5 mm in length. The rear wheels start at a $|z|$ of 5134 mm with a length of 961 mm, leaving a 2 mm gap between the wheels. Each front wheel module contains 25 parallel copper plates which are each 25 mm thick and are evenly spaced in z and arrayed transverse to the beamline. The rear wheel modules each contain 17 parallel copper plates which are 50 mm thick and are also evenly spaced in z and arrayed transverse to the beamline. This means that all of the plates are separated by 8.5 mm gaps which are filled with liquid argon. Three electrodes are evenly spaced in each gap with the outer two electrodes held at 2000 V and the central electrode providing the signal for amplification and processing. All of the plates have an outer radius of 2090 mm and the first 9 plates of the front wheels has an inner radius of 372 mm while the remaining plates all have an inner radius of 475 mm, providing coverage in the region $1.5 < |\eta| < 3.2$~\cite{HCAL_HEC}. 

The final sub-system of the ATLAS hadronic calorimeter is the forward calorimeter. This system covers the region $3.1 < |\eta| < 4.9$ and resides entirely inside the 475 mm inner radius of the hadronic endcap calorimeter. This region is extremely harsh with very high radiation densities and many design compromises were necessary to ensure the forward calorimeter could survive and operate in this environment. The forward calorimeter is composed of three sections at each end of the detector. These sections are cylindrical and are arranged coaxially along the length of the beam pipe as seen in Figure~\ref{FIGURE-ATLAS-CALO}. Each section is made of an absorber matrix cylinder with holes along its length in a honeycomb pattern. Each of these holes contains a thin walled electrode tube and an electrode rod of slightly smaller radius. The small gap between the electrode rod and tube is filled with liquid argon and the electrode rod is held at 250 V relative to the electrode tube. In the section on either side of the detector which is nearest the interaction point the absorber matrix and the electrode rod are both made of copper. In the remaining sections the absorber matrix and electrode rods are made of tungsten. These materials were choosen due to their densities as well as their thermal properties, ability to be produced to the necessary specifications, and hadronic shower sizes. In all of the modules the electrode tube is made of copper and the electrical signal is read out from each absorber rod for amplification and processing. The liquid argon gaps are smaller than is common in LAr detectors, being 269 $\mu m$, 376 $\mu m$, and 508 $\mu m$ in the three sections at each end of the detector and increasing with the distance from the interaction point. This is necessary to prevent charge accumulation in the liquid argonwhich would degrade performance and is caused by the high radiation density of the region, which decreases with distance from the interaction point. The overall layout of the segments is approximately projective from the interaction point, with the inner radius of the segments increasing proportional to $|z|$, the electrode spacing increasing from 7.50 mm to 9.00 mm across the three segments, and the number of electrodes decreasing from 12,260 tubes in each module nearest the interaction point to 8224 electrodes in each module furthest from the interation point~\cite{HCAL_FCAL1,HCAL_FCAL2}.

\subsection{Muon Spectrometer}
\label{SECTION-ATLAS-MUON}

The ATLAS Muon Spectrometer (MS) is the outermost of the ATLAS detector systems and accounts for a majority of the detector's volume. The purpose of this system is to detect muons as they traverse the ATLAS detector and making precision position measurements at three different detector layers to calculate the momentum of each muon based on the curvature of the muon's trajectory as it travels through the ATLAS magnetic field. To accomplish this goal the muon spectrometer has four subsystems which employ differing detector technologies as needed in the various regions of the ATLAS detector. Monitored Drift Tubes (MDTs) and Cathode Strip Chambers (CSCs) provide high precision tracking information over the large area of the muon spectrometer in three concentric layers, called stations. The MDT system uses gas drift tube technology and covers the region $|\eta| < 2.7$, while the CSC system uses multiwire proportional chambers with a cathode strip readout and covers $2.0 < |\eta| < 2.7$. Both the MDT and CSC systems have long response times and are not capable of being used in the Level 1 trigger system as described in Section~\ref{SECTION-ATLAS-TDAQ}, so two additional muon spectrometer systems are employed for the initial detection of muons. The Resistive Plate Chamber (RPC) covers the region $|\eta| < 1.05$ using resistive plate capacitors which locally discharge when their internal gas is ionized while the Thin Gap Chamber (TGC) systems covers $1.05 < |\eta| < 2.4$ using multiwire proportional chambers with a smaller geometry than the CSC system~\cite{MS_TDR}. The overall layout of these systems is shown in Figure~\ref{FIGURE-ATLAS-MUON}.

\VLARGEFIG{Muon}{Cutaway diagram of the ATLAS muon spectrometer and toroid magnet systems~\cite{Figure-Muon}.}{FIGURE-ATLAS-MUON}

\subsubsection{Monitored drift tubes}
\label{SECTION-ATLAS-MUON-MDT}
The monitored drift tube (MDT) chambers provide the majority of the precision muon tracking capability in ATLAS. MDT modules are arranged into barrel and end-cap regions, with the barrel composed of three concentric cylinders with radii of 5, 7.5, and 10 m with coverage of $|\eta| < 1.0$ and the endcap regions containing four disks each at $|z|$ of 7, 10, 14, 22 m respectively and covering the range $1.0 < |\eta| < 2.7$. Each chamber is composed of two sets of drift tube multilayers on either side of a rigid support structure which is 150 mm thick. The multilayers in MDT chambers in the stations nearest the interaction point have four layers of drift tubes while all other MDT chamber multilayers have three layers of drift tubes. Each drift tube is 30 mm in diameter and is filled with a 91$\%$ Ar, 4$\%$ $N_2$, 5$\%$ $CH_4$ gas mixture. Each tube is read out from a central 50 $\mu m$ tungsten wire. The wire is held at 3270 V and gives a spacial resolution of of 80 $\mu m$ with a maximum drift time of approximately 500 ns. Because of this long drift time it is necessary to correlate the signals from the MDT system with corresponding signals in the RPC and TGC systems which provide much more prompt results in order to determine which bunch crossing the MDT signals originate from~\cite{MS_TDR}.

\subsubsection{Cathode strip chambers}
\label{SECTION-ATLAS-MUON-CSC}
The cathode strip chambers (CSCs) are multiwire proportional chambers used for precision muon position measurements in the region of highest radiation density, $2.0 < |\eta| < 2.7$ in the station nearest to the interaction point. Similar to the MDT, the CSC consists of two multilayers with each multilayer containing four monolayers. Each monolayer is a 5.08 mm gas gap containing a 30$\%$ Ar, 50$\%$ $CO_2$ and 20$\%$ $CF_4$ gas mixture. In the center of each gas gap ia a plane of parallel anode wires. The anode wires are 30 $\mu m$ diameter tungsten wires separated by 2.54 mm and held at 2600 V. The walls forming the gap are copper-clad and etched to form thin cathode strips. The cathode strips on one of the walls run orthogonal to the anode wires and provide the precision coordinate measurement, while the cathode strips on the other wall are coarser and run parallel to the anode wires to provide the transverse coordinate measurement. For the precision strips it is only necessary to read out every third strip in order to exceed the resolution of the MDT by using charge interpolation between the strips, and these read-out strips are separated by 5.08 mm. The final resolution in the bending direction is 60 $\mu m$ for a monolayer~\cite{MS_TDR}.
 
\subsubsection{Resistive plate chambers}
\label{SECTION-ATLAS-MUON-RPC}
The resistive plate chambers (RPCs) are designed to complement the MDT system in the barrel region ($|\eta| < 1.05$). Each RPC chamber is a simple design; two resistive plates form a capacitor and are held at 8900 V with a 2 mm gap filled with 97$\%$ $C_2H_2F_4$ and 3$\%$ $C_4H_{10}$. An incident muon will ionize the gas and cause a local discharge of the capacitor. This discharge is read out via capacitative coupling by metal strips running in orthoganal directions on both sides of the resistive capacitor. The RPC chambers are placed two thick at each of three stations. The two middle stations are directly inside and outside of the middle MDT barrel station, and the far station is directly inside of the outer MDT barrel station. This system provides prompt muon detection with a delay of less than 10 ns and a timing uncertainty of 1.5 ns. The signal position is known to within a resolution of 1 cm which is used by the level 1 trigger system and provides a complementary position measurement to the MDT~\cite{MS_TDR}.

\subsubsection{Thin gap chambers}
\label{SECTION-ATLAS-MUON-TGC}
The thin gap chambers (TGCs) fill a role similar to the RPC, prompt detection of muons for use in level 1 triggering and a complementary position measurement to the MDT, but in the endcap region ($1.05 < |\eta| < 2.4$). The TGCs are based on multiwire proportional chamber technology similar to the CSCs but with a smaller geometry and faster readout time. Each TGC gas gap is 2.8 mm wide and is filled with a highly quenching 55$\%$ $CO_2$ and 45$\%$ $n-C_{5}H_{12}$ gas mixture. A central plane of 50 $\mu m$ tungsten anode wires are spaced 1.8 mm apart and are held at 3100 V. The signals from these wires are read out with 4-20 wires forming an individual readout channel depending on $\eta$. Signals are also read out from etched copper strips on one of the walls of each gap to provide a measurement of the azimuthal angle for each track. This configuration gives each gap a position resolution of approximately 9 mm and a time response of 7 ns, which is sufficient for bunch identification and use by the level 1 trigger system. TGC modules are made up of either gas gap doublets or triplets with 20 mm of separation between consecutive gas gaps. The inner wheel at $|z|\ =\ 7\ m$ of each endcap has a layer of doublet chambers and the middle layer wheel at $|z|\ =\ 14\ m$ has two layers of doublet chambers and a layer of triplet chambers, giving the total system a depth of nine gaps~\cite{MS_TDR}. 

\subsection{Triggering and data acquisition}
\label{SECTION-ATLAS-TDAQ}
As described in Section~\ref{SECTION-ATLAS-LHC}, proton bunch crossings occur inside the ATLAS detector every 25 ns. With the size and complexity of the ATLAS detector (the average event is 1.3 Mbytes of data~\cite{TDR1}) it is not possible to read out and store the detector response for every bunch crossing, thus a trigger and data aquisition system (TDAQ) has been implemented to identify and record the most interesting events. The trigger system is divided into three levels, each of which takes as input the accepted events of the previous level and itself reduces the rate of accepted events using increasingly complex algorithms. The level 1 (L1) trigger uses local information in the calorimeter and muon systems to reduce the accepted event rate from 40 MHz to 75 kHz, the level 2 (L2) trigger uses more precise information including tracking from the inner detector for the region of interest (RoI) that caused the level 1 acceptance to further reduce the accepted event rate from 75 kHz to 3.5 kHz, and the event filter (EF) is the final trigger level which uses the highest granularity information from the entire detector to further reduce the accepted event rate from 3.5 kHz to the 200 Hz which is saved for analysis.

The level 1 trigger has an event input rate of 40 MHz and a maximum event acceptance rate of 75 kHz with a total latency of 2.0 $\mu s$. The 40 MHz input event rate means that no single part of the trigger decision can take more than 25 ns, which is achieved by using a highly parallelized hardware implementation. The electromagnetic liquid-argon calorimeter and hadronic tile calorimeter systems as well as the RPCs and TGCs in the muon spectrometer each have their signals read out to the level 1 trigger system. The calorimeter signals are processed by hardware located in the ATLAS counting room adjacent to the cavern which houses ATLAS, while the muon system signals are processed by hardware located on the ATLAS detector. These level 1 trigger processors each only process their local detector area and operate at a lower granularity than the systems are ultimately capable of. The processors look for energy clusters above a variety of set thresholds depending on the system and region of the detector, with an above threshold area forming a region of interest (RoI). The exception to the local scope of the level 1 trigger system is a special processor which calculates the total transverse energy of each event as well as the missing transverse energy of each event and compares them to a variety of thresholds. All of the processors send a list of surpassed thresholds to the central trigger processor (CTP) which correlates and counts the multiplicity of surpassed thesholds and determines a level 1 trigger decision for each event based on a programable trigger menu~\cite{Level1_Trigger}.

The level 2 trigger has an event input rate of 75 kHz from the level 1 trigger and event acceptance rate of 3.5 kHz with a total latency of 10 ms. Unlike the level 1 trigger, the level 2 trigger system uses all of the ATLAS detector systems and is implemented in software. For each event, the detector signals for all of the systems are read out in each of the RoIs identified by the level 1 trigger system to a node in the level 2 server farm. A series of algorithms are then applied in software to the event depending on the exact level 1 trigger conditions in order to refine the measurements. A final level 2 decision is reached based on the outcome of these algorithms~\cite{HLT_TDR}.

The final level of the trigger system is the event filter. The event filter has an event input rate of 3.5 kHz and a final event acceptance rate of 200 Hz with a latency of 1 s. This trigger level is very similar to the level 2 trigger system however rather than only calculating a trigger decision based on the RoIs, the event filter calculates a decision based on the entire event. Each event accepted by the level 2 system has all of the detector systems read out to a node in the event filter server farm. Based on the complete event information a lengthier and more precise callibration is performed, and based on this more detailed information an event filter decision is calculated. Events which are accepted by the event filter are read out from ATLAS to be saved for analysis~\cite{Level1_Trigger}.

\chapter{Object Definitions}
\label{SECTION-OBJ}


\section{Electron definition}
\label{SECTION-OBJ-EL}


\section{Muon definition}
\label{SECTION-OBJ-MU}


\section{Jet definition}
\label{SECTION-OBJ-JET}

\begin{equation}
\label{EQ-OBJ-ANTIKT}
d_{ij} = min(p^{-2}_{T,i},p^{-2}_{T,j})\frac{\Delta\eta^{2}_{ij}+\Delta\phi^{2}_{ij}}{R^{2}}
\end{equation}

\begin{equation}
\label{EQ-OBJ-ANTIKTBEAM}
d_{i} = p^{-2}_{T}
\end{equation}


\subsection{Jet b-tagging}
\label{SECTION-OBJ-JET-BTAG}


\section{Missing transverse energy definition}
\label{SECTION-OBJ-MET}

\chapter{Background Simulation}
\label{SECTION-BG}

In order to devise and optimize the analysis strategy both signal and background events are modeled. Most of these events are simulated using Monte Carlo (MC) techniques where each event is generated, showered and hadronized, run through a detector simulation, and reconstructed using a variety of software packages. The exception to this are the W+jets and multijet backgrounds which are modeled using either partially or wholely data driven techniques as described in Sections~\ref{SECTION-BG-DD-WJETS} and~\ref{SECTION-BG-DD-QCD} respectively.

\section{Monte Carlo simulation}
\label{SECTION-BG-MC}
The MC simulation of events is broken down into four stages. Event generation simulates the initial physics event and its decay. Showering and hadronization simulate the formation of jets from any bare quarks or gluons in the generated events. Detector simulation models the interaction of the physics event with the ATLAS detector using  a \GEANT~\cite{GEANT4} simulation of the ATLAS detector, resulting in a detector response for the event. The final step is event reconstruction where the same algorithms used to analyze data events are applied to the simulated detector responses to build analysis objects.

There are a plethora of software packages available to perform MC simulation of events, and these packages make a variety of different assumptions and simplifications of the physics they are simulating. This leads to the situation that different packages are able to more accurately simulate different physics processes and careful consideration and investigation is necessary to ensure the simulations used in the analysis are as accurate as possible. Since \Wprimechan\ is a single top process it was extremely useful to consult the extensive work already done comparing the different MC generator and showering programs for each process by the ATLAS single top group. 

For all processes except \Wprime\ the current group recomendation has been used. For the \Wprime\ signal processes the \MADGRAPH~\cite{MADGRAPH} generator has been used due to its ease of implementation and handling of spin correlations of decays. The \Wprime\ events were showered with \Pythia~\cite{PYTHIA} similar to most of the background signals. Table~\ref{TABLE-BG-MC} shows which programs were chosen to simulate each sample's generation and showering~\cite{MADGRAPH}~\cite{PYTHIA}~\cite{POWHEG}~\cite{HERWIG}~\cite{ALPGEN}~\cite{ACERMC}. With the exception of the data driven methods described in Section~\ref{SECTION-BG-DD}, the background and signal samples are normalized using their theoretical cross-sections ($\sigma$), the total luminosity (\Lumi), and a k-factor (k) which estimates the higher order corrections to the cross-section. Equation~\ref{EQ-BG-NORM} gives the normalized number of events expected for each sample (N). The cross-section and k-factor values for the signal and background samples are  given in Table~\ref{TABLE-BG-SIGNAL} and Table~\ref{TABLE-BG-MC} respectively.

\begin{equation}
\label{EQ-BG-NORM}
N = k\sigma\Lumi
\end{equation}


\begin{table}
\begin{center}
\begin{tabular}{|l|cc|cc|}
\hline
\Wprime\ Mass [GeV] & \WprimeL\ $\sigma$ [pb] & \WprimeL\ k & \WprimeR\ $\sigma$ [pb] & \WprimeR\ k \\[1mm]
\hline
500 & 12.333 & 1.3684 & 17.510 & 1.2906 \\[1mm]
750 & 2.7223 & 1.3144 & 3.7174 & 1.2779 \\[1mm]
1000 & 0.81915 & 1.2564 & 1.0652 & 1.2796 \\[1mm]
1250 & 0.28025 & 1.2405 & 0.37278 & 1.2260 \\[1mm]
1500 & 0.10618 & 1.2202 & 0.13932 & 1.2183 \\[1mm]
1750 & 0.043693 & 1.1893 & 0.055667 & 1.2062 \\[1mm]
2000 & 0.018551 & 1.1774 & 0.023718 & 1.1740 \\[1mm]
2250 & 0.0082073 & 1.1638 & 0.010283 & 1.1669 \\[1mm]
2500 & 0.0038171 & 1.1512 & 0.0046794 & 1.1485 \\[1mm]
2750 & 0.0018512 & 1.1529 & 0.0021970 & 1.1522 \\[1mm]
3000 & 0.00095811 & 1.1687 & 0.0011035 & 1.1592 \\[1mm]
\hline
\end{tabular}
\caption{Cross-sections and k-factors for generated \Wprime\ samples.}
\label{TABLE-BG-SIGNAL}
\end{center}
\end{table}


\begin{table}[htdp]
\begin{center}
\begin{tabular}{|l|cc|rr|}
\hline
Process & $\sigma$ [pb] & k & Generator & Showering \\[1mm]
\hline 
single top s-channel & 1.6424 & 1.1067 & \POWHEG\ & \Pythia\ \\[1mm]
single top t-channel & 25.750 & 1.1042 & \AcerMC\ & \Pythia\ \\[1mm]
single top Wt-channel & 20.461 & 1.0933 & \POWHEG\ & \Pythia\ \\[1mm]
$t\bar{t}$ & 114.51 & 1.1992 & \POWHEG\ & \Pythia\ \\[1mm]
W+lf & 31994 & 1.133 & \ALPGEN\ & \Pythia\ \\[1mm]
W+c & 1126.0 & 1.52 & \ALPGEN\ & \Pythia\ \\[1mm]
W+cc & 403.44 & 1.133 & \ALPGEN\ & \Pythia\ \\[1mm]
W+bb & 133.99 & 1.133 & \ALPGEN\ & \Pythia\ \\[1mm]
Z+jets & 2804.4 & 1.229 & \ALPGEN\ & \HERWIG\ \\[1mm] %Alpgen+Pythia in v11
diboson & 17.075 & 1.7223 & \HERWIG\ & \HERWIG\ \\[1mm]
\hline
\end{tabular}
\caption{Simulated background samples with associated cross-sections, k-factors, generating programs and showering programs.}
\label{TABLE-BG-MC}
\end{center}
\end{table}


\section{Data driven estimates} 
\label{SECTION-BG-DD}
While the above method works well to simulate many background processes, it is sometimes useful to use control regions of data to estimate some backgrounds. For W+jets it is necessary to correct the overall normalization as well as the relative abundance of the simulated samples based on the flavor associated jet. Multijets has a very high rate of occurence and a very low acceptance making it very difficult to predict, so this analysis uses the matrix method to estimate this background from data.

\subsection{W+jets normalization} 
\label{SECTION-BG-DD-WJETS}
The W+jets samples in this analysis are globally normalized using the charge asymmetry method in the region $m(W')\ <\ 330\ GeV$. This region has a signal contamination $<$ 5\% for all signal mass points considered in the analysis. This method normalizes the W+jets sample in each analysis channel using the theoretical asymmetry ratio $r_{MC} = \frac{W^{+}}{W^{-}}$ to account for the observed asymmetry in data. The ratio between the observed asymmetery and the expected asymmetry is applied as a normalization factor to the entire channel, as shown in Equation~\ref{EQ-BG-DD-WJETS}.

\begin{equation}
\label{EQ-BG-DD-WJETS}
N_{W^{+}} + N_{W^{-}} = \frac{r_{MC} + 1}{r_{MC} - 1}(D^{+} - D^{-})
\end{equation}

\noindent
$N_{W^{+}}\ +\ N_{W^{-}}$ is the normalized W+jets yield and $D^{+}$ and $D^{-}$ are the number of data events with positive and negative leptons respectively. The fraction of W+jets composed by W+lf, W+c, W+cc, and W+bb is determined by simultaneously varying the fraction of the total W+jets sample each sub-channel composes and fitting the MET distribution. For this fit the W+cc and W+bb samples are merged into a single W+hf sample and so they recieve the same normalization factor. The normalization factors for each sample are given in Table~\ref{TABLE-BG-DD-WJETS}.

\begin{table}[htdp]
\begin{center}
\begin{tabular}{|l|cccc|}
\hline
Process & 2jets 1tag & 2jets 2tag & 3jets 1tag & 3jets 2tag \\[1mm]
\hline 
W+lf & 0.941462 & 1.31867 & 0.883688 & 1.96718 \\[1mm]
W+c & 0.801521 & 1.12266 & 0.752335 & 1.67477 \\[1mm]
W+cc & 1.39795 & 1.95806 & 1.31217 & 2.92102 \\[1mm]
W+bb & 1.39795 & 1.95806 & 1.31217 & 2.92102 \\[1mm]
\hline
\end{tabular}
\caption{W+jets normalization factors.}
\label{TABLE-BG-DD-WJETS}
\end{center}
\end{table}

\subsection{Multijets estimate} 
\label{SECTION-BG-DD-QCD}
The contribution of the multijet process to this analysis is estimated using the matrix method. The matrix method uses data events which have passed the event selection in Chapter~\ref{SECTION-SELECTION} except with a loose lepton which has relaxed requirements compared to the tight lepton required for the signal region. Both the loose and tight lepton are defined in Chapter~\ref{SECTION-OBJ}. For both electrons and muons

\begin{equation}
\label{EQN-BG-DD-QCD1}
N^{loose} = N^{loose}_{real} + N^{loose}_{fake}
\end{equation}
\begin{equation}
\label{EQN-BG-DD-QCD2}
N^{tight} = \epsilon_{real}N^{loose}_{real} + \epsilon_{fake}N^{loose}_{fake}
\end{equation}

\noindent
where N is the number of data events containing a lepton of the indicated type. $\epsilon_{real} = \frac{N^{tight}_{real}}{N^{loose}_{real}}$ and $\epsilon_{fake} = \frac{N^{tight}_{fake}}{N^{loose}_{fake}}$ are the conversion efficiencies for loose leptons to tight leptons. $\epsilon_{real}$ is estimated using the tag and probe method on $Z \rightarrow ll$ events, while $\epsilon_{fake}$ is estimated using a multijets enhanced data sample where the lepton isolation criteria have been removed. With the total number of events with loose and tight leptons known from the dataset, Equation~\ref{EQN-BG-DD-QCD1} and Equation~\ref{EQN-BG-DD-QCD2} can be inverted and combined with the fakes conversion efficiency to solve for $N^{tight}_{fake}$ which is the multijets estimate for the analysis.

\chapter{Event selection}
\label{SECTION-SELECTION}
Before the final analysis described in Chapter~\ref{SECTION-ANALYSIS} can be performed, an event selection specific to the signal kinematics is applied. This event selection is designed to remove background events while having minimal impact on the signal and defines the control regions used to perform the data-driven background estimates described in Section~\ref{SECTION-BG-DD}. Events are also separated into different channels by the event selection and these channels are individually optimized.

\section{Composite objects}
\label{SECTION-SELECTION-COMPOBJ}
While Chapter~\ref{SECTION-OBJ} details how basic analysis objects are reconstructed from the raw detector response, Figure~\ref{FIGURE-SELECTION-WPRIME} shows several intermediate particles that can also be reconstructed. These intermediate states of the W boson, top quark, and ultimately the \Wprime\ boson are what define this channel as unique from any other process with the same final state, such as $Wbb$. These intermediate particles also have unique kinematics that distinguish \Wprimechan\ from other processes.

\VLARGEFIG{Wprime}{Illustration of the \Wprimechan\ process.}{FIGURE-SELECTION-WPRIME}

\subsection{W boson and neutrino reconstruction}
\label{SECTION-SELECTION-COMPOBJ-W}
The W boson in Figure~\ref{FIGURE-SELECTION-WPRIME} is the only intermediate particle composed entirely of final state objects and its reconstruction is as simple as adding the 4-momenta of the lepton and neutrino together. The complication with this is that the 4-momentum of the neutrino is not known. Section~\ref{SECTION-OBJ-MET} describes how the neutrino's $p_T$ can be determined from the MET by assuming that the momentum is balanced in the transverse plane. This same technique cannot be used to determine $p_z$ for the neutrino because there is no reason the interacting partons should have the same momentum along the beamline as each other. Instead the W boson and neutrino are defined simultaneously by requiring that the lepton (a single lepton selection is applied in Section~\ref{SECTION-SELECTION-CUTS}) and neutrino combine to form an on-shell W boson with a mass of 80.4 GeV. Both the lepton and neutrino are assumed to be massless and the neutrino's $p_T$ is assumed to be equivalent to the MET. This gives rise to a quadratic equation for the neutrino's $p_z$, with solutions given by Equation~\ref{EQ-SELECTION-NEUTRINO}. 

\begin{equation}
\label{EQ-SELECTION-NEUTRINO}
p_{z,\nu} = \frac{\mu p_{z,l}}{p_{T,l}^2} \pm \sqrt{\frac{\mu^2p_{z,l}^2}{p_{T,l}^4} - \frac{E_l^2p_{T,\nu}^2-\mu^2}{p_{T,l}^2}}
\end{equation}
\begin{equation}
\label{EQ-SELECTION-NEUTRINOMU}
\mu = \frac{M_W^2}{2} + cos(\Delta\phi_{l,\nu})p_{T,\nu}p_{T,l}
\end{equation}

\noindent
In Equation~\ref{EQ-SELECTION-MTW}, $p_{T,l}$ and $p_{T,\nu}$ are the transverse momenta of the lepton and neutrino respectively,and $p_{z,l}$ and $p_{z,\nu}$ are the z-momenta of the lepton and neutrino. $\Delta\phi_{l,\nu}$ is the difference in $\phi$ between the lepton and neutrino. There are three possible categories of solution to Equation~\ref{EQ-SELECTION-NEUTRINO} based on the sign of the discriminant. If the discriminant is positive then there are two real solutions to Equation~\ref{EQ-SELECTION-NEUTRINO} and the solution with the lowest $|p_z|$ is chosen to define the neutrino, creating a less energetic final state. If the discriminant is 0 then there is only one $p_z$ solution then the neutrino is uniquely defined. If the discriminant is negative then the solutions for $p_z$ are imaginary, in this case the $p_T$ of the neutrino is rescaled so that the discriminant becomes 0, then the neutrino $p_z$ is uniquely defined and the neutrino $p_T$ is taken to be the rescaled value.

\subsection{Top quark reconstruction}
\label{SECTION-SELECTION-COMPOBJ-T}
While it is possible to reconstruct the top quark in Figure~\ref{FIGURE-SELECTION-WPRIME}, there is an ambiguity about which jet originated from the top quark decay. The indeterminacy is resolved differently depending on which channel the event belongs to. If the event contains only 1 b-tagged jet then the invariant mass of each jet and the reconstructed W boson is calculated and the combination with a mass closest to the top quark mass of 172.5 GeV forms the reconstructed top quark. For events that contain 2 b-tagged jets the mass of the W boson and each b-tagged jet is calculated, with the pair producing a mass closest to 172.5 GeV forming the top quark. The cut flow for each channel is described in greater detail in Section~\ref{SECTION-SELECTION-CUTS}.

\subsection{\Wprime\ reconstruction}
\label{SECTION-SELECTION-COMPOBJ-WPRIME}
Similar to how the top quark is reconstructed, the \Wprime\ boson is reconstructed differently depending on which analysis channel the event falls into. For events that contain 2 b-tagged jets the \Wprime\ boson is reconstructed by combining the reconstructed top quark with the b-tagged jet that was not used to reconstruct the top quark. For events with 1 b-tagged jet the \Wprime\ boson is reconstructed by combining the reconstructed top quark with the highest $p_T$ jet not used to reconstruct the top quark, requiring that the b-tagged jet is included in the \Wprime\ reconstruction. This means that for events where the b-tagged jet was included in the top quark reconstruction that the jet combined with the top quark is not b-tagged. For events where the top quark reconstruction does not include the b-tagged jet, the jet combined with the top quark to form the \Wprime\ boson must be b-tagged.

\section{Data triggers}
\label{SECTION-SELECTION-TRIG}
In order for an event to be recorded by the ATLAS detector and included in an analysis it must pass the trigger selection described in Section~\ref{SECTION-ATLAS-TDAQ}. To search for \Wprimechan\ the ATLAS single lepton triggers are used. The single electron triggers require that electrons either have an $E_T$ $>$ 24 GeV and pass medium isolation requirements for the hadronic leakage, shower width in $\eta$, and track isolation as described in Section~\ref{SECTION-OBJ-EL} or have an $E_T$ $>$ 60 GeV without any isolation requirement. The single muon triggers require that muons either have a $p_T$ $>$ 24 GeV and pass medium isolation requirements for the ID track isolation described in Section~\ref{SECTION-OBJ-MU} or have a $p_T$ $>$ 36~GeV without any isolation requirement. The complete set of requirements for electrons and muons detailed in Chapter~\ref{SECTION-OBJ} is applied offline, after the data has been recorded. Events must also have been taken during an LHC stable beam period and during a time when all of the ATLAS subsystems were properly operating. The combination of these requirements corresponds to an integrated luminosity of \LUMI.

\section{Cut flow}
\label{SECTION-SELECTION-CUTS}
Before performing the multivariate analysis described in Chapter~\ref{SECTION-ANALYSIS}, it is useful to apply a set of event selection cuts. These cuts are designed to remove background events with large kinematical differences from the signal samples so that the multivariate analysis can be more focused on discriminating between the hard to classify events. The event selection also defines the separate analysis channels which will undergo individually optimized multivariate analyses. The event selection cuts are as follows:

\begin{list}{$\bullet$}{}
\item Exactly 1 lepton.
\item Lepton $p_T$ $>$ 35 GeV.
\item MET $>$ 35 GeV
\item W boson transverse mass ($m_T(W)$) + MET $>$ 60 GeV, where $m_T(W)$ is defined in Equation~\ref{EQ-SELECTION-MTW}.
\item Exactly 2 or 3 jets.
\item Exactly 1 or 2 b-tagged jets.
\item \Wprime\ boson mass ($m(\Wprime)$) $>$ 330 GeV.
\end{list}

\begin{equation}
\label{EQ-SELECTION-MTW}
m_T(W) = \sqrt{2p_{T,l}p_{T,\nu}(1-cos(\Delta\phi_{l,\nu}))}
\end{equation}

\noindent
The number of jets and the number of b-tagged jets defines a unique analysis channel which is referred to by the number of jets and b-tagged jets in events in that particular channel, for example the 2jets 1tag channel contains events with exactly 2 jets and exactly 1 b-tagged jet. This produces four separate analysis channels, 2jets 1tag, 2jets 2tag, 3jets 1tag, and 3jets 2tag. 

The cuts on the lepton number, lepton $p_T$, and MET are chosen to match the decay channel seen in Figure~\ref{FIGURE-SELECTION-WPRIME} where we expect a single high $p_T$ lepton and large MET from the W boson decay. The cut on $m_T(W)$ + MET is called the triangular cut and is commonly used in single top analyses to discriminate against the multijets background. The cut on $m(\Wprime)$ is chosen to define a control region used to perform a data driven normalization of the W+jets background as described in Section~\ref{SECTION-BG-DD-WJETS}. The cut value of 330 GeV was chosen to maximize the size of the control region while keeping the signal contamination to less than 5\% for all of the signal samples. 

Both of the 1tag channels have significantly larger backgrounds than the 2tag channels so two additional cuts are applied to the 1tag channels only:

\begin{list}{$\bullet$}{}
\item $E_T$ of the leading jet ($E_T(jet1)$) $>$ 140 GeV.
\item Transverse energy of the reconstructed top quark ($E_T(Top)$) $>$ 175 GeV.
\end{list}

\noindent
These cuts are chosen by ranking a list of event kinematics variables by their discrimination power after performing the initial event selection cuts. The discrimination power of each variable is determined by mapping the signal efficiency ($\epsilon_S$) versus the background efficiency ($\epsilon_B$) for successively raised cuts on the variable. The area between the curve this process maps out and the line of $\epsilon_S\ =\ \epsilon_B$ is defined to be the discrimination power of the variable. For the two most discriminating variables, $p_T(jet1)$ and $E_T(Top)$, the cut is chosen to be at least 95\% efficient for all signal samples. The final event yields are shown in Table~\ref{TABLE-SELECTION-YIELDS}.


\begin{table}
\begin{center}
\begin{tabular}{|c|cccc|}
\hline
Sample & 2jets 1tag & 3jets 1tag & 2jets 2tag & 3jets 2tag \\
\hline
\WprimeR 500 & 12601.14 & 5599.62 & 8874.10 & 5120.80 \\ 
\WprimeR 750 & 4018.08 & 2723.38 & 2468.55 & 2172.28 \\ 
\WprimeR 1000 & 1117.67 &  937.33 &  606.33 &  657.60 \\ 
\WprimeR 1250 &  337.63 &  311.93 &  155.44 &  189.40 \\ 
\WprimeR 1500 &  101.52 &  107.72 &   41.97 &   57.29 \\ 
\WprimeR 1750 &   32.09 &   36.60 &   12.11 &   18.28 \\ 
\WprimeR 2000 &   11.07 &   13.18 &    4.12 &    6.20 \\ 
\WprimeR 2250 &    4.09 &    5.01 &    1.48 &    2.06 \\ 
\WprimeR 2500 &    1.71 &    1.94 &    0.63 &    0.84 \\ 
\WprimeR 2750 &    0.80 &    0.86 &    0.34 &    0.40 \\ 
\WprimeR 3000 &    0.42 &    0.42 &    0.19 &    0.22 \\ 
\hline
\WprimeL 500 & 6680.34 & 3078.17 & 5235.86 & 3129.40 \\ 
\WprimeL 750 & 2307.41 & 1551.99 & 1556.07 & 1310.33 \\ 
\WprimeL 1000 &  681.85 &  577.36 &  411.47 &  438.10 \\ 
\WprimeL 1250 &  229.29 &  213.66 &  112.84 &  138.45 \\ 
\WprimeL 1500 &   75.38 &   77.88 &   32.26 &   44.19 \\ 
\WprimeL 1750 &   25.97 &   30.55 &   10.65 &   14.45 \\ 
\WprimeL 2000 &    9.75 &   11.16 &    3.43 &    4.99 \\ 
\WprimeL 2250 &    3.81 &    4.55 &    1.37 &    1.98 \\ 
\WprimeL 2500 &    1.65 &    1.89 &    0.60 &    0.78 \\ 
\WprimeL 2750 &    0.73 &    0.82 &    0.30 &    0.40 \\ 
\WprimeL 3000 &    0.40 &    0.41 &    0.18 &    0.21 \\ 
\hline
single top s-channel  &  138.10 &   73.92 &   98.82 &   58.71 \\  
single top t-channel  & 1957.69 & 1080.64 &  242.73 &  373.00 \\  
single top Wt-channel &  624.74 &  979.43 &   80.10 &  270.67 \\  
$t\bar{t}$ & 4586.93 & 9410.07 & 1480.34 & 5108.30 \\  
W+lf & 2950.59 & 1255.28 &   45.78 &   45.27 \\  
W+c  & 4877.90 & 1989.33 &   78.18 &   71.61 \\  
W+cc & 3471.35 & 2470.32 &   81.00 &  127.04 \\  
W+bb & 3395.41 & 2086.27 &  455.80 &  675.72 \\  
Z+jets &  361.64 &  379.02 &    2.32 &    9.45 \\  
diboson &  214.10 &  119.48 &   16.34 &   15.36 \\  
multijets & 1132.34 &  540.02 &   59.54 &   58.68 \\  
\hline
total background & 23710.80 & 20383.78 & 2640.95 & 6813.83 \\ 
\hline
data & 21106.00 & 18317.00 & 2632.00 & 6666.00 \\ 
\hline
\end{tabular}
\caption{Event yields for signal samples, background samples, and data by analysis channel.}
\label{TABLE-SELECTION-YIELDS}
\end{center}
\end{table}

\chapter{Analysis}
\label{SECTION-ANALYSIS} 

Let's think the unthinkable, let's do the undoable. Let us prepare to grapple with the ineffable itself, and see if we may not eff it after all. -Douglas Adams

\vspace{5mm} %5mm vertical space

After the \az, the \at, the \aw~from the \at~decay, and the neutrino from the \aw~decay are reconstructed as described in Chapter~\ref{SECTION-OBJ}, they are used to help separate \tz~from the various backgrounds. The energies and momenta of each of these objects in our detector, as well as the multiplicity of the objects, are used to achieve this seperation. The decisions made in the preselection and cut flow are informed by the kinematic properties of the \tz~process. The \tz~Feynman diagram is shown in Figure~\ref{FIGURE-tZ}. 



\section{Preselection}
\label{SECTION-PRESELECTION}

One goal in setting up an analysis is to understand the background model in relation to the observed data. To accomplish this, defining characteristics of the signal region are determined in order to limit the number of \MC~samples needed. Because the signal has three leptons and a \az, cuts on the number of leptons and \azhyph~mass (for instance) are applied to limit any contribution from certain low lepton multiplicity non-\azhyph~sources such as \aw +jets and multijets. 

The following cuts are optimized by maximizing $S/\sqrt{B}$ where $S$ is the total expected signal contribution and $B$ is the total expected background contribution. This is done to improve agreement between data and the background model while maintaining as much signal statistics as possible. 

\begin{itemize}
\item Exactly 3 leptons with \PT~\textgreater~10~GeV. Exactly 3 leptons is a defining feature of this analysis. Two of the leptons come from the \az~decay, while the third comes from the \at~decay. Because these leptons are required to be electrons or muons their distributions are mirrors of each other by definition which can be seen in Figure~\ref{FIGURE-NMUON}.
\item At least one OSSF pair. Because we are concerned with processes that contain a real \az, this requirement ensures that we can always attempt to reconstruct a valid \az~candidate even if it is an event that is mis-identified as containing a \az. 
\item Leading lepton \PT~\textgreater~40~GeV. The leading lepton's threshold is higher than the second and third due to being more likely that it is the candidate that is required to pass the single lepton trigger or a candidate in the case of a multi-lepton trigger. Figure~\ref{TRIPFIG1} shows that this cut removes some background, but little signal is lost.
\item Second lepton \PT~\textgreater~20~GeV. This lepton is not required to have passed a single lepton trigger, but may have been required to pass the di-electron trigger threshold. Figure~\ref{TRIPFIG1} shows that this cut removes some background, but little signal is lost. Tightening this cut is also investigated because \zjets~and \TTB~peak at a lower momentum than the signal, but in the interest of maintaining statistics a lower \PT~threshold is chosen. 
\item Third lepton \PT~\textgreater~10~GeV. The \PT~of this lepton is significantly lower than the rest. 10 GeV is chosen due to the thresholds that define how electrons are reconstructed. Figure~\ref{TRIPFIG1} shows that the signal peaks at higher \PT~than \zjets~and \TTB~, and tightening this cut is investigated, but in the interest of maintaining statistics, a lower \PT~threshold is chosen.
\item 2, 3, or 4 jets with \PT~\textgreater~25~GeV. Figure~\ref{TRIPFIG2} shows that the one jet region contains virtually no signal, so removing it eliminates background at no cost, while events with $>=$ 5 jets have little signal and are not as well modeled.
\item Leading jet \PT~\textgreater~40~GeV. Figure~\ref{TRIPFIG1} shows that below 40 GeV, there is less than 1 expected signal events, so very little is removed.
\item Exactly 1 \bjet. Figure~\ref{TRIPFIG2} shows that there is little signal outside the one \bjet~region. The signal and top backgrounds have one \bjet~from the top decay while non-top backgrounds are unlikely to have one.
\item 80 GeV \textless~\azhyph~mass \textless~100~GeV. The \az~is a defining feature of this analysis. The signal and all backgrounds except \TTB~and single \at~production have a real \az. Figure~\ref{TRIPFIG3} shows how much ttbar is off the $Z$ peak and a cut here will give substantial gains in removing \TTB~background without removing a significant amount of signal events. 
\item \met~\textgreater~20~GeV. This cut defines processes that have a real source of \met~such as the neutrino from \athyph~decays. Figure~\ref{TRIPFIG2} shows that the low \met~region is populated heavily by \zjets~with little signal.
\item If \wtm~\textless~40~GeV then \met~\textgreater~40~GeV is required. If viewed in the two dimentional plane, cases where jets are mis-reconstructed as leptons are expected to have low \wtm~and low \met. Distributions for \wtm~and \met~can be seen independently in Figure~\ref{TRIPFIG2}. The 2D plane of \wtm~vs. \met~is shown for both signal and data in Figure~\ref{TRIPFIG3}. Here we can see that the signal peaks above 20~GeV in \met~and above 40~GeV in \wtm~while data preferentially resides in the region where both \wtm~and \met~are below 40~GeV. This cut is primarily targeted at \zjets~events (as well as potential backgrounds with mis-identified \aw s) which heavily populate the low \wtm~region. This cut is referred to as the notch cut because of its unique shape. 
\end{itemize}


\QUADFIGLEG{LeadingLeptonPt}{SecondLeptonPt}{ThirdLeptonPt}{LeadingJetPt}{Distributions of Lepton \PT~for (a) leading, (b) second, and (c) third leptons as well as (d) leading jet \PT~with preselection applied except the cuts on minimum \PT~thresholds shown which are 40~GeV for the leading lepton, 20~GeV for the second lepton, 10~GeV for the third lepton, or 40 GeV for the leading jet. There are minimum \PT~reconstruction thresholds for these objects which are 25~GeV for the leading lepton and leading jet, and 10~GeV for the second and third leptons.}{TRIPFIG1}

\QUADFIGLEG{njet}{nbjet}{Wtransversemass}{met}{Distributions of (a) number of jets, (b) number of \bjet s, (c) \wtm, (d) \met. At least one jet is required at this level in all cases, but the cut on the variable shown is omitted in order to assess the full distribution. The distribution of the number of jets does not include the cut on the number of jets, the distribution of the number of \bjet s does not include the cut on the number of \bjet s, and the \wtm~and \MET~distributions do not contain the \MET~or the notch cuts.}{TRIPFIG2}

\TRPFIGLEG{METvsWtmSignal}{METvsWtmData}{Zmass}{Distributions of (a) two-dimentional map of \wtm~vs \met~for the signal, (b) two dimentional map of \wtm~vs \met~for the data, and (c) invariant mass of the \az. For both (a) and (b) the \met~cut and the notch cut are not applied and for (c) the \azhyph~mass window cut is not applied in order to show the full distribution.}{TRIPFIG3}


\clearpage

\section{Control Regions}
\label{SECTION-CONTROL-REGIONS}

Three control regions are considered for the three primary backgrounds to ensure that the background model describes the data well. The control regions are for \TTB, Diboson, and \zjets~and their yields are summarized in Table~\ref{tab:CRyields} where it can be seen that \tz~contamination is small and that the control regions are fairly pure in their respective backgrounds. The control region for \TTB~is defined by the preselection cuts with the exception of the \azhyph~mass window which is inverted. This has the effect of cutting out large contributions which contain a real \az, leaving primarily \TTB. In every distribution shown in Figures~\ref{FIGURE-CRttbar} and~\ref{FIGURE-CRttbar2}, there is good agreement between data and simulated events with a quite pure sample of \TTB. In order to isolate Diboson and \zjets , we begin with the preselection again, but instead of requiring exactly 1 \bjet, we require exactly 0 \bjet s in order to eliminate \athyph~contributions. This defines an intermediate control region with Diboson and \zjets~mixed as shown in Figures~\ref{FIGURE-CRint} and~\ref{FIGURE-CRint2}. This is expected because both Diboson and \zjets~have a real \az~and do not have a \ab~that would come from a \athyph~decay. To isolate Diboson more precisely, a cut is placed on \wtm~to constrain it to higher than 80~GeV as shown in Figures~\ref{FIGURE-CRDiboson} and~\ref{FIGURE-CRDiboson2}. This provides a region with high Diboson purity to evaluate the quality of its modeling. In order to isolate \zjets~from the intermediate control region a cut on \met~is made to constrain it to lower than 60~GeV as shown in Figures~\ref{FIGURE-CRZjets} and~\ref{FIGURE-CRZjets2}. This region has lower purity in \zjets~when compared to the \TTB~control region and the Diboson control region, and shows areas of mis-modeling in low to mid \wtm~(less than 70~GeV). Low lepton \PT~also seems to be poorly modeled (20-40~GeV for each of the three leptons). This is likely because the third lepton must be a mis-reconstructed one. Despite also having a mis-reconstructed lepton, due to only having two real leptons, \TTB~does not show similar mis-modeling for two primary reasons. The first reason is that \TTB~MC statistics is much better than \zjets. The second reason is that \TTB~has more hard objects (extra jets) stemming from the primary interactions, while \zjets~has extra hard objects come from initial-state or final-state radiation. This mis-modeling is mitigated by cuts on \PT, \wtm, and \met~as well as cuts made to the signal region. Even with these measures taken the mis-modeling reflects itself as large uncertainties on \zjets~which is shown in Chapter~\ref{SECTION-RESULTS}. Collectively these control regions give insight to the contribution of the largest backgrounds to this analysis. 



\QUADFIGLEG{LeadingLeptonPtCRttbar}{SecondLeptonPtCRttbar}{ThirdLeptonPtCRttbar}{LeadingJetPtCRttbar}{Distributions of transverse momenta for (a) the leading lepton, (b) the second lepton, (c) the third lepton, and (d) the leading jet in the control region for \TTB.}{FIGURE-CRttbar}
\QUADFIGLEG{njetCRttbar}{nbjetCRttbar}{WtransversemassCRttbar}{metCRttbar}{Distributions of (a) jet multiplicity, (b) \bjet~multiplicity, (c) \wtm, and (d) \met~in the control region for \TTB.}{FIGURE-CRttbar2}



\QUADFIGLEG{LeadingLeptonPtCRint}{SecondLeptonPtCRint}{ThirdLeptonPtCRint}{LeadingJetPtCRint}{Distributions of transverse momenta for (a) the leading lepton, (b) the second lepton, (c) the third lepton, and (d) the leading jet in the intermediate control region for Diboson and \zjets.}{FIGURE-CRint}
\QUADFIGLEG{njetCRint}{nbjetCRint}{WtransversemassCRint}{metCRint}{Distributions of (a) jet multiplicity, (b) \bjet~multiplicity, (c) \wtm, and (d) \met~in the intermediate control region for Diboson and \zjets.}{FIGURE-CRint2}



\QUADFIGLEG{LeadingLeptonPtCRDiboson}{SecondLeptonPtCRDiboson}{ThirdLeptonPtCRDiboson}{LeadingJetPtCRDiboson}{Distributions of transverse momenta for (a) the leading lepton, (b) the second lepton, (c) the third lepton, and (d) the leading jet in the control region for Diboson.}{FIGURE-CRDiboson}
\QUADFIGLEG{njetCRDiboson}{nbjetCRDiboson}{WtransversemassCRDiboson}{metCRDiboson}{Distributions of (a) jet multiplicity, (b) \bjet~multiplicity, (c) \wtm, and (d) \met~in the control region for Diboson.}{FIGURE-CRDiboson2}



\QUADFIGLEG{LeadingLeptonPtCRZjets}{SecondLeptonPtCRZjets}{ThirdLeptonPtCRZjets}{LeadingJetPtCRZjets}{Distributions of transverse momenta for (a) the leading lepton, (b) the second lepton, (c) the third lepton, and (d) the leading jet in the control region for \zjets.}{FIGURE-CRZjets}
\QUADFIGLEG{njetCRZjets}{nbjetCRZjets}{WtransversemassCRZjets}{metCRZjets}{Distributions of (a) jet multiplicity, (b) \bjet~multiplicity, (c) \wtm, and (d) \met~in the control region for \zjets.}{FIGURE-CRZjets2}



\begin{table} [ht!]
\setlength{\tabcolsep}{2pt}
\footnotesize
\centering
\begin{tabular}{| l | c | c | c | c | c | c |}
\hline
\hline
Event Yields & Preselection & \TTB~CR & intermediate CR & Diboson CR & \zjets~CR & final selection\\

\hline
\hline

$t\bar{t}$ & 45 & 196 & 16 & 4.0 & 5.7 & 10 $\pm$ 45\%\\
single \at & 1.4 & 7.7 & 0.85 & 0.26 & 0.30 & 0.34 $\pm$ 66\%\\
$ttV$ & 4.4 & 2.7 & 1.0 & 0.38 & 0.30 & 0.61 $\pm$ 66\%\\
$Z$ + jets & 32 & 10 & 110 & 4.0 & 78 & 1.7 $\pm$ 413\%\\
Diboson & 18 & 5.0 & 100 & 31 & 48 & 3.3 $\pm$ 32\%\\

\hline

$tZ$ & 5.7 & 0.63 & 1.6 & 0.4 & 0.68 & 2.9 $\pm$ 11\%\\

\hline

Total Expected & 108 & 223 & 232 & 41 & 134 & 19 $\pm$ 71\%\\
Data Observed & 108 & 237 & 214 & 52 & 131 & 22\\

\hline

 S/B & 0.06 & 0.00 & 0.01 & 0.01 & 0.01 & 0.18 \\ 
 S/$\surd$B & 0.57 & 0.04 & 0.10 & 0.07 & 0.06 & 0.71 \\ 

\hline
\hline

\end{tabular}
\caption{Event yields for various stages of analysis to compare with control region (CR) yields. The final selection is described in Section ~\ref{SECTION-SELECTION-CUTS} and uncertainties provided on the final selection are described in Chapter~\ref{SECTION-RESULTS} taken in quadrature for each sample. They are provided here for reference.}
\label{tab:CRyields}
\end{table}



\clearpage

\section{Cut Flow}
\label{SECTION-SELECTION-CUTS}

Once the preselection region is defined, our goal is to improve the sensitivity of the analysis. We do this by searching for kinematic variables where the shape of the signal distribution significantly differs from the shape of one or all of the background distributions and evaluating its effect on the value $S/\sqrt{B}$. The variable $S/\sqrt{B}$ is used to optimize because it ensures both strong signal to background ratios while also ensuring that we limit the contribution of statistical errors. Many distributions are considered for their background rejection, and/or physical motivations but distributions of special interest are the angular variables and \athyph~mass shown in Figure~\ref{FIGURE-TRIPFIG-FINAL2} because they display the properties of the \at. The polarization of the \at~is most notable in Figure~\ref{FIGURE-TRIPFIG-FINAL2} where the optimal basis shows both \TTB~and \tz~have a distribution favoring values closer to 1, while Diboson is comparatively flat. In principle these variables could be used to distinguish backgrounds without a \at~from the signal which does. In practice the discrimination power of these variables is not as strong as that of others. The variables with the best discriminating power are shown in Table~\ref{tab:eventyieldFullSelec} and are, 

\begin{itemize}

\item \wtm~\textgreater~50~GeV. This selects for events with higher energy \aw s. 

\item Leading non-\bjet~|\eta|~\textgreater~1.5. This selects for events with a forward jet as is the case with single top \tchan~and \tz.

\item $\Delta R$ between the \bjet~and Leading non-\bjet~\textgreater~2.5. $\Delta R$ is calculated as the $\Delta \eta$ and the $\Delta \phi$ added in quadrature. These two objects are expected to not be near each other in the signal selecting for events where the jets do not both come from the same source.

\end{itemize}

 The distributions of these variables are shown in Figure~\ref{cutfig} and are re-optimized sequentially to show that any correlations are minor, and to ensure optimal sensitivity. Table~\ref{tab:eventyieldFullSelec} also shows what background each cut is preferentially removing. The \wtm~cut targets \zjets, while also eliminating \TTB~and some Diboson. The cut on the leading non-\bjet~\eta~is less obviously targeted at a specific background, but is removing approximately half of all backgrounds while removing comparatively little signal. This is due to the forward jet, a characteristic kinematic property of single \athyph~production. The cut on the $\Delta R$ between the \bjet~and leading non-\bjet~performs well because the \bjet~and the leading non-\bjet~are coming from opposite legs of the hard interaction. This creates a distribution where the \at~and its decay products (in this case the \bjet) come out preferentially far apart in $\Delta R$ in the signal compared to the backgrounds. 








\begin{table} [ht!]
\setlength{\tabcolsep}{2pt}
\footnotesize
\centering
\begin{tabular}{| l | r | r | r | r |}
\hline
\hline
Process & Preselection & \wtm~& Leading-non \bjet~\eta & full selection \\ 
\hline
$t\bar{t}$ & 45 & 28 & 15 & 10 $\pm$ 45\% \\ 
single \at & 1.4 & 1.0 & 0.49 & 0.34 $\pm$ 66\% \\ 
$ttV$ & 4.4 & 3.1 & 1.0 & 0.61 $\pm$ 66\%\\ 
$Z$ + jets & 32 & 5.3 & 2.3 & 1.7 $\pm$ 413\%\\ 
Diboson & 18 & 13 & 5.2 & 3.3 $\pm$ 32\%\\ 
\hline
$tZ$ & 5.7 & 4.3 & 3.2 & 2.9 $\pm$ 11\%\\ 
\hline
Total Expected & 108 & 55 & 27 & 19 $\pm$ 71\% \\ 
Data Observed & 108 & 62 & 29 & 22 \\ 
\hline
 S/B & 0.06 & 0.08  & 0.13 & 0.18 \\ 
 S/$\surd$B & 0.57 & 0.60 & 0.66 & 0.71 \\ 

\hline
\hline
\end{tabular}
\caption{Event yields after selection cuts are applied. Uncertainties provided on the final selection are the uncertainties described in Chapter~\ref{SECTION-RESULTS} taken in quadrature for each sample.}
\label{tab:eventyieldFullSelec}
\end{table}


\TRPFIGLEG{W_transverse_mass}{LeadingNonb-jetEta}{b-jet+LeadingNonb-jetdR}{Distributions of (a) \wtm which is required to be~\textgreater~50~GeV, (b) the \eta~of the leading non b-tagged jet which is required to be~\textgreater~1.5, and (c) the $\Delta R$ between the \bjet~and leading non-\bjet which is required to be~\textgreater~2.5. Each has the entire selection applied except the variable plotted to view the full distribution.}{cutfig}


Once we have applied the full cut flow, we are left with the remaining distributions to analyze. These represent the kinematic properties of events selected by this analysis which are shown in Figures~\ref{FIGURE-SigRegion} and~\ref{FIGURE-SigRegion2}. The application of the full selection takes us from an $S/B$ of 0.06 to 0.18. These efforts are to improve the sensitivity of our analysis as shown in the next chapter. There is reasonable agreement throughout the signal region, and in \PT~distributions it can be seen that the signal peaks higher when the cuts are placed. These cuts were chosen because they optimized $S/\sqrt{B}$ which in this case prioritized preserving statistics over improving signal purity. With more data collected these cuts could be tightened to further improve $S/\sqrt{B}$. 



\QUADFIGLEG{FinalLeadingLeptonPt}{FinalSecondLeptonPt}{FinalThirdLeptonPt}{FinalLeadingJetPt}{Distributions of transverse momenta for (a) the leading lepton, (b) the second lepton, (c) the third lepton, and (d) the leading jet in the signal region.}{FIGURE-SigRegion}
\QUADFIGLEG{Finalnjet}{Finalnbjet}{WtransM}{Finalmet}{Distributions of (a) jet multiplicity, (b) \bjet~multiplicity, (c) \wtm, and (d) \met~in the signal region.}{FIGURE-SigRegion2}


\DBLFIGLEG{nelec}{nmuon}{(a) Number of electrons and (b) number of muons.}{FIGURE-NMUON}


\QUADFIGLEG{TopPolOpt}{TopPolHel}{Whelicity}{HISTO-TOPMASS}{Distributions of the \athyph~polarization in the (a) Optimal basis and (b) the helicity basis, (c) the \awhyph~helicity, and (d) the mass of the \at.}{FIGURE-TRIPFIG-FINAL2}



\chapter{Results}
\label{SECTION-RESULTS}

Bayesian address the question everyone is interested in by using assumptions no one believes. Frequentest use impeccable logic to deal with an issue of no interest to anyone. - L.Lyons

\vspace{5mm} %5mm vertical space

A single bin profile likelihood calculation is performed to extract limits on the \tz~\xs~at the 95\% confidence limit using roostats~\cite{ROOSTAT}. Profile likelihood calculations can produce confidence intervals on non-normal distributions more accurately than maximum or partial likelihood functions~\cite{st0132} and for this reason they have become a popular statistical method for high energy physics. However, before we can do this evaluation, a series of systematic uncertainties must be adressed and evaluated. The expected sensitivity to larger data-sets from the LHC is also evaluated. 

\section{Systematic Uncertainties}
\label{SECTION-systematics}

Systematic uncertainties on the object reconstruction, event reconstruction, normalization, and theoretical modeling affect the acceptance and expected event yield for each source. Tables~\ref{tab:systematicsMODEL} and~\ref{tab:jetsystematics} contain evaluated uncertainties. Some uncertainties are symmetric in nature, while others have distinct up and down variations. Nearly all uncertainties have been symmetrized either because of practical reasons (their effect is small so we can simplify them or they happen to come out symmetric) or for theoretical reasons (there is a physical motivation for them to be symmetric). For these symmetric systematic uncertainties both up and down variations are considered and the greater of the two is used~\cite{TOPCOMMONSYSTEMATICS}. 

\begin{itemize}

\item Luminosity - The uncertainty on the integrated luminosity is $\pm$2.1\%. It is obtained from Van Der Meer scans are performed in which the beam positions in the $x-y$ plane are varied~\cite{Balagura:2011yw,Aad:2013ucp}.

\item Pile Up - Pile up is discussed briefly in Chapter~\ref{SECTION-TRIGGERS}. Here we need to evaluate how well we estimate the degree to which pile up interferes with our ability to distinguish events from each other. This is one of the few uncertainties that was not symmetrized having different uncertainties for the up and down variations~\cite{JETUNCERTAINTIES}. 

\item Lepton efficiency scale factors - Leptons from our simulated Monte Carlo samples are needed to replicate our data in identification criteria (Electron ID, Muon ID Systematic, and Muon ID Statistics), isolation criteria, and trigger simulation (Electron Trigger, Muon Trigger). A prescription for how to assess this uncertainty is provided by the EGamma (which evaluates electrons and photons) and Muon groups which are derived from $Z->\ell\ell$ samples.~\cite{ELECTRON-RECO,muonSFWiki}

\item Electron calibration - Electron momentum scale (Electron Scale) and resolution (Electron Resolution) are handled separately from lepton efficiency scale factors. Scale corrections are derived for data and smearing corrections for Monte Carlo. These corrections assess the systematic uncertainties associated with the processing of photons, and in this case, electrons~\cite{Fayard:2060328}. %(EG_SCALE_ALL) (EG_RESOLUTION_ALL)

\item Muon calibration - Muon momentum scale and resolution are handled separately from lepton efficiency scale factors. Muon track identification (Muon track ID), transverse momentum scale (Muon Scale), and resolution (Muon Resolution) are corrected as well as~\cite{muonSFWiki}. %MUONS SCALE MUONS MS MUONS ID

\item \met~calibration - Lepton and jet energy and momentum scale and resolution uncertainties propagate into calculations of \met~(giving MET Scale and MET Resolution). How we include soft tracks into this calculation corresponds to a source of uncertainty~\cite{METWiki}.

%There are three sources of uncertainties that manifest from comparing the soft track components of \met~to the hard track components. The vector sum of hard objects creates an axis known as ptHard. The three sources are the scale of the soft tracks shifted along ptHard, the \PT~smearing of soft tracks along ptHard, and the \PT~smearing of soft tracks perpendicular to ptHard. The MissingETUtility tool is used to evaluate these effects.~\cite{METWiki}

\item Jet energy scale (JES) - JES  and  its  uncertainty  are  derived  combining  information from test-beam data, collision data, and simulation. The JES uncertainty is split into several orthogonal components using $in situ$ techniques resulting in independent effective uncertainties. This is determined with 8 TeV data and extrapolated to 13 TeV running conditions~\cite{ATL-PHYS-PUB-2015-015}.
%NPScinario1 Grouped NP stuff

\item Jet energy resolution (JER) - The precision with which a jet's energy is measured has an uncertainty associated with it. A mis-modeling of this energy resolution can lead to varying acceptances in final state kinematics~\cite{ATL-PHYS-PUB-2015-015,JERWiki}.

\item \bjet~tagging - \btag ing scale factors are used on a per-event basis to correct \btag ing efficiency. This is determined with 8 TeV data and extrapolated to 13 TeV running conditions using three independent eigenvectors for the efficiency of \bjet s, \cjet s, and light jets as well as two parameters to account for the extrapolation from 8 to 13~TeV~\cite{Aad:2015ydr}.

%A 6 element eigenvector is needed for the efficiency of \bjet s, a 4 element eigenvector is needed for the efficiency of $c$-jets, and a 12 element eigenvector is needed for the efficiency of light jets. 

\item Initial-state radiation and final-state radiation (ISR/FSR) - ISR/FSR is evaluated on the \TTB~sample by varying the renormalization and factorization scales up and down by a factor of two from the nominal value of 1. This process is done to \TTB~because it is the dominant background to single top analyses and is small when compared to other uncertainties that affect the other backgrounds. 

%(RADHI and RADLO) 

\item NLO subtraction - The uncertainties of how the NLO subtraction method is applied is evaluated on the \TTB~sample. Powheg and aMC@NLO are two tools that are used to calculate higher order corrections. A comparison of the two tools applied to \TTB~is used to estimate this uncertainty.
% aMcAtNloHerwigpp

\item Parton showering (PS) and Hadronization - The uncertainty on parton showering and hadronization is evaluated by comparing the cluster model in Herwig and the Lund string model in Pythia applied to \TTB. A comparison of the two techniques implemented in these tools is used to estimate the uncertainty on this process. 
%PowhegHerwigppEvtGen

\item Parton Distribution Function (PDF) - The uncertainties that come from the choice of PDF is evaluated on the \TTB~sample by comparing PDF4LHC15 and CT10. 
%PowhegPythia8

\item Normalization - Normalization uncertainties for Diboson and \zjets~are estimated from control regions. For \TTB~\cite{ttbarxsecUNCERT}, single top~\cite{sgtopxsecUNCERT}, and \ttz~\cite{ttVxsecUNCERT}, theory uncertainties on scale variations, PDF, and top-quark mass are used.

\item MC Statistics - The uncertainty due to limited statistics in our simulated samples is assessed by taking the sum of the square of the weights of each event in each sample. When selecting for a narrow piece of phase space in order to look for small signals as is done in this analysis, it becomes increasingly difficult to both separate signal from background and maintain meaningful statistics both for MC and data.


\end{itemize}




\begin{table} [ht!]
\setlength{\tabcolsep}{2pt}
\footnotesize
\centering
\begin{tabular}{| l | c | c | c | c | c | c | c |}
\hline
\hline
Systematics & $t\bar{t}$ & Other Top & $Z$ + jets & Diboson & $tZ$ & background total \\
\hline
\hline

Pile Up UP & -11\% & 64\% & 94\% & 10\% & -2.4\% & 5.8\% \\
Pile Up DOWN & 2.1\% & -13\% & 6.8\% & 1.8\% & 0.69\% & 1.3\% \\

\hline
Normalization & $\pm $ 5.5\% & $\pm $ 10\% & $\pm $ 20\% & $\pm $ 20\% & $\pm $ - & $\pm $ 10\% \\
\hline

MC Statistics & $\pm $ 8.3\% & $\pm $ 9.2\% & $\pm $ 47\% & $\pm $ 3.0\% & $\pm $ 1.6\% & $\pm $ 9.9\% \\

\hline

PDF & $\pm $ 4.3\% & - & - & - & - & $\pm $ 2.3\% \\
PS and Hadronization & $\pm $ 0.86\% & - & - & - & - & $\pm $ 0.46\% \\
NLO subtraction & $\pm $ 20\% & - & - & - & - & $\pm $ 11\% \\
ISR/FSR RadLo & 27\% & - & - & - & - & 14\% \\
ISR/FSR RadHi & -24\% & - & - & - & - & -11\% \\

\hline
\hline

\end{tabular}
\caption{Systematic uncertainties related to background normalization and theory modeling. Other Top is the combination of \TTB V and single top.}
\label{tab:systematicsMODEL}
\end{table}



\clearpage



\begin{table} [ht!]
\setlength{\tabcolsep}{2pt}
\footnotesize
\centering
\begin{tabular}{| l | c | c | c | c | c | c | c |}
\hline
\hline
Systematics & $t\bar{t}$ & Other Top & $Z$ + jets & Diboson & $tZ$ & background total \\
\hline
\hline


Muon ID Systematic & $\pm $ 0.57\% & $\pm $ 0.90\% & $\pm $ 1.1\% & $\pm $ 0.90\% & $\pm $ 1.0\% & $\pm $ 0.77\% \\
\hline

Muon ID Statistics & $\pm $ 0.48\% & $\pm $ 0.90\% & $\pm $ 0.57\% & $\pm $ 0.60\% & $\pm $ 0.69\% & $\pm $ 0.570\% \\
\hline

Electron ID & $\pm $ 2.6\% & $\pm $ 1.8\% & $\pm $ 1.1\% & $\pm $ 1.5\% & $\pm $ 1.7\% & $\pm $ 2.1\% \\
\hline

Electron Trigger & $\pm $ 2.1\% & $\pm $ 5.4\% & $\pm $ 6.8\% & $\pm $ 1.8\% & $\pm $ 0.69\% & $\pm $ 1.3\% \\
\hline

Electron Reconstruction & $\pm $ 1.2\% & $\pm $ 0.90\% & $\pm $ 1.1\% & $\pm $ 0.90\% & $\pm $ 0.69\% & $\pm $ 1.0\% \\
\hline

Electron Scale & $\pm $ 1.7\% & $\pm $ 3.8\% & $\pm $ 0\% & $\pm $ 1.2\% & $\pm $ 0\% & $\pm $ 0.99\% \\
\hline

Electron Resolution & $\pm $ 0.86\% & $\pm $ 2.9\% & $\pm $ 0\% & $\pm $ 0.30\% & $\pm $ 0.34\% & $\pm $ 0.72\% \\
\hline

Muon Scale & $\pm $ 0.48\% & $\pm $ 2.1\% & $\pm $ 0.57\% & $\pm $ 0.60\% & $\pm $ 0.69\% & $\pm $ 0.57\% \\
\hline
Muon Resolution & $\pm $ 0.48\% & $\pm $ 7.0\% & $\pm $ 2.8\% & $\pm $ 0.90\% & $\pm $ 0.69\% & $\pm $ 0.41\% \\
\hline
Muon track ID & $\pm $ 1.3\% & $\pm $ 1.9\% & $\pm $ 0.57\% & $\pm $ 0.60\% & $\pm $ 0.69\% & $\pm $ 0.82\% \\
\hline

MET Scale & $\pm $ 0.67\% & $\pm $ 0\% & $\pm $ 0\% & $\pm $ 0.60\% & $\pm $ 0.34\% & $\pm $ 0.46\% \\
\hline
MET Resolution & $\pm $ 1.3\% & $\pm $ 0\% & $\pm $ 0\% & $\pm $ 0\% & $\pm $ 0.34\% & $\pm $ 0.77\% \\
\hline

JER & $\pm $ 4.7\% & $\pm $ 4.5\% & $\pm $ 130\% & $\pm $ 8.4\% & $\pm $ 1.40\% & $\pm $ 10\% \\
\hline

bTagSF \bjet s& $\pm $ 5.7\% & $\pm $ 1.2\% & $\pm $ 1.1\% & $\pm $ 0.30\% & $\pm $ 1.2\% & $\pm $ 3.2\% \\

\hline

bTagSF \cjet s & $\pm $ 0.48\% & $\pm $ 1.3\% & $\pm $ 6.0\% & $\pm $ 10\% & $\pm $ 0\% & $\pm $ 2.0\% \\

\hline

bTagSF light jets & $\pm $ 1.7\% & $\pm $ 2.1\% & $\pm $ 12\% & $\pm $ 13\% & $\pm $ 1.0\% & $\pm $ 2.4\% \\

\hline

JES 1 up & 1.2\% & -3.0\% & 150\% & 9.6\% & 0\% & 15\% \\
JES 1 down & -4.7\% & -3.0\% & 130\% & 8.4\% & 1.3\% & 10\% \\

JES 2 up & 0.76\% & -2.1\% & 290\% & 4.5\% & 0.34\% & 27\% \\
JES 2 down & -2.1\% & -1.2\% & 0\% & -3.4\% & 0\% & -1.7\% \\

JES 3 up & 4.1\% & -0.90\% & 190\% & 10\% & 0.69\% & 20.90\% \\
JES 3 down & -4.6\% & 2.1\% & 0\% & -10.0\% & -0.69\% & -4.1\% \\

\hline
\hline
\end{tabular}
\caption{Systematic uncertainties related to object identification, resolution, and scale. Other Top is the combination of \TTB V and single top.}
\label{tab:jetsystematics}
\end{table}



\clearpage



\section{Statistical Analysis}
\label{SECTION-stats}

Maximum likelihood ratio tests are among the most used methods in statistics because of their strength in hypothesis testing and generality. A popular variant of this method is the profile likelihood ratio test which considers nuisance parameters which are not of primary interest ($\theta$) to be functions of the  parameter which is of interest ($\beta$). The parameter of interest, $\beta$, in this case is defined as the ratio of the measured cross section to the standard model cross section. The nuisance parameters, $\theta$, are measures of systematic uncertainties which are modeled by Gaussian statistics. By profiling we simplify the problem of finding $\beta$ and $\theta$ which optimizes the likelihood function in Equation~\ref{EQUATION-MaxLike} to constrain $\theta = f(\beta )$ so that we can optimize Equation~\ref{EQUATION-ProfLike} which is often a preferable procedure when only one nuisance parameter is important.

\begin{equation}
\mathcal{L}(\beta ,\theta |data) 
\label{EQUATION-MaxLike}
\end{equation}

\begin{equation}
\mathcal{L}(\beta ,f(\beta )|data)
\label{EQUATION-ProfLike}
\end{equation}

Because we have only one parameter which needs to be optimized for a profile likelihood fit is performed. This procedure is further simplified by only considering a distribution of a single bin. This simplification makes the profiling of the nuisance parameters easy, as each is simply a Gaussian, not dependent on the parameter of interest at all. These nuisance parameters are treated as correlated between sources of signal and background in the optimization procedure. The signal \xs~is then extracted from the likelihood function. The extracted \xs~measurement is $\sigma_{tZ}$~=~448~$\pm$~672~(stat)~$\pm$~448~(syst)~fb. This is 1.9 times the expected Standard Model \xs~of 236~fb which is due to the data excess over the expected background shown in Table~\ref{tab:eventyieldFullSelec}. Because of the large uncertainties this is still in agreement with the standard model expectation. This corresponds to an upper bound at the 95\% confidence limit on the \tz~\xs~of $\sigma_{tZ}$ = 1345~fb. A lower bound can not be set due to large systematic uncertainties. The most notable systematic uncertainties in this analysis are the experimental uncertainties JES and JER, MC statistics for samples that have a mis-reconstructed lepton, and normalization uncertainties.  

\section{Outlook}
\label{SECTION-outlook}

In order to estimate the potential sensitivity to \tz~with increased data collection, a series of simplified statistical analyses is performed. Systematic uncertainties are removed in order to see the effects of increased statistics in an idealized way. The expected yields obtained by this analysis are scaled up by a factor of 10 to estimate the expected precision of the cross-section measurement with the full 2016 data set, which corresponds to an integrated luminosity of approximately 30~\fb. The expected yields are then also scaled up by a factor of 100 in order to estimate the expected precision of the cross-section measurement with the full Run~2 and Run~3 data set, which corresponds to an integrated luminosity of approximately 300~\fb. When this is performed with the event yields of this analysis we can get an expected uncertainty on the \xs~of 150\%. With the full 2016 data set the expected uncertainty drops to 50\%. With the full set of run 2 data the expected uncertainty falls to 20\%. This analysis is currently statistics limited, but with the full run 2 data set we will become systematics limited. The most immediate gains can be made from increasing MC statistics, with longer-term gains to be made from better understanding JES and JER. To improve the sensitivity of this analysis, more complex multivariate analysis methods could be employed, the profile likelihood could be performed on a strong discriminating distribution, and/or control regions could be fit and included in the statistical analysis. Beyond that we will need to wait for the LHC to deliver more data in order to put further constraints on the \tz~\xs. 



\chapter{Conclusion}
We have analyzed \LUMI\ of data collected with the ATLAS detector. In our search for the \Wtchan\ we have seen a statistically significant excess of 3.3$\sigma$. This is sufficient to claim evidence, and although this does not meet the $>5\sigma$ criteria to claim observation, it is a significant step to verifying the Standard Model prediction. The estimated cross-section is also extracted from the data, giving a result of $\sigma(pp\rightarrow Wt + X) = 16.8 ^{+2.9}_{-2.9} \mathrm{(stat)} ^{+4.9}_{-4.9} \mathrm{(syst)}~pb$. This analysis also allowed us to make measurements of other Standard Model parameters. The CKM matrix element $V_{tb}$ is measured to be $|V_{tb}| = 1.03^{+0.16}_{-0.19}$. The width of the top quark is measured at $\Gamma_{t}^{obs.} = 1.4\pm 0.5~\rm GeV$ (Note the increase in the percent uncertainty due to the $|V_{tb}|^2$ dependence), giving a lifetime of $\tau_{t}=(4.7^{+1.2}_{-1.2})\times 10^{-25}~s$. These measurements are all consistent with theoretical Standard Model predictions and other experimental measurements. This analysis is published in Physics Letters B~\cite{WTEVIDENCE}.

In this analysis I implemented the BDT used, which includes the variable selection and testing, the training procedure, and the parameter optimization. I implemented the ATLAS and top group recommendations for the object definitions, event selection, and studied most of the systematics (the jet energy scale, jet reconstruction, jet ID, lepton ID, lepton resolution, \MET, and pile-up uncertainties). The data-driven \Ztt\ normalization is estimated by me. I prepared the plots of the BDT and plots of the variables used. During the preparation of the paper and the associated note, I gave many single top working group talks and the approval talk to the top working group. I also collaborated with Huaqiao Zhang to perform many cross-checks while going through review.

With time the systematic uncertainties will be better understood and in the future this analysis will be repeated with more data. However, there is ample room for improvement in the analysis procedure itself. Note that the BDT optimization is done using only the nominal \MC. A look at the uncertainty composition of the final cross-section measurement will reveal that this analysis is quite systematically limited. A BDT optimization using information from the systematically shifted datasets could bring significant improvement to the result as a whole. This is not a trivial undertaking, as the existing toolsets are not equipped to do this kind of optimization out of the box, however implementing a systematics-sensitive optimization has the potential to greatly increase the significance.

This evidence for the existence of the \Wtchan\ was also confirmed independently by the CMS collaboration~\cite{CMSEVIDENCE}. Both the CMS and ATLAS collaborations will continue to update these analyses with better analysis techniques, a better understanding of the systematic uncertainties, and more data. The discovery of the \Wtchan\ is not the end, of course. Precision measurements of $V_{tb}$ and the top quark properties and searches for new physics in the \Wtchan\ signal region are all exciting new analyses waiting to be explored.

The LHC era is already showing its promise, giving exciting results like the recent Higgs discovery~\cite{HiggsATLAS,HiggsCMS} and confirming the predictions of the Standard Model. Even with the Higgs boson discovered, there remains much discovery ahead. The LHC will be running for years, pushing our understanding forward. With each collision we strive for a better understanding of our universe, and with time and hard work, these efforts will be rewarded.

%\newpage
%\vspace*{\fill}
%\begin{center}
%\Huge \textbf{APPENDICIES}
%\end{center}
%\vfill
%\newpage
%\appendix
%\chapter{Data/MC Agreement in Control Regions}
\label{APPENDIX-CONTROLREGIONS}
This appendix shows the BDT variables in the background-enhanced 2-jet and 3-jet regions. The 2-jet and 3-jet regions clearly show how dominant of a background \ttbar\ is for this analysis. Due to the strong \ttbar\ contribution we are able to use these regions to constrain the \ttbar\ normalization, which would otherwise be a dominating uncertainty. Selected variables are also shown in the three dilepton channels: $ee$, $e\mu$, and $\mu\mu$. The dilepton subchannels show that the good data-simulation agreement does not break down when these subchannels are examined independently. 

\section{2-jet events}
\label{APPENDIX-CONTROLREGIONS-2J}
\SEXFIGLEG{paper_ll2j_LP2fb_v4_pT_sys_flat}{paper_ll2j_LP2fb_v4_pT_sys_sig_flat}{paper_ll2j_LP2fb_v4_AllJetsLepton_Centrality_flat}{paper_ll2j_LP2fb_v4_ThrustEta_flat}{paper_ll2j_LP2fb_v4_SystemLep1Lep2_eta_flat}{legend}{The top five variables in the BDT ranked by separation power, comparing the signal and background estimate to the data in the 2-jet bin.}{FIGURE-CONTROL-2JVARIABLES1}

\SEXFIGLEG{paper_ll2j_LP2fb_v4_eta_sys_lepsJet1_flat}{paper_ll2j_LP2fb_v4_LeadingLeptonEta_flat}{paper_ll2j_LP2fb_v4_SystemLep1Lep2_E_flat}{paper_ll2j_LP2fb_v4_HT_AllJets_flat}{paper_ll2j_LP2fb_v4_pT_sys_lepsJet1_flat}{legend}{The 6th-10th top variables in the BDT ranked by separation power, comparing the signal and background estimate to the data in the 2-jet bin.}{FIGURE-CONTROL-2JVARIABLES2}

\SEXFIGLEG{paper_ll2j_LP2fb_v4_Thrust_flat}{paper_ll2j_LP2fb_v4_InvariantMass_Lep2Jet1_flat}{paper_ll2j_LP2fb_v4_SystemLep1Jet1_eta_flat}{paper_ll2j_LP2fb_v4_SubLeadingLeptonEta_flat}{paper_ll2j_LP2fb_v4_Jet1Eta_flat}{legend}{The 11th-15th top variables in the BDT ranked by separation power, comparing the signal and background estimate to the data in the 2-jet bin.}{FIGURE-CONTROL-2JVARIABLES3}

\SEXFIGLEG{paper_ll2j_LP2fb_v4_DeltaMinPhiLeptonLeadingJet_flat}{paper_ll2j_LP2fb_v4_InvariantMass_Lep1Jet1_flat}{paper_ll2j_LP2fb_v4_DeltaPhi_SLep1Jet1_Lep2_flat}{paper_ll2j_LP2fb_v4_MET_flat}{paper_ll2j_LP2fb_v4_DeltaEtaLeadingLeptonLeadingJet_flat}{legend}{The 16th-20th top variables in the BDT ranked by separation power, comparing the signal and background estimate to the data in the 2-jet bin.}{FIGURE-CONTROL-2JVARIABLES4}

\TRPFIGLEG{paper_ll2j_LP2fb_v4_DeltaRSubLeadingLeptonLeadingJet_flat}{paper_ll2j_LP2fb_v4_InvariantMass_MaxLepJet1_flat}{legend}{The 21st and 22nd top variables in the BDT ranked by separation power, comparing the signal and background estimate to the data in the 2-jet bin.}{FIGURE-CONTROL-2JVARIABLES5}

\newpage

\section{3-jet inclusive events}
\label{APPENDIX-CONTROLREGIONS-3J}
\SEXFIGLEG{paper_ll3jinc_LP2fb_v4_pT_sys_flat}{paper_ll3jinc_LP2fb_v4_pT_sys_sig_flat}{paper_ll3jinc_LP2fb_v4_AllJetsLepton_Centrality_flat}{paper_ll3jinc_LP2fb_v4_ThrustEta_flat}{paper_ll3jinc_LP2fb_v4_SystemLep1Lep2_eta_flat}{legend}{The top five variables in the BDT ranked by separation power, comparing the signal and background estimate to the data in the 3-jet inclusive bin.}{FIGURE-CONTROL-3JVARIABLES1}

\SEXFIGLEG{paper_ll3jinc_LP2fb_v4_eta_sys_lepsJet1_flat}{paper_ll3jinc_LP2fb_v4_LeadingLeptonEta_flat}{paper_ll3jinc_LP2fb_v4_SystemLep1Lep2_E_flat}{paper_ll3jinc_LP2fb_v4_HT_AllJets_flat}{paper_ll3jinc_LP2fb_v4_pT_sys_lepsJet1_flat}{legend}{The 6th-10th top variables in the BDT ranked by separation power, comparing the signal and background estimate to the data in the 3-jet inclusive bin.}{FIGURE-CONTROL-3JVARIABLES2}

\SEXFIGLEG{paper_ll3jinc_LP2fb_v4_Thrust_flat}{paper_ll3jinc_LP2fb_v4_InvariantMass_Lep2Jet1_flat}{paper_ll3jinc_LP2fb_v4_SystemLep1Jet1_eta_flat}{paper_ll3jinc_LP2fb_v4_SubLeadingLeptonEta_flat}{paper_ll3jinc_LP2fb_v4_Jet1Eta_flat}{legend}{The 11th-15th top variables in the BDT ranked by separation power, comparing the signal and background estimate to the data in the 3-jet inclusive bin.}{FIGURE-CONTROL-3JVARIABLES3}

\SEXFIGLEG{paper_ll3jinc_LP2fb_v4_DeltaMinPhiLeptonLeadingJet_flat}{paper_ll3jinc_LP2fb_v4_InvariantMass_Lep1Jet1_flat}{paper_ll3jinc_LP2fb_v4_DeltaPhi_SLep1Jet1_Lep2_flat}{paper_ll3jinc_LP2fb_v4_MET_flat}{paper_ll3jinc_LP2fb_v4_DeltaEtaLeadingLeptonLeadingJet_flat}{legend}{The 16th-20th top variables in the BDT ranked by separation power, comparing the signal and background estimate to the data in the 3-jet inclusive bin.}{FIGURE-CONTROL-3JVARIABLES4}

\TRPFIGLEG{paper_ll3jinc_LP2fb_v4_DeltaRSubLeadingLeptonLeadingJet_flat}{paper_ll3jinc_LP2fb_v4_InvariantMass_MaxLepJet1_flat}{legend}{The 21st and 22nd top variables in the BDT ranked by separation power, comparing the signal and background estimate to the data in the 3-jet inclusive bin.}{FIGURE-CONTROL-3JVARIABLES5}
\newpage
\section {Dilepton subchannels}
This section contains selected variables of the different dilepton final states.  This illustrates that our backgrounds are well modeled for each of the final states individually.

\SEXFIGLEG{paper_ee1+j_LP2fb_v4_NJets_flat}{paper_ee1+j_LP2fb_v4_Jet1Pt_flat}{paper_ee1+j_LP2fb_v4_HT_AllJets_flat}{paper_ee1+j_LP2fb_v4_MET_flat}{paper_ee1+j_LP2fb_v4_LeadingLeptonPt_flat}{legend}{Distributions of variables comparing the signal and background estimate to the data  in the $ee$ channel. (a) Jet multiplicity, (b) Leading jet \pT, (c)$H_T(jet)$, (d) \MET, (e) Leading lepton \pT}{FIGURE-PRESEL-EE}
\SEXFIGLEG{paper_em1+j_LP2fb_v4_NJets_flat}{paper_em1+j_LP2fb_v4_Jet1Pt_flat}{paper_em1+j_LP2fb_v4_HT_AllJets_flat}{paper_em1+j_LP2fb_v4_MET_flat}{paper_em1+j_LP2fb_v4_LeadingLeptonPt_flat}{legend}{Distributions of variables comparing the signal and background estimate to the data  in the $e\mu$ channel. (a) Jet multiplicity, (b) Leading jet \pT, (c)$H_T(jet)$, (d) \MET, (e) Leading lepton \pT}{FIGURE-PRESEL-EM}
\SEXFIGLEG{paper_mm1+j_LP2fb_v4_NJets_flat}{paper_mm1+j_LP2fb_v4_Jet1Pt_flat}{paper_mm1+j_LP2fb_v4_HT_AllJets_flat}{paper_mm1+j_LP2fb_v4_MET_flat}{paper_mm1+j_LP2fb_v4_LeadingLeptonPt_flat}{legend}{Distributions of variables comparing the signal and background estimate to the data  in the $\mu\mu$ channel. (a) Jet multiplicity, (b) Leading jet \pT, (c)$H_T(jet)$, (d) \MET, (e) Leading lepton \pT}{FIGURE-PRESEL-MM}

%\chapter{\bstar\ search}
\label{SECTION-BPRIME}

This appendix will describe another analysis I worked on. In this analysis I implemented the object definitions, the event selection, and most of the systematic uncertainties. I studied the potential templates we considered using and attempted to reconstruct the neutrinos using invariant mass constraints, although this is not effective enough to make it into the paper. This analysis has been accepted for publication in Physics Letters B, and will be published in the near future (preprint~\cite{BPRIMEPREPRINT}). It is a search for a hypothetical \bstar\ excited state using $4.7\ fb^{-1}$ of integrated luminosity. This search uses ATLAS data in the same final state as the \Wtchan\ analysis, hence the object definitions and event selection criteria will be similar to the \Wtchan\ analysis. In addition, this appendix will give an overview of the analysis with the focus being the significant differences between the two. As a result, some of the details in common with the \Wtchan\ analysis will be glossed over. For a full description of this search, please consult the ATLAS note for this analysis~\cite{BPRIMEINT}.

\section{Introduction to \bstar}

This analysis is motivated in part by the fine-tuning problem, which is illustrated by examining the Standard Model Higgs mass to a one loop correction~\cite{PDG}
 
\begin{equation}
m_{H}^2 = m_{H_0}^2 + \frac{kg^2\Lambda^2}{16\pi^2}.
\end{equation}

\noindent

where $m_{H}$ is the observed Higgs mass, $m_{H_0}$ is an unmeasured fundamental parameter, $g$ is the electroweak coupling, $k$ is a constant expected to be $\mathcal{O}(1)$, and $\Lambda^2$ is tge energy scale of new physics. If $\Lambda$ is large, such as the Planck scale, then the $m_{H_0}$ parameter must be carefully balanced with the second term to cancel it out to give the observed Higgs mass. This is referred to as the fine-tuning problem in high energy physics. This amount of fine-tuning seems unnatural, thus it is suspected that there is other physics at work here. Theorists have made significant efforts to address this problem with models that modify the Standard Model to avoid the fine-tuning. Supersymmetry models describing massive supersymmetric partners~\cite{PDG} for every particle currently in the Standard Model are an example of such efforts.

Instead of a new family of massive particles, smaller additions to the Standard Model are often considered~\cite{Nutter}. Because the largest corrections to the Higgs mass arise from the top quark in loops such as that shown in Fig.~\ref{FIGURE-HIGGSLOOP}, an excited state of the top quark can cancel out those corrections. In addition, if an excited top quark is added, an associated excited bottom quark should also exist. We may expect that the mass hierarchy of these excited states would mirror the hierarchy we see in the Standard Model, hence in this analysis we search for a single theoretical excited state of the bottom quark that will be referred to as \bstar. 

\LARGEFIG{HiggsLoop}{A correction to the Higgs mass from the top quark.}{FIGURE-HIGGSLOOP}

The experimental constraints on this \bstar\ state require it to be much more massive than the Standard Model particles. Due to this high mass some of the \bstar-state's most common decays lead to high mass final states. In general, the most common decay modes are expected to be $\bstar \to Zb$, $\bstar \to bg$, $\bstar \to bH$, and $\bstar \to Wt$. This analysis searches for the decay mode $\bstar \to Wt$, illustrated in Fig.~\ref{FIGURE-BPRIME-FEYNMAN}. This decay mode varies in branching ratio from about 20\% at low mass (200 GeV) to approximately 40\% at high \bstar\ masses (400 GeV). The theoretical cross-section for $p\bar{p} \to \bstar \to Wt$ production in the model~\cite{Nutter} at the LHC at 7 TeV are shown in Table~\ref{BprimeCrossSection}.

\begin{table}[htdp]
\begin{center}
\begin{tabular}{r r@{.}l|r r@{.}l}  \hline \hline
mass point [$\GeV$] & \multicolumn{2}{c}{cross-section [pb]} & mass point [$\GeV$] & \multicolumn{2}{c}{cross-section [pb]}\\
\hline
300 & 181&2& 900 & 0&804  \\
400 & 69&21& 1000& 0&394  \\
500 & 24&45& 1100& 0&201  \\
600 & 9&366& 1200& 0&106  \\
700 & 3&884& 1300& 0&057 \\
800 & 1&719& 1400& 0&031 \\
\hline\hline
\end{tabular}
\caption{The total cross-section of $\bstar \rightarrow Wt$ in a mass range of 300 GeV to 1400 GeV.}
\label{BprimeCrossSection}
\end{center}
\end{table}

\FIG{bprime}{A Feynman diagram illustrated the \bstar\ decay investigated in this analysis.}{FIGURE-BPRIME-FEYNMAN}

This analysis is constructed to be sensitive to generic resonances in the \Wt\ final state and observed deviations from the Standard Model may also be caused by other resonances. In addition, coupling limits are calculated for three potential \bstar\ models: a \bstar-state with only left-handed couplings, a \bstar-state with only right-handed couplings, and a vector \bstar-state with both right and left-handed couplings with equal magnitude. These limits are calculated on a two-dimensional plane along with the mass of the \bstar-state. An example of this plane can be seen in Fig.~\ref{FIGURE-BPRIME-LIMIT3} in Section~\ref{SECTION-BPRIME-MEASUREMENT}.

Like the \Wtchan\ analysis, this analysis looks at the dilepton final state. This analysis uses the full 2011 dataset with updated simulation and systematic implementations. Another analysis was performed by a second group looking at the leptons+jets final state~\cite{BSTAR-LEPJETS}. These two analyses then collaborated to produce a unified result. The methods used to combine these two analysis will be discussed in Section~\ref{SECTION-BPRIME-MEASUREMENT}.

\section{Simulation}
\label{SECTION-BPRIME-SIMULATION}
Because the final state in this analysis is the same as the final state in the \Wtchan\ dilepton analysis, the backgrounds for these analyses are identical, except that the \Wtchan\ is a Standard Model background to the \bstar\ process. The signal in this analysis is simulated using Madgraph5~\cite{MADGRAPH} for the generation and Pythia~\cite{PYTHIA} for the hadronization. In total 12 simulated samples are generated representing \bstar\ with masses from 300 GeV to 1400 GeV in 100 GeV increments. The cross-section of \bstar\ production is dependent on the mass point, and these cross-sections are given in Table~\ref{BprimeCrossSection}. In addition, dedicated simulation samples are generated to study the impact of the uncertainty in the initial and final state radiation modeling.
The backgrounds are modeled using the same general scheme as the \Wt\ analysis, but updated to match the full 2011 ATLAS recommendations, described in the note~\cite{BPRIMEINT}. The full list of simulated samples is shown in Tables~\ref{TABLE-MCSAMPLES1},~\ref{TABLE-MCSAMPLES2}, and~\ref{TABLE-MCSAMPLESDiBoson}.


\begin{table}[htdp]
\begin{center}
\begin{tabular}{lrrrr}
\hline
Description         & $\sigma$ [pb]  & $L_{int}$ [$fb^{-1}$] &  $N_{MC}$& Generator+Shower \\[1mm]
\hline \hline
$\bstar\to Wt$, $M_{\bstar}=300 \ GeV$, & 61.6 & 3.2 & 200k &MadGraph+Pythia \\[1mm]
$\bstar\to Wt$, $M_{\bstar}=400 \ GeV$, & 23.5 & 8.5 & 200k &MadGraph+Pythia \\[1mm]
$\bstar\to Wt$, $M_{\bstar}=500 \ GeV$, & 8.31 & 24 & 200k &MadGraph+Pythia \\[1mm]
$\bstar\to Wt$, $M_{\bstar}=600 \ GeV$, & 3.18 & 63 & 200k &MadGraph+Pythia \\[1mm]
$\bstar\to Wt$, $M_{\bstar}=700 \ GeV$, & 1.32 & 150 & 200k &MadGraph+Pythia \\[1mm]
$\bstar\to Wt$, $M_{\bstar}=800 \ GeV$, & 0.58 & 350 & 200k &MadGraph+Pythia \\[1mm]
$\bstar\to Wt$, $M_{\bstar}=900 \ GeV$, & 0.27 & 740 & 200k &MadGraph+Pythia \\[1mm]
$\bstar\to Wt$, $M_{\bstar}=1000\ GeV$, & 0.13 & 1500 & 200k &MadGraph+Pythia \\[1mm]
$\bstar\to Wt$, $M_{\bstar}=1100\ GeV$, & 0.07 & 2900 & 200k &MadGraph+Pythia \\[1mm]
$\bstar\to Wt$, $M_{\bstar}=1200\ GeV$, & 0.04 & 5000 & 200k &MadGraph+Pythia \\[1mm]
$\bstar\to Wt$, $M_{\bstar}=1300\ GeV$, & 0.02 & 10000 & 200k &MadGraph+Pythia \\[1mm]
$\bstar\to Wt$, $M_{\bstar}=1400\ GeV$, & 0.01 & 20000 & 200k &MadGraph+Pythia \\[1mm]
\hline
\hline\hline
\end{tabular}
\caption{\bstar\ simulated samples for the analysis. The cross-section column includes branching ratios. All \bstar\ simulated samples are generated with at least one leptonic \Wboson\ boson decay.
}
\label{TABLE-MCSAMPSIG1}
\end{center}
\end{table}


\begin{table}[htdp]
\begin{center}
\begin{tabular}{lrrrr}
\hline
Description         & $\sigma$ [pb]  & $L_{int}$ [$fb^{-1}$]&  $N_{MC}$& Generator+Shower \\[1mm]
\hline \hline
$\bstar\to Wt$, $M_{\bstar}=300 \ GeV$, ISRFSR- & 61.6 & 3.3 & 200k & MadGraph+Pythia \\[1mm]
$\bstar\to Wt$, $M_{\bstar}=400 \ GeV$, ISRFSR- & 23.5 & 8.5 & 200k & MadGraph+Pythia \\[1mm]
$\bstar\to Wt$, $M_{\bstar}=500 \ GeV$, ISRFSR- & 8.31 & 23 & 200k & MadGraph+Pythia \\[1mm]
$\bstar\to Wt$, $M_{\bstar}=600 \ GeV$, ISRFSR- & 3.18 & 63 & 200k & MadGraph+Pythia \\[1mm]
$\bstar\to Wt$, $M_{\bstar}=700 \ GeV$, ISRFSR- & 1.32 & 150 & 200k & MadGraph+Pythia \\[1mm]
$\bstar\to Wt$, $M_{\bstar}=800 \ GeV$, ISRFSR- & 0.58 & 340 & 200k & MadGraph+Pythia \\[1mm]
$\bstar\to Wt$, $M_{\bstar}=900 \ GeV$, ISRFSR- & 0.27 & 740 & 200k & MadGraph+Pythia \\[1mm]
$\bstar\to Wt$, $M_{\bstar}=1000\ GeV$, ISRFSR- & 0.13 &  1500 & 200k & MadGraph+Pythia \\[1mm]
$\bstar\to Wt$, $M_{\bstar}=1100\ GeV$, ISRFSR- & 0.07 & 2900 & 200k & MadGraph+Pythia \\[1mm]
$\bstar\to Wt$, $M_{\bstar}=1200\ GeV$, ISRFSR- & 0.04 & 5000 & 200k & MadGraph+Pythia \\[1mm]
$\bstar\to Wt$, $M_{\bstar}=1300\ GeV$, ISRFSR- & 0.02 & 10000 & 200k & MadGraph+Pythia \\[1mm]
$\bstar\to Wt$, $M_{\bstar}=1400\ GeV$, ISRFSR- & 0.01 & 20000 & 200k & MadGraph+Pythia \\[1mm]
\hline
$\bstar\to Wt$, $M_{\bstar}=300 \ GeV$, ISRFSR+ & 61.6 & 3.3 & 200k & MadGraph+Pythia \\[1mm]
$\bstar\to Wt$, $M_{\bstar}=400 \ GeV$, ISRFSR+ & 23.5 & 8.5 & 200k & MadGraph+Pythia \\[1mm]
$\bstar\to Wt$, $M_{\bstar}=500 \ GeV$, ISRFSR+ & 8.31 & 23 & 200k & MadGraph+Pythia \\[1mm]
$\bstar\to Wt$, $M_{\bstar}=600 \ GeV$, ISRFSR+ & 3.18 & 63 & 200k & MadGraph+Pythia \\[1mm]
$\bstar\to Wt$, $M_{\bstar}=700 \ GeV$, ISRFSR+ & 1.32 & 150 & 200k & MadGraph+Pythia \\[1mm]
$\bstar\to Wt$, $M_{\bstar}=800 \ GeV$, ISRFSR+ & 0.58 & 340 & 200k & MadGraph+Pythia \\[1mm]
$\bstar\to Wt$, $M_{\bstar}=900 \ GeV$, ISRFSR+ & 0.27 & 740 & 200k & MadGraph+Pythia \\[1mm]
$\bstar\to Wt$, $M_{\bstar}=1000\ GeV$, ISRFSR+ & 0.13 & 1500 & 200k & MadGraph+Pythia \\[1mm]
$\bstar\to Wt$, $M_{\bstar}=1100\ GeV$, ISRFSR+ & 0.07 & 2900  & 200k & MadGraph+Pythia \\[1mm]
$\bstar\to Wt$, $M_{\bstar}=1200\ GeV$, ISRFSR+ & 0.04 &  5000 & 200k & MadGraph+Pythia \\[1mm]
$\bstar\to Wt$, $M_{\bstar}=1300\ GeV$, ISRFSR+ & 0.02 &  10000 & 200k & MadGraph+Pythia \\[1mm]
$\bstar\to Wt$, $M_{\bstar}=1400\ GeV$, ISRFSR+ & 0.01 &  20000 & 200k & MadGraph+Pythia \\[1mm]
\hline
\hline\hline
\end{tabular}
\caption{\bstar\ simulated samples for the analysis. The cross-section column includes branching ratios. All \bstar\ simulated events are generated with at least one leptonic \Wboson\ boson decay.
}
\label{TABLE-MCSAMPSIG2}
\end{center}
\end{table}


\begin{table}[htdp]
\begin{center}
\begin{tabular}{lrrrr}
\hline
Description         & $\sigma$ [pb]  & $L_{int}$ [$fb^{-1}$] &  $N_{MC}$& Generator+Shower \\[1mm]
\hline \hline
\Wt\   all decays             & 15.74   & 13  & 200k         & MC@NLO+Herwig      \\[1mm]
$Wt$ Less ISRFSR              & 15.74   & 19  & 300k         & ACERMC+Pythia      \\[1mm]
$Wt$ More ISRFSR              & 15.74   & 19  & 300k         & ACERMC+Pythia      \\[1mm]
$t\bar{t}$ no fully hadronic  & 89.71   & 17  & 1,500k         & MC@NLO+Herwig      \\[1mm]
$t\bar{t}$ no fully hadronic  & 89.4    & 34  & 3,000k         & POWHEG+Herwig      \\[1mm]
$t\bar{t}$ no fully hadronic  & 89.4    & 34  & 3,000k         & POWHEG+Pythia      \\[1mm]
$t\bar{t}$ no fully hadronic Less ISRFSR  & 89.4 & 11 & 1,000k  & ACERMC+Pythia      \\[1mm]
$t\bar{t}$ no fully hadronic More ISRFSR  & 89.4 & 11 & 1,000k  & ACERMC+Pythia      \\[1mm]
\hline\hline
\end{tabular}
\caption{Top quark event simulated samples for the analysis. The cross-section column includes $k$-factors and branching ratios. All NLO simulated samples have been simulated with \pileup\ corresponding to 50~ns bunch trains.}
\label{TABLE-MCSAMPLES1}
\end{center}
\end{table}

\begin{table}[phtdp]
\begin{center}
\begin{tabular}{lrrrr}
\hline
Description         & $\sigma$ [pb]  & $L_{in}$ [$fb^{-1}$] &  $N_{MC}$& Generator+Shower \\[1mm]
\hline \hline
$Z\to \ell\ell$ + 0 parton   & 827.4  &  8.0          &  6,600k & ALPGEN+HERWIG \\[1mm]     
$Z\to \ell\ell$ + 1 partons  & 166.6  &  8.0          &  1,340k & ALPGEN+HERWIG \\[1mm]     
$Z\to \ell\ell$ + 2 partons  & 50.4   &  5.7          &    285k & ALPGEN+HERWIG \\[1mm]     
$Z\to \ell\ell$ + 3 partons  & 14.0   &  7.9          &    110k & ALPGEN+HERWIG \\[1mm]  
$Z\to \ell\ell$ + 4 partons  & 3.4    &  8.8          &     30k & ALPGEN+HERWIG \\[1mm]  
$Z\to \ell\ell$ + 5 partons  & 1.0    &  9.0          &     9k & ALPGEN+HERWIG \\[1mm]  
\hline
$W\to \ell\nu$ + 0 parton   & 8,296   &  0.4         &  3,500k & ALPGEN+HERWIG \\[1mm]     
$W\to \ell\nu$ + 1 partons  & 1,551   &  1.6         &  2,500k & ALPGEN+HERWIG \\[1mm]     
$W\to \ell\nu$ + 2 partons  &   452   &  8.3         &  3,770k & ALPGEN+HERWIG \\[1mm]     
$W\to \ell\nu$ + 3 partons  &   121   &  8.3         &  1,000k & ALPGEN+HERWIG \\[1mm]  
$W\to \ell\nu$ + 4 partons  &  30.3   &  8.3         &    250k & ALPGEN+HERWIG \\[1mm]  
$W\to \ell\nu$ + 5 partons  &   8.3   &  8.4         &     70k & ALPGEN+HERWIG \\[1mm]  
\hline
$W\to \ell\nu+b\bar{b}$ + 0 parton   & 54.7  &  8.7  &    475k & ALPGEN+HERWIG \\[1mm]     
$W\to \ell\nu+b\bar{b}$ + 1 partons  & 40.4  &  5.1  &    205k & ALPGEN+HERWIG \\[1mm]     
$W\to \ell\nu+b\bar{b}$ + 2 partons  & 20.0  &  8.8  &    175k & ALPGEN+HERWIG \\[1mm]     
$W\to \ell\nu+b\bar{b}$ + 3 partons  & 7.6   &  9.2  &     70k & ALPGEN+HERWIG \\[1mm] \hline
\hline
$W\to \ell\nu+c$ + 0 parton   & 517.6 &  1.7   &  860k & ALPGEN+HERWIG \\[1mm]     
$W\to \ell\nu+c$ + 1 partons  & 192.1 &  1.7   &  318k & ALPGEN+HERWIG \\[1mm]     
$W\to \ell\nu+c$ + 2 partons  & 51.0  &  1.7 &     85k& ALPGEN+HERWIG \\[1mm]     
$W\to \ell\nu+c$ + 3 partons  & 11.9  &  1.7    &  20k & ALPGEN+HERWIG \\[1mm] 
$W\to \ell\nu+c$ + 4 partons  & 2.8   &  1.8   &    5k & ALPGEN+HERWIG \\[1mm] 
\hline\hline
\end{tabular}
\caption{Background simulated samples. Cross-sections include $k$-factor. 
All NLO simulated samples have been simulated with \pileup\ corresponding to 
50~ns bunch trains. }
\label{TABLE-MCSAMPLES2}
\end{center}
\end{table}

\begin{table}[phtdp]
\begin{center}
\begin{tabular}{lrrrr}
\hline
 Description         & $\sigma$ [pb]  & $L_{int}$ [$fb^{-1}$] &  $N_{MC}$& Generator+Shower \\[1mm]
\hline \hline
$WW\to l\nu l\nu$ + 0 parton   & 2.0950    &    95     &  200k & ALPGEN+Herwig \\[1mm]     
$WW\to l\nu l\nu$ + 1 partons  & 0.9962    &    100    &  100k & ALPGEN+Herwig \\[1mm]     
$WW\to l\nu l\nu$ + 2 partons  & 0.4547    &     130    &  60k  & ALPGEN+Herwig \\[1mm]     
$WW\to l\nu l\nu$ + 3 partons  & 0.1758    &     230    &  40k  & ALPGEN+Herwig \\[1mm]  
\hline
$WZ\to \ell\nu \ell\ell$ + 0 parton   & 0.6718  &  89          &  60k & ALPGEN+Herwig \\[1mm]     
$WZ\to \ell\nu \ell\ell$ + 1 partons  & 0.4138  &  97          &  40k & ALPGEN+Herwig \\[1mm]     
$WZ\to \ell\nu \ell\ell$ + 2 partons  & 0.2249  &  89          &  20k & ALPGEN+Herwig \\[1mm]     
$WZ\to \ell\nu \ell\ell$ + 3 partons  & 0.0950  &  210          &  20k & ALPGEN+Herwig \\[1mm]  
\hline
$ZZ\to inclusive + \ell\ell$ + 0 parton   & 0.5086  &  79        &  40k & ALPGEN+Herwig \\[1mm]     
$ZZ\to inclusive + \ell\ell$ + 1 partons  & 0.2342  &  85         &  20k & ALPGEN+Herwig \\[1mm]     
$ZZ\to inclusive + \ell\ell$ + 2 partons  & 0.0886  &  230         &  20k & ALPGEN+Herwig \\[1mm]     
$ZZ\to inclusive + \ell\ell$ + 3 partons  & 0.0314  &  320         &  10k & ALPGEN+Herwig \\[1mm]  
\hline\hline
\end{tabular}
\caption{Background simulated samples. Cross-sections include $K$-factor. 
All NLO simulated samples have been simulated with a pile-up corresponding to a 
50~ns bunch trains (tag {\it r2920}). }
\label{TABLE-MCSAMPLESDiBoson}
\end{center}
\end{table}
\section{Object definition}
\label{SECTION-BPRIME-OBJECTS}
As in the \Wt\ analysis, the same basic objects types are considered: electrons, muons, jets, and missing transverse energy. These objects are constructed in the same manner as described in the main text, with some refinements that will be discussed below.

The electron definition remains mostly the same with a few exceptions. A new electron identification criterion is used, called ``tightPP'' (tight plus plus). This is the result of re-optimizing the same tight algorithms using more data and a better understanding of the ATLAS triggering systems, giving an overall increase in detection efficiency. An additional step has also been added to the jet-electron overlap removal algorithm. After applying the old jet-electron cut of removing a single jet if there exists one within $dR < 0.2$ of an electron, electrons within $dR < 0.4$ of any jet are rejected. This makes the electron signal cleaner by removing electrons that may be contaminated by nearby jets.

The muon definition remains the the same with optimizations to the quality definitions using new performance data.

The jet definition adds a cut on the jet vertex fraction (JVF). This variable corresponds to how certain we are that a jet originated from the primary vertex. As jets are sensitive to \pileup, this cut reduces the impact of \pileup\ on the analysis. While \pileup\ was not a problem in the \Wt\ analysis, the data added when considering the full 2011 dataset contains many runs with much higher instantaneous luminosity, which increases the impact of the \pileup\ systematic uncertainty. While implementing this cut we also add a scale factor to renormalize the simulated samples. These scale factors are calculated using a tag and probe method choosing a selection which results in a high likelihood of having a high $p_T$ jet from the primary interaction. The difference between the predicted efficiency and the observed efficiency in this region are parametrized as a scale factor as a function of jet $p_T$. This scale factor also comes with a corresponding additional systematic uncertainty, described in Section~\ref{SECTION-BPRIME-MEASUREMENT}. 

The \MET\ definition is also updated with the new data, taking into account the changes in the identification of the electrons, muons, and jets. 

\section{Event selection}
\label{SECTION-BPRIME-SELECTION}
This analysis uses $4.7\ fb^{-1}$ of data at $\sqrt{s}= 7\ TeV$ collected with the ATLAS detector. The data are filtered to select only events during which all detectors were functioning normally with stable beam from the LHC. Like the \Wt\ analysis, events are selected from dielectron ($ee$), dimuon ($\mu\mu$), and electron-muon ($e\mu$) channels, and then eventually combined into one channel for the final analysis. 

The same general event quality filtering is applied to the events as in the \Wt\ analysis, but several of the details have been updated in the full 2011 dataset. The cut due to malfunction in the LAr detectors during data taking is no longer explicitly made in the selection cuts, instead being accounted for in the generation of the simulated events. 

The trigger selection and matching has been updated to account for the changing triggering conditions while running, and also to add trigger selection and matching criteria for the muons. The triggers for various periods are given in Table~\ref{TABLE-BPRIME-TRIGGER}.

There is also an additional selection cut  of $M_{\ell\ell}>15~\gev$ added to the analysis. This cut has little impact on the selected events, but is required to allow an improvement in the \multijet\ estimation technique discussed in Section~\ref{SECTION-BPRIME-BACKGROUND}.

\begin{table}[htdp]
\begin{center}
\begin{tabular}{l|l}
\hline
Electrons &\\
\hline
Before period K & EF\_e20\_medium \\
Period K & EF\_e22\_medium \\
After period K & EF\_e22VHF\_medium1 OR EF\_e45\_medium1\\
\hline\hline
Muons &\\
\hline
Before period J & EF\_mu18 \\
Period J and later & EF\_mu18\_medium\\
\hline
\end{tabular}
\caption{The triggers for the electrons and muons for each data-taking period.}
\label{TABLE-BPRIME-TRIGGER}
\end{center}
\end{table}


\section{Background estimation}
\label{SECTION-BPRIME-BACKGROUND}
In this analysis the backgrounds were simulated using the same software as the \Wtchan\ analysis, with updated simulations of the ATLAS running conditions. The \ttbar, \Wt, and diboson backgrounds remain estimated using \MC\ techniques, while the \multijet, $Z \to \ell\bar{\ell}$, and \Ztt\ backgrounds use data-driven estimates to determine the normalization and simulated events to estimate the distribution shapes. The methodology used for the $Z \to \ell\bar{\ell}$ and \Ztt\ backgrounds is identical to that used for the \Wtchan\ analysis, but with an updated input dataset using the full $4.7 fb^{-1}$ luminosity. The \multijet\ estimation procedure is almost identical, but is improved by adding an additional requirement of $M_{\ell\ell}>15~\gev$ to minimize contamination from  $J/\Psi$ and $Y$.

After selection in the 1-jet bin, 2190 events are expected and 2259 are observed, a good agreement between data and simulation within two $\sigma$ of data statistical uncertainty. This agreement also extends to each of the $ee$, and $\mu\mu$ subchannels, as shown in Table~\ref{TABLE-SELECTION-1JET}. The $\mumu$ channel has some disagreement, but it is consistent when data statistical uncertainties and $t\bar{t}$ theoretical modeling systematic uncertainties are considered (the generator, parton shower, and normalization uncertainties). Agreement in the kinematics of the event is also good, as shown in Figs.~\ref{FIGURE-BPRIME-KINEMATICS1} and~\ref{FIGURE-BPRIME-KINEMATICS2}.
 

\begin{table}[htdp]
\begin{center}
   \begin{tabular}{lrrrr}
    \hline
    Process & $ee$ & $\mu\mu$ & $e\mu$ & all combined \\[1mm]
    \hline 

\hline
    $\bstar_{400\ GeV}$ &     187.1 $\pm$ 3.6 &      394.5 $\pm$ 5.5 &      663.8 $\pm$ 6.9 &     1245.5 $\pm$ 9.6\\
    $\bstar_{600\ GeV}$ &      34.4 $\pm$ 0.6 &       70.3 $\pm$ 0.9 &      105.9 $\pm$ 1.0 &      210.7 $\pm$ 1.4\\
    $\bstar_{800\ GeV}$ &       6.9 $\pm$ 0.1 &       13.6 $\pm$ 0.2 &       20.1 $\pm$ 0.2 &       40.6 $\pm$ 0.3\\
    $\bstar_{1000\ GeV}$ &       1.5 $\pm$ 0.0 &        3.0 $\pm$ 0.0 &        4.4 $\pm$ 0.0 &        8.9 $\pm$ 0.1\\
    $\bstar_{1200\ GeV}$ &       0.4 $\pm$ 0.0 &        0.7 $\pm$ 0.0 &        1.1 $\pm$ 0.0 &        2.1 $\pm$ 0.0\\
\hline
    $Wt$         &      42.8 $\pm$ 1.8 &       97.6 $\pm$ 2.9 &      152.7 $\pm$ 3.5 &      293.2 $\pm$ 4.8\\
    \TTB\        &     196.5 $\pm$ 2.3 &      470.2 $\pm$ 3.6 &      713.0 $\pm$ 4.4 &     1379.7 $\pm$ 6.1\\
    Diboson  &      31.6 $\pm$ 1.2 &       96.6 $\pm$ 2.2 &      126.3 $\pm$ 2.5 &      254.6 $\pm$ 3.5\\
    \Zee\    &      41.1 $\pm$ 4.1 &                negl. &                negl. &       41.1 $\pm$ 4.1\\
    \Zmm\    &               negl. &      118.0 $\pm$11.8 &                negl. &      118.0 $\pm$11.8\\
    \Ztt\    &       1.5 $\pm$ 0.7 &        3.7 $\pm$ 0.9 &        7.8 $\pm$ 1.3 &       14.2 $\pm$ 1.8\\
    Fake lepton   &      78.0 $\pm$78.0 &        8.6 $\pm$ 8.6 &        3.2 $\pm$ 3.2 &       89.8 $\pm$89.8\\
\hline
    Total Bkg. Expected &     391.5 $\pm$78.2 &      794.9 $\pm$13.3 &     1003.0 $\pm$10.6 &     2190.5 $\pm$91.1\\
    Total Observed &               347.0 $\pm$18.6 &                805.0 $\pm$28.4&               1107.0 $\pm$33.3 &               2259.0$\pm$47.5\\

  \hline\hline
   \end{tabular}
 \caption{Observed and predicted event yields in the 1-jet bin after the preselection with an integrated luminosity of \LUMI. Fake dilepton and \Zjets\ background event yields are estimated from the data-driven techniques applied to the 1-jet bin. The errors shown include statistical error only (top pair, signal, dibosons) or statistical + systematic uncertainties (Drell-Yan, fakes).}
\label{TABLE-SELECTION-1JET}
\end{center}
\end{table}

\FIVEFIGLEG{paper_ll1j_MC11c_v11_LeadingLeptonPt_flat}{paper_ll1j_MC11c_v11_LeadingLeptonEta_flat}{paper_ll1j_MC11c_v11_SubLeadingLeptonPt_flat}{paper_ll1j_MC11c_v11_SubLeadingLeptonEta_flat}{legend}{Kinematic distributions of the signal region comparing data and background. (a) Leading lepton $\pT$, (b) Leading lepton $\eta$, (c) Sub leading lepton $\pT$ and (d) Sub leading lepton $\eta$ .}{FIGURE-BPRIME-KINEMATICS1}

\FIVEFIGLEG{paper_ll1j_MC11c_v11_Jet1Pt_flat}{paper_ll1j_MC11c_v11_Jet1Eta_flat}{paper_ll1j_MC11c_v11_DeltaPhiLep1Lep2_flat}{paper_ll1j_MC11c_v11_DeltaRLep1Lep2_flat}{legend}{Kinematic distributions of the signal region comparing data and background. (a) Leading jet \pT, (b) Leading jet $\eta$, (c) $\Delta\phi$ between the two leptons and (d) $\Delta$R between the two leptons.}{FIGURE-BPRIME-KINEMATICS2}

\section{Discriminant variable selection}
\label{SECTION-BPRIME-DISCRIMINANT}
After selection a discrimination template is chosen to analyze. For the \Wtchan\ analysis the template was the BDT distribution histogram, but this analysis does not use MVA techniques. This analysis is intended to be quicker and more straightforward than the \Wtchan\ analysis and adding a MVA technique requires a lot of cross-checks. It also is more difficult to do a MVA analysis when there are multiple mass points for the signal. Instead of training on a single signal sample, either a different methodology has to be developed to train for each mass point, or only one mass point is trained on, decreasing overall sensitivity.

The choice of variable is critical to maximizing sensitivity, as its bins will be the only information the statistical tools will have as input. Consequently, we want to choose a variable with good signal/background separation. For the \bstar\ signal, the most obvious feature that stands out is the high mass of the resonance particle. Though the interacting particle itself is not directly detected by the ATLAS detector, this high mass is seen indirectly as a high transverse mass of the system. However, calculating the transverse mass of the system requires information of each individual particle in the system, which is not available for the neutrinos. As a result, we can only choose variables that approximate the transverse mass. Five of the most promising candidates for the discriminant are defined below, shown in order of increasing complexity:

\begin{enumerate}
\item \HT\ is defined as the scalar sum of all of the \pT\ of the jets, leptons and the \MET. This is the same variable as one of the input variables for the BDT for the \Wtchan\ analysis.
\item $M_{T}^1 = \sqrt{\HT^2-(\pT^{sys})^2}$
\item $M_{T}^2 = \sqrt{  \pT^{leptons+jet}\MET\   -  (\pT^{sys})^2 }$
\item $M_{T}^3 = \sqrt{  E_{T}^{leptons+jet}\MET\   -  (\pT^{sys})^2}$ \newline where $E_{T}^{leptons+jet} = \sqrt{ (\pT^{leptons+jet})^2+(M^{leptons+jet})^2 }$, and leptons+jet represents the system composed of both leptons and the jets.
\item $M_{T}^4 = \sqrt{ \left(\pT^{lep1}+\pT^{lep2}+\pT^{jet}+\frac{\MET}{cos(\Delta\phi(lep1,\MET))} +  \frac{\MET}{cos(\Delta\phi(lep2,\MET))}\right)^2-(\pT^{sys})^2}$
\end{enumerate}

These five variables are shown in Fig.~\ref{FIGURE-BPRIME-DISCRIMINANTS}. The sensitivity for each of these templates is evaluated using the template fitting procedure described in Section~\ref{SECTION-BPRIME-MEASUREMENT}. It is found that there is no improvement from any of the $M_T^n$ variables over \HT. Since \HT\ is straightforward and has an intuitive physical interpretation, this variable is used as the discrimination variable.
\SEXFIGLEG{paper_ll1j_MC11c_v11_HT_flat}{paper_ll1j_MC11c_v11_M_T1_flat}{paper_ll1j_MC11c_v11_M_T2_flat}{paper_ll1j_MC11c_v11_M_T3_flat}{paper_ll1j_MC11c_v11_M_T4_flat}{legend}{The variables considered to be the discrimination template for the \bstar\ search.}{FIGURE-BPRIME-DISCRIMINANTS}
\section{Measurement}
\label{SECTION-BPRIME-MEASUREMENT}
The systematics investigated in this analysis were applied with similar procedures as in the \Wtchan\ analysis. For details specific to this analysis, please see the note~\cite{BPRIMEINT}. This analysis has one additional systematic uncertainty that did not exist in the \Wtchan\ analysis. It is described below.\\

{\noindent{\bf Jet Vertex Fraction}}

The jet vertex fraction (JVF) is an estimate of the probability that a given jet originated from the primary vertex. If it did not originate from the primary vertex, it is assumed that it is a \pileup\ effect and is ignored. If the JVF cut applied to our events, an additional scale factor must be applied to match the simulated events to the observed data. This scale factor has an uncertainty associated with it, calculated by the {\sc TopJetUtils} package. These uncertainty scale factors are applied to the nominal sample, creating an alternate set of JVF systematic events.\\
\\
In this analysis a template shape fitting procedure is used to set limits on mass points and couplings. We do a binned likelihood analysis using the Bayesian Analysis Toolkit software package~\cite{Caldwell:2008fw}. This distribution is shown in both flat and log scale in Fig.~\ref{FIGURE-BPRIME-HT}. Figure~\ref{FIGURE-BPRIME-HTCOMPARE} compares the \HT\ signal distribution to the background distribution for selected \bstar\ mass points and Fig.~\ref{FIGURE-BPRIME-HTSYS} shows the effect of the JES systematic on the background compared to the observed data. The likelihood function is constructed by taking the product of the likelihood for each bin, shown in equation~\ref{eqn:lhoodtemp}.

\TRPFIGLEG{paper_ll1j_MC11c_v11_HT_flat}{paper_ll1j_MC11c_v11_HT_logy}{legendbprime1}{(a) Comparison of data and predicted background $H_T$. (b) Comparison of data and predicted background $H_T$ at log scale.}{FIGURE-BPRIME-HT}

\DBLFIGLEG{hackybstarHTratio}{legendbprime2}{Data and predicted background $H_T$ are shown. In addition, several signal-only $HT$ distributions at $M_{\bstar}$ = 300, 700, 1100 GeV are shown.}{FIGURE-BPRIME-HTCOMPARE}
\DBLFIGLEG{hackybstarHTsysjes}{legendbprime3}{Comparison of JES shifted background $H_T$ with data.}{FIGURE-BPRIME-HTSYS}

\begin{equation}
  {\cal L}(data|\sigma_{pp\to \bstar \to Wt},\theta_i) = \prod_{k=1}^{N_{bin}} \frac{\mu_k^{n_k} e^{-\mu_k}}{n_k!}\prod_{i=1}^{N_{sys}}G(\theta_i,0,1) \hspace{0.3cm}
  ,where \hspace{0.3cm} \mu_k = s_k + b_k
\label{eqn:lhoodtemp} 
\end{equation}

\noindent
Here the index $k$ loops over the bins of the \HT\ distribution, $\mu_k = s_k + b_k$ is the sum of the expected signal and background yield, $n_k$ is the number of observed events, the index $\imath$ loops over the systematics, and $G_i$ is a Gaussian model for each systematic. The prior probability for the cross-section is taken to be uniform. By integrating over the systematic nuisance parameters, the likelihood function becomes parametrized in terms of only the \bstar\ cross-section~\ref{eqn:lhoodn}.

\begin{equation}
  {\cal L}(data|\sigma_{pp\to \bstar \to Wt}) = \int  {\cal L}(\sigma_{pp\to \bstar \to Wt},\theta_1, ..., \theta_N)d\theta_1, ..., d\theta_N 
\label{eqn:lhoodn}
\end{equation}

\noindent
This likelihood function is converted to a posterior probability density using Bayes Theorem using our assumption that the prior probability of the cross-section is uniform. This posterior probability density is shown in equation~\ref{eqn:postprob}.

\begin{equation}
{\cal L}(\sigma_{pp\to \bstar \to Wt}|data) = {\cal L}(data|\sigma_{pp\to \bstar \to Wt})\pi(\sigma_{pp\to \bstar \to Wt}) 
\label{eqn:postprob}
\end{equation}

\noindent
 This posterior probability density has a maximum at the most likely cross-section given the data. However, in this analysis we do not expect to see a signal, and instead want to set exclusion limits. To do this we take the ratio of the integral of the posterior probability density from zero to $\sigma'$ to the integral of the posterior probability density from zero to infinity, and find the value of $\sigma'$ such that this ratio is equal to our exclusion criteria, in this case 0.95.

\begin{equation}
0.95 = \frac{\int_{0}^{\sigma'}
{\cal L}(data|\sigma_{pp\to \bstar\to Wt})\pi(\sigma_{pp\to \bstar\to Wt}) d(\sigma_{pp\to \bstar\to Wt})} {\int_0^\infty
{\cal L}(data|\sigma_{pp\to \bstar\to Wt})\pi(\sigma_{pp\to \bstar\to Wt}) d(\sigma_{pp\to \bstar\to Wt})}.
\end{equation}

\noindent
This gives a 95\% cross-section limit for each mass point. These cross-section limits are interpolated using the theoretical relationship between the cross-sections and the \bstar\ mass. This procedure is performed using both the observed dataset and ensembles of pseudoexperiments from the background estimates to give observed and expected limits. This procedure combines the results from both the dilepton and lepton+jets analyses. The intersection between the observed (expected) cross-section limit and the theoretical cross-section gives the observed (expected) \bstar-quark mass limit. The cross-section limit for a maximal left-handed coupling is 870 GeV observed (910 GeV expected) and the associated exclusion plot is shown in Fig.~\ref{FIGURE-BPRIME-LIMITLEFTDILEP}.

\VLARGEFIG{MC11c_v11HTcomblimit_vsMass}{$\bstar$ mass limit from the combined analysis, with an observed limit of $M_{\bstar} > 870\ GeV$ and expected limit of
$M_{\bstar}> 910\ GeV$.}{FIGURE-BPRIME-LIMITLEFTDILEP}

The cross-section limit is also calculated for the case where \bstar\ has only a maximal right-handed coupling and when it couples maximally both right- and left-handed. Here the cross-section limit in the right-handed case is 920 GeV observed (950 GeV expected). For the case where it has both maximal left and right-handed couplings, the cross-section limit is 1030 GeV observed (1030 GeV expected).

We can also make our limits more general by allowing the $b\bstar$$\to g$ ($k^b_{L/R}$) and $\bstar$$\to Wt$ couplings ($g_{L/R}$) to vary independently. Here we investigate three cases: the case where we assume only left-handed couplings, the case where we assume only right-handed couplings, and the case where we assume equal right- and left-handed couplings. The two dimensional limits for each of these cases are given in Figs.~\ref{FIGURE-BPRIME-LIMIT1},~\ref{FIGURE-BPRIME-LIMIT2}, and~\ref{FIGURE-BPRIME-LIMIT3}.
\VLARGEFIG{MC11c_v11HTcombobv_limit_vsMass3D}{The two dimensional coupling and mass limits for left-handed coupling \bstar.}{FIGURE-BPRIME-LIMIT1}
\VLARGEFIG{MC11c_v11HTcombobv_limit_vsMass3DRight}{The two dimensional coupling and mass limits for right-handed coupling \bstar.}{FIGURE-BPRIME-LIMIT2}
\VLARGEFIG{MC11c_v11HTcombobv_limit_vsMass3DVLQ}{The two dimensional coupling and mass limits for a combined left and right-handed coupling \bstar.}{FIGURE-BPRIME-LIMIT3}

%\newpage
%\appendix
%\part*{Appendices}
%\addcontentsline{toc}{part}{Appendices}
%\chapter{Data/MC Agreement in Control Regions}
\label{APPENDIX-CONTROLREGIONS}
This appendix shows the BDT variables in the background-enhanced 2-jet and 3-jet regions. The 2-jet and 3-jet regions clearly show how dominant of a background \ttbar\ is for this analysis. Due to the strong \ttbar\ contribution we are able to use these regions to constrain the \ttbar\ normalization, which would otherwise be a dominating uncertainty. Selected variables are also shown in the three dilepton channels: $ee$, $e\mu$, and $\mu\mu$. The dilepton subchannels show that the good data-simulation agreement does not break down when these subchannels are examined independently. 

\section{2-jet events}
\label{APPENDIX-CONTROLREGIONS-2J}
\SEXFIGLEG{paper_ll2j_LP2fb_v4_pT_sys_flat}{paper_ll2j_LP2fb_v4_pT_sys_sig_flat}{paper_ll2j_LP2fb_v4_AllJetsLepton_Centrality_flat}{paper_ll2j_LP2fb_v4_ThrustEta_flat}{paper_ll2j_LP2fb_v4_SystemLep1Lep2_eta_flat}{legend}{The top five variables in the BDT ranked by separation power, comparing the signal and background estimate to the data in the 2-jet bin.}{FIGURE-CONTROL-2JVARIABLES1}

\SEXFIGLEG{paper_ll2j_LP2fb_v4_eta_sys_lepsJet1_flat}{paper_ll2j_LP2fb_v4_LeadingLeptonEta_flat}{paper_ll2j_LP2fb_v4_SystemLep1Lep2_E_flat}{paper_ll2j_LP2fb_v4_HT_AllJets_flat}{paper_ll2j_LP2fb_v4_pT_sys_lepsJet1_flat}{legend}{The 6th-10th top variables in the BDT ranked by separation power, comparing the signal and background estimate to the data in the 2-jet bin.}{FIGURE-CONTROL-2JVARIABLES2}

\SEXFIGLEG{paper_ll2j_LP2fb_v4_Thrust_flat}{paper_ll2j_LP2fb_v4_InvariantMass_Lep2Jet1_flat}{paper_ll2j_LP2fb_v4_SystemLep1Jet1_eta_flat}{paper_ll2j_LP2fb_v4_SubLeadingLeptonEta_flat}{paper_ll2j_LP2fb_v4_Jet1Eta_flat}{legend}{The 11th-15th top variables in the BDT ranked by separation power, comparing the signal and background estimate to the data in the 2-jet bin.}{FIGURE-CONTROL-2JVARIABLES3}

\SEXFIGLEG{paper_ll2j_LP2fb_v4_DeltaMinPhiLeptonLeadingJet_flat}{paper_ll2j_LP2fb_v4_InvariantMass_Lep1Jet1_flat}{paper_ll2j_LP2fb_v4_DeltaPhi_SLep1Jet1_Lep2_flat}{paper_ll2j_LP2fb_v4_MET_flat}{paper_ll2j_LP2fb_v4_DeltaEtaLeadingLeptonLeadingJet_flat}{legend}{The 16th-20th top variables in the BDT ranked by separation power, comparing the signal and background estimate to the data in the 2-jet bin.}{FIGURE-CONTROL-2JVARIABLES4}

\TRPFIGLEG{paper_ll2j_LP2fb_v4_DeltaRSubLeadingLeptonLeadingJet_flat}{paper_ll2j_LP2fb_v4_InvariantMass_MaxLepJet1_flat}{legend}{The 21st and 22nd top variables in the BDT ranked by separation power, comparing the signal and background estimate to the data in the 2-jet bin.}{FIGURE-CONTROL-2JVARIABLES5}

\newpage

\section{3-jet inclusive events}
\label{APPENDIX-CONTROLREGIONS-3J}
\SEXFIGLEG{paper_ll3jinc_LP2fb_v4_pT_sys_flat}{paper_ll3jinc_LP2fb_v4_pT_sys_sig_flat}{paper_ll3jinc_LP2fb_v4_AllJetsLepton_Centrality_flat}{paper_ll3jinc_LP2fb_v4_ThrustEta_flat}{paper_ll3jinc_LP2fb_v4_SystemLep1Lep2_eta_flat}{legend}{The top five variables in the BDT ranked by separation power, comparing the signal and background estimate to the data in the 3-jet inclusive bin.}{FIGURE-CONTROL-3JVARIABLES1}

\SEXFIGLEG{paper_ll3jinc_LP2fb_v4_eta_sys_lepsJet1_flat}{paper_ll3jinc_LP2fb_v4_LeadingLeptonEta_flat}{paper_ll3jinc_LP2fb_v4_SystemLep1Lep2_E_flat}{paper_ll3jinc_LP2fb_v4_HT_AllJets_flat}{paper_ll3jinc_LP2fb_v4_pT_sys_lepsJet1_flat}{legend}{The 6th-10th top variables in the BDT ranked by separation power, comparing the signal and background estimate to the data in the 3-jet inclusive bin.}{FIGURE-CONTROL-3JVARIABLES2}

\SEXFIGLEG{paper_ll3jinc_LP2fb_v4_Thrust_flat}{paper_ll3jinc_LP2fb_v4_InvariantMass_Lep2Jet1_flat}{paper_ll3jinc_LP2fb_v4_SystemLep1Jet1_eta_flat}{paper_ll3jinc_LP2fb_v4_SubLeadingLeptonEta_flat}{paper_ll3jinc_LP2fb_v4_Jet1Eta_flat}{legend}{The 11th-15th top variables in the BDT ranked by separation power, comparing the signal and background estimate to the data in the 3-jet inclusive bin.}{FIGURE-CONTROL-3JVARIABLES3}

\SEXFIGLEG{paper_ll3jinc_LP2fb_v4_DeltaMinPhiLeptonLeadingJet_flat}{paper_ll3jinc_LP2fb_v4_InvariantMass_Lep1Jet1_flat}{paper_ll3jinc_LP2fb_v4_DeltaPhi_SLep1Jet1_Lep2_flat}{paper_ll3jinc_LP2fb_v4_MET_flat}{paper_ll3jinc_LP2fb_v4_DeltaEtaLeadingLeptonLeadingJet_flat}{legend}{The 16th-20th top variables in the BDT ranked by separation power, comparing the signal and background estimate to the data in the 3-jet inclusive bin.}{FIGURE-CONTROL-3JVARIABLES4}

\TRPFIGLEG{paper_ll3jinc_LP2fb_v4_DeltaRSubLeadingLeptonLeadingJet_flat}{paper_ll3jinc_LP2fb_v4_InvariantMass_MaxLepJet1_flat}{legend}{The 21st and 22nd top variables in the BDT ranked by separation power, comparing the signal and background estimate to the data in the 3-jet inclusive bin.}{FIGURE-CONTROL-3JVARIABLES5}
\newpage
\section {Dilepton subchannels}
This section contains selected variables of the different dilepton final states.  This illustrates that our backgrounds are well modeled for each of the final states individually.

\SEXFIGLEG{paper_ee1+j_LP2fb_v4_NJets_flat}{paper_ee1+j_LP2fb_v4_Jet1Pt_flat}{paper_ee1+j_LP2fb_v4_HT_AllJets_flat}{paper_ee1+j_LP2fb_v4_MET_flat}{paper_ee1+j_LP2fb_v4_LeadingLeptonPt_flat}{legend}{Distributions of variables comparing the signal and background estimate to the data  in the $ee$ channel. (a) Jet multiplicity, (b) Leading jet \pT, (c)$H_T(jet)$, (d) \MET, (e) Leading lepton \pT}{FIGURE-PRESEL-EE}
\SEXFIGLEG{paper_em1+j_LP2fb_v4_NJets_flat}{paper_em1+j_LP2fb_v4_Jet1Pt_flat}{paper_em1+j_LP2fb_v4_HT_AllJets_flat}{paper_em1+j_LP2fb_v4_MET_flat}{paper_em1+j_LP2fb_v4_LeadingLeptonPt_flat}{legend}{Distributions of variables comparing the signal and background estimate to the data  in the $e\mu$ channel. (a) Jet multiplicity, (b) Leading jet \pT, (c)$H_T(jet)$, (d) \MET, (e) Leading lepton \pT}{FIGURE-PRESEL-EM}
\SEXFIGLEG{paper_mm1+j_LP2fb_v4_NJets_flat}{paper_mm1+j_LP2fb_v4_Jet1Pt_flat}{paper_mm1+j_LP2fb_v4_HT_AllJets_flat}{paper_mm1+j_LP2fb_v4_MET_flat}{paper_mm1+j_LP2fb_v4_LeadingLeptonPt_flat}{legend}{Distributions of variables comparing the signal and background estimate to the data  in the $\mu\mu$ channel. (a) Jet multiplicity, (b) Leading jet \pT, (c)$H_T(jet)$, (d) \MET, (e) Leading lepton \pT}{FIGURE-PRESEL-MM}

%\chapter{\bstar\ search}
\label{SECTION-BPRIME}

This appendix will describe another analysis I worked on. In this analysis I implemented the object definitions, the event selection, and most of the systematic uncertainties. I studied the potential templates we considered using and attempted to reconstruct the neutrinos using invariant mass constraints, although this is not effective enough to make it into the paper. This analysis has been accepted for publication in Physics Letters B, and will be published in the near future (preprint~\cite{BPRIMEPREPRINT}). It is a search for a hypothetical \bstar\ excited state using $4.7\ fb^{-1}$ of integrated luminosity. This search uses ATLAS data in the same final state as the \Wtchan\ analysis, hence the object definitions and event selection criteria will be similar to the \Wtchan\ analysis. In addition, this appendix will give an overview of the analysis with the focus being the significant differences between the two. As a result, some of the details in common with the \Wtchan\ analysis will be glossed over. For a full description of this search, please consult the ATLAS note for this analysis~\cite{BPRIMEINT}.

\section{Introduction to \bstar}

This analysis is motivated in part by the fine-tuning problem, which is illustrated by examining the Standard Model Higgs mass to a one loop correction~\cite{PDG}
 
\begin{equation}
m_{H}^2 = m_{H_0}^2 + \frac{kg^2\Lambda^2}{16\pi^2}.
\end{equation}

\noindent

where $m_{H}$ is the observed Higgs mass, $m_{H_0}$ is an unmeasured fundamental parameter, $g$ is the electroweak coupling, $k$ is a constant expected to be $\mathcal{O}(1)$, and $\Lambda^2$ is tge energy scale of new physics. If $\Lambda$ is large, such as the Planck scale, then the $m_{H_0}$ parameter must be carefully balanced with the second term to cancel it out to give the observed Higgs mass. This is referred to as the fine-tuning problem in high energy physics. This amount of fine-tuning seems unnatural, thus it is suspected that there is other physics at work here. Theorists have made significant efforts to address this problem with models that modify the Standard Model to avoid the fine-tuning. Supersymmetry models describing massive supersymmetric partners~\cite{PDG} for every particle currently in the Standard Model are an example of such efforts.

Instead of a new family of massive particles, smaller additions to the Standard Model are often considered~\cite{Nutter}. Because the largest corrections to the Higgs mass arise from the top quark in loops such as that shown in Fig.~\ref{FIGURE-HIGGSLOOP}, an excited state of the top quark can cancel out those corrections. In addition, if an excited top quark is added, an associated excited bottom quark should also exist. We may expect that the mass hierarchy of these excited states would mirror the hierarchy we see in the Standard Model, hence in this analysis we search for a single theoretical excited state of the bottom quark that will be referred to as \bstar. 

\LARGEFIG{HiggsLoop}{A correction to the Higgs mass from the top quark.}{FIGURE-HIGGSLOOP}

The experimental constraints on this \bstar\ state require it to be much more massive than the Standard Model particles. Due to this high mass some of the \bstar-state's most common decays lead to high mass final states. In general, the most common decay modes are expected to be $\bstar \to Zb$, $\bstar \to bg$, $\bstar \to bH$, and $\bstar \to Wt$. This analysis searches for the decay mode $\bstar \to Wt$, illustrated in Fig.~\ref{FIGURE-BPRIME-FEYNMAN}. This decay mode varies in branching ratio from about 20\% at low mass (200 GeV) to approximately 40\% at high \bstar\ masses (400 GeV). The theoretical cross-section for $p\bar{p} \to \bstar \to Wt$ production in the model~\cite{Nutter} at the LHC at 7 TeV are shown in Table~\ref{BprimeCrossSection}.

\begin{table}[htdp]
\begin{center}
\begin{tabular}{r r@{.}l|r r@{.}l}  \hline \hline
mass point [$\GeV$] & \multicolumn{2}{c}{cross-section [pb]} & mass point [$\GeV$] & \multicolumn{2}{c}{cross-section [pb]}\\
\hline
300 & 181&2& 900 & 0&804  \\
400 & 69&21& 1000& 0&394  \\
500 & 24&45& 1100& 0&201  \\
600 & 9&366& 1200& 0&106  \\
700 & 3&884& 1300& 0&057 \\
800 & 1&719& 1400& 0&031 \\
\hline\hline
\end{tabular}
\caption{The total cross-section of $\bstar \rightarrow Wt$ in a mass range of 300 GeV to 1400 GeV.}
\label{BprimeCrossSection}
\end{center}
\end{table}

\FIG{bprime}{A Feynman diagram illustrated the \bstar\ decay investigated in this analysis.}{FIGURE-BPRIME-FEYNMAN}

This analysis is constructed to be sensitive to generic resonances in the \Wt\ final state and observed deviations from the Standard Model may also be caused by other resonances. In addition, coupling limits are calculated for three potential \bstar\ models: a \bstar-state with only left-handed couplings, a \bstar-state with only right-handed couplings, and a vector \bstar-state with both right and left-handed couplings with equal magnitude. These limits are calculated on a two-dimensional plane along with the mass of the \bstar-state. An example of this plane can be seen in Fig.~\ref{FIGURE-BPRIME-LIMIT3} in Section~\ref{SECTION-BPRIME-MEASUREMENT}.

Like the \Wtchan\ analysis, this analysis looks at the dilepton final state. This analysis uses the full 2011 dataset with updated simulation and systematic implementations. Another analysis was performed by a second group looking at the leptons+jets final state~\cite{BSTAR-LEPJETS}. These two analyses then collaborated to produce a unified result. The methods used to combine these two analysis will be discussed in Section~\ref{SECTION-BPRIME-MEASUREMENT}.

\section{Simulation}
\label{SECTION-BPRIME-SIMULATION}
Because the final state in this analysis is the same as the final state in the \Wtchan\ dilepton analysis, the backgrounds for these analyses are identical, except that the \Wtchan\ is a Standard Model background to the \bstar\ process. The signal in this analysis is simulated using Madgraph5~\cite{MADGRAPH} for the generation and Pythia~\cite{PYTHIA} for the hadronization. In total 12 simulated samples are generated representing \bstar\ with masses from 300 GeV to 1400 GeV in 100 GeV increments. The cross-section of \bstar\ production is dependent on the mass point, and these cross-sections are given in Table~\ref{BprimeCrossSection}. In addition, dedicated simulation samples are generated to study the impact of the uncertainty in the initial and final state radiation modeling.
The backgrounds are modeled using the same general scheme as the \Wt\ analysis, but updated to match the full 2011 ATLAS recommendations, described in the note~\cite{BPRIMEINT}. The full list of simulated samples is shown in Tables~\ref{TABLE-MCSAMPLES1},~\ref{TABLE-MCSAMPLES2}, and~\ref{TABLE-MCSAMPLESDiBoson}.


\begin{table}[htdp]
\begin{center}
\begin{tabular}{lrrrr}
\hline
Description         & $\sigma$ [pb]  & $L_{int}$ [$fb^{-1}$] &  $N_{MC}$& Generator+Shower \\[1mm]
\hline \hline
$\bstar\to Wt$, $M_{\bstar}=300 \ GeV$, & 61.6 & 3.2 & 200k &MadGraph+Pythia \\[1mm]
$\bstar\to Wt$, $M_{\bstar}=400 \ GeV$, & 23.5 & 8.5 & 200k &MadGraph+Pythia \\[1mm]
$\bstar\to Wt$, $M_{\bstar}=500 \ GeV$, & 8.31 & 24 & 200k &MadGraph+Pythia \\[1mm]
$\bstar\to Wt$, $M_{\bstar}=600 \ GeV$, & 3.18 & 63 & 200k &MadGraph+Pythia \\[1mm]
$\bstar\to Wt$, $M_{\bstar}=700 \ GeV$, & 1.32 & 150 & 200k &MadGraph+Pythia \\[1mm]
$\bstar\to Wt$, $M_{\bstar}=800 \ GeV$, & 0.58 & 350 & 200k &MadGraph+Pythia \\[1mm]
$\bstar\to Wt$, $M_{\bstar}=900 \ GeV$, & 0.27 & 740 & 200k &MadGraph+Pythia \\[1mm]
$\bstar\to Wt$, $M_{\bstar}=1000\ GeV$, & 0.13 & 1500 & 200k &MadGraph+Pythia \\[1mm]
$\bstar\to Wt$, $M_{\bstar}=1100\ GeV$, & 0.07 & 2900 & 200k &MadGraph+Pythia \\[1mm]
$\bstar\to Wt$, $M_{\bstar}=1200\ GeV$, & 0.04 & 5000 & 200k &MadGraph+Pythia \\[1mm]
$\bstar\to Wt$, $M_{\bstar}=1300\ GeV$, & 0.02 & 10000 & 200k &MadGraph+Pythia \\[1mm]
$\bstar\to Wt$, $M_{\bstar}=1400\ GeV$, & 0.01 & 20000 & 200k &MadGraph+Pythia \\[1mm]
\hline
\hline\hline
\end{tabular}
\caption{\bstar\ simulated samples for the analysis. The cross-section column includes branching ratios. All \bstar\ simulated samples are generated with at least one leptonic \Wboson\ boson decay.
}
\label{TABLE-MCSAMPSIG1}
\end{center}
\end{table}


\begin{table}[htdp]
\begin{center}
\begin{tabular}{lrrrr}
\hline
Description         & $\sigma$ [pb]  & $L_{int}$ [$fb^{-1}$]&  $N_{MC}$& Generator+Shower \\[1mm]
\hline \hline
$\bstar\to Wt$, $M_{\bstar}=300 \ GeV$, ISRFSR- & 61.6 & 3.3 & 200k & MadGraph+Pythia \\[1mm]
$\bstar\to Wt$, $M_{\bstar}=400 \ GeV$, ISRFSR- & 23.5 & 8.5 & 200k & MadGraph+Pythia \\[1mm]
$\bstar\to Wt$, $M_{\bstar}=500 \ GeV$, ISRFSR- & 8.31 & 23 & 200k & MadGraph+Pythia \\[1mm]
$\bstar\to Wt$, $M_{\bstar}=600 \ GeV$, ISRFSR- & 3.18 & 63 & 200k & MadGraph+Pythia \\[1mm]
$\bstar\to Wt$, $M_{\bstar}=700 \ GeV$, ISRFSR- & 1.32 & 150 & 200k & MadGraph+Pythia \\[1mm]
$\bstar\to Wt$, $M_{\bstar}=800 \ GeV$, ISRFSR- & 0.58 & 340 & 200k & MadGraph+Pythia \\[1mm]
$\bstar\to Wt$, $M_{\bstar}=900 \ GeV$, ISRFSR- & 0.27 & 740 & 200k & MadGraph+Pythia \\[1mm]
$\bstar\to Wt$, $M_{\bstar}=1000\ GeV$, ISRFSR- & 0.13 &  1500 & 200k & MadGraph+Pythia \\[1mm]
$\bstar\to Wt$, $M_{\bstar}=1100\ GeV$, ISRFSR- & 0.07 & 2900 & 200k & MadGraph+Pythia \\[1mm]
$\bstar\to Wt$, $M_{\bstar}=1200\ GeV$, ISRFSR- & 0.04 & 5000 & 200k & MadGraph+Pythia \\[1mm]
$\bstar\to Wt$, $M_{\bstar}=1300\ GeV$, ISRFSR- & 0.02 & 10000 & 200k & MadGraph+Pythia \\[1mm]
$\bstar\to Wt$, $M_{\bstar}=1400\ GeV$, ISRFSR- & 0.01 & 20000 & 200k & MadGraph+Pythia \\[1mm]
\hline
$\bstar\to Wt$, $M_{\bstar}=300 \ GeV$, ISRFSR+ & 61.6 & 3.3 & 200k & MadGraph+Pythia \\[1mm]
$\bstar\to Wt$, $M_{\bstar}=400 \ GeV$, ISRFSR+ & 23.5 & 8.5 & 200k & MadGraph+Pythia \\[1mm]
$\bstar\to Wt$, $M_{\bstar}=500 \ GeV$, ISRFSR+ & 8.31 & 23 & 200k & MadGraph+Pythia \\[1mm]
$\bstar\to Wt$, $M_{\bstar}=600 \ GeV$, ISRFSR+ & 3.18 & 63 & 200k & MadGraph+Pythia \\[1mm]
$\bstar\to Wt$, $M_{\bstar}=700 \ GeV$, ISRFSR+ & 1.32 & 150 & 200k & MadGraph+Pythia \\[1mm]
$\bstar\to Wt$, $M_{\bstar}=800 \ GeV$, ISRFSR+ & 0.58 & 340 & 200k & MadGraph+Pythia \\[1mm]
$\bstar\to Wt$, $M_{\bstar}=900 \ GeV$, ISRFSR+ & 0.27 & 740 & 200k & MadGraph+Pythia \\[1mm]
$\bstar\to Wt$, $M_{\bstar}=1000\ GeV$, ISRFSR+ & 0.13 & 1500 & 200k & MadGraph+Pythia \\[1mm]
$\bstar\to Wt$, $M_{\bstar}=1100\ GeV$, ISRFSR+ & 0.07 & 2900  & 200k & MadGraph+Pythia \\[1mm]
$\bstar\to Wt$, $M_{\bstar}=1200\ GeV$, ISRFSR+ & 0.04 &  5000 & 200k & MadGraph+Pythia \\[1mm]
$\bstar\to Wt$, $M_{\bstar}=1300\ GeV$, ISRFSR+ & 0.02 &  10000 & 200k & MadGraph+Pythia \\[1mm]
$\bstar\to Wt$, $M_{\bstar}=1400\ GeV$, ISRFSR+ & 0.01 &  20000 & 200k & MadGraph+Pythia \\[1mm]
\hline
\hline\hline
\end{tabular}
\caption{\bstar\ simulated samples for the analysis. The cross-section column includes branching ratios. All \bstar\ simulated events are generated with at least one leptonic \Wboson\ boson decay.
}
\label{TABLE-MCSAMPSIG2}
\end{center}
\end{table}


\begin{table}[htdp]
\begin{center}
\begin{tabular}{lrrrr}
\hline
Description         & $\sigma$ [pb]  & $L_{int}$ [$fb^{-1}$] &  $N_{MC}$& Generator+Shower \\[1mm]
\hline \hline
\Wt\   all decays             & 15.74   & 13  & 200k         & MC@NLO+Herwig      \\[1mm]
$Wt$ Less ISRFSR              & 15.74   & 19  & 300k         & ACERMC+Pythia      \\[1mm]
$Wt$ More ISRFSR              & 15.74   & 19  & 300k         & ACERMC+Pythia      \\[1mm]
$t\bar{t}$ no fully hadronic  & 89.71   & 17  & 1,500k         & MC@NLO+Herwig      \\[1mm]
$t\bar{t}$ no fully hadronic  & 89.4    & 34  & 3,000k         & POWHEG+Herwig      \\[1mm]
$t\bar{t}$ no fully hadronic  & 89.4    & 34  & 3,000k         & POWHEG+Pythia      \\[1mm]
$t\bar{t}$ no fully hadronic Less ISRFSR  & 89.4 & 11 & 1,000k  & ACERMC+Pythia      \\[1mm]
$t\bar{t}$ no fully hadronic More ISRFSR  & 89.4 & 11 & 1,000k  & ACERMC+Pythia      \\[1mm]
\hline\hline
\end{tabular}
\caption{Top quark event simulated samples for the analysis. The cross-section column includes $k$-factors and branching ratios. All NLO simulated samples have been simulated with \pileup\ corresponding to 50~ns bunch trains.}
\label{TABLE-MCSAMPLES1}
\end{center}
\end{table}

\begin{table}[phtdp]
\begin{center}
\begin{tabular}{lrrrr}
\hline
Description         & $\sigma$ [pb]  & $L_{in}$ [$fb^{-1}$] &  $N_{MC}$& Generator+Shower \\[1mm]
\hline \hline
$Z\to \ell\ell$ + 0 parton   & 827.4  &  8.0          &  6,600k & ALPGEN+HERWIG \\[1mm]     
$Z\to \ell\ell$ + 1 partons  & 166.6  &  8.0          &  1,340k & ALPGEN+HERWIG \\[1mm]     
$Z\to \ell\ell$ + 2 partons  & 50.4   &  5.7          &    285k & ALPGEN+HERWIG \\[1mm]     
$Z\to \ell\ell$ + 3 partons  & 14.0   &  7.9          &    110k & ALPGEN+HERWIG \\[1mm]  
$Z\to \ell\ell$ + 4 partons  & 3.4    &  8.8          &     30k & ALPGEN+HERWIG \\[1mm]  
$Z\to \ell\ell$ + 5 partons  & 1.0    &  9.0          &     9k & ALPGEN+HERWIG \\[1mm]  
\hline
$W\to \ell\nu$ + 0 parton   & 8,296   &  0.4         &  3,500k & ALPGEN+HERWIG \\[1mm]     
$W\to \ell\nu$ + 1 partons  & 1,551   &  1.6         &  2,500k & ALPGEN+HERWIG \\[1mm]     
$W\to \ell\nu$ + 2 partons  &   452   &  8.3         &  3,770k & ALPGEN+HERWIG \\[1mm]     
$W\to \ell\nu$ + 3 partons  &   121   &  8.3         &  1,000k & ALPGEN+HERWIG \\[1mm]  
$W\to \ell\nu$ + 4 partons  &  30.3   &  8.3         &    250k & ALPGEN+HERWIG \\[1mm]  
$W\to \ell\nu$ + 5 partons  &   8.3   &  8.4         &     70k & ALPGEN+HERWIG \\[1mm]  
\hline
$W\to \ell\nu+b\bar{b}$ + 0 parton   & 54.7  &  8.7  &    475k & ALPGEN+HERWIG \\[1mm]     
$W\to \ell\nu+b\bar{b}$ + 1 partons  & 40.4  &  5.1  &    205k & ALPGEN+HERWIG \\[1mm]     
$W\to \ell\nu+b\bar{b}$ + 2 partons  & 20.0  &  8.8  &    175k & ALPGEN+HERWIG \\[1mm]     
$W\to \ell\nu+b\bar{b}$ + 3 partons  & 7.6   &  9.2  &     70k & ALPGEN+HERWIG \\[1mm] \hline
\hline
$W\to \ell\nu+c$ + 0 parton   & 517.6 &  1.7   &  860k & ALPGEN+HERWIG \\[1mm]     
$W\to \ell\nu+c$ + 1 partons  & 192.1 &  1.7   &  318k & ALPGEN+HERWIG \\[1mm]     
$W\to \ell\nu+c$ + 2 partons  & 51.0  &  1.7 &     85k& ALPGEN+HERWIG \\[1mm]     
$W\to \ell\nu+c$ + 3 partons  & 11.9  &  1.7    &  20k & ALPGEN+HERWIG \\[1mm] 
$W\to \ell\nu+c$ + 4 partons  & 2.8   &  1.8   &    5k & ALPGEN+HERWIG \\[1mm] 
\hline\hline
\end{tabular}
\caption{Background simulated samples. Cross-sections include $k$-factor. 
All NLO simulated samples have been simulated with \pileup\ corresponding to 
50~ns bunch trains. }
\label{TABLE-MCSAMPLES2}
\end{center}
\end{table}

\begin{table}[phtdp]
\begin{center}
\begin{tabular}{lrrrr}
\hline
 Description         & $\sigma$ [pb]  & $L_{int}$ [$fb^{-1}$] &  $N_{MC}$& Generator+Shower \\[1mm]
\hline \hline
$WW\to l\nu l\nu$ + 0 parton   & 2.0950    &    95     &  200k & ALPGEN+Herwig \\[1mm]     
$WW\to l\nu l\nu$ + 1 partons  & 0.9962    &    100    &  100k & ALPGEN+Herwig \\[1mm]     
$WW\to l\nu l\nu$ + 2 partons  & 0.4547    &     130    &  60k  & ALPGEN+Herwig \\[1mm]     
$WW\to l\nu l\nu$ + 3 partons  & 0.1758    &     230    &  40k  & ALPGEN+Herwig \\[1mm]  
\hline
$WZ\to \ell\nu \ell\ell$ + 0 parton   & 0.6718  &  89          &  60k & ALPGEN+Herwig \\[1mm]     
$WZ\to \ell\nu \ell\ell$ + 1 partons  & 0.4138  &  97          &  40k & ALPGEN+Herwig \\[1mm]     
$WZ\to \ell\nu \ell\ell$ + 2 partons  & 0.2249  &  89          &  20k & ALPGEN+Herwig \\[1mm]     
$WZ\to \ell\nu \ell\ell$ + 3 partons  & 0.0950  &  210          &  20k & ALPGEN+Herwig \\[1mm]  
\hline
$ZZ\to inclusive + \ell\ell$ + 0 parton   & 0.5086  &  79        &  40k & ALPGEN+Herwig \\[1mm]     
$ZZ\to inclusive + \ell\ell$ + 1 partons  & 0.2342  &  85         &  20k & ALPGEN+Herwig \\[1mm]     
$ZZ\to inclusive + \ell\ell$ + 2 partons  & 0.0886  &  230         &  20k & ALPGEN+Herwig \\[1mm]     
$ZZ\to inclusive + \ell\ell$ + 3 partons  & 0.0314  &  320         &  10k & ALPGEN+Herwig \\[1mm]  
\hline\hline
\end{tabular}
\caption{Background simulated samples. Cross-sections include $K$-factor. 
All NLO simulated samples have been simulated with a pile-up corresponding to a 
50~ns bunch trains (tag {\it r2920}). }
\label{TABLE-MCSAMPLESDiBoson}
\end{center}
\end{table}
\section{Object definition}
\label{SECTION-BPRIME-OBJECTS}
As in the \Wt\ analysis, the same basic objects types are considered: electrons, muons, jets, and missing transverse energy. These objects are constructed in the same manner as described in the main text, with some refinements that will be discussed below.

The electron definition remains mostly the same with a few exceptions. A new electron identification criterion is used, called ``tightPP'' (tight plus plus). This is the result of re-optimizing the same tight algorithms using more data and a better understanding of the ATLAS triggering systems, giving an overall increase in detection efficiency. An additional step has also been added to the jet-electron overlap removal algorithm. After applying the old jet-electron cut of removing a single jet if there exists one within $dR < 0.2$ of an electron, electrons within $dR < 0.4$ of any jet are rejected. This makes the electron signal cleaner by removing electrons that may be contaminated by nearby jets.

The muon definition remains the the same with optimizations to the quality definitions using new performance data.

The jet definition adds a cut on the jet vertex fraction (JVF). This variable corresponds to how certain we are that a jet originated from the primary vertex. As jets are sensitive to \pileup, this cut reduces the impact of \pileup\ on the analysis. While \pileup\ was not a problem in the \Wt\ analysis, the data added when considering the full 2011 dataset contains many runs with much higher instantaneous luminosity, which increases the impact of the \pileup\ systematic uncertainty. While implementing this cut we also add a scale factor to renormalize the simulated samples. These scale factors are calculated using a tag and probe method choosing a selection which results in a high likelihood of having a high $p_T$ jet from the primary interaction. The difference between the predicted efficiency and the observed efficiency in this region are parametrized as a scale factor as a function of jet $p_T$. This scale factor also comes with a corresponding additional systematic uncertainty, described in Section~\ref{SECTION-BPRIME-MEASUREMENT}. 

The \MET\ definition is also updated with the new data, taking into account the changes in the identification of the electrons, muons, and jets. 

\section{Event selection}
\label{SECTION-BPRIME-SELECTION}
This analysis uses $4.7\ fb^{-1}$ of data at $\sqrt{s}= 7\ TeV$ collected with the ATLAS detector. The data are filtered to select only events during which all detectors were functioning normally with stable beam from the LHC. Like the \Wt\ analysis, events are selected from dielectron ($ee$), dimuon ($\mu\mu$), and electron-muon ($e\mu$) channels, and then eventually combined into one channel for the final analysis. 

The same general event quality filtering is applied to the events as in the \Wt\ analysis, but several of the details have been updated in the full 2011 dataset. The cut due to malfunction in the LAr detectors during data taking is no longer explicitly made in the selection cuts, instead being accounted for in the generation of the simulated events. 

The trigger selection and matching has been updated to account for the changing triggering conditions while running, and also to add trigger selection and matching criteria for the muons. The triggers for various periods are given in Table~\ref{TABLE-BPRIME-TRIGGER}.

There is also an additional selection cut  of $M_{\ell\ell}>15~\gev$ added to the analysis. This cut has little impact on the selected events, but is required to allow an improvement in the \multijet\ estimation technique discussed in Section~\ref{SECTION-BPRIME-BACKGROUND}.

\begin{table}[htdp]
\begin{center}
\begin{tabular}{l|l}
\hline
Electrons &\\
\hline
Before period K & EF\_e20\_medium \\
Period K & EF\_e22\_medium \\
After period K & EF\_e22VHF\_medium1 OR EF\_e45\_medium1\\
\hline\hline
Muons &\\
\hline
Before period J & EF\_mu18 \\
Period J and later & EF\_mu18\_medium\\
\hline
\end{tabular}
\caption{The triggers for the electrons and muons for each data-taking period.}
\label{TABLE-BPRIME-TRIGGER}
\end{center}
\end{table}


\section{Background estimation}
\label{SECTION-BPRIME-BACKGROUND}
In this analysis the backgrounds were simulated using the same software as the \Wtchan\ analysis, with updated simulations of the ATLAS running conditions. The \ttbar, \Wt, and diboson backgrounds remain estimated using \MC\ techniques, while the \multijet, $Z \to \ell\bar{\ell}$, and \Ztt\ backgrounds use data-driven estimates to determine the normalization and simulated events to estimate the distribution shapes. The methodology used for the $Z \to \ell\bar{\ell}$ and \Ztt\ backgrounds is identical to that used for the \Wtchan\ analysis, but with an updated input dataset using the full $4.7 fb^{-1}$ luminosity. The \multijet\ estimation procedure is almost identical, but is improved by adding an additional requirement of $M_{\ell\ell}>15~\gev$ to minimize contamination from  $J/\Psi$ and $Y$.

After selection in the 1-jet bin, 2190 events are expected and 2259 are observed, a good agreement between data and simulation within two $\sigma$ of data statistical uncertainty. This agreement also extends to each of the $ee$, and $\mu\mu$ subchannels, as shown in Table~\ref{TABLE-SELECTION-1JET}. The $\mumu$ channel has some disagreement, but it is consistent when data statistical uncertainties and $t\bar{t}$ theoretical modeling systematic uncertainties are considered (the generator, parton shower, and normalization uncertainties). Agreement in the kinematics of the event is also good, as shown in Figs.~\ref{FIGURE-BPRIME-KINEMATICS1} and~\ref{FIGURE-BPRIME-KINEMATICS2}.
 

\begin{table}[htdp]
\begin{center}
   \begin{tabular}{lrrrr}
    \hline
    Process & $ee$ & $\mu\mu$ & $e\mu$ & all combined \\[1mm]
    \hline 

\hline
    $\bstar_{400\ GeV}$ &     187.1 $\pm$ 3.6 &      394.5 $\pm$ 5.5 &      663.8 $\pm$ 6.9 &     1245.5 $\pm$ 9.6\\
    $\bstar_{600\ GeV}$ &      34.4 $\pm$ 0.6 &       70.3 $\pm$ 0.9 &      105.9 $\pm$ 1.0 &      210.7 $\pm$ 1.4\\
    $\bstar_{800\ GeV}$ &       6.9 $\pm$ 0.1 &       13.6 $\pm$ 0.2 &       20.1 $\pm$ 0.2 &       40.6 $\pm$ 0.3\\
    $\bstar_{1000\ GeV}$ &       1.5 $\pm$ 0.0 &        3.0 $\pm$ 0.0 &        4.4 $\pm$ 0.0 &        8.9 $\pm$ 0.1\\
    $\bstar_{1200\ GeV}$ &       0.4 $\pm$ 0.0 &        0.7 $\pm$ 0.0 &        1.1 $\pm$ 0.0 &        2.1 $\pm$ 0.0\\
\hline
    $Wt$         &      42.8 $\pm$ 1.8 &       97.6 $\pm$ 2.9 &      152.7 $\pm$ 3.5 &      293.2 $\pm$ 4.8\\
    \TTB\        &     196.5 $\pm$ 2.3 &      470.2 $\pm$ 3.6 &      713.0 $\pm$ 4.4 &     1379.7 $\pm$ 6.1\\
    Diboson  &      31.6 $\pm$ 1.2 &       96.6 $\pm$ 2.2 &      126.3 $\pm$ 2.5 &      254.6 $\pm$ 3.5\\
    \Zee\    &      41.1 $\pm$ 4.1 &                negl. &                negl. &       41.1 $\pm$ 4.1\\
    \Zmm\    &               negl. &      118.0 $\pm$11.8 &                negl. &      118.0 $\pm$11.8\\
    \Ztt\    &       1.5 $\pm$ 0.7 &        3.7 $\pm$ 0.9 &        7.8 $\pm$ 1.3 &       14.2 $\pm$ 1.8\\
    Fake lepton   &      78.0 $\pm$78.0 &        8.6 $\pm$ 8.6 &        3.2 $\pm$ 3.2 &       89.8 $\pm$89.8\\
\hline
    Total Bkg. Expected &     391.5 $\pm$78.2 &      794.9 $\pm$13.3 &     1003.0 $\pm$10.6 &     2190.5 $\pm$91.1\\
    Total Observed &               347.0 $\pm$18.6 &                805.0 $\pm$28.4&               1107.0 $\pm$33.3 &               2259.0$\pm$47.5\\

  \hline\hline
   \end{tabular}
 \caption{Observed and predicted event yields in the 1-jet bin after the preselection with an integrated luminosity of \LUMI. Fake dilepton and \Zjets\ background event yields are estimated from the data-driven techniques applied to the 1-jet bin. The errors shown include statistical error only (top pair, signal, dibosons) or statistical + systematic uncertainties (Drell-Yan, fakes).}
\label{TABLE-SELECTION-1JET}
\end{center}
\end{table}

\FIVEFIGLEG{paper_ll1j_MC11c_v11_LeadingLeptonPt_flat}{paper_ll1j_MC11c_v11_LeadingLeptonEta_flat}{paper_ll1j_MC11c_v11_SubLeadingLeptonPt_flat}{paper_ll1j_MC11c_v11_SubLeadingLeptonEta_flat}{legend}{Kinematic distributions of the signal region comparing data and background. (a) Leading lepton $\pT$, (b) Leading lepton $\eta$, (c) Sub leading lepton $\pT$ and (d) Sub leading lepton $\eta$ .}{FIGURE-BPRIME-KINEMATICS1}

\FIVEFIGLEG{paper_ll1j_MC11c_v11_Jet1Pt_flat}{paper_ll1j_MC11c_v11_Jet1Eta_flat}{paper_ll1j_MC11c_v11_DeltaPhiLep1Lep2_flat}{paper_ll1j_MC11c_v11_DeltaRLep1Lep2_flat}{legend}{Kinematic distributions of the signal region comparing data and background. (a) Leading jet \pT, (b) Leading jet $\eta$, (c) $\Delta\phi$ between the two leptons and (d) $\Delta$R between the two leptons.}{FIGURE-BPRIME-KINEMATICS2}

\section{Discriminant variable selection}
\label{SECTION-BPRIME-DISCRIMINANT}
After selection a discrimination template is chosen to analyze. For the \Wtchan\ analysis the template was the BDT distribution histogram, but this analysis does not use MVA techniques. This analysis is intended to be quicker and more straightforward than the \Wtchan\ analysis and adding a MVA technique requires a lot of cross-checks. It also is more difficult to do a MVA analysis when there are multiple mass points for the signal. Instead of training on a single signal sample, either a different methodology has to be developed to train for each mass point, or only one mass point is trained on, decreasing overall sensitivity.

The choice of variable is critical to maximizing sensitivity, as its bins will be the only information the statistical tools will have as input. Consequently, we want to choose a variable with good signal/background separation. For the \bstar\ signal, the most obvious feature that stands out is the high mass of the resonance particle. Though the interacting particle itself is not directly detected by the ATLAS detector, this high mass is seen indirectly as a high transverse mass of the system. However, calculating the transverse mass of the system requires information of each individual particle in the system, which is not available for the neutrinos. As a result, we can only choose variables that approximate the transverse mass. Five of the most promising candidates for the discriminant are defined below, shown in order of increasing complexity:

\begin{enumerate}
\item \HT\ is defined as the scalar sum of all of the \pT\ of the jets, leptons and the \MET. This is the same variable as one of the input variables for the BDT for the \Wtchan\ analysis.
\item $M_{T}^1 = \sqrt{\HT^2-(\pT^{sys})^2}$
\item $M_{T}^2 = \sqrt{  \pT^{leptons+jet}\MET\   -  (\pT^{sys})^2 }$
\item $M_{T}^3 = \sqrt{  E_{T}^{leptons+jet}\MET\   -  (\pT^{sys})^2}$ \newline where $E_{T}^{leptons+jet} = \sqrt{ (\pT^{leptons+jet})^2+(M^{leptons+jet})^2 }$, and leptons+jet represents the system composed of both leptons and the jets.
\item $M_{T}^4 = \sqrt{ \left(\pT^{lep1}+\pT^{lep2}+\pT^{jet}+\frac{\MET}{cos(\Delta\phi(lep1,\MET))} +  \frac{\MET}{cos(\Delta\phi(lep2,\MET))}\right)^2-(\pT^{sys})^2}$
\end{enumerate}

These five variables are shown in Fig.~\ref{FIGURE-BPRIME-DISCRIMINANTS}. The sensitivity for each of these templates is evaluated using the template fitting procedure described in Section~\ref{SECTION-BPRIME-MEASUREMENT}. It is found that there is no improvement from any of the $M_T^n$ variables over \HT. Since \HT\ is straightforward and has an intuitive physical interpretation, this variable is used as the discrimination variable.
\SEXFIGLEG{paper_ll1j_MC11c_v11_HT_flat}{paper_ll1j_MC11c_v11_M_T1_flat}{paper_ll1j_MC11c_v11_M_T2_flat}{paper_ll1j_MC11c_v11_M_T3_flat}{paper_ll1j_MC11c_v11_M_T4_flat}{legend}{The variables considered to be the discrimination template for the \bstar\ search.}{FIGURE-BPRIME-DISCRIMINANTS}
\section{Measurement}
\label{SECTION-BPRIME-MEASUREMENT}
The systematics investigated in this analysis were applied with similar procedures as in the \Wtchan\ analysis. For details specific to this analysis, please see the note~\cite{BPRIMEINT}. This analysis has one additional systematic uncertainty that did not exist in the \Wtchan\ analysis. It is described below.\\

{\noindent{\bf Jet Vertex Fraction}}

The jet vertex fraction (JVF) is an estimate of the probability that a given jet originated from the primary vertex. If it did not originate from the primary vertex, it is assumed that it is a \pileup\ effect and is ignored. If the JVF cut applied to our events, an additional scale factor must be applied to match the simulated events to the observed data. This scale factor has an uncertainty associated with it, calculated by the {\sc TopJetUtils} package. These uncertainty scale factors are applied to the nominal sample, creating an alternate set of JVF systematic events.\\
\\
In this analysis a template shape fitting procedure is used to set limits on mass points and couplings. We do a binned likelihood analysis using the Bayesian Analysis Toolkit software package~\cite{Caldwell:2008fw}. This distribution is shown in both flat and log scale in Fig.~\ref{FIGURE-BPRIME-HT}. Figure~\ref{FIGURE-BPRIME-HTCOMPARE} compares the \HT\ signal distribution to the background distribution for selected \bstar\ mass points and Fig.~\ref{FIGURE-BPRIME-HTSYS} shows the effect of the JES systematic on the background compared to the observed data. The likelihood function is constructed by taking the product of the likelihood for each bin, shown in equation~\ref{eqn:lhoodtemp}.

\TRPFIGLEG{paper_ll1j_MC11c_v11_HT_flat}{paper_ll1j_MC11c_v11_HT_logy}{legendbprime1}{(a) Comparison of data and predicted background $H_T$. (b) Comparison of data and predicted background $H_T$ at log scale.}{FIGURE-BPRIME-HT}

\DBLFIGLEG{hackybstarHTratio}{legendbprime2}{Data and predicted background $H_T$ are shown. In addition, several signal-only $HT$ distributions at $M_{\bstar}$ = 300, 700, 1100 GeV are shown.}{FIGURE-BPRIME-HTCOMPARE}
\DBLFIGLEG{hackybstarHTsysjes}{legendbprime3}{Comparison of JES shifted background $H_T$ with data.}{FIGURE-BPRIME-HTSYS}

\begin{equation}
  {\cal L}(data|\sigma_{pp\to \bstar \to Wt},\theta_i) = \prod_{k=1}^{N_{bin}} \frac{\mu_k^{n_k} e^{-\mu_k}}{n_k!}\prod_{i=1}^{N_{sys}}G(\theta_i,0,1) \hspace{0.3cm}
  ,where \hspace{0.3cm} \mu_k = s_k + b_k
\label{eqn:lhoodtemp} 
\end{equation}

\noindent
Here the index $k$ loops over the bins of the \HT\ distribution, $\mu_k = s_k + b_k$ is the sum of the expected signal and background yield, $n_k$ is the number of observed events, the index $\imath$ loops over the systematics, and $G_i$ is a Gaussian model for each systematic. The prior probability for the cross-section is taken to be uniform. By integrating over the systematic nuisance parameters, the likelihood function becomes parametrized in terms of only the \bstar\ cross-section~\ref{eqn:lhoodn}.

\begin{equation}
  {\cal L}(data|\sigma_{pp\to \bstar \to Wt}) = \int  {\cal L}(\sigma_{pp\to \bstar \to Wt},\theta_1, ..., \theta_N)d\theta_1, ..., d\theta_N 
\label{eqn:lhoodn}
\end{equation}

\noindent
This likelihood function is converted to a posterior probability density using Bayes Theorem using our assumption that the prior probability of the cross-section is uniform. This posterior probability density is shown in equation~\ref{eqn:postprob}.

\begin{equation}
{\cal L}(\sigma_{pp\to \bstar \to Wt}|data) = {\cal L}(data|\sigma_{pp\to \bstar \to Wt})\pi(\sigma_{pp\to \bstar \to Wt}) 
\label{eqn:postprob}
\end{equation}

\noindent
 This posterior probability density has a maximum at the most likely cross-section given the data. However, in this analysis we do not expect to see a signal, and instead want to set exclusion limits. To do this we take the ratio of the integral of the posterior probability density from zero to $\sigma'$ to the integral of the posterior probability density from zero to infinity, and find the value of $\sigma'$ such that this ratio is equal to our exclusion criteria, in this case 0.95.

\begin{equation}
0.95 = \frac{\int_{0}^{\sigma'}
{\cal L}(data|\sigma_{pp\to \bstar\to Wt})\pi(\sigma_{pp\to \bstar\to Wt}) d(\sigma_{pp\to \bstar\to Wt})} {\int_0^\infty
{\cal L}(data|\sigma_{pp\to \bstar\to Wt})\pi(\sigma_{pp\to \bstar\to Wt}) d(\sigma_{pp\to \bstar\to Wt})}.
\end{equation}

\noindent
This gives a 95\% cross-section limit for each mass point. These cross-section limits are interpolated using the theoretical relationship between the cross-sections and the \bstar\ mass. This procedure is performed using both the observed dataset and ensembles of pseudoexperiments from the background estimates to give observed and expected limits. This procedure combines the results from both the dilepton and lepton+jets analyses. The intersection between the observed (expected) cross-section limit and the theoretical cross-section gives the observed (expected) \bstar-quark mass limit. The cross-section limit for a maximal left-handed coupling is 870 GeV observed (910 GeV expected) and the associated exclusion plot is shown in Fig.~\ref{FIGURE-BPRIME-LIMITLEFTDILEP}.

\VLARGEFIG{MC11c_v11HTcomblimit_vsMass}{$\bstar$ mass limit from the combined analysis, with an observed limit of $M_{\bstar} > 870\ GeV$ and expected limit of
$M_{\bstar}> 910\ GeV$.}{FIGURE-BPRIME-LIMITLEFTDILEP}

The cross-section limit is also calculated for the case where \bstar\ has only a maximal right-handed coupling and when it couples maximally both right- and left-handed. Here the cross-section limit in the right-handed case is 920 GeV observed (950 GeV expected). For the case where it has both maximal left and right-handed couplings, the cross-section limit is 1030 GeV observed (1030 GeV expected).

We can also make our limits more general by allowing the $b\bstar$$\to g$ ($k^b_{L/R}$) and $\bstar$$\to Wt$ couplings ($g_{L/R}$) to vary independently. Here we investigate three cases: the case where we assume only left-handed couplings, the case where we assume only right-handed couplings, and the case where we assume equal right- and left-handed couplings. The two dimensional limits for each of these cases are given in Figs.~\ref{FIGURE-BPRIME-LIMIT1},~\ref{FIGURE-BPRIME-LIMIT2}, and~\ref{FIGURE-BPRIME-LIMIT3}.
\VLARGEFIG{MC11c_v11HTcombobv_limit_vsMass3D}{The two dimensional coupling and mass limits for left-handed coupling \bstar.}{FIGURE-BPRIME-LIMIT1}
\VLARGEFIG{MC11c_v11HTcombobv_limit_vsMass3DRight}{The two dimensional coupling and mass limits for right-handed coupling \bstar.}{FIGURE-BPRIME-LIMIT2}
\VLARGEFIG{MC11c_v11HTcombobv_limit_vsMass3DVLQ}{The two dimensional coupling and mass limits for a combined left and right-handed coupling \bstar.}{FIGURE-BPRIME-LIMIT3}

%\clearpage
%\newpage
%% Put the body of your dissertation here. 
%% DO NOT include  the bibliography
%% If you wish to include one or more appendices, remove the "%" from the 
%% following eight (8) lines.
%\newpage
%\vspace*{\fill}
%\begin{center}
%\Huge \textbf{APPENDICIES}
%\end{center}
%\vfill
%\newpage
%\appendix
%%To start your first appendix, which will be labeled as Appendix A  
%% just type \chapter{<appendix 1 name>}
%%%%%%% A NOTE ABOUT APPENDICES %%%%%%%%%
%% Some appendices may be single spaced such as survey examples or letters.
%% Contact the Graduate School for details.
%% To single space an appendix first remove the % from 
%% the following two lines.
% \end{doublespace}
% \chapter{<appendix  name>}
%% After entering the appendix remove the % from 
%% the following line
% \begin{doublespace}
%% Any text entered now will be double spaced.
\end{doublespace}

%%Bibliography 

\bibliographystyle{atlasnote}
%\bibliographystyle{unsrt}
\bibliography{bibmain}

%% A bibliography is required. It may be made using BibTeX.
%% If it's made from scratch,
%% remove the "%" in front of \begin{thebibliography}{???}
%% replacing the ??? with the appropriate entry and 
%% remove the "%" in front of \end{thebibliography}
% \begin{thebibliography}{???}
%%  Enter the bibliography here.
% \end{thebibliography}
\end{document}
