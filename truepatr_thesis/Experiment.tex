\chapter{ATLAS and the LHC}
\label{SECTION-ATLAS}
The search for \Wprimechan\ requires a very large and extensive experimental setup. In order to set limits competitive to those currently in the literature, particles need to be collided with atleast several TeV of energy, and in order to correctly identify \Wprime\ events from the background the products of these collisions need to be carefully measured. The ATLAS (A Toroidal Lhc ApparatuS) experiment meets these criteria, it is the largest collider detector ever built and is capable of very precise measurements of the products of particle collisions. The collisions it measures are produced by the Large Hadron Collider (LHC) which is designed to collide particles with up to 14 TeV center of mass energy.

\section{The Large Hadron Collider}
\label{SECTION-ATLAS-LHC}
\subsection{The Accelerator Chain}
The LHC is only the final accelerator in a chain designed to take ions from rest, accelerate and collide them at up to 14 TeV center of mass energy. This acceleration occurs in stages, with protons being accelerated through a separate chain than other ions such as lead. Since my analysis uses only proton-proton collisions I will detail only their acceleration here. The proton source is a bottle of hydrogen gas, which is stripped of its electrons and accelerated to a 50 MeV proton beam by the Linac2 linear accelerator~\cite{Linac2}. This 50 MeV proton beam is then passed to the Proton Synchroton Booster (PSB) which accelerates the beam to 1.4 GeV in four superimposed synchrotron rings before injecting the bunches into the Proton Synchrotron (PS). By adjusting the timings of the four superimposed rings of the PSB and varying which rings the PS is filled from a plethora of bunch patterns can be selected based on the current operating goals~\cite{PSB}. The PS accelerates the proton beam from 1.4 GeV to 25 GeV in a 628 meter circumference synchrotron. The PS also does the final bunch splitting, creating the bunch pattern which will be kept through the remainder of the beam acceleration and collision~\cite{PS}. After being accelerated and bunched by the PS, the beams enter the Super Proton Synchrotron (SPS) for final acceleration and tuning before injection into the LHC. The SPS is a synchrotron nearly 7 km in circumference which accelerates the proton beam up to 450 GeV before injecting it into the LHC~\cite{SPS}.

The final stage of the accelerator chain is the LHC itself. The LHC is a 27 km circumference synchrotron with 2 superimposed rings which resides in what was previously the Large Electron-Positron collider (LEP) tunnel at CERN. It consists of 1104 superconducting dipole magnets designed to reach a peak field of 8.33 T to bend the proton beams around the ring, and 384 quadrupole magnets per ring to control the focusing of the beams. Each ring has a further 536 quadrupole, 1608 sextupole, 784 octupole, and 616 decapole magnets to control the beams and correct instabilities in the beams due to couplings during acceleration and collision. Nominally the LHC is designed to collide proton bunches at ATLAS every 25 nanoseconds with a center of mass energy of 14 TeV, however it is still early in the LHC program and these were not the conditions that the 2012 dataset was taken under. Due to difficulties with the magnet fault protection system the collisions took place with 8 TeV center of mass energy, and because the accelerator and beams are being actively studied a variety of beam configurations were used with bunches separated by 25-75 nanoseconds~\cite{LHC}.   

\section{The ATLAS detector}
\label{SECTION-ATLAS-DET}
The ATLAS detector is one of two large general purpose experiments which studies collisions produced by the LHC. It is designed to able to perform a wide variety of searches and measurements by collecting as much information as possible about the products of each collision. ATLAS uses a multilayered design that has become standard for large collider experiments and can be seen in Figure~\ref{FIGURE-ATLAS}. The innermost portion is called the inner detector which provides fine granularity tracking of charged particles. Moving radially outwards from the interaction point the next detectors are the calorimeters which measure the energy of the particles, and the outermost portion of the detector consists of the muon system which detects and tracks muons traveling through ATLAS. Each of these portions of ATLAS are made up of sub-detector systems designed to work together with the other systems and provide more information than any single technology detector~\cite{TDR1}.

\VLARGEFIG{Atlas}{Cutaway diagram of the ATLAS detector~\cite{Figure-Atlas}.}{FIGURE-ATLAS}

\subsection{Detector geometry}
\label{SECTION-ATLAS-GEO}
Before detailing each detector system that makes up ATLAS it is useful to discuss the coordinate system used in the experiment. The center of the detector is taken to be the origin and the z-axis extends along the beam line with positive being counterclockwise around the LHC when viewed from above. The x-axis points towards the center of the LHC and the y-axis points vertically upwards. While this forms a complete basis to describe the detector and it is sometimes used, it is not the most common coordinate system. Ignoring gravitational effects all directions transverse to the beams are equivalent and can be described by an angle $\phi$ taken to be 0 along the x-axis and increasing right-handedly with respect to the z-axis. The angle from the beam line is a common parameter for describing decays. However because objects are produced with Lorentz boosts in the z direction ranging from 0 to nearly 1 it is more useful to use a relativistic invariant to describe this angle. The equation for the Lorentz invariant rapidity ($y$) is:

\begin{equation}
y = \frac{1}{2}ln\left(\frac{E+p_z}{E-p_z}\right)
\end{equation}

While useful, the rapidity of a particle is dependent on the particle's mass and a different rapidity coordinate system to describe the detector would be necessary for each mass. Almost all particles produced by the LHC have $m << E$ so we can calculate rapidity with $m=0$ and it is approximately the rapidity for all particles produced by the LHC. This is called the pseudorapidity ($\eta$).

\begin{equation}
 \eta = \frac{1}{2}ln\left(\frac{\left|\vec{p}\right|+p_z}{\left|\vec{p}\right|-p_z}\right)
\end{equation}

\noindent
Which can be rewritten using the angle from the z-axis ($\theta$) as:

\begin{equation}
 \eta = -ln\left(tan\left(\frac{\theta}{2}\right)\right)
\end{equation}

Thus pseudorapidity is a purely geometric quantity, with $\eta = 0$ corresponding to the transverse plane and $|\eta| = \infty$ corresponding to the beamline. In detector parlance regions with small $|\eta|$ are called ``central'' and regions with larger $|\eta|$ are called ``forward.''


\subsection{Magnet system}
\label{SECTION-ATLAS-MAGNETS}
The ATLAS detector has three large superconducting magnet systems, the superconducting solenoid, the barrel toroid, and the endcap toroids as shown in Figure~\ref{FIGURE-MAGNETS}. The purpose of these magnets is to bend the path of charged particles as they propagate through the detector. With careful tracking of a charged particle's path through the magnetic field, it is possible to determine the particle's momentum~\cite{TDR1}.

The superconducting solenoid is a cylinder 5.3 m long and 2.63 m in diameter. It has 1173 turns of superconducting wire in a single layer along its length with an operating current of 7.6 kA. The inner volume contained by the solenoid has a central magnetic field of 2 T with a peak field of 2.6 T at the superconducting wire. The solenoid is designed to be as thin as possible in order to minimize the interaction between itself and particles from physics events. The particles pass through the 19 cm thick (at most 0.66 radiation lengths) solenoid before they enter the calorimeters~\cite{MAGNET}. 

The barrel toroid consists of 8 coils each of which is a 25.3 m long and 5.35 m wide ``racetrack'' design. These magnets run the length of ATLAS with their long dimension running parallel to z and their short dimension running radially. They are spaced evenly around the detector with their outer edge at a radius of 10.05 m. Each coil contains 120 turns of superconducting wire with an operating current of 20.5 kA which produces a peak field of 3.9 T~\cite{MAGNET}.

The two endcap toroids complete the ATLAS magnet system. Each endcap contains 8 coils of a racetrack design similar to the barrel toroid, however these coils are 4.5 m in the radial direction and 5 m in the z direction. The endcap coils are offset from the barrel toroid coils by $22.5^\circ$ in $\phi$ so that they bisect the angle between adjacent barrel toroid coils. They are aligned in z to share a common outer edge with the barrel toroid, and are placed radially from 0.825 m to 5.35 m. With 116 turns per coil of superconducting wire carrying 20 kA, the endcap toroids produce a peak field of 4.1 T~\cite{MAGNET}.

\VLARGEFIG{Magnets}{Illustration of the ATLAS magnet system, showing the barrel solenoid, barrel toroid, and endcap toroid coils~\cite{ATLAS-EXP}.}{FIGURE-MAGNETS}


\subsection{Inner detector}
\label{SECTION-ATLAS-ID}
The ATLAS inner detector is designed to provide excellent tracking information for charged particles with $|\eta| < 2.5$ produced by the LHC and is comprised of three concentric subsystems as shown in Figure~\ref{FIGURE-ATLAS-ID}. The pixel detector is nearest the beamline and provides the most precise position information with 97 million channels across three layers in the barrel region and with 43 million channels across five disk layers at both ends. Moving radially outwards from the beamline the next subsystem is the semiconductor tracker (SCT) which uses eight layers of thin slicon microstrip sensors in the barrel and 44 sensor layers in each endcap, with alternating layers at a 40 mrad angle to each other to allow full determination of position. The final subsystem of the inner detector is the transition radiation tracker (TRT) which is a straw tube system consisting of a barrel section containing axial straws and 18 radial straw wheel segments in each endcap, designed so that most particles will transverse 36 detecting straws~\cite{TDR1}.

\VLARGEFIG{InnerDetector}{Cutaway diagram of the ATLAS inner detector~\cite{Figure-InnerDetector}.}{FIGURE-ATLAS-ID}

\subsubsection{Pixel detector}
\label{SECTION-ATLAS-ID-PIXEL}
The pixel detector has the highest granularity and offers the best positioning and tracking information of charged particles in ATLAS. The system contains three barrel layers with three transverse disk layers at each end. The barrel layers are all 801 mm long, with the innermost layer located at a mean radius of 50.5 mm, the middle layer at 88.5 mm, and the outermost layer at 122.5 mm. The disk layers are all identical annuli with an inner radius of 89 mm and outer radius of 150 mm. These disks are placed at a mean $|z|$ of 495 mm, 580 mm, and 650 mm. This gives the pixel detector a total detecting area of 1.7 $m^{2}$ and coverage of $|\eta| \le 2.5$~\cite{TDR1}~\cite{PIXEL}. 

The active medium in the detector is silicon sensor cells 50 $\mu$m $\times$ 400 $\mu$m in size. In the barrel layers the long dimension is in the z direction and in the disk layers the long dimension is radial. These sensor cells are bump bonded to readout chips with each chip reading an 18 $\times$ 160 cell array. The signal from each cell is amplified and compared to a programmable threshold on each chip. If the signal exceeds the threshold the location is stored in a buffer on the chip to be read out via optical link in the case of a level 1 trigger acceptance, as detailed in Section~\ref{SECTION-ATLAS-TDAQ}~\cite{TDR1,PIXEL}.

\subsubsection{Semiconductor tracker}
\label{SECTION-ATLAS-ID-SCT}
While the pixel detector provides the highest resolution for tracking particles, the technology is not cost-effective to use to cover the larger areas corresponding to larger radii. The next subdetector is the semiconductor tracker (SCT) with four cylinders at radii of 299 mm, 371 mm, 443 mm, and 514 mm and nine disks at both endcaps with mean $|z|$ of 853.8 mm - 2720.2 mm. Each cylinder is 1492 mm long and contains two layers of silicon microstrip sensors at a 40 mrad angle to each other. The microstrip sensors are each 63.6 mm wide and 64 mm long rectangles divided into 768 microstrips each 16 $\mu$m wide~\cite{SCT_Barrel}. 
The endcap disks have a more complicated geometry with each disk containing 1-3 ``rings'' of sensor modules depending on position. Each endcap module has two layers of microstrip silicon sensors at a 40 mrad angle to eachother, similar to the barrel modules, however unlike the barrel modules the endcap microstrip sensors are tapered to form trapezoidal segments rather than rectangular. This tapering also causes variable microstrip widths of 16 $\mu$m - 20 $\mu$m. Each module for both the endcap and barrel regions has four silicon sensors (two per layer) attached to central logic circuits which amplify the signals from each microstrip and compare them to a programable threshold. Similar to the pixel detector, the channels with signals exceeding the threshold are stored in a buffer to be read out if the event is accepted by the level 1 trigger system~\cite{TDR1,SCT_Barrel,SCT_Endcap}.

%While the pixel detector provides the highest resolution for tracking particles, the technology is not feasible to use to cover the larger areas corresponding to larger radii, thus the next subdetector is the SCT with 4 cylinders at radii of 299 mm, 371 mm, 443 mm, and 514 mm and 9 disks at both endcaps with mean $|z|$ of 853.8 mm, 934 mm, 1091.5 mm, 1299.9 mm, 1399.7 mm, 1771.4 mm, 2115.2 mm, 2505 mm, and 2720.2 mm.

\subsubsection{Transition radiation tracker}
\label{SECTION-ATLAS-ID-TRT}
The final subdetector of the inner detector is the TRT. While both the pixel detector and SCT use variations of silicon detector technology, the TRT uses a modification of drift tube technology to detect particles. The TRT is divided into one barrel and two endcap sections. The barrel section consists of 52544 straw-tubes arranged in 73 layers parallel to the beam axis. Each straw-tube is a drift tube 1441 mm long and 4 mm in diameter and contains a 70\% Xe, 27\% $CO_{2}$ and 3\% $O_{2}$ gas mixture. Each straw-tube also contains a central 31 $\mu$m diameter gold-plated W-Re wire which is held at a potential of -1.53 kV relative to the straw-tube wall~\cite{TRT_Barrel}. 

The endcaps are each made up of 122,880 straw-tubes arranged radially in 160 layers. These straw-tubes are identical to those used in the barrel except that they are each 370 mm long. These endcap straw-tubes are bundled into modules called wheels of 8 layers each, and the wheels are distributed with $848 mm \le |z| \le 2710 mm$ to give nearly uniform coverage in $\eta$~\cite{TRT_Endcap}. Overall the barrel covers $|\eta| < 1.0$ and the endcaps cover $1.0 < |\eta| < 2.0$ with most particles traversing a total of 30 straw-tubes. 

As a charged particle traverses each straw it causes primary ionization within the gas, which undergoes avalanche multiplication as it accumulates toward the wire giving an amplification factor of $2.5 \times 10^{4}$ with the operating gas mixture and voltage. The unique feature of the ATLAS TRT is that surrounding each straw-tube is a layer of transition radiation (TR) material. The TR material is made up of many layers of polypropylene and polyethylene, and is designed to maximize the production of transition radiation produced by charged particles traversing the boundary between the two materials with different dielectric constants. The transition radiation produced in the TR material is generally a soft x-ray photon which is absorbed by the xenon in the straw-tubes, ionizing the xenon and producing an energy cascade much larger than a typical ionizing particle does when traversing a straw tube. This is particularly useful because electrons and charged pions are difficult to discriminate between, however the energy of transition radiation is proportional to $\gamma=E/m$ which allows for an additional rejection factor of 50-100 depending on the electron quality definition as described in Section~\ref{SECTION-OBJ-EL}~\cite{TDR1,TRT}.

\subsection{Calorimeters}
\label{SECTION-ATLAS-CALO}
Having measured the positions of particles as precisely as possible in the inner detector, the next detector systems particles will encounter are designed to measure their energy. The electromagnetic (EM) calorimeter is nearest the beamline covering $|\eta| < 3.2$ and uses liquid argon (LAr) technology with lead absorber plates in a distinctive accordion pattern. The hadronic calorimeter resides around the EM calorimeter, using scintillating tiles with iron absorbers in the barrel region of $|\eta| < 1.7$ and using LAr technology with copper and tungsten absorbers in the $1.5 < |\eta| < 3.2$ and $3.1 < |\eta| < 4.9$ regions respectively. The layout of these systems can be seen in Figure~\ref{FIGURE-ATLAS-CALO}. It is important that the calorimeter system provides the best containment of particles possible while maintaining good energy resolution so that the total energy of events can be determined~\cite{TDR1}.
  
\VLARGEFIG{Calorimeters}{Cutaway diagram of the ATLAS calorimeter systems~\cite{Figure-Calo}.}{FIGURE-ATLAS-CALO}

\subsubsection{Electromagnetic Calorimeter}
\label{SECTION-ATLAS-CALO-EM}
The ATLAS EM calorimeter is divided into a barrel section, a presampler, and two endcap sections. The barrel calorimeter is made up of two half barrels which surround the superconducting solenoid and covers the range $|\eta| < 1.475$ with one half barrel covering $\eta > 0$ and the other half barrel covering $\eta < 0$.  Each half barrel is a cylinder 3.2 m long and has a 2.8 m inner radius and 4 m outer radius. There are 1024 accordion-shaped absorber plates arrayed radially in each half barrel with the oscillations increasing in amplitude as radius increases which provides uniform density in $\phi$. The absorbers are 1.53 mm thick lead for $|\eta| < 0.8$ and 1.13 mm thick lead for $0.8 < |\eta| < 1.475$ with 0.2 mm thick stainless steel sheets glued to each side to provide structural support. Centered between consecutive absorbers is a readout electrode held at 2 kV relative to the absorber with the 2 mm gap between the electrode and absorber filled with liquid argon. Electrically charged incident particles will shower via Bremsstrahlung in the absorber and this shower will exit the thin absorber layer and enter the liquid argon. The shower ionizes the liquid argon and this ionization is collected at the electrode where it is amplified and read out at both the inner and outer edges of the calorimeter~\cite{EMCAL_Barrel}. 

The presampler is a 22 mm thick detector covering the interior of the barrel calorimeter. It is similar to the barrel detector in that uses liquid argon with 1.9-2.0 mm gaps between electrodes, however unlike the barrel calorimeter the presampler has no absorbers. The purpose of the presampler is to measure showers produced by interactions with the material between the interaction point and the EM calorimeter and improve the energy resolution of the EM calorimeter~\cite{EMCAL_Presampler}. 

The endcap sections are each a wheel 630 mm thick with a 330 mm inner radius and 2098 mm outer radius which covers $1.375 < |\eta| < 3.2$. Each wheel is further divided into an inner wheel and an outer wheel by a 3 mm gap located at $|\eta| = 2.5$. The endcaps have a design similar to the barrel calorimeter, with accordion-shaped absorber plates placed radially and readout electrodes interleaved. Each outer wheel contains 768 absorbers of 1.7 mm thick lead, while the inner wheels each have 256 absorbers of 2.2 mm thick lead. The endcap sections also have an 11 mm thick presampler of similar design to that used in in the barrel section~\cite{TDR1,EMCAL_Endcap}.

\subsubsection{Hadronic Calorimeter}
\label{SECTION-ATLAS-CALO-HAD}
The hadronic calorimeter makes up the remainder of the ATLAS calorimeter system and is comprised of four sub-systems; the tile barrel calorimeter and tile extended barrel calorimeter are both based on using iron absorber plates with plastic scintillator tiles interspersed, while the hadronic endcap calorimeter and the forward calorimeter are both based on LAr technology similar to the EM calorimeter. The tile barrel calorimeter has an inner radius of 1144 mm and an outer radius of 2115mm and a length of 5640 mm. The tile barrel calorimeter consists of 64 modules, each of which is a radial slice of the detector. Each module consists of 64 steel plates that are each 5 mm thick and run the radial length of the module. Between consecutive full length plates there are 11 alternating layers of scintillating plastic tiles and steel spacing tiles which is 4 mm thick. These layers progressively increase in length from the inner radius to the outer radius in order to provide high precision measurements while maintaining the necessary depth of interaction lengths to contain very energetic jets. A 1.5 mm gap along both edges of each alternating layer contains a wavelength shifting fiber which carries the scintillation light to photomultiplier tubes located along the outer radius of the modules where the signals are amplified, digitized and processed by readout electronics. The extended barrel calorimeter consists of two sections, one at each end of the tile barrel calorimeter. Each of these sections is 2900 mm long but otherwise follows the same general design as the tile barrel calorimeter with minor modifications to 12 of the 64 modules in each extended barrel calorimeter to accommodate necessary structural supports for the LAr cryostat. A gap region exists between the tile barrel calorimeter and the extended barrel calorimeter on each side. This gap is necessary to provide services to the LAr calorimeters and the inner detector, and while approximately 750 mm wide it is adjusted as needed to accomodate these necessary services. The gap region contains the intermediate tile calorimeter which consists of an irregular arrangement of absorber and scintilator tiles used to estimate the energy lost in the dead material of the gap region. In total the tile barrel calorimeter covers the $|\eta| < 1.0$ region while the extended barrel calorimeter covers $0.8 < |\eta| < 1.7$ and the intermediate tile calorimeter covers $0.8 < |\eta| < 1.0$~\cite{HCAL_Barrel}.

The hadronic endcap calorimeter consists of two wheels located outside of the electromegnetic endcap calorimeters at both ends of the detector, for a total of four wheels. Each of these wheels is further made up of 32 identical wedge-shaped modules. The front wheel on each side starts at a $|z|$ of 4,277 mm and is 816.5 mm in length. The rear wheels start at a $|z|$ of 5134 mm with a length of 961 mm, leaving a 2 mm gap between the wheels. Each front wheel module contains 25 parallel copper plates which are each 25 mm thick and are evenly spaced in z and arrayed transverse to the beamline. The rear wheel modules each contain 17 parallel copper plates which are 50 mm thick and are also evenly spaced in z and arrayed transverse to the beamline. This means that all of the plates are separated by 8.5 mm gaps which are filled with liquid argon. Three electrodes are evenly spaced in each gap with the outer two electrodes held at 2000 V and the central electrode providing the signal for amplification and processing. All of the plates have an outer radius of 2090 mm and the first 9 plates of the front wheels has an inner radius of 372 mm while the remaining plates all have an inner radius of 475 mm, providing coverage in the region $1.5 < |\eta| < 3.2$~\cite{HCAL_HEC}. 

The final sub-system of the ATLAS hadronic calorimeter is the forward calorimeter. This system covers the region $3.1 < |\eta| < 4.9$ and resides entirely inside the 475 mm inner radius of the hadronic endcap calorimeter. This region is extremely harsh with very high radiation densities and many design compromises were necessary to ensure the forward calorimeter could survive and operate in this environment. The forward calorimeter is composed of three sections at each end of the detector. These sections are cylindrical and are arranged coaxially along the length of the beam pipe as seen in Figure~\ref{FIGURE-ATLAS-CALO}. Each section is made of an absorber matrix cylinder with holes along its length in a honeycomb pattern. Each of these holes contains a thin walled electrode tube and an electrode rod of slightly smaller radius. The small gap between the electrode rod and tube is filled with liquid argon and the electrode rod is held at 250 V relative to the electrode tube. In the section on either side of the detector which is nearest the interaction point the absorber matrix and the electrode rod are both made of copper. In the remaining sections the absorber matrix and electrode rods are made of tungsten. These materials were choosen due to their densities as well as their thermal properties, ability to be produced to the necessary specifications, and hadronic shower sizes. In all of the modules the electrode tube is made of copper and the electrical signal is read out from each absorber rod for amplification and processing. The liquid argon gaps are smaller than is common in LAr detectors, being 269 $\mu m$, 376 $\mu m$, and 508 $\mu m$ in the three sections at each end of the detector and increasing with the distance from the interaction point. This is necessary to prevent charge accumulation in the liquid argonwhich would degrade performance and is caused by the high radiation density of the region, which decreases with distance from the interaction point. The overall layout of the segments is approximately projective from the interaction point, with the inner radius of the segments increasing proportional to $|z|$, the electrode spacing increasing from 7.50 mm to 9.00 mm across the three segments, and the number of electrodes decreasing from 12,260 tubes in each module nearest the interaction point to 8224 electrodes in each module furthest from the interation point~\cite{HCAL_FCAL1,HCAL_FCAL2}.

\subsection{Muon Spectrometer}
\label{SECTION-ATLAS-MUON}

The ATLAS Muon Spectrometer (MS) is the outermost of the ATLAS detector systems and accounts for a majority of the detector's volume. The purpose of this system is to detect muons as they traverse the ATLAS detector and making precision position measurements at three different detector layers to calculate the momentum of each muon based on the curvature of the muon's trajectory as it travels through the ATLAS magnetic field. To accomplish this goal the muon spectrometer has four subsystems which employ differing detector technologies as needed in the various regions of the ATLAS detector. Monitored Drift Tubes (MDTs) and Cathode Strip Chambers (CSCs) provide high precision tracking information over the large area of the muon spectrometer in three concentric layers, called stations. The MDT system uses gas drift tube technology and covers the region $|\eta| < 2.7$, while the CSC system uses multiwire proportional chambers with a cathode strip readout and covers $2.0 < |\eta| < 2.7$. Both the MDT and CSC systems have long response times and are not capable of being used in the Level 1 trigger system as described in Section~\ref{SECTION-ATLAS-TDAQ}, so two additional muon spectrometer systems are employed for the initial detection of muons. The Resistive Plate Chamber (RPC) covers the region $|\eta| < 1.05$ using resistive plate capacitors which locally discharge when their internal gas is ionized while the Thin Gap Chamber (TGC) systems covers $1.05 < |\eta| < 2.4$ using multiwire proportional chambers with a smaller geometry than the CSC system~\cite{MS_TDR}. The overall layout of these systems is shown in Figure~\ref{FIGURE-ATLAS-MUON}.

\VLARGEFIG{Muon}{Cutaway diagram of the ATLAS muon spectrometer and toroid magnet systems~\cite{Figure-Muon}.}{FIGURE-ATLAS-MUON}

\subsubsection{Monitored drift tubes}
\label{SECTION-ATLAS-MUON-MDT}
The monitored drift tube (MDT) chambers provide the majority of the precision muon tracking capability in ATLAS. MDT modules are arranged into barrel and end-cap regions, with the barrel composed of three concentric cylinders with radii of 5, 7.5, and 10 m with coverage of $|\eta| < 1.0$ and the endcap regions containing four disks each at $|z|$ of 7, 10, 14, 22 m respectively and covering the range $1.0 < |\eta| < 2.7$. Each chamber is composed of two sets of drift tube multilayers on either side of a rigid support structure which is 150 mm thick. The multilayers in MDT chambers in the stations nearest the interaction point have four layers of drift tubes while all other MDT chamber multilayers have three layers of drift tubes. Each drift tube is 30 mm in diameter and is filled with a 91$\%$ Ar, 4$\%$ $N_2$, 5$\%$ $CH_4$ gas mixture. Each tube is read out from a central 50 $\mu m$ tungsten wire. The wire is held at 3270 V and gives a spacial resolution of of 80 $\mu m$ with a maximum drift time of approximately 500 ns. Because of this long drift time it is necessary to correlate the signals from the MDT system with corresponding signals in the RPC and TGC systems which provide much more prompt results in order to determine which bunch crossing the MDT signals originate from~\cite{MS_TDR}.

\subsubsection{Cathode strip chambers}
\label{SECTION-ATLAS-MUON-CSC}
The cathode strip chambers (CSCs) are multiwire proportional chambers used for precision muon position measurements in the region of highest radiation density, $2.0 < |\eta| < 2.7$ in the station nearest to the interaction point. Similar to the MDT, the CSC consists of two multilayers with each multilayer containing four monolayers. Each monolayer is a 5.08 mm gas gap containing a 30$\%$ Ar, 50$\%$ $CO_2$ and 20$\%$ $CF_4$ gas mixture. In the center of each gas gap ia a plane of parallel anode wires. The anode wires are 30 $\mu m$ diameter tungsten wires separated by 2.54 mm and held at 2600 V. The walls forming the gap are copper-clad and etched to form thin cathode strips. The cathode strips on one of the walls run orthogonal to the anode wires and provide the precision coordinate measurement, while the cathode strips on the other wall are coarser and run parallel to the anode wires to provide the transverse coordinate measurement. For the precision strips it is only necessary to read out every third strip in order to exceed the resolution of the MDT by using charge interpolation between the strips, and these read-out strips are separated by 5.08 mm. The final resolution in the bending direction is 60 $\mu m$ for a monolayer~\cite{MS_TDR}.
 
\subsubsection{Resistive plate chambers}
\label{SECTION-ATLAS-MUON-RPC}
The resistive plate chambers (RPCs) are designed to complement the MDT system in the barrel region ($|\eta| < 1.05$). Each RPC chamber is a simple design; two resistive plates form a capacitor and are held at 8900 V with a 2 mm gap filled with 97$\%$ $C_2H_2F_4$ and 3$\%$ $C_4H_{10}$. An incident muon will ionize the gas and cause a local discharge of the capacitor. This discharge is read out via capacitative coupling by metal strips running in orthoganal directions on both sides of the resistive capacitor. The RPC chambers are placed two thick at each of three stations. The two middle stations are directly inside and outside of the middle MDT barrel station, and the far station is directly inside of the outer MDT barrel station. This system provides prompt muon detection with a delay of less than 10 ns and a timing uncertainty of 1.5 ns. The signal position is known to within a resolution of 1 cm which is used by the level 1 trigger system and provides a complementary position measurement to the MDT~\cite{MS_TDR}.

\subsubsection{Thin gap chambers}
\label{SECTION-ATLAS-MUON-TGC}
The thin gap chambers (TGCs) fill a role similar to the RPC, prompt detection of muons for use in level 1 triggering and a complementary position measurement to the MDT, but in the endcap region ($1.05 < |\eta| < 2.4$). The TGCs are based on multiwire proportional chamber technology similar to the CSCs but with a smaller geometry and faster readout time. Each TGC gas gap is 2.8 mm wide and is filled with a highly quenching 55$\%$ $CO_2$ and 45$\%$ $n-C_{5}H_{12}$ gas mixture. A central plane of 50 $\mu m$ tungsten anode wires are spaced 1.8 mm apart and are held at 3100 V. The signals from these wires are read out with 4-20 wires forming an individual readout channel depending on $\eta$. Signals are also read out from etched copper strips on one of the walls of each gap to provide a measurement of the azimuthal angle for each track. This configuration gives each gap a position resolution of approximately 9 mm and a time response of 7 ns, which is sufficient for bunch identification and use by the level 1 trigger system. TGC modules are made up of either gas gap doublets or triplets with 20 mm of separation between consecutive gas gaps. The inner wheel at $|z|\ =\ 7\ m$ of each endcap has a layer of doublet chambers and the middle layer wheel at $|z|\ =\ 14\ m$ has two layers of doublet chambers and a layer of triplet chambers, giving the total system a depth of nine gaps~\cite{MS_TDR}. 

\subsection{Triggering and data acquisition}
\label{SECTION-ATLAS-TDAQ}
As described in Section~\ref{SECTION-ATLAS-LHC}, proton bunch crossings occur inside the ATLAS detector every 25 ns. With the size and complexity of the ATLAS detector (the average event is 1.3 Mbytes of data~\cite{TDR1}) it is not possible to read out and store the detector response for every bunch crossing, thus a trigger and data aquisition system (TDAQ) has been implemented to identify and record the most interesting events. The trigger system is divided into three levels, each of which takes as input the accepted events of the previous level and itself reduces the rate of accepted events using increasingly complex algorithms. The level 1 (L1) trigger uses local information in the calorimeter and muon systems to reduce the accepted event rate from 40 MHz to 75 kHz, the level 2 (L2) trigger uses more precise information including tracking from the inner detector for the region of interest (RoI) that caused the level 1 acceptance to further reduce the accepted event rate from 75 kHz to 3.5 kHz, and the event filter (EF) is the final trigger level which uses the highest granularity information from the entire detector to further reduce the accepted event rate from 3.5 kHz to the 200 Hz which is saved for analysis.

The level 1 trigger has an event input rate of 40 MHz and a maximum event acceptance rate of 75 kHz with a total latency of 2.0 $\mu s$. The 40 MHz input event rate means that no single part of the trigger decision can take more than 25 ns, which is achieved by using a highly parallelized hardware implementation. The electromagnetic liquid-argon calorimeter and hadronic tile calorimeter systems as well as the RPCs and TGCs in the muon spectrometer each have their signals read out to the level 1 trigger system. The calorimeter signals are processed by hardware located in the ATLAS counting room adjacent to the cavern which houses ATLAS, while the muon system signals are processed by hardware located on the ATLAS detector. These level 1 trigger processors each only process their local detector area and operate at a lower granularity than the systems are ultimately capable of. The processors look for energy clusters above a variety of set thresholds depending on the system and region of the detector, with an above threshold area forming a region of interest (RoI). The exception to the local scope of the level 1 trigger system is a special processor which calculates the total transverse energy of each event as well as the missing transverse energy of each event and compares them to a variety of thresholds. All of the processors send a list of surpassed thresholds to the central trigger processor (CTP) which correlates and counts the multiplicity of surpassed thesholds and determines a level 1 trigger decision for each event based on a programable trigger menu~\cite{Level1_Trigger}.

The level 2 trigger has an event input rate of 75 kHz from the level 1 trigger and event acceptance rate of 3.5 kHz with a total latency of 10 ms. Unlike the level 1 trigger, the level 2 trigger system uses all of the ATLAS detector systems and is implemented in software. For each event, the detector signals for all of the systems are read out in each of the RoIs identified by the level 1 trigger system to a node in the level 2 server farm. A series of algorithms are then applied in software to the event depending on the exact level 1 trigger conditions in order to refine the measurements. A final level 2 decision is reached based on the outcome of these algorithms~\cite{HLT_TDR}.

The final level of the trigger system is the event filter. The event filter has an event input rate of 3.5 kHz and a final event acceptance rate of 200 Hz with a latency of 1 s. This trigger level is very similar to the level 2 trigger system however rather than only calculating a trigger decision based on the RoIs, the event filter calculates a decision based on the entire event. Each event accepted by the level 2 system has all of the detector systems read out to a node in the event filter server farm. Based on the complete event information a lengthier and more precise callibration is performed, and based on this more detailed information an event filter decision is calculated. Events which are accepted by the event filter are read out from ATLAS to be saved for analysis~\cite{Level1_Trigger}.
