\chapter{Multivariate analysis}
\label{SECTION-ANALYSIS}
The event selection described in Chapter~\ref{SECTION-SELECTION} does not sufficiently optimize the statistical significance of a potential signal in each of the analysis channels. The method used to optimize the significance in this analysis is a boosted decision tree (BDT) multivariate analysis (MVA). The details of the BDT algorithm are discussed in Section~\ref{SECTION-ANALYSIS-BDT}, but the basic premise is that events are sorted into background-like and signal-like bins using a series of cuts called a decision tree. This decision tree optimizes the separation of signal and background events at each node by placing a cut on a single variable, with both output possibilities forming a new decision node iteratively until the specified end conditions are met. Misclassified events, that is signal events in a background-like bin or background events in a signal-like bin, have their relative weight ''boosted'' and a new decision tree is trained. This training and boosting process is iterated many times and all of the decision trees combined form the BDT. This boosting serves to increase the importance of hard to classify events and make them more likely to be correctly categorized in subsequent decision trees. A final decision weight is assigned to each event based on the aggregate purity of the bins it was sorted into in each decision tree, which represents how signal-like or background-like each event is. This iterative process allows for a more nuanced handling of variable correlations than a simple cuts based analysis which is effectively a single decision tree. The BDT also has the convenient property that it condenses the discrimination power from many variables into a single highly discriminating weight which is more easily understood by the human analyst.

\section{Boosted decision trees}
\label{SECTION-ANALYSIS-BDT}
The boosted decision tree (BDT) algorithm~\cite{BDT} is designed to optimize the discrimination between signal and background events. The implementation of the BDT algorithm used in this analysis is done by the Toolkit for Multivariate Data Analysis with ROOT (TMVA)~\cite{TMVA} and while the principles of the analysis are general to all BDTs, the details are specific to this software package. Each of the analysis channels (2jets 1tag, 2jets 2tag, 3jets 1tag, and 3jets 2tag) has an independent BDT trained with the variables for each channel described in Section~\ref{SECTION-ANALYSIS-BDT-VARS}, and the BDT settings for each channel are described in Section~\ref{SECTION-ANALYSIS-BDT-PARAM}.

The BDT training begins with a sample of signal events and background events, with each event having a weight equal to its contribution to the overall normalization and a set of kinematic variables defined to be used in the BDT. An optimal cut on one of these variables is determined by maximizing $\Delta G$ as defined in Equation~\ref{EQ-ANALYSIS-DELTAGINI} for all possible cuts on these variables where $G_{pre-cut}$ is calculated for all of the events and $G_{passed\ cut}$ and $G_{failed\ cut}$ are calculated for the events which pass and fail the cut respectively.

\begin{equation}
\label{EQ-ANALYSIS-DELTAGINI}
\Delta G = G_{pre-cut}-G_{passed\ cut}-G_{failed\ cut}
\end{equation}

\begin{equation}
\label{EQ-ANALYSIS-GINI}
G = \sum\limits_{events}WP(1-P)
\end{equation}

\noindent
In Equation~\ref{EQ-ANALYSIS-GINI} G is called the Gini index, W is the event weight, and P is the purity which is fraction of events belonging to the signal. With the optimal separation cut defined all of the events are sorted into two bins: events that pass the cut,  and events that fail the cut. The process of finding the optimal cut and sorting the events based on that cut forms a decision node.

Each of the two output bins from the first decision node then form their own decision nodes to be applied to only those events. A new optimal cut is determined for each of these nodes and the events are sorted into four bins (two for each decision node). This process is continued iteratively, further subdividing the sample into more bins at each iteration, until a termination condition is met. A minimum number of events is required to be in a bin in order for that bin to become a new decision node, and there is a maximum depth in the decision tree for a node to become a decision node. The initial decision node is at a depth of 1, the bins output from that decision node are at a depth of 2, and so on. Any bins that do not form decision nodes instead form termination nodes and are defined as signal or background nodes based on the majority of events in that termination node. The decision and termination nodes together form a decision tree, an example of which is given in Figure~\ref{FIGURE-DECISIONTREE}.

\VLARGEFIG{DecisionTreeGraphic}{An example decision tree which sorts input events into signal and background bins. The variables used in the decision tree are described in Section~\ref{SECTION-ANALYSIS-BDT-VARS}}{FIGURE-DECISIONTREE}

While a single decision tree can provide some discrimination power between signal and background events, the discrimination power can generally be increased by training multiple decision trees and increasing the weight of events misclassified as either signal or background by each tree. A misclassified event is a signal event that is sorted into a background termination node or a background event that is sorted into a signal termination node, and this analysis uses the Adaboost algorithm to increase the weight of such events when training subsequent decision trees. The new weight of a misclassified event is given by $w_{boosted}$ in Equation~\ref{EQ-ANALYSIS-ADABOOST}, where $\beta$ is the boosting strength and is on of the parameters optimized in Section~\ref{SECTION-ANALYSIS-BDT-PARAM} and $w_{pre-boosted}$ is the event weight before being boosted.

\begin{equation}
\label{EQ-ANALYSIS-ADABOOST}
w_{boosted} = \left(\frac{1 - err}{err}\right)^{\beta}w_{pre-boosted}
\end{equation}

\begin{equation}
\label{EQ-ANALYSIS-ERR}
err = \frac{sum\ of\ misclassified\ event\ weights}{sum\ of\ all\ event\ weights}
\end{equation}

\noindent
After the misclassified events have their weights boosted, all events have their weights scaled to keep the total sum of events weights normalized and a new decision tree can be trained. The total number of decision trees trained is one of the parameters optimized in Section~\ref{SECTION-ANALYSIS-BDT-PARAM}.

With each event being sorted by multiple decision trees, an aggregate BDT decision weight ($y_{Boost}$) is formed by taking a weighted average of the individual decision tree responses as given by Equation~\ref{EQ-ANALYSIS-BDTWEIGHT}.

\begin{equation}
\label{EQ-ANALYSIS-BDTWEIGHT}
y_{Boost} = \frac{1}{N_{Trees}}\sum\limits_{i=1}^{N_{Trees}}ln\left(\frac{1 - err_i}{err_i}\right)h_i
\end{equation}

\noindent
For Equation~\ref{EQ-ANALYSIS-BDTWEIGHT}, $N_{Trees}$ is the total number of decision trees trained and $err_i$ is given by Equation~\ref{EQ-ANALYSIS-ERR} for each tree. The result of each tree for the event is represented by $h_i$, which is $+1$ or $-1$ if the event was sorted into a signal or background termination node by that tree. Thus the final BDT decision weight for each event is a value between $-1$ and $+1$, where $-1$ represents an ideal background event and $+1$ represents an ideal signal event.

\subsection{Overtraining}
\label{SECTION-ANALYSIS-BDT-OVERTRAINING}
If the BDT has too many degrees of freedom compared to the statistical size of the generated samples during training then the BDT begins to optimize the decision trees based on the specific variable values of individual events rather than on the overall distributions. This is called overtraining and it causes the BDT output to be drastically different if the trained BDT is applied to a new sample simulated under identical conditions. In order to check if a BDT is overtrained all of the event samples are split in half before training. One half, called the training sample, is used to train the BDT and is run through the resulting BDT to compute the BDT decision weight distribution. The remaining half of the events, called the testing sample, is run through the already trained BDT and produces an independant BDT decision weight distribution. The training and testing distributions for both the signal and background samples should be similar if the BDT is not overtrained, and they are compared using a Kolmgorov-Smirnov (KS) test~\cite{KS}. The KS test provides a probability that two samples come from the same distribution, and if the KS result is $<$ 0.5 for either the signal or background distributions then the BDT is determined to be overtrained.

\subsection{Variable selection}
\label{SECTION-ANALYSIS-BDT-VARS}
The variables that are used in the BDT to separate the signal and background events has a large impact on the separation power of the BDT. Variables should be chosen that can discriminate between signal and background events, where the discrimination power is defined the same as in Section~\ref{SECTION-SELECTION-CUTS}. Because the correlations between several different variables are not always obvious for all of the different processes included in the signal and background samples, the analysis begins with a large list of variables in an attempt to be as comprehensive as possible. For each sample the analysis objects, as defined in Chapter~\ref{SECTION-OBJ}, are the lepton (lep), reconstructed neutrino (neutrino), reconstructed W boson (W), reconstructed top quark (top), reconstructed \Wprime\ boson (W'), the leading jet (jet1), the sub-leading jet (jet2), the leading b-tagged jet (bjet1), and in the appropriate analysis channels the third jet (jet3) and sub-leading b-tagged jet (bjet2). For each of these objects the energy, mass, pseudorapidity, azimuthal angle, transverse momentum, and transverse energy are considered as possible variables for inclusion in the BDT. In addition a set of two body variables is included in this list which consists of the mass, difference in pseudorapidity ($\Delta \eta$), difference in azimuthal angle ($\Delta \phi$), and opening angle ($\Delta R$) for every pair of analysis objects. A final set of global event variables consisting of the MET, the azimuthal angle of the MET ($\phi (MET)$), the total sum of the transverse energy in the event (Sum $E_T$), the total sum of the transverse energy of the analysis object in the event ($H_T$), the aplanarity of the event (a measure of how much of the energy in the event resides in one plane), and the sphericity of the event (a measure of how evenly the energy of an event is ditributed around the interaction point). This forms an initial list of approximately 200 variables depending on the analysis channel, however past experience in the single-top analyses suggests that fewer than 20 variables are necessary for a BDT to achieve good separation of signal and background events. In order to remove unnecessary variables, variables with a discrimination power $<$ 20$\%$ are removed from the list for each channel which leaves approximately 50 variables for each channel. It is important that these variables be well modeled so the background distributions are compared to the data and any variables with a KS value $<$ 0.5 are removed. A BDT is then trained using the remaining variables and the variables are ranked by importance, which is how frequently they are used in the individual decision trees. The five least important variables are removed from the variable list and the process is repeated until only 20 variables remain. At this point the training and variable removal iterations continue but only a single variable is removed in each iteration until the removal of a variable would cause the separation power of the BDT to degrade to $<$ 90$\%$ of the maximum value obtained during these iterations. Table~\ref{TABLE-ANALYSIS-VARIABLES} lists the final variables for each of the analysis channels in order of importance, and Figures~\ref{FIG-ANALYSIS-VARIABLES1}-\ref{FIG-ANALYSIS-VARIABLES13} show the data-Monte Carlo comparison plots for these variables.

\begin{table}
\begin{center}
\begin{tabular}{|c|c|c|c|}
\hline
2jets 1tag & 3jets 1tag & 2jets 2tag & 3jets 2tag \\
\hline
$p_T$(top) & mass(jet1,W) & mass(\Wprime) & mass(\Wprime) \\
mass(\Wprime) & $H_T$ & $\Delta$R(lepton,top) & sphericity \\
E(\Wprime) & mass(\Wprime) & aplanarity & $p_T$(top) \\
mass(jet1,jet2) & mass(jet1,jet2) & $H_T$ & $\Delta\eta$(lepton,W) \\
$\Delta\eta$(lepton,W) & MET & $p_T$(bjet1) & $E_T$(bjet1) \\
Sum $E_T$ & $\Delta$R(neutrino,top) & $p_T$(bjet2) & $\Delta$R(bjet1,top) \\
mass(jet1,W) & E(lepton) & mass(bjet1,bjet2) & $\Delta$R(lepton,bjet1) \\
$\Delta$R(lepton,top) & $p_T$(top) & $E_T$(W) & $\Delta\phi$(W,top) \\
mass(lepton,jet1) & $p_T$(W) & $\Delta$R(lepton,bjet2) & $\Delta\eta$(lepton,top) \\
$\Delta\eta$(neutrino,top) & E(\Wprime) & $\Delta$R(bjet1,bjet2) & mass(lepton,bjet1) \\
$E_T$(lepton) & $\Delta\eta$(lepton,W) & $\Delta$R(lepton,W) & $\Delta$R(bjet1,W) \\
$p_T$(W) & mass(lepton,jet1) & $p_T$(top) & $p_T$(lepton) \\
$H_T$ & $\Delta$R(lepton,top) & sphericity & aplanarity \\
MET & mass(jet1,top) & $p_T$(lepton) &  \\
$\Delta\eta$(neutrino,jet2) & $\Delta\eta$(neutrino,W) & & \\
$\Delta$R(jet2,W) & & & \\
$\Delta$R(lepton,jet2) & & & \\
mass(jet1,top) & & & \\
\hline
\end{tabular}
\caption{Boosted decision tree variable lists for the four analysis channels. Variables are ranked by importance.}
\label{TABLE-ANALYSIS-VARIABLES}
\end{center}
\end{table}

\SEXFIGLEG{BDT_variables/BDTvar_Toppt_opt_2jets_1tag_MV1_All}{BDT_variables/BDTvar_Wprimemass_opt_2jets_1tag_MV1_All}{BDT_variables/BDTvar_WprimeE_opt_2jets_1tag_MV1_All}{BDT_variables/BDTvar_Mass_Jet1Jet2_2jets_1tag_MV1_All}{BDT_variables/BDTvar_DeltaEta_LeptonW_2jets_1tag_MV1_All}{BDT_variables/legend}{Comparison of data to Monte Carlo prediction of the 1st-5th variables by importance in the 2jet 1tag BDT}{FIG-ANALYSIS-VARIABLES1}

\SEXFIGLEG{BDT_variables/BDTvar_MissingEt_sumet_2jets_1tag_MV1_All}{BDT_variables/BDTvar_Mass_Jet1W_2jets_1tag_MV1_All}{BDT_variables/BDTvar_DeltaR_LeptonTop_2jets_1tag_MV1_All}{BDT_variables/BDTvar_Mass_LeptonJet1_2jets_1tag_MV1_All}{BDT_variables/BDTvar_DeltaEta_NeutrinoTop_2jets_1tag_MV1_All}{BDT_variables/legend}{Comparison of data to Monte Carlo prediction of the 6th-10th variables by importance in the 2jet 1tag BDT}{FIG-ANALYSIS-VARIABLES2}

\SEXFIGLEG{BDT_variables/BDTvar_Lepton_Et_2jets_1tag_MV1_All}{BDT_variables/BDTvar_W_pt_2jets_1tag_MV1_All}{BDT_variables/BDTvar_Ht_AllJetsLeptonMissingEt_2jets_1tag_MV1_All}{BDT_variables/BDTvar_MissingEt_et_2jets_1tag_MV1_All}{BDT_variables/BDTvar_DeltaEta_NeutrinoJet2_2jets_1tag_MV1_All}{BDT_variables/legend}{Comparison of data to Monte Carlo prediction of the 11th-15th variables by importance in the 2jet 1tag BDT}{FIG-ANALYSIS-VARIABLES3}

\QUADFIGLEG{BDT_variables/BDTvar_DeltaR_Jet2W_2jets_1tag_MV1_All}{BDT_variables/BDTvar_DeltaR_LeptonJet2_2jets_1tag_MV1_All}{BDT_variables/BDTvar_Mass_Jet1Top_2jets_1tag_MV1_All}{BDT_variables/legend}{Comparison of data to Monte Carlo prediction of the 16th-18th variables by importance in the 2jet 1tag BDT}{FIG-ANALYSIS-VARIABLES4}

\SEXFIGLEG{BDT_variables/BDTvar_Mass_Jet1W_3jets_1tag_MV1_All}{BDT_variables/BDTvar_Ht_AllJetsLeptonMissingEt_3jets_1tag_MV1_All}{BDT_variables/BDTvar_Wprimemass_opt_3jets_1tag_MV1_All}{BDT_variables/BDTvar_Mass_Jet1Jet2_3jets_1tag_MV1_All}{BDT_variables/BDTvar_MissingEt_et_3jets_1tag_MV1_All}{BDT_variables/legend}{Comparison of data to Monte Carlo prediction of the 1st-5th variables by importance in the 3jet 1tag BDT}{FIG-ANALYSIS-VARIABLES5}

\SEXFIGLEG{BDT_variables/BDTvar_DeltaR_NeutrinoTop_3jets_1tag_MV1_All}{BDT_variables/BDTvar_Lepton_E_3jets_1tag_MV1_All}{BDT_variables/BDTvar_Toppt_opt_3jets_1tag_MV1_All}{BDT_variables/BDTvar_W_pt_3jets_1tag_MV1_All}{BDT_variables/BDTvar_WprimeE_opt_3jets_1tag_MV1_All}{BDT_variables/legend}{Comparison of data to Monte Carlo prediction of the 6th-10th variables by importance in the 3jet 1tag BDT}{FIG-ANALYSIS-VARIABLES6}

\SEXFIGLEG{BDT_variables/BDTvar_DeltaEta_LeptonW_3jets_1tag_MV1_All}{BDT_variables/BDTvar_Mass_LeptonJet1_3jets_1tag_MV1_All}{BDT_variables/BDTvar_DeltaR_LeptonTop_3jets_1tag_MV1_All}{BDT_variables/BDTvar_Mass_Jet1Top_3jets_1tag_MV1_All}{BDT_variables/BDTvar_DeltaEta_NeutrinoW_3jets_1tag_MV1_All}{BDT_variables/legend}{Comparison of data to Monte Carlo prediction of the 11th-15th variables by importance in the 3jet 1tag BDT}{FIG-ANALYSIS-VARIABLES7}

\SEXFIGLEG{BDT_variables/BDTvar_Wprimemass_opt_2jets_2tag_MV1_All}{BDT_variables/BDTvar_DeltaR_LeptonTop_2jets_2tag_MV1_All}{BDT_variables/BDTvar_Aplanarity_AllJets_2jets_2tag_MV1_All}{BDT_variables/BDTvar_Ht_AllJetsLeptonMissingEt_2jets_2tag_MV1_All}{BDT_variables/BDTvar_BJet1_Jet_pt_2jets_2tag_MV1_All}{BDT_variables/legend}{Comparison of data to Monte Carlo prediction of the 1st-5th variables by importance in the 2jet 2tag BDT}{FIG-ANALYSIS-VARIABLES8}

\SEXFIGLEG{BDT_variables/BDTvar_BJet2_Jet_pt_2jets_2tag_MV1_All}{BDT_variables/BDTvar_Mass_BJet1BJet2_2jets_2tag_MV1_All}{BDT_variables/BDTvar_W_Et_2jets_2tag_MV1_All}{BDT_variables/BDTvar_DeltaR_LeptonBJet2_2jets_2tag_MV1_All}{BDT_variables/BDTvar_DeltaR_BJet1BJet2_2jets_2tag_MV1_All}{BDT_variables/legend}{Comparison of data to Monte Carlo prediction of the 6th-10th variables by importance in the 2jet 2tag BDT}{FIG-ANALYSIS-VARIABLES9}

\FIVEFIGLEG{BDT_variables/BDTvar_DeltaR_LeptonW_2jets_2tag_MV1_All}{BDT_variables/BDTvar_Toppt_opt_2jets_2tag_MV1_All}{BDT_variables/BDTvar_Sphericity_AllJets_2jets_2tag_MV1_All}{BDT_variables/BDTvar_Lepton_pt_2jets_2tag_MV1_All}{BDT_variables/legend}{Comparison of data to Monte Carlo prediction of the 11th-14th variables by importance in the 2jet 2tag BDT}{FIG-ANALYSIS-VARIABLES10}

\SEXFIGLEG{BDT_variables/BDTvar_Wprimemass_opt_3jets_2tag_MV1_All}{BDT_variables/BDTvar_Sphericity_AllJets_3jets_2tag_MV1_All}{BDT_variables/BDTvar_Toppt_opt_3jets_2tag_MV1_All}{BDT_variables/BDTvar_DeltaEta_LeptonW_3jets_2tag_MV1_All}{BDT_variables/BDTvar_BJet1_Jet_Et_3jets_2tag_MV1_All}{BDT_variables/legend}{Comparison of data to Monte Carlo prediction of the 1st-5th variables by importance in the 3jet 2tag BDT}{FIG-ANALYSIS-VARIABLES11}

\SEXFIGLEG{BDT_variables/BDTvar_DeltaR_BJet1Top_3jets_2tag_MV1_All}{BDT_variables/BDTvar_DeltaR_LeptonBJet1_3jets_2tag_MV1_All}{BDT_variables/BDTvar_DeltaPhi_WTop_3jets_2tag_MV1_All}{BDT_variables/BDTvar_DeltaEta_LeptonTop_3jets_2tag_MV1_All}{BDT_variables/BDTvar_Mass_LeptonBJet1_3jets_2tag_MV1_All}{BDT_variables/legend}{Comparison of data to Monte Carlo prediction of the 6th-10th variables by importance in the 3jet 2tag BDT}{FIG-ANALYSIS-VARIABLES12}

\QUADFIGLEG{BDT_variables/BDTvar_DeltaR_BJet1W_3jets_2tag_MV1_All}{BDT_variables/BDTvar_Lepton_pt_3jets_2tag_MV1_All}{BDT_variables/BDTvar_Aplanarity_AllJets_3jets_2tag_MV1_All}{BDT_variables/legend}{Comparison of data to Monte Carlo prediction of the 11th-13th variables by importance in the 3jet 2tag BDT}{FIG-ANALYSIS-VARIABLES13}

\subsection{BDT parameter optimization}
\label{SECTION-ANALYSIS-BDT-PARAM}
The BDT parameters are the specific values used in the BDT training algorithm described in Section~\ref{SECTION-ANALYSIS-BDT}. These values have a strong impact on the resulting BDT and have a complicated relationship with the variables chosen for the BDT as well as any possible overtraining of the BDT. Because of this it is important to optimize these values but this is not a straight forward process. For each value of these parameters to be tested a new BDT must be trained which makes it computationally impractical to optimize the parameters simultaneous with the variable selection. Instead an iterative approach is taken, with each parameter value being optimized individually by scanning a range of values in discreet steps after the variable selection has been performed. An optimal parameter is chosen based on the BDT separation power after any overtrained trees have been removed. The variable selection and subsequent BDT parameter optimization are then redone if any parameter value changed by more than one step during the optimization. While it is possible for this method to get stuck in local optimizations, it has produced stable results with the available computational resources. A description of the parameters is provided below in the order they are optimized along with the initial training value of each parameter and the range and step size of the optimization scan. The final value of these parameters is provided in Table~\ref{TABLE-ANALYSIS-PARAMETERS}.

\begin{list}{$\bullet$}{}
\item \textbf{nTrees} is the number of individual decision trees to be trained by the BDT. This affects both the performance and possible overtraining of the BDT by allowing more degrees of freedome in the optimization. The range 60-200 is scanned with a step size of 20 and a default value of 120.
\item \textbf{nEventsMin} is the minimum number of events for a decision node to be formed instead of a termination node. This affects the BDT performance and any possible overtraining by limiting how complex the individual decision trees can become. The range 100-300 is scanned with a step size of 20 and a default value of 200.
\item \textbf{maxDepth} is the maximum decision depth of each decision tree in the BDT. Similar to nEventsMin, this affects the BDT performance and any possible overtraining by limiting how complex the individual decision trees can become. The range 6-12 is scanned with a step size of 1 and a default value of 9.
\item \textbf{AdaBoost} is the boosting strength $\beta$ in Equation~\ref{EQ-ANALYSIS-ADABOOST}. This primarily affects the performance of the BDT by adjusting how important the hard to classify events are. The range 0.4-1.0 is scanned with a step size of 0.1 and a default value of 0.7.
\item \textbf{nCuts} is how many possible cuts are tested for each variable when determining the optimal cut for a decision node. This affects the BDT performance by ensuring the decision nodes have adequate granularity in their optimization. The range 8-16 is scanned with a step size of 2 and a default value of 12.
\end{list}

\begin{table}
\begin{center}
\begin{tabular}{|c|c|c|c|c|}
\hline
Parameter & 2jets 1tag & 3jets 1tag & 2jets 2tag & 3jets 2tag \\
\hline
\textbf{nTrees} & 160 & 100 & 100 & 100\\
\textbf{nEventsMin} & 200 & 200 & 180 & 200\\
\textbf{maxDepth} & 10 & 10 & 8 & 8\\
\textbf{AdaBoost} & 0.7 & 0.7 & 0.5 & 0.7\\
\textbf{nCuts} & 10 & 12 & 14 & 12\\
\hline
\end{tabular}
\caption{Optimized Boosted decision tree parameters for each of the four analysis channels.}
\label{TABLE-ANALYSIS-PARAMETERS}
\end{center}
\end{table}

\subsection{BDT output distributions}
\label{SECTION-ANALYSIS-BDT-OUTPUTS}
The BDT output distributions for the four analysis channels are shown in Figure~\ref{FIG-ANALYSIS-BDTRESPONSE}.

\QUADFIGLEG{BDT_variables/2jet_1tag_WprimeR_BDTResponse}{BDT_variables/3jet_1tag_WprimeR_BDTResponse}{BDT_variables/2jet_2tag_WprimeR_BDTResponse}{BDT_variables/3jet_2tag_WprimeR_BDTResponse}{The BDT output distribution with the signal and background processes split into testing and training samples for (a) the 2jet 1tag analysis channel (b) the 2jet 2tag analysis channel (c) the 3jet 1tag analysis channel (d) the 3jet 2tag analysis channel.}{FIG-ANALYSIS-BDTRESPONSE}
