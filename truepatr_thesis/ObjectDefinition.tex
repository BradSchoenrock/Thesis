\chapter{Object Definitions}
\label{SECTION-OBJ}
In order to perform a search for \Wprimechan\ each event needs to be reconstructed from the raw ATLAS data. This raw data is a collection of energy deposits in the calorimeters and tracking hits from the inner detector and muon spectrometer which needs to be refined into a more useable form. The raw data is reconstructed into analysis objects which generally correspond to the particles that passed through the detector. Different types of particles will interact with the various detector systems in different ways, leaving distinct signatures:

\begin{list}{$\bullet$}{}
\item Electrons leave a track in the inner detector and an energy shower in the EM calorimeter. 
\item Muons leave a track in the inner detector and in the muon spectrometer with minimal energy deposited in the calorimeters.
\item Jets leave tracks in the inner detector and energy showers in the EM and hadronic calorimeters.
\item Photons do not leave a track in the inner detector and produce only an energy shower in the EM calorimeter.
\item Neutrinos do not interact with the detector but their presence can be inferred from an imbalance in the total event momentum in the transverse plane.
\end{list}

\noindent
For the \Wprimechan\ analysis the objects of interest are electrons, muons, jets, and the missing transverse energy (MET) corresponding to a neutrino. It is possible for ``fake'' objects to be created due to detector resolution effects or by other particles interacting with the detector in rare or unexpected ways. For example, jets that deposit all of their energy in the EM calorimeter before reaching the hadronic calorimeter would appear to be electrons while electrons that do not lose all of their energy in the EM calorimeter and ``punch-through'' to the hadronic calorimeter can appear to be jets. The object definitions are chosen to balance the rejection of fakes with acceptance of real objects.

\section{Electron definition}
\label{SECTION-OBJ-EL}
Electrons are a key component of the \Wprimechan\ and their reconstruction uses a complex algorithm to identify them at high efficiency while keeping the fakes rate low. In order to have access to higher efficiency or higher purity samples as needed the electrons are reconstructed with increasingly stringent requirements to form three qualities. The requirements for the 3 electron qualities of loose, medium and tight are summarized in Table~\ref{TABLE-OBJ-ELQUALITY}. The reconstruction starts by performing a sliding window search of the middle layer of the EM calorimeter, where a $3 \times 5$ ($\eta \times \phi$) window of calorimeter cells (each $0.025 \times 0.025$ in $\eta \times \phi$) is moved about the calorimeter to find the local maxima of energy enclosed. Maxima with an energy above 2.5 GeV are called seed clusters. Seed clusters are then checked against the tracking information, and clusters with a track within $\Delta\eta$ and $\Delta\phi$ requirements determined by the electron quality are considered electron candidates~\cite{Electrons}.

Electron candidates have their energy recomputed using a $3 \times 7$ ($\eta \times \phi$) window of calorimeter cells with corrections applied based on the position and energy and are assigned a four-momentum based on the tracking and corrected energy. A final set of cuts shown in Table~\ref{TABLE-OBJ-ELQUALITY} is applied to electron candidates based on the quality, with cut values optimized in 10 bins of $\eta$ and 11 bins of cluster transverse energy ($E_{T}$), where $E_T\ = \frac{cluster\ E}{cosh(track\ \eta)}$~\cite{Electrons}. In addition to these requirements the tight electrons are required to pass an enhanced set of cuts. Electron candidates are rejected if they are in the EM calorimeter crack region of $1.37\ <\ |\eta|\ <\ 1.52$ because the calorimeter performance is degraded. The $E_T$ is required to be greater than 25 GeV because this analysis is focused on high energy events. Electron candidates are rejected if they have $\Delta R\ <\ 0.4$ with a jet, where $\Delta R\ =\ \sqrt{\Delta\eta^2\ +\ \Delta\phi^2}$. The final requirement for tight electrons is that they are isolated in the tracker and calorimeter as defined by cutting on the parameters Ptcone30 and Etcone20 respectively, with both cuts being energy dependent with 90\% efficiency. Ptcone30 is the sum of the $p_T$ of all tracks in a cone with half opening angle of 0.3 minus the $p_T$ of the candidate's track. Similarly, Etcone20 is the sum of the $E_T$ in a cone with half opening angle of 0.2 minus the $E_T$ of the cluster. 

\begin{table}
\begin{center}
\begin{tabular}{|p{4cm}|p{11cm}|}
\hline
Type & Cut Description \\
\hline
\multicolumn{2}{|c|}{\textbf{Loose electrons}}\\
\hline
Detector acceptance & $\bullet$ $|\eta|\ <\ 2.47$ \\
\hline
Hadronic leakage & $\bullet$ Ratio of the $E_{T}$ in the first layer of the hadronic calorimeter to the EM cluster $E_T$ ($|\eta|\ <\ 0.8\ and\ |\eta|\ >\ 1.37$) \\
                 & $\bullet$ Ratio of the $E_T$ in all layers of the hadronic calorimeter to the EM cluster $E_T$ ($0.8\ <\ |\eta|\ <\ 1.37$) \\
\hline
EM calorimeter & $\bullet$ Ratio of the 3 $\times$ 7 cell energy to the 7 $\times$ 7 cell energy \\
middle layer   & $\bullet$ Shower width in $\eta$ \\
\hline
\multicolumn{2}{|c|}{\textbf{Medium electrons (including Loose cuts)}} \\
\hline
EM calorimeter first & $\bullet$ Total shower width \\
layer                & $\bullet$ Ratio of the difference in the largest and second largest energy deposits to the sum of those energies \\
\hline
Track quality & $\bullet$ Number of hits in the pixel detector ($\ge\ 1$) \\
              & $\bullet$ Sum of hits in the pixel detector and SCT ($\ge\ 7$) \\
              & $\bullet$ Transverse impact parameter ($<$ 5 mm) \\
\hline
Track matching & $\bullet$ $\Delta\eta$ between the track and cluster ($<$ 0.01) \\
\hline
\multicolumn{2}{|c|}{\textbf{Tight electrons (including Medium cuts)}} \\
\hline
Track quality & $\bullet$ Number of hits in the first layer of the pixel detector ($\ge\ 1$) \\
              & $\bullet$ Transverse impact parameter cut ($<$ 1 mm) \\
\hline
Track matching & Ratio of the cluster energy to the track momentum \\
               & $\bullet$ $\Delta\phi$ between the track and cluster ($<$ 0.02) \\
               & $\bullet$ $\Delta\eta$ between the track and cluster ($<$ 0.005) \\
\hline
TRT & $\bullet$ Number of hits in the TRT \\
    & $\bullet$ Ratio of the number of high-threshold hits to the total number of hits \\
\hline
Photon conversion & $\bullet$ Matches to reconstructed photon conversions are rejected \\
\hline
\end{tabular}
\caption{Definition of variables used for electron identification cuts~\cite{Electrons}.}
\label{TABLE-OBJ-ELQUALITY}
\end{center}
\end{table}

\section{Muon definition}
\label{SECTION-OBJ-MU}
Muons are of approximately equal importance to the analysis as electrons, but thankfully they are much easier to identify and reconstruct in ATLAS. As described in Section~\ref{SECTION-ATLAS-MUON}, the muon spectrometer is the largest ATLAS sub-detector and is dedicated to identifying and measuring muons. Muons are reconstructed with different qualities similar to how electrons are reconstructed. This analysis only uses muons reconstructed with the ``muidcombined'' quality so only that algorithm is described here. Muidcombined muon candidates are formed by independently reconstructing a track in the muon spectrometer (MS) and inner detector (ID), and if these tracks match within $\Delta R\ <\ 0.05$ then a combined track is reconstructed from both systems. Several cuts are detailed in Table~\ref{TABLE-OBJ-MU} which ensure that only well-defined tracks that lie in the most sensitive regions of the detector and that are isolated from other activity are included in the analysis. Two new variables are introduced in these cuts, nTRT is the sum of the number of TRT hits and the number of TRT outliers while MiniIso10\_4 is the sum of the $p_T$ of all objects inside a cone with half opening angle of 0.1 minus the muon $p_T$ with a maximum of 40 GeV.

\begin{table}
\begin{center}
\begin{tabular}{|p{4cm}|p{11cm}|}
\hline
Type & Cut Description \\
\hline
Muon energy & $\bullet$ $p_T\ >\ 25\ GeV$ \\
\hline
Detector acceptance & $\bullet$ $0.1\ <\ |\eta|\ <\ 2.5$ \\
\hline
Track quality & $\bullet$ Number of pixel hits + number of crossed dead pixel cells $>\ 0$ \\
              & $\bullet$ Number of SCT hits + number of crossed dead SCT strips $\ge\ 5$ \\
              & $\bullet$ Number of crossed dead pixel cells + number of crossed dead SCT strips $<\ 3$ \\
              & $\bullet$ For $0.1\ <\ |\eta|\ <\ 1.9$: number of TRT hits + number of TRT outliers $>\ 5$ and $\frac{number\ of\ TRT\ outliers}{nTRT}\ <\ 0.9$ \\
              & $\bullet$ Distance along z from track to primary vertex $<\ 2\ mm$ \\
\hline
Isolation & $\bullet$ $\frac{MiniIso10 \textunderscore 4}{muon\ p_T}\ <\ 0.05$ \\
          & $\bullet$ Muon and all jets with $p_T\ >\ 25\ GeV$ have $\Delta R\ >\ 0.4$ \\
\hline
\end{tabular}
\caption{Definition of muidcombined muon reconstruction cuts.}
\label{TABLE-OBJ-MU}
\end{center}
\end{table}


\section{Jet definition}
\label{SECTION-OBJ-JET}
As described in Section~\ref{}, bare quarks and gluons undergo hadronization before they can interact with the detector. This forms a multitude of tracks and calorimeter enrgy deposits in a spread, which is treated as a single object called a jet. Being of such a composite nature, jets are complicated objects and there are many different ways to define and reconstruct them.

This analysis uses the $anti-k_t$ algorithm~\cite{AntiKt} to define and reconstruct jets. The $anti-k_t$ algorithm starts with a list of all objects, in this case the calorimeter cell energies. From this list of objects, a list of all distances is computed where the distance between objects is defined in Equation~\ref{EQ-OBJ-ANTIKT}, and the distance between the object and the beam as defined in Equation~\ref{EQ-OBJ-ANTIKTBEAM}. If the minimum distance is between two objects then they are merged to form a new object, the original objects are removed from the list and the new object is added to the list, then all distances are recalculated. If the minimum distance is between an object and the beam then the object is classified as a jet and removed from the list. This process is repeated recursively until the object list is exhausted. In Equation~\ref{EQ-OBJ-ANTIKT} the paramter R is the characteristic size of the jet. Larger values of R produce fewer, wider jets which are more likely to contain products from more than one parton. Conversely smaller values of R produce more, smaller jets that may not contain all of the products of individual partons. For this analysis $R\ =\ 0.4$, consistent with other ATLAS top quark analyses. 

\begin{equation}
\label{EQ-OBJ-ANTIKT}
d_{ij} = min(p^{-2}_{T,i},p^{-2}_{T,j})\frac{\Delta\eta^{2}_{ij}+\Delta\phi^{2}_{ij}}{R^{2}}
\end{equation}

\begin{equation}
\label{EQ-OBJ-ANTIKTBEAM}
d_{i} = p^{-2}_{T}
\end{equation}

After forming jets with the $anti-k_t$ algorithm a correction is applied to each jet based on the jet's position and $p_T$ to correct for the specific response of each region of the detector. The corrected jets then have a series of quality cuts applied. Any jets with negative energy are removed as these are unphysical. If any jets are within $\Delta R\ <\ 0.2$ of an electron, the jet nearest the electron is removed because it is likely a double counting of the electron as a jet.

\subsection{Jet b-tagging}
\label{SECTION-OBJ-JET-BTAG}
Jets that originate from different particles can often exhibit different kinematics, and by analyzing the kinematics of a given jet it is possible to predict the flavor of the parton that produced the jet. This process is generally referred to as ``tagging'', with jets that pass the tagging criteria called ``tagged'' and jets that fail the tagging criteria called ``untagged''. In this analysis b-tagging is employed to sort jets based on how likely they are to have originated from a b quark (b-jets). Specifically, the MV1~\cite{MV1} b-tagging algorithm is used. 

MV1 is a neural network analysis of ATLAS b-tagging algorithms SV0, IP3D+SV1, and JetFitterCombNN. SV0, IP3D+SV1, and JetFitterCombNN all use the secondary vertex caused by the b quark's relatively long hadronization time, as discussed in Section~\ref{}, to distinguish b-jets from other jets. This secondary vertex can be parameterized into the transverse and longitudinal impact parameters ($d_0$ and $z_0$) which are the distances between the secondary and primary vertex in the radial or z projection respectively, or the decay length ($L_0$) which is the distance between the primary and secondary vertex. SV0, IP3D+SV1, and JetFitterCombNN use varying mixtures of these parameters as well as the parameters' significances, which are defined as the ratio of the parameter and its uncertainty, to discriminate between b-jets and all other jets~\cite{BTAGGING}. MV1 uses the outputs of these other b-tagging algorithms to produce a single weight that corresponds to how likely a jet is to have originated from a b quark. For this analysis a tagging cut on each jet is placed at the $70\%$ b tagging efficiency value, which means that $70\%$ of b quark initiated jets are expected to pass the cut and be tagged.

\section{Missing transverse energy definition}
\label{SECTION-OBJ-MET}
With particles colliding nearly head-on along the beam line, the sum of the products' momenta in the transverse plane should be approximately 0. However, neutrinos and some theoretical exotic particles are not expected to interact with the ATLAS detector, causing the measured final state to have an imbalance in the transverse momentum. The negative of this measured imbalance is called the missing transverse energy (MET). Unfortunately MET is sensitive to any mismeasurement in the detector as well as the possibility of being faked by an interacting particle missed by the detector by traveling through cracks or dead regions. To correct for these different sources individual calibrations are applied to soft jets, hard jets, electrons, muons, and cell-out energy fragments. Hard jets are jets as defined in Section~\ref{SECTION-OBJ-JET} with $p_T\ >\ 20\ GeV$. Soft jets are jets as defined in Section~\ref{SECTION-OBJ-JET} with $20\ GeV\ >\ p_T\ >\ 10\ GeV$. Electrons and muons are defined using the definitions in Section~\ref{SECTION-OBJ-EL} and Section~\ref{SECTION-OBJ-MU} respectively. Cell-out energy fragments are any energy in the calorimeter which is not included in the other objects~\cite{MET}.


